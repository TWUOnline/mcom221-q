% Options for packages loaded elsewhere
\PassOptionsToPackage{unicode}{hyperref}
\PassOptionsToPackage{hyphens}{url}
\PassOptionsToPackage{dvipsnames,svgnames,x11names}{xcolor}
%
\documentclass[
  letterpaper,
  DIV=11,
  numbers=noendperiod]{scrreprt}

\usepackage{amsmath,amssymb}
\usepackage{iftex}
\ifPDFTeX
  \usepackage[T1]{fontenc}
  \usepackage[utf8]{inputenc}
  \usepackage{textcomp} % provide euro and other symbols
\else % if luatex or xetex
  \usepackage{unicode-math}
  \defaultfontfeatures{Scale=MatchLowercase}
  \defaultfontfeatures[\rmfamily]{Ligatures=TeX,Scale=1}
\fi
\usepackage{lmodern}
\ifPDFTeX\else  
    % xetex/luatex font selection
\fi
% Use upquote if available, for straight quotes in verbatim environments
\IfFileExists{upquote.sty}{\usepackage{upquote}}{}
\IfFileExists{microtype.sty}{% use microtype if available
  \usepackage[]{microtype}
  \UseMicrotypeSet[protrusion]{basicmath} % disable protrusion for tt fonts
}{}
\makeatletter
\@ifundefined{KOMAClassName}{% if non-KOMA class
  \IfFileExists{parskip.sty}{%
    \usepackage{parskip}
  }{% else
    \setlength{\parindent}{0pt}
    \setlength{\parskip}{6pt plus 2pt minus 1pt}}
}{% if KOMA class
  \KOMAoptions{parskip=half}}
\makeatother
\usepackage{xcolor}
\ifLuaTeX
  \usepackage{luacolor}
  \usepackage[soul]{lua-ul}
\else
  \usepackage{soul}
  
\fi
\setlength{\emergencystretch}{3em} % prevent overfull lines
\setcounter{secnumdepth}{5}
% Make \paragraph and \subparagraph free-standing
\makeatletter
\ifx\paragraph\undefined\else
  \let\oldparagraph\paragraph
  \renewcommand{\paragraph}{
    \@ifstar
      \xxxParagraphStar
      \xxxParagraphNoStar
  }
  \newcommand{\xxxParagraphStar}[1]{\oldparagraph*{#1}\mbox{}}
  \newcommand{\xxxParagraphNoStar}[1]{\oldparagraph{#1}\mbox{}}
\fi
\ifx\subparagraph\undefined\else
  \let\oldsubparagraph\subparagraph
  \renewcommand{\subparagraph}{
    \@ifstar
      \xxxSubParagraphStar
      \xxxSubParagraphNoStar
  }
  \newcommand{\xxxSubParagraphStar}[1]{\oldsubparagraph*{#1}\mbox{}}
  \newcommand{\xxxSubParagraphNoStar}[1]{\oldsubparagraph{#1}\mbox{}}
\fi
\makeatother


\providecommand{\tightlist}{%
  \setlength{\itemsep}{0pt}\setlength{\parskip}{0pt}}\usepackage{longtable,booktabs,array}
\usepackage{calc} % for calculating minipage widths
% Correct order of tables after \paragraph or \subparagraph
\usepackage{etoolbox}
\makeatletter
\patchcmd\longtable{\par}{\if@noskipsec\mbox{}\fi\par}{}{}
\makeatother
% Allow footnotes in longtable head/foot
\IfFileExists{footnotehyper.sty}{\usepackage{footnotehyper}}{\usepackage{footnote}}
\makesavenoteenv{longtable}
\usepackage{graphicx}
\makeatletter
\newsavebox\pandoc@box
\newcommand*\pandocbounded[1]{% scales image to fit in text height/width
  \sbox\pandoc@box{#1}%
  \Gscale@div\@tempa{\textheight}{\dimexpr\ht\pandoc@box+\dp\pandoc@box\relax}%
  \Gscale@div\@tempb{\linewidth}{\wd\pandoc@box}%
  \ifdim\@tempb\p@<\@tempa\p@\let\@tempa\@tempb\fi% select the smaller of both
  \ifdim\@tempa\p@<\p@\scalebox{\@tempa}{\usebox\pandoc@box}%
  \else\usebox{\pandoc@box}%
  \fi%
}
% Set default figure placement to htbp
\def\fps@figure{htbp}
\makeatother

\KOMAoption{captions}{tableheading,figureheading}
\makeatletter
\@ifpackageloaded{tcolorbox}{}{\usepackage[skins,breakable]{tcolorbox}}
\@ifpackageloaded{fontawesome5}{}{\usepackage{fontawesome5}}
\definecolor{quarto-callout-color}{HTML}{909090}
\definecolor{quarto-callout-note-color}{HTML}{0758E5}
\definecolor{quarto-callout-important-color}{HTML}{CC1914}
\definecolor{quarto-callout-warning-color}{HTML}{EB9113}
\definecolor{quarto-callout-tip-color}{HTML}{00A047}
\definecolor{quarto-callout-caution-color}{HTML}{FC5300}
\definecolor{quarto-callout-color-frame}{HTML}{acacac}
\definecolor{quarto-callout-note-color-frame}{HTML}{4582ec}
\definecolor{quarto-callout-important-color-frame}{HTML}{d9534f}
\definecolor{quarto-callout-warning-color-frame}{HTML}{f0ad4e}
\definecolor{quarto-callout-tip-color-frame}{HTML}{02b875}
\definecolor{quarto-callout-caution-color-frame}{HTML}{fd7e14}
\makeatother
\makeatletter
\@ifpackageloaded{bookmark}{}{\usepackage{bookmark}}
\makeatother
\makeatletter
\@ifpackageloaded{caption}{}{\usepackage{caption}}
\AtBeginDocument{%
\ifdefined\contentsname
  \renewcommand*\contentsname{Table of contents}
\else
  \newcommand\contentsname{Table of contents}
\fi
\ifdefined\listfigurename
  \renewcommand*\listfigurename{List of Figures}
\else
  \newcommand\listfigurename{List of Figures}
\fi
\ifdefined\listtablename
  \renewcommand*\listtablename{List of Tables}
\else
  \newcommand\listtablename{List of Tables}
\fi
\ifdefined\figurename
  \renewcommand*\figurename{Figure}
\else
  \newcommand\figurename{Figure}
\fi
\ifdefined\tablename
  \renewcommand*\tablename{Table}
\else
  \newcommand\tablename{Table}
\fi
}
\@ifpackageloaded{float}{}{\usepackage{float}}
\floatstyle{ruled}
\@ifundefined{c@chapter}{\newfloat{codelisting}{h}{lop}}{\newfloat{codelisting}{h}{lop}[chapter]}
\floatname{codelisting}{Listing}
\newcommand*\listoflistings{\listof{codelisting}{List of Listings}}
\makeatother
\makeatletter
\makeatother
\makeatletter
\@ifpackageloaded{caption}{}{\usepackage{caption}}
\@ifpackageloaded{subcaption}{}{\usepackage{subcaption}}
\makeatother

\usepackage{bookmark}

\IfFileExists{xurl.sty}{\usepackage{xurl}}{} % add URL line breaks if available
\urlstyle{same} % disable monospaced font for URLs
\hypersetup{
  pdftitle={MCOM 221},
  pdfauthor={TWU Online},
  colorlinks=true,
  linkcolor={blue},
  filecolor={Maroon},
  citecolor={Blue},
  urlcolor={Blue},
  pdfcreator={LaTeX via pandoc}}


\title{MCOM 221}
\author{TWU Online}
\date{Feb 3, 2025}

\begin{document}
\maketitle

\renewcommand*\contentsname{Table of contents}
{
\hypersetup{linkcolor=}
\setcounter{tocdepth}{2}
\tableofcontents
}

\bookmarksetup{startatroot}

\chapter*{Welcome}\label{welcome}
\addcontentsline{toc}{chapter}{Welcome}

\markboth{Welcome}{Welcome}

This is the course book for \textbf{MCOM 221: Digital Filmmaking}. This
book is divided into thematic units of study to help you engage with the
materials. The course resources and learning activities are designed not
only to help prepare you for the course assessments, but also to give
you opportunities to practice various skills.

\begin{tcolorbox}[enhanced jigsaw, opacityback=0, colframe=quarto-callout-note-color-frame, leftrule=.75mm, colback=white, toprule=.15mm, breakable, arc=.35mm, rightrule=.15mm, bottomrule=.15mm, left=2mm]
\begin{minipage}[t]{5.5mm}
\textcolor{quarto-callout-note-color}{\faInfo}
\end{minipage}%
\begin{minipage}[t]{\textwidth - 5.5mm}

Please read the full course syllabus located on the Course Home page in
Moodle. It includes key information about the course schedule,
assignments, and policies.

\end{minipage}%
\end{tcolorbox}

\section*{Course Activities}\label{course-activities}
\addcontentsline{toc}{section}{Course Activities}

\markright{Course Activities}

Below is some key information on features you may see throughout the
course.

\begin{tcolorbox}[enhanced jigsaw, opacityback=0, colframe=quarto-callout-note-color-frame, leftrule=.75mm, arc=.35mm, rightrule=.15mm, colbacktitle=quarto-callout-note-color!10!white, titlerule=0mm, colback=white, toprule=.15mm, bottomtitle=1mm, breakable, toptitle=1mm, title={Learning Activity}, coltitle=black, bottomrule=.15mm, left=2mm, opacitybacktitle=0.6]

This box will prompt you to engage in course concepts by:

\begin{itemize}
\tightlist
\item
  Viewing resources and reflecting on your experience and/or learning.
\item
  Checking your understanding to make sure you are ready for what
  follows. Ways to check your learning might include self-check quizzes
  or questions for discussion.
\end{itemize}

\begin{tcolorbox}[enhanced jigsaw, opacityback=0, colframe=quarto-callout-note-color-frame, leftrule=.75mm, colback=white, toprule=.15mm, breakable, arc=.35mm, rightrule=.15mm, bottomrule=.15mm, left=2mm]

Working through course activities will help you to meet the learning
outcomes and successfully complete your assessments.

\end{tcolorbox}

\end{tcolorbox}

Below is an accordion.

\begin{tcolorbox}[enhanced jigsaw, opacityback=0, colframe=quarto-callout-note-color-frame, leftrule=.75mm, colback=white, toprule=.15mm, breakable, arc=.35mm, rightrule=.15mm, bottomrule=.15mm, left=2mm]

\vspace{-3mm}\textbf{This is an accordion. Click/tap this banner to show/hide the content.}\vspace{3mm}

An accordion may contain extra content such as worked examples or sample
answers.

\end{tcolorbox}

\bookmarksetup{startatroot}

\chapter{Project Excellence and
Professionalism}\label{project-excellence-and-professionalism}

\section*{Overview}\label{overview}
\addcontentsline{toc}{section}{Overview}

\markright{Overview}

Welcome to MCOM 221! You are about to embark on a creative adventure.
The specific goal of this journey is to create attention-grabbing and
alluring short films. However, the general goal is far more valuable as
you learn things about yourself you never knew. This course will equip
you with the knowledge to develop transferable life skills that will
help you in your career and professional life after you complete your
education.

In addition, there is a larger element of this course that connects us
to people from the past, present, and future. Namely, storytelling.

According to communications theorist Walter Fischer, human beings are
more than \emph{Homo sapiens}, we are Homo \emph{narrans}, man the
storyteller. Storytelling is hardwired into our brains and our cultures
and history. Stories are how we make sense of our life and our world.
This is why religion, philosophy, literature, and myth have been vital
to human culture and understanding. Even science is a story---it tells
us how the world works. But it cannot tell us why. Only story-based
meaning can reveal why life is important, how we should live our lives,
and why there is something rather than nothing. Think of the Bible. If
one were to take out the stories, it would be a thin book.

Don't let these ideas scare you. This is a creative course not a
philosophic one. The point here is that as you increase your
storytelling skills---a vital part of this course---you will join the
community of narrators throughout human history who have shared their
wisdom, insight, and understanding. This knowledge serves as the
foundation for building identities, developing traditions, and
discovering the types of meaning that make life worth living across all
cultures.

This course will focus on one aspect of narratives, namely visual
storytelling. How do we use pictures (and sounds) to create meaning and
emotionally connect people to ourselves and each other? This will be the
deeper theme operating beneath the surface of this course. (In film
terms it will be the subtext beneath the text.)

In practical terms, the better you are at communicating stories
(relative to each profession), the better you will be at your job.
Fasten your seat belt, you are about to find out why we have focused on
the above to start this course.

\subsection*{Topics}\label{topics}
\addcontentsline{toc}{subsection}{Topics}

This unit is divided into the following topics:

\begin{enumerate}
\def\labelenumi{\arabic{enumi}.}
\tightlist
\item
  Course Introduction
\item
  Excellence and Professional and Personal Development
\item
  15 Tips for Doing Your Best
\item
  Transferable Skills
\end{enumerate}

\subsection*{Learning Outcomes}\label{learning-outcomes}
\addcontentsline{toc}{subsection}{Learning Outcomes}

When you have completed this unit, you will be able to:

\begin{itemize}
\tightlist
\item
  Describe excellence and why it is important
\item
  Define what constitutes an excellent film project
\item
  Articulate the big picture of why stories and creativity are important
\item
  Determine potential transferable life skills
\item
  Self-assess your strengths and weaknesses
\item
  Plan what you want to focus on during the course in terms of your
  professional and personal growth.
\end{itemize}

\subsection*{Learning Activities}\label{learning-activities}
\addcontentsline{toc}{subsection}{Learning Activities}

Here is a checklist of learning activities you will benefit from in
completing this unit. You may find it useful for planning your work.

\begin{itemize}
\tightlist
\item
  The Importance of the Film Journal: Write your first two Film Journal
  entries for this unit.
\item
  Read and Reflect: Read the Introduction which sets up the course ahead
  and Chapter One ``Project Excellence and Professionalism'' in the
  course textbook.
\item
  Reflect on the 15 Tips for Doing Your Best
\end{itemize}

\begin{tcolorbox}[enhanced jigsaw, opacityback=0, colframe=quarto-callout-note-color-frame, leftrule=.75mm, colback=white, toprule=.15mm, breakable, arc=.35mm, rightrule=.15mm, bottomrule=.15mm, left=2mm]
\begin{minipage}[t]{5.5mm}
\textcolor{quarto-callout-note-color}{\faInfo}
\end{minipage}%
\begin{minipage}[t]{\textwidth - 5.5mm}

Working through course activities will help you to meet the learning
outcomes and successfully complete your assessments.

\end{minipage}%
\end{tcolorbox}

\subsection*{Assessment}\label{assessment}
\addcontentsline{toc}{subsection}{Assessment}

\textbf{Course Journal}

After completing this unit, including the learning activities, you are
asked to make sure you are doing journal entries and when appropriate to
share your responses with your facilitator and classmates when you meet.

Note that entries are expected after every unit. Your journal
reflections are submitted at the end of the course as part of the
Journal One: Personal Journal and self assessment.

\begin{tcolorbox}[enhanced jigsaw, opacityback=0, colframe=quarto-callout-note-color-frame, leftrule=.75mm, colback=white, toprule=.15mm, breakable, arc=.35mm, rightrule=.15mm, bottomrule=.15mm, left=2mm]
\begin{minipage}[t]{5.5mm}
\textcolor{quarto-callout-note-color}{\faInfo}
\end{minipage}%
\begin{minipage}[t]{\textwidth - 5.5mm}

Please see the Assessment section in Moodle for assignment details as
well as the grading criteria.

\end{minipage}%
\end{tcolorbox}

\subsection*{Resources}\label{resources}
\addcontentsline{toc}{subsection}{Resources}

Here are the resources you will need to complete this unit.

\begin{itemize}
\tightlist
\item
  Introduction and Chapter One of the course text: \emph{Digital
  Filmmaking: A Beginner's Guide to Mastering the Craft}, by Ned
  Vankevich (e-text)
\item
  \href{https://karenbanes.com/how-to-start-a-creative-journal}{How to
  Start a Creative Journal} (Be sure to click on the internal links on
  this website)
\item
  \href{https://www.youtube.com/watch?v=hUTWo7_W0lc}{Watch: \emph{How to
  Journal Every Day for Increased Productivity, Clarity, and Mental
  Health}}
\item
  \href{https://brandyourself.com/blog/guide/how-to-grow-professionally/}{How
  To Grow Professionally}
\item
  \href{https://www.goodtherapy.org/learn-about-therapy/issues/creative-blocks}{Creative
  blocks website}
\item
  Other resources will be provided in the unit.
\end{itemize}

\section{Course Introduction}\label{course-introduction}

We begin Unit 1 marvelling at the magic of film and cinema. (Read the
course text Preface.) The ability of visually-centered storytelling to
cross cultures and to captivate, educate, and entertain audiences is a
universal contemporary phenomenon. For instance, how a series of
flickering lights and images projected at 24, 25, and 30 frames a second
can create emotions and experiences we all share is a wondrous mystery.

Less mysterious is how this is done. There are techniques, rules,
guidelines, and practices that can help us reach people in the ways that
films, TV shows, streaming Internet programs, news, and the host of
other visually-centered can make us laugh, cry, and emotionally move and
engage us. The worst thing that can be said by an audience is that ``I
want my five, ten, sixty, or ninety minutes back (depending on the
length of what we have viewed). Learning the methods to avoid this and
to engage our audience will be central to this course.

\textbf{WHAT LIES AHEAD}

This course will be divided up into ten segments or units. Each segment
(or chapter) is self-contained but each section is inter-connected and
vital for the others. A word to the wise---do note not skip a section,
even if you think that you already know it. Each section of the course
will build upon the previous one.

See the overview of the course and the ten units in the Introduction to
the course text: \emph{Introduction to Digital Filmmaking: A Beginner's
Guide to Mastering the Craft}.

Of special note, Unit Ten and Chapter Ten will be a summary and a
celebration of what you have learned during the course. It will also be
a time for awards to be handed out for outstanding and excellent work
such as Awards for Best film, Best Director, Best Cinematography, Best
Editing., Best Story, and Most Imaginative and Most Improved Filmmaker,
etc. This is noted here to give you a goal to work towards: namely, to
win one of the top awards.

As can be gleaned from the above, this course will start with the most
basic elements and proceed to creating a short film with a strong
beginning, middle, and end. As you move through the course keep the
mindset that this will be a fun, enjoyable, and exciting adventure.

\subsection{Activity: The Importance of the Film
Journal}\label{activity-the-importance-of-the-film-journal}

\begin{tcolorbox}[enhanced jigsaw, opacityback=0, colframe=quarto-callout-note-color-frame, leftrule=.75mm, arc=.35mm, rightrule=.15mm, colbacktitle=quarto-callout-note-color!10!white, titlerule=0mm, colback=white, toprule=.15mm, bottomtitle=1mm, breakable, toptitle=1mm, title={Learning Activity}, coltitle=black, bottomrule=.15mm, left=2mm, opacitybacktitle=0.6]

During this course you are encouraged to keep a ``Film Journal.'' This
is important for many reasons, including helping you to brainstorm
ideas, as well as ponder and process what you are learning. The journal
will also help you to keep a record of your course activities, and note
feedback of what does and does not work in film projects of your fellow
course members. In addition, some of the exercises for this course will
not be graded (for reasons that will be explained ahead) but this does
not mean you will not be accountable for doing them. Your journal will
be submitted at the end of the course and will play a role in the grade
you receive. In short, students who engage with the course well will in
general receive higher marks.

Also, note that you may be asked to use your journal entries to
participate in discussions, presentations, and other learning activities
in the FAR Centre Facilitated Learning Labs. Please check with your
facilitator about specific due dates for activities, including journal
responses.

\textbf{Unit 1 Film Journal Entry:}

Based on the importance of the course journal make your first two
entries for this unit:

\begin{itemize}
\tightlist
\item
  Entry One: After reading the course text Introduction, describe your
  best take-aways of what you have learned.
\item
  Entry Two:Study the recommended how-to-write-a-journal resources and
  log what you learned from them and how it will help you during the
  course.

  \begin{itemize}
  \tightlist
  \item
    \href{https://karenbanes.com/how-to-start-a-creative-journal/}{How
    to Start a Creative Journal}
  \item
    \href{https://www.youtube.com/watch?v=hUTWo7_W0lc}{Watch: \emph{How
    to Journal Every Day for Increased Productivity, Clarity, and Mental
    Health}}
  \end{itemize}
\end{itemize}

\url{https://www.youtube-nocookie.com/embed/hUTWo7_W0lc}

\begin{tcolorbox}[enhanced jigsaw, opacityback=0, colframe=quarto-callout-note-color-frame, leftrule=.75mm, colback=white, toprule=.15mm, breakable, arc=.35mm, rightrule=.15mm, bottomrule=.15mm, left=2mm]

\emph{Note: Your journal entries do not have to be long but they should
be detailed and specific enough to indicate that you have engaged the
course material and projects.}

\end{tcolorbox}

\end{tcolorbox}

\subsection{Activity: Film Terms and Vocabulary Needed for this
Course}\label{activity-film-terms-and-vocabulary-needed-for-this-course}

\begin{tcolorbox}[enhanced jigsaw, opacityback=0, colframe=quarto-callout-note-color-frame, leftrule=.75mm, arc=.35mm, rightrule=.15mm, colbacktitle=quarto-callout-note-color!10!white, titlerule=0mm, colback=white, toprule=.15mm, bottomtitle=1mm, breakable, toptitle=1mm, title={Learning Activity}, coltitle=black, bottomrule=.15mm, left=2mm, opacitybacktitle=0.6]

The course text Introduction also draws attention to the fact that you
will learn many new terms and concepts during this course and that it is
important for you to find resources that you can consult throughout the
course to re-enforce their meaning and application to the course.
Several websites will be cited here such as:

\begin{itemize}
\tightlist
\item
  studiobinder movie film terms
\item
  A more comprehensive glossary can be found at Wikipedia's Glossary of
  Motion Picture terms ; and
\item
  AMC Film Terms Glossary
\end{itemize}

Commit to learning \textbf{\emph{five}} new film terms each day by
starting with A and going to Z. The more you focus on this the better
you will do in the course. Don't worry if you do not understand a term.
We will be covering many of them during the course. The important thing
is to become familiar with important film terms and vocabulary.

\textbf{Questions to Consider}

\begin{itemize}
\tightlist
\item
  Why is this course important to you?
\item
  What do you hope to learn and how can it aid your future?
\end{itemize}

\begin{tcolorbox}[enhanced jigsaw, opacityback=0, colframe=quarto-callout-note-color-frame, leftrule=.75mm, colback=white, toprule=.15mm, breakable, arc=.35mm, rightrule=.15mm, bottomrule=.15mm, left=2mm]

\emph{Note: Your journal entries do not have to be long but they should
be detailed and specific enough to indicate that you have engaged the
course material and projects.}

\end{tcolorbox}

\end{tcolorbox}

\section{Excellence and Professional and Personal
Development}\label{excellence-and-professional-and-personal-development}

This course will focus on the classical approach to filmmaking, that is,
making motivated film stories and executing them in ways that help
audiences to fully engage the story and characters without unnecessary
distractions. In doing so, the course has a lofty goal: to aim for
excellence.

Excellence means a high standard of being good. This is of course
relative to being an Introductory course with students who are just
beginning their filmmaking journey, or are taking this course as an
elective. But like any course, if you shoot for an A you will do better
than settling for average and mediocrity.

Excellence is important in filmmaking because our competition is great.
Film is a public medium. It is designed to be shown to audiences and the
quality of our projects, like many things we do in life, will be judged
whether we want them to be or not.

One of the transferable life skills of this course is to help us be more
professional. Thus the more we shoot for excellence, the more we will
develop abilities that will serve us better in our careers ahead.

At the same time we must not confuse excellence with perfection. Our
beginning films and projects will have lots of mistakes and things that
do not work. This should not deter us from our goal of excellence. All
great artists, athletes, business leaders, and a host of other
professional make mistakes. The key is to learn from them and to keep
improving our craft.

\begin{tcolorbox}[enhanced jigsaw, opacityback=0, colframe=quarto-callout-note-color-frame, leftrule=.75mm, colback=white, toprule=.15mm, breakable, arc=.35mm, rightrule=.15mm, bottomrule=.15mm, left=2mm]

HELPFUL HINT: For insight into how great artists, athletes, and
performers use mistakes to grow in their art and craft see the following
book by Daniel Coyle: \href{http://danielcoyle.com/the-talent-code/}{The
Talent Code: Greatness Isn'T Born. It'S Grown. Here'S How}.

\end{tcolorbox}

Our goal then is to learn how to seek to produce visually-driven,
digital film stories that are technically and artistically proficient,
meaningfully engaging, motivated, and fraught with excellence.

As you can see excellence and professionalism go hand in hand. In order
to grow and excel in a craft we need to grow and excel at being
professional. Implied here is that we also need to mature and grow
personally as we develop the character traits necessary for being a
mature, moral, and responsible person.

\subsection{Activity: Read and Reflect}\label{activity-read-and-reflect}

\begin{tcolorbox}[enhanced jigsaw, opacityback=0, colframe=quarto-callout-note-color-frame, leftrule=.75mm, arc=.35mm, rightrule=.15mm, colbacktitle=quarto-callout-note-color!10!white, titlerule=0mm, colback=white, toprule=.15mm, bottomtitle=1mm, breakable, toptitle=1mm, title={Learning Activity}, coltitle=black, bottomrule=.15mm, left=2mm, opacitybacktitle=0.6]

Read Chapter One ``Project Excellence and Professionalism'' in the
course textbook.

As you study and ponder what excellence and what professional and
personal development mean, log in your journal why they are important
and how you hope to focus on them during the course. Which character
traits are your strongest and which do you need to work on? How will you
do this? To help you with these entries study this resource:

\href{https://brandyourself.com/blog/guide/how-to-grow-professionally/}{How
To Grow Professionally}

\end{tcolorbox}

\section{15 Tips for Doing Your Best}\label{tips-for-doing-your-best}

The following will help you to not just do well in this Film project,
but will make it fun and more enjoyable and hopefully be one of your
favourite courses ever:

\begin{enumerate}
\def\labelenumi{\arabic{enumi}.}
\tightlist
\item
  \textbf{\emph{Keep your eye on the prize.}} Creating an excellent
  project and growing as a professional and better person through the
  process.
\item
  \textbf{\emph{Meeting Deadlines.}} Being on time is vital in any
  profession, especially in filmmaking where the cost of feature film
  shoots can be thousands of dollars per hour. Plan your time and
  schedule accordingly so that you make your due date. Failure to do so
  will result in reduced marks on your assignment.
\item
  \textbf{\emph{Avoiding Distractions.}} In our Internet-cell phone era
  we are inundated and surrounded by 24/7 distractions. If we succumb to
  them we will not meet our deadlines and thus delay our professional
  and personal development. Learning to discern what is vital and what
  is the tyranny of the unimportant and unnecessary urgent is a critical
  skill to develop today. Focus on it and you will go further in the
  course. (\textbf{Tip:} \emph{turn off all notifications on your cell
  phone. In fact, turn off or put your cell phone away while you are
  working on your Film course. You will find that this will help you to
  focus on and finish your assignments quicker and with more
  creativity}).
\item
  \textbf{\emph{Focussing on the positive}}. Most of the assignments
  ahead will be challenging. Cultivate a ``can do attitude.'' For
  context think about your taking a piano lesson course. You would not
  expect to sit down and immediately play chords and songs. You would
  have to learn to play the notes, then chords, then songs over a period
  of time. Filmmaking is similar. There are notes to learn (film shots),
  chords to learn (film scenes), before we can create melodies (film
  sequences) and songs (full films).
\item
  \textbf{\emph{It takes time and practice to master film}}. If you find
  you have time, or if you are truly interested in learning the world of
  film or becoming a filmmaker, I encourage you to go beyond the course
  exercises. For example, continue to find and shoot interesting visual
  compositions (they are all around you) during the course, or practice
  fascinating and more complicated motion shots and two and three person
  shots. The more you do this the faster you will grow in your craft.
  Think about musicians. They practice over and over everyday.
\item
  \textbf{\emph{Study the practical tips and helpful hints in these
  units}}. Pay attention to them. They will help fast-track your skills
  growth.
\item
  \textbf{\emph{Find and do what you love}}. During this course you will
  discover things you love to do and those that you do not. As a
  filmmaker, every skill set is vital for the final production -- but
  you will discover which roles you are better at and enjoy the most.
  Even if you do not go into filmmaking as a career this self-insight
  will be invaluable as you make life decisions ahead.
\item
  \textbf{\emph{Make a firm commitment to succeed}}. When we are
  double-minded or not sure of what we want, we will waiver and fall
  prey to distraction and self-sabotage. Resolve not to give up as you
  make your projects despite what ``Murphy's Law'' tosses your way.
  (\emph{To be covered later.})
\item
  \textbf{\emph{Strive for balance}}. The emphasis on excellence and
  professional and personal development must not make us ``lose our
  soul.'' Too many people get caught up in goals and ambition and end up
  failing in their marriage and relationships. Remember we are physical,
  mental, spiritual, emotional, and relational beings and must find a
  balance that lets us grow in each part equally. This is the key to
  personal flourishing.
\item
  \textbf{\emph{Create strong and healthy relationships}}. During this
  course you will discover a lot about yourself and others. Some people
  will be easy to work with and some won't. Some will be diligent and
  some won't. Choose your teammates wisely and if you have a winning
  rapport with them, you might want to work with them on the next
  project. One of the keys to success in the professional world is
  cultivating strong and fruitful relationships. Start the process here.
\item
  \textbf{\emph{Follow your intuition}}. If you've never experimented
  with following your deeper instincts or ``gut feelings'' try it during
  this course. Often this will take you to new creative places. It might
  not always work but you will learn a lot from doing so.
\item
  \textbf{\emph{Take risks}}. We don't grow unless we try new things and
  fail. They key is not to get discouraged but to shake off a loss and
  fight to win again.
\item
  \textbf{\emph{Practice, practice, practice. Learn, learn, learn}}.
  Your best competition is continually practicing and learning their
  craft. So should you. You need to cultivate both a strong work ethic
  and a smart ethic. Take the time to think things through before
  launching into something.
\item
  \textbf{\emph{Take responsibility for yourself and your future}}.
  Playing the blame game or the victim or continually making excuses
  will not get you ahead in life.
\item
  \textbf{\emph{Find inspiring quotes and pin them to your computer or
  wall}}. Most of us need encouragement especially when the going gets
  tough. Seek out inspiration that helps keep you going. The following
  strike some of the themes of this course:
\end{enumerate}

\begin{quote}
\emph{``Success is not final; failure is not fatal: It is the courage to
continue that counts.''} - Winston S. Churchill
\end{quote}

\begin{quote}
\emph{``It is better to fail in originality than to succeed in
imitation.''} - Herman Melville
\end{quote}

\begin{quote}
\emph{``You will become clever through your mistakes.''} - German
Proverb in ``The Talent Code''
\end{quote}

\subsection{Activity: Reflection}\label{activity-reflection}

\begin{tcolorbox}[enhanced jigsaw, opacityback=0, colframe=quarto-callout-note-color-frame, leftrule=.75mm, arc=.35mm, rightrule=.15mm, colbacktitle=quarto-callout-note-color!10!white, titlerule=0mm, colback=white, toprule=.15mm, bottomtitle=1mm, breakable, toptitle=1mm, title={Learning Activity}, coltitle=black, bottomrule=.15mm, left=2mm, opacitybacktitle=0.6]

Which of the tips in this section appeal to you most? Which do you need
to focus on during the course? Write your response in your journal and
share with your facilitator and/or classmates which are most important
and why. Find an inspiring quote and share it with your peers.

\end{tcolorbox}

\section{Transferable Skills}\label{transferable-skills}

This class is about more than making films. If followed diligently, you
will learn a lot about yourself, others, and the world around you. You
will also learn skills and traits that will play an important role in
your life journey ahead. Some of the transferable skills can include:

\begin{itemize}
\tightlist
\item
  Growing in your creative skills
\item
  Growing in your teamwork skills
\item
  Growing in your critiquing skills
\item
  Growing in your ability to overcome obstacles
\item
  Meeting deadlines and taking responsibility
\item
  Becoming a leader
\item
  Learning to take constructive criticism
\item
  Committing to continual learning and improvement
\item
  Developing a positive, can-do attitude
\item
  Crafting what you want to say and communicating it well
\item
  Learning the power of story and how it impacts many areas in our
  lives.
\end{itemize}

\subsection{Activity: Reflection on Life
Skills}\label{activity-reflection-on-life-skills}

\begin{tcolorbox}[enhanced jigsaw, opacityback=0, colframe=quarto-callout-note-color-frame, leftrule=.75mm, arc=.35mm, rightrule=.15mm, colbacktitle=quarto-callout-note-color!10!white, titlerule=0mm, colback=white, toprule=.15mm, bottomtitle=1mm, breakable, toptitle=1mm, title={Learning Activity}, coltitle=black, bottomrule=.15mm, left=2mm, opacitybacktitle=0.6]

\begin{itemize}
\tightlist
\item
  Which of the life skills above are important to you and why? Log this
  in your journal.
\item
  Which life skill(s) can you add to the list? Share your responses with
  your Facilitator and classmates.
\end{itemize}

\begin{tcolorbox}[enhanced jigsaw, opacityback=0, colframe=quarto-callout-note-color-frame, leftrule=.75mm, colback=white, toprule=.15mm, breakable, arc=.35mm, rightrule=.15mm, bottomrule=.15mm, left=2mm]

\emph{Note: Be sure to revisit these journal entries on Tips to Do Well
and Transferable Life Skills as you monitor your progress during the
course. One of the actives in Unit Ten will involve your writing a
summary of the best lessons you learned from the course and about
yourself. Again, this is a reminder of how important our course journal
will be.}

\end{tcolorbox}

\end{tcolorbox}

\section*{Summary}\label{summary}
\addcontentsline{toc}{section}{Summary}

\markright{Summary}

In this first unit, you have had the opportunity to learn about what
this course entails, why it is important, and the role that professional
and personal development will play. You also had a chance to respond to
tips of how to make the most of this course and what skills you can
transfer from it to your life.

\begin{tcolorbox}[enhanced jigsaw, opacityback=0, colframe=quarto-callout-note-color-frame, leftrule=.75mm, arc=.35mm, rightrule=.15mm, colbacktitle=quarto-callout-note-color!10!white, titlerule=0mm, colback=white, toprule=.15mm, bottomtitle=1mm, breakable, toptitle=1mm, title={Checking Your Learning}, coltitle=black, bottomrule=.15mm, left=2mm, opacitybacktitle=0.6]

Before you move on to the next unit, you may want to check to make sure
that you are able to:

\begin{itemize}
\tightlist
\item
  Describe excellence and why it is important.
\item
  Define what constitutes an excellent film project.
\item
  Articulate the big picture of why stories and creativity are
  important.
\item
  Determine potential transferable life skills.
\item
  Self-assessing your strengths and weaknesses.
\item
  Plan what you want to focus on during the course in terms of
  professional and personal growth.
\end{itemize}

\end{tcolorbox}

\bookmarksetup{startatroot}

\chapter{The Filmmaking Process}\label{the-filmmaking-process}

\section*{Overview}\label{overview-1}
\addcontentsline{toc}{section}{Overview}

\markright{Overview}

Filmmaking is exciting. We get to create and challenge ourselves as we
make projects that can wow and inspire people. It is also a challenging
art and craft. Film as a medium incorporates many other arts such as
acting (like theatre); sets and production design (like painting and
architecture); rhythm and timing (like music); scripts (like
literature); movement (like dance), plasticity of form (like sculpture);
as well as its unique combination of these many other arts.

Given the many skills needed to make a great film over time, filmmakers
have devised a system to make it easier for filmmakers. This process is
so well honed that some filmmakers, like experimental filmmakers, can do
it all: produce, direct, act, shoot, and edit their film. However, the
larger the film production and the bigger the budget, the more
complicated the process becomes and the more people need to do it.

Like in a story that has a beginning, middle and end, the filmmaking
process is divided into three sections or segments: pre-production,
production, and post-production. (There is a fourth component dealing
with marketing and distribution, but this will not be covered in this
course.) We start with an overall look at the filmmaking process. This
will give us the big picture of what will lie ahead so that we do not
get lost in the details of the many elements of filmmaking, and so that
we can place what we will be doing and practicing in a larger framework.

\subsection*{Topics}\label{topics-1}
\addcontentsline{toc}{subsection}{Topics}

This unit is divided into the following topics:

\begin{enumerate}
\def\labelenumi{\arabic{enumi}.}
\tightlist
\item
  The Three Phases of Filmmaking
\item
  Pre-Production
\item
  Production
\item
  Post-Production
\end{enumerate}

\subsection*{Learning Outcomes}\label{learning-outcomes-1}
\addcontentsline{toc}{subsection}{Learning Outcomes}

When you have completed this unit, you will be able to:

\begin{itemize}
\tightlist
\item
  Describe the big picture of the film production process
\item
  Determine what is involved in each phase and why it is important
\item
  Capture an overview of the processes you will do during the course
\item
  Define why excellence should be the benchmark of each phase of
  filmmaking.
\end{itemize}

\subsection*{Learning Activities}\label{learning-activities-1}
\addcontentsline{toc}{subsection}{Learning Activities}

Here is a checklist of learning activities you will benefit from in
completing this unit. You may find it useful for planning your work.

\begin{itemize}
\tightlist
\item
  Watch the video ``Shooting a Film- START to FINISH!''
\item
  Watch the video ``Pitching and Pre-Production'' to learn why pitching
  is important.
\item
  Read and study Chapter 2, and answer to the questions provided
\item
  Share in your journal what you learn from shooting a film from start
  to finish.
\end{itemize}

\begin{tcolorbox}[enhanced jigsaw, opacityback=0, colframe=quarto-callout-note-color-frame, leftrule=.75mm, colback=white, toprule=.15mm, breakable, arc=.35mm, rightrule=.15mm, bottomrule=.15mm, left=2mm]
\begin{minipage}[t]{5.5mm}
\textcolor{quarto-callout-note-color}{\faInfo}
\end{minipage}%
\begin{minipage}[t]{\textwidth - 5.5mm}

Working through course activities will help you to meet the learning
outcomes and successfully complete your assessments.

\end{minipage}%
\end{tcolorbox}

\subsection*{Assessment}\label{assessment-1}
\addcontentsline{toc}{subsection}{Assessment}

\textbf{Course Journal}

After completing this unit, including the learning activities, you are
asked to make sure you are doing journal entries and when appropriate to
share your responses with your facilitator and classmates when you meet.

Note that entries are expected after every unit. Your journal
reflections are submitted at the end of the course as part of the
Journal One: Personal Journal and self assessment.

\begin{tcolorbox}[enhanced jigsaw, opacityback=0, colframe=quarto-callout-note-color-frame, leftrule=.75mm, colback=white, toprule=.15mm, breakable, arc=.35mm, rightrule=.15mm, bottomrule=.15mm, left=2mm]
\begin{minipage}[t]{5.5mm}
\textcolor{quarto-callout-note-color}{\faInfo}
\end{minipage}%
\begin{minipage}[t]{\textwidth - 5.5mm}

Please see the Assessment section in Moodle for assignment details as
well as the grading criteria.

\end{minipage}%
\end{tcolorbox}

\subsection*{Resources}\label{resources-1}
\addcontentsline{toc}{subsection}{Resources}

Here are the resources you will need to complete this unit.

\begin{itemize}
\tightlist
\item
  Chapter Two of \ul{Digital Filmmaking: A Beginner's Guide to Mastering
  the Craft}, by Ned Vankevich (e-text)
\item
  \href{https://www.youtube.com/watch?v=JE53JL60ihc}{\textbf{Pitching
  and Pre-Production: Crash Course Film Production \#2}}
\item
  \href{https://www.youtube.com/watch?v=8NCLf9rF6IQ}{\textbf{Shooting a
  film - START to FINISH!}}
\item
  \href{https://www.youtube.com/watch?v=oj_Blr8JE1I}{\textbf{Filmmaking
  From Beginning to End: Preproduction to Production}} (Don't worry if
  you do not understand everything in these tutorials. You will at the
  right time as you follow this course.)
\item
  \href{https://www.youtube.com/watch?v=y9_LW5H2EC4}{\textbf{How to
  Shoot a Scene - Blocking Actors}}
\item
  Other resources will be provided in the unit.
\end{itemize}

\section{The Three Phases of
Filmmaking}\label{the-three-phases-of-filmmaking}

In Unit 1 we focused on project excellence and great storytelling and
why they are important. This unit will help obtain these goals by
breaking the filmmaking process down into three phases. Each phase
depends upon the other and if we fail to understand and execute each one
well, the other phases and the film itself will suffer.

The overall filmmaking process has not changed much over the past
century. In short, it involves three phases: Pre-Production---preparing
to shoot a project; Production---shooting or filming the project;
Post-Production---editing or putting all of the production elements
together in a unified final project.

Understanding these three phases is vital to making a great film. You
will not have to master each phase, no one can because there are too
many elements and skills involved depending on the complexity of the
film, its story, and its execution.

\subsection{Activity: Shooting a Film}\label{activity-shooting-a-film}

\begin{tcolorbox}[enhanced jigsaw, opacityback=0, colframe=quarto-callout-note-color-frame, leftrule=.75mm, arc=.35mm, rightrule=.15mm, colbacktitle=quarto-callout-note-color!10!white, titlerule=0mm, colback=white, toprule=.15mm, bottomtitle=1mm, breakable, toptitle=1mm, title={Learning Activity}, coltitle=black, bottomrule=.15mm, left=2mm, opacitybacktitle=0.6]

To get a visual representation of the overall filmmaking process view
the following resource:

\href{https://youtu.be/8NCLf9rF6IQ?si=5NNDABPg-kEZYxAY}{Watch:
\emph{Shooting a film - START to FINISH!}}

\url{https://www.youtube-nocookie.com/embed/8NCLf9rF6IQ}

Don't worry if you do not know all of the terms he is using such as
blocking, dolly, tilt up, etc. These terms will be covered in the units
ahead. For now, just take in the whole process and you will be amazed
how much easier it will be to understand the later units as you explore
the details of each phase.

For now, just watch the video and enjoy the process.

\begin{tcolorbox}[enhanced jigsaw, opacityback=0, colframe=quarto-callout-note-color-frame, leftrule=.75mm, colback=white, toprule=.15mm, breakable, arc=.35mm, rightrule=.15mm, bottomrule=.15mm, left=2mm]

\textbf{Helpful Hint:} \emph{Don't multi-task during the viewing of
resources. It will divide your attention and you will not absorb as much
content. As with most things in this course---be in the moment and focus
on what is at hand.}

\end{tcolorbox}

\end{tcolorbox}

\section{Pre-Production}\label{pre-production}

Like most things in life, if you do not have a strong foundation, what
follows after will wobble. Proper pre-production is the foundation of an
excellent film (and your up-coming exercises). The better you plan
something, the better the result.

There are many elements to pre-production. When followed properly they
enhance the 5 Cs of successful filmmaking (see the \textbf{Why
Pre-Production is Important} section in Chapter 2 of the course text):

\begin{itemize}
\tightlist
\item
  Creativity
\item
  Calmness
\item
  Co-ordination
\item
  Coherence
\item
  Consistency
\end{itemize}

Proper pre-production involves many important general steps (see
\textbf{What Proper Pre-Production Entails} section in Chapter 2 of the
course text).

\begin{itemize}
\tightlist
\item
  Planning
\item
  Anticipation
\item
  Strategizing
\item
  Testing
\end{itemize}

As can be gleaned from above, pre-production helps ensure that a film
project maximizes the creative process and minimizes chaos, confusion,
and uncertainty---the enemies of a good film.

There are many steps in the prep-production process highlighted in
Chapter 2. (See \textbf{Successful Steps of Pre-Production}.)

They involve:

\begin{itemize}
\tightlist
\item
  Creating a viable concept for a project
\item
  Knowing who the audience is and genre requirements (people who like
  comedy want to laugh, people who like horror want to be scared)
\item
  Creating a script to film
\item
  Creating storyboards and shot lists
\item
  Pitching your project to get funding, actors, crew members, etc.
\item
  Budgeting the cost of the film
\item
  Getting your actors and the right crew
\item
  Finding locations and props
\item
  Scheduling the shoot
\item
  Testing your gear to make sure you know how to use it and that it
  works properly.
\end{itemize}

As can be seen, there is a lot involved before we film. Luckily, we will
start with short exercises which will take minimal pre-production and
gradually build to the more detailed aspects later.

\begin{tcolorbox}[enhanced jigsaw, opacityback=0, colframe=quarto-callout-note-color-frame, leftrule=.75mm, colback=white, toprule=.15mm, breakable, arc=.35mm, rightrule=.15mm, bottomrule=.15mm, left=2mm]

Note: The lion's share of this unit and Chapter Two in the course book
is devoted to pre-production to emphasize how important this phase is.
Most people might think production is the most important part of the
filming process. In fact, each phase is. In Hollywood the development
process (creating the screenplay) and pre-production for a film often
takes far longer than production. Post-Production is also a longer
process than production when a major film is involved.

\end{tcolorbox}

\subsection{Activity: Pre-Production
Planning}\label{activity-pre-production-planning}

\begin{tcolorbox}[enhanced jigsaw, opacityback=0, colframe=quarto-callout-note-color-frame, leftrule=.75mm, arc=.35mm, rightrule=.15mm, colbacktitle=quarto-callout-note-color!10!white, titlerule=0mm, colback=white, toprule=.15mm, bottomtitle=1mm, breakable, toptitle=1mm, title={Learning Activity}, coltitle=black, bottomrule=.15mm, left=2mm, opacitybacktitle=0.6]

\begin{itemize}
\tightlist
\item
  Read the Pre-Production section of the course text. What part of the
  pre-production process appeals to you most? Why? Log this in your
  journal and share your responses with your facilitator and classmates
  when you meet for this unit.
\item
  View this resource and share with your facilitator and course mates
  why pitching is important:
\end{itemize}

\href{https://youtu.be/JE53JL60ihc?si=oVHz_PQEsKXc3GL_}{Watch:
\emph{Pitching and Pre-Production: Crash Course Film Production with
Lily Gladstone}}

\url{https://www.youtube-nocookie.com/embed/JE53JL60ihc}

\textbf{Questions to Consider}

\begin{itemize}
\tightlist
\item
  Murphy's Law states that ``what can go wrong will go wrong'' but many
  people forget the last component ``at the worse possible moment.''
  Have you ever experienced it? Share this experience with your
  facilitator and classmates, the result and if and how it was overcome.
\end{itemize}

\end{tcolorbox}

\section{Production}\label{production}

The production phase involves the shooting of the film, what is often
referred to as ``principle photography''. (Sometimes pick-up shots,
re-dos, and B-roll footage are done during post-production.) For many
filmmakers this is the most exciting part of filmmaking when they get a
chance to go on location and watch actors do their magic.

Like pre-production, the production phase must be attended to
thoughtfully and diligently or you will not have what you need to edit
your film and make your story work. This is why a large portion of this
course and the course text are dedicated to this phase of filmmaking,
which will only be covered briefly here.

In addition to filming the actors, locations, action sequences, etc.
production also includes recording on-location audio for the project.
Capturing, recording, and creating good sound, like cinematography,
takes lots of time and practice.

Most of the production process takes place on real locations such as
streets, schoolyards, restaurants, etc., and sets which are built for
the filming. If you have the budget, filming on sets such as kitchens,
living, rooms, police stations, etc. is easier than real locations. The
reason for this is that you can control the lighting and use of space
better and will not have extraneous and disruptive noises occurring in
the environment.

Production can also include green screen and CGI (computer generated
imagery) work, but these are normally for bigger budget productions
since they are specialized skills and it's costly to do them well. They
are also time consuming. (If you have After Effects and other special
effects software skills please consider using them in this course.)

Performance is a crucial part of the filming process. Many filmmakers
forget this and get so caught up in the shooting process that they
overlook or miss bad acting. Don't fall into this trap. You want to work
on having your actors deliver believable, motivated, non-melodramatic
performances. More on this later.

\section{Post-Production}\label{post-production}

The final phase of the filmmaking process involves post-production where
all elements of pre-production and production are ``cut'' or stitched
together to create a finished film.

Like production, the post process is hands-on and labor intensive. If
done digitally, it involves using software such as Adobe Premier Pro,
Apple Final Cut, Avid, DaVinici Resolve, etc. to edit the footage and
make transitions such as fades, dissolves, wipes, etc. (to be covered
later). This can be easy but a lot of work goes into choosing the best
and rights shots and editing them together.

There are four phases to cutting the picture of a film:

\begin{itemize}
\tightlist
\item
  Assembly Cut: where the shots are placed together in proper order
  without trimming them.
\item
  Rough Cut: where you begin to ``trim the fat'' from the beginning and
  end of shots to get a feel for how the story will flow.
\item
  Fine Cuts: which will involve multiple versions as you trim or cut
  shots and scenes that do not work. You can further hone the film of
  any ``excess'' fat that does not add to the effective presentation of
  the story and characters.
\item
  Final Cut: which is the ``locked'' picture version that the composer
  and sound design people need to add music and sound effects to precise
  moments in the story.
\end{itemize}

In addition to cutting the picture, post-production also entails:

\begin{itemize}
\tightlist
\item
  \textbf{Sound Design} where the right music and sound effects are
  chosen and added.
\item
  \textbf{Titling and Graphics} when the opening and ending film credits
  appear and any special written material that will appear on the screen
  such as: FIVE YEARS EARLIER.
\item
  \textbf{Test Screening} your film to those you trust to make sure the
  story works and is clear, and to test how an audience will respond.
\item
  \textbf{Color Correcting} where the flow of the color and light and
  dark nature of the images appears seamless and appropriate. This is
  where you correct over exposed and under expose shots or those with
  the wrong color temperature. (\emph{Look these terms up in the film
  glossary you are using during this course. Refer to Activity 1.2 in
  Unit 1})
\item
  \textbf{Audio Mixing} which involves finding the right levels and
  balance between the sound elements such as dialogue, music, sound
  effects, room tone, etc.
\item
  \textbf{Format Delivery}: What resolution will you use to export your
  film project? The wrong one can undermine the quality of your film.As
  can be gleaned, there is a lot to post-production but by carefully
  studying and practicing the units ahead it will make it easier.
\end{itemize}

If you are feeling overwhelmed and intimidated, you will not be alone.
This is a lot to do and learn and this is why the bigger film projects
can be done with team members.

\subsection{Activity: Reflect and
Share}\label{activity-reflect-and-share}

\begin{tcolorbox}[enhanced jigsaw, opacityback=0, colframe=quarto-callout-note-color-frame, leftrule=.75mm, arc=.35mm, rightrule=.15mm, colbacktitle=quarto-callout-note-color!10!white, titlerule=0mm, colback=white, toprule=.15mm, bottomtitle=1mm, breakable, toptitle=1mm, title={Learning Activity}, coltitle=black, bottomrule=.15mm, left=2mm, opacitybacktitle=0.6]

Based on this unit and your reading of Chapter 2 in the course text,
share in your journal why knowing the overall process is important and
what you learned most from this unit.

Share also if you are feeling intimidated. Remember all but one of the
graded assignments will be done in teams so you will not have to do it
all alone.

\end{tcolorbox}

\section*{Summary}\label{summary-1}
\addcontentsline{toc}{section}{Summary}

\markright{Summary}

In this unit, you learned about the overall process of making a film as
well as the steps involved in the following three phases:

\begin{itemize}
\tightlist
\item
  Pre-Production: this must be done before making a film to save time
  and help ensure quality, as you write the script, plan the production,
  and work out the logistics such as casting and finding locations and
  props.
\item
  Production: this involves shooting the film and where you use camera
  angles, camera motions, and the blocking of actors to reveal the
  meaning of your story and its characters.
\item
  Post-Production: this involves the editing and completing of your film
  where you cut the shots and scenes together, and includes adding
  titling, credits, special effects, music and sound design, and colour
  grading.
\end{itemize}

\begin{tcolorbox}[enhanced jigsaw, opacityback=0, colframe=quarto-callout-note-color-frame, leftrule=.75mm, arc=.35mm, rightrule=.15mm, colbacktitle=quarto-callout-note-color!10!white, titlerule=0mm, colback=white, toprule=.15mm, bottomtitle=1mm, breakable, toptitle=1mm, title={Checking Your Learning}, coltitle=black, bottomrule=.15mm, left=2mm, opacitybacktitle=0.6]

Before you move on to the next unit, you may want to check to make sure
that you are able to:

\begin{itemize}
\tightlist
\item
  Describe the big picture of the film production process
\item
  Determine what is involved in each phase and why it is important
\item
  Capture an overview of what lies ahead
\item
  Define why excellence should be the bench mark of each phase
\end{itemize}

\end{tcolorbox}

\bookmarksetup{startatroot}

\chapter{Visual Composition}\label{visual-composition}

\section*{Overview}\label{overview-2}
\addcontentsline{toc}{section}{Overview}

\markright{Overview}

We will now make our theory more practical. So far, we have focused on
project excellence and the three phases of filmmaking we will use to
create it. In this unit we will concentrate on where it all begins
visually -- the single image and how to craft it well.

This unit will also give you a chance to apply what you are learning in
a creative way. Don't worry if you think you are not creative or
artistic We will focus on simple elements and reveal how they can help
you see the world and capture it in exciting ways. We will do so by
focusing on composition---the way in which the elements in a picture or
image are arranged.

Visual composition will play an important role in this course, so taking
the time to learn it is well worth the investment.

\subsection*{Topics}\label{topics-2}
\addcontentsline{toc}{subsection}{Topics}

This unit is divided into the following topics:

\begin{enumerate}
\def\labelenumi{\arabic{enumi}.}
\tightlist
\item
  Elements of Composition
\item
  Principles of Composition
\item
  Photographic Themes
\end{enumerate}

\subsection*{Learning Outcomes}\label{learning-outcomes-2}
\addcontentsline{toc}{subsection}{Learning Outcomes}

When you have completed this unit, you should be able to:

\begin{itemize}
\tightlist
\item
  Describe the elements used in visual compositions
\item
  Define composition principles
\item
  Study the works of noted photographers and apply composition elements
  and principles to photographs you create
\item
  Create photos that reveal your understanding of the chapter's content
\end{itemize}

\subsection*{Learning Activities}\label{learning-activities-2}
\addcontentsline{toc}{subsection}{Learning Activities}

Here is a checklist of learning activities you will benefit from in
completing this unit. You may find it useful for planning your work.

\begin{itemize}
\tightlist
\item
  Read and study Chapter 3 of course e-text: \emph{Digital Filmmaking: A
  Beginner's Guide to Mastering the Craft.}
\item
  Read the article ``11 Thoughts: An Introduction to Photographic
  Composition'' by Todd Vorenkamp and reflect on the question provided.
\item
  Read the article ``20 Composition Techniques That Will Improve Your
  Photos'', and practice and apply techniques and principles of design.
\item
  Review the suggested websites to enrich your understanding of
  photographic themes and find images that stand out in intriguing,
  startling, fascinating, and emotionally moving ways.
\end{itemize}

\begin{tcolorbox}[enhanced jigsaw, opacityback=0, colframe=quarto-callout-note-color-frame, leftrule=.75mm, colback=white, toprule=.15mm, breakable, arc=.35mm, rightrule=.15mm, bottomrule=.15mm, left=2mm]
\begin{minipage}[t]{5.5mm}
\textcolor{quarto-callout-note-color}{\faInfo}
\end{minipage}%
\begin{minipage}[t]{\textwidth - 5.5mm}

Working through course activities will help you to meet the learning
outcomes and successfully complete your assessments.

\end{minipage}%
\end{tcolorbox}

\subsection*{Assessment}\label{assessment-2}
\addcontentsline{toc}{subsection}{Assessment}

\textbf{Course Journal}

After completing this unit, including the learning activities, you are
asked to make sure you are doing journal entries and when appropriate to
share your responses with your facilitator and classmates when you meet.

Note that entries are expected after every unit. Your journal
reflections are submitted at the end of the course as part of the
Journal One: Personal Journal and self assessment.

\textbf{Assignment 2: Composition Exercise}

\emph{This activity is ungraded, but part of journal entry}

After you have done some composition research and explored great
photographers and photos and their compositional techniques and
aesthetics, review Chapter 3 and go out and find and create five
outstanding visual compositions with your camera (cell phone, DSLR,
etc.). Share these with your family, friends, and peers and get their
feedback about what worked and did not work. When you have found your
best images upload them to the course online folder where the class
assignments will be stored.

Remember, the name of the game is to learn not to be perfect.

\begin{tcolorbox}[enhanced jigsaw, opacityback=0, colframe=quarto-callout-note-color-frame, leftrule=.75mm, colback=white, toprule=.15mm, breakable, arc=.35mm, rightrule=.15mm, bottomrule=.15mm, left=2mm]
\begin{minipage}[t]{5.5mm}
\textcolor{quarto-callout-note-color}{\faInfo}
\end{minipage}%
\begin{minipage}[t]{\textwidth - 5.5mm}

Please see the Assessment section in Moodle for assignment details as
well as the grading criteria.

\end{minipage}%
\end{tcolorbox}

\subsection*{Resources}\label{resources-2}
\addcontentsline{toc}{subsection}{Resources}

Here are the resources you will need to complete this unit.

\begin{itemize}
\tightlist
\item
  Chapter Three of course text: \emph{Digital Filmmaking: A Beginner's
  Guide to Mastering the Craft}, by Ned Vankevich (e-text)
\item
  \href{http://www.artyfactory.com/art_appreciation/visual-elements/visual-elements.html}{The
  Visual Elements: The Building Blocks of Composition in Art}
\item
  \href{https://petapixel.com/2016/09/14/20-composition-techniques-will-improve-photos/}{20
  Composition Techniques That Will Improve Your Photos}
\item
  \href{https://www.themuse.com/advice/taking-constructive-criticism-like-a-champ}{Taking
  Constructive Criticism Like a Champ}
\item
  \href{https://artplusmarketing.com/how-criticism-triggers-growth-52d775e97557}{How
  Criticism triggers Growth\ldots{}}
\item
  \href{https://www.getty.edu/education/teachers/building_lessons/principles_design.pdf}{Principles
  of Design}
\item
  Guidelines for Visual Composition Assignment
\item
  Other resources will be provided in the unit.
\end{itemize}

\section{Elements of Composition}\label{elements-of-composition}

At heart, visual composition means the intentional arrangement or
conscious activity of constructing the ``ingredients'' of an image. As
we grow in our filmmaking craft, we must continually be aware of what
takes place within the frame (the border of a picture or a film shot).
Focusing on a single image can help train us to do this.

\begin{tcolorbox}[enhanced jigsaw, opacityback=0, colframe=quarto-callout-note-color-frame, leftrule=.75mm, colback=white, toprule=.15mm, breakable, arc=.35mm, rightrule=.15mm, bottomrule=.15mm, left=2mm]
\begin{minipage}[t]{5.5mm}
\textcolor{quarto-callout-note-color}{\faInfo}
\end{minipage}%
\begin{minipage}[t]{\textwidth - 5.5mm}

\textbf{Important Tip}: Before continuing below, study this overview of
the building blocks of composition in art:
\href{http://www.artyfactory.com/art_appreciation/visual-elements/visual-elements.html}{The
Visual Elements: The Building Blocks of Composition in Art}

\end{minipage}%
\end{tcolorbox}

The following are some of the formal elements that make up the design
and structure of an image. However, it is important to remember that
subject matter and composition are linked and therefore do not get
locked in to rigidly adhering to what you are learning. Many masters
break the rules but they know them well before they do so.

\begin{itemize}
\tightlist
\item
  Line
\item
  Shape (geometric, organic, natural)
\item
  Colour (hues, intensities, symbolism)
\item
  Texture (tactile feel)
\item
  Value (shadows and shading)
\item
  Form (3-D, 2-D)
\item
  Space (positive/objects); (negative/space between objects)
\item
  Depth (foreground/mid-ground/background)
\end{itemize}

\section{Principles of Composition}\label{principles-of-composition}

In addition to elements of design, there are also principles or general
rules that help give structure to visual compositions and which can also
lend meaning or a theme to an image. Elements are specific things
whereas principles are more general. Here are four that the course text
focuses on:

\textbf{1. Balance} \textbf{2. Rule of Thirds} \textbf{3. Repetition and
Patterns} \textbf{4. Combinations}

There are many other principles such as the use of triangles and frame
within a frame in compositions.

\subsection{Activity: Read and Study}\label{activity-read-and-study}

\begin{tcolorbox}[enhanced jigsaw, opacityback=0, colframe=quarto-callout-note-color-frame, leftrule=.75mm, arc=.35mm, rightrule=.15mm, colbacktitle=quarto-callout-note-color!10!white, titlerule=0mm, colback=white, toprule=.15mm, bottomtitle=1mm, breakable, toptitle=1mm, title={Learning Activity}, coltitle=black, bottomrule=.15mm, left=2mm, opacitybacktitle=0.6]

Read and study Chapter 3. Be sure to study each of the images provided
in this section and how the elements are used. This will increase your
understanding of how they function in visual compositions. Before
exploring the rest of this chapter, study this essay to help guide you
during the process we are exploring:
\href{https://www.bhphotovideo.com/explora/photography/tips-and-solutions/11-thoughts-introduction-photographic-composition}{11
Thoughts: An Introduction to Photographic Composition}, by Todd
Vorenkamp.

\begin{tcolorbox}[enhanced jigsaw, opacityback=0, colframe=quarto-callout-note-color-frame, leftrule=.75mm, colback=white, toprule=.15mm, breakable, arc=.35mm, rightrule=.15mm, bottomrule=.15mm, left=2mm]

\textbf{\emph{Helpful Hint:}} New York-based B and H Photo is a great
resource for buying and learning about photographic and film and video
gear.

\end{tcolorbox}

\textbf{Question to Consider}

\begin{itemize}
\tightlist
\item
  Based on your study of this section, what are some observations you
  can share in your journal and with you facilitator and classmates?
\end{itemize}

\end{tcolorbox}

\subsection{Activity: Readings on Techniques and Principles of
Composition}\label{activity-readings-on-techniques-and-principles-of-composition}

\begin{tcolorbox}[enhanced jigsaw, opacityback=0, colframe=quarto-callout-note-color-frame, leftrule=.75mm, arc=.35mm, rightrule=.15mm, colbacktitle=quarto-callout-note-color!10!white, titlerule=0mm, colback=white, toprule=.15mm, bottomtitle=1mm, breakable, toptitle=1mm, title={Learning Activity}, coltitle=black, bottomrule=.15mm, left=2mm, opacitybacktitle=0.6]

\begin{itemize}
\item
  Study these helpful resources:
  \href{https://petapixel.com/photography-composition-techniques/}{\emph{28
  Composition Techniques That Will Improve Your Photos}}, and
  \href{https://www.getty.edu/education/teachers/building_lessons/principles_design.pdf}{\emph{Principles
  of Design}}.
\item
  Read the next section of Chapter 3, Principles of Composition, and see
  if you have a richer understanding of composition. Log your
  observations in your journal.
\item
  Find an example of the use of frame-within-a-frame and triangular
  composition and share them with your facilitator and classmates.
\end{itemize}

\end{tcolorbox}

\section{Photographic Themes}\label{photographic-themes}

The subjects of a visual image includes the objects, people, or items in
the frame. E.g., a mountainous landscape, a portrait of twins, etc.

A more advanced approach to composition involves the use of creating a
theme or deeper meaning to a composition, or where you take a topic such
as clothes on clotheslines and photograph a series of images that reveal
their beauty or what they say about the culture, people, or environment
where they were shot.

As you grow in your composition skills you will discover how to add more
depth, interest, and meaning to your photographs and images.

\subsection{Activity: Read and Explore}\label{activity-read-and-explore}

\begin{tcolorbox}[enhanced jigsaw, opacityback=0, colframe=quarto-callout-note-color-frame, leftrule=.75mm, arc=.35mm, rightrule=.15mm, colbacktitle=quarto-callout-note-color!10!white, titlerule=0mm, colback=white, toprule=.15mm, bottomtitle=1mm, breakable, toptitle=1mm, title={Learning Activity}, coltitle=black, bottomrule=.15mm, left=2mm, opacitybacktitle=0.6]

You do not have to create a theme for your photographs in this unit.
However, studying websites like the following can help enrich your
understanding and find images that stand out in intriguing, startling,
fascinating, and emotionally moving ways.

\begin{itemize}
\tightlist
\item
  \href{https://www.moma.org/artists/1777}{Walker Evans MoMA Exhibit}
\item
  The outstanding Depression Era works of Dorothea Lange are also rich
  with theme. See
  \href{http://www.historyplace.com/unitedstates/lange/}{The History
  Place Dorothea Lange}
\item
  \href{https://www.boredpanda.com/top-10-photographers-for-travel-portraits/?utm_source=google&utm_medium=organic&utm_campaign=organic}{Top
  10 Most Famous Portrait Photographers In The World}
\end{itemize}

\begin{tcolorbox}[enhanced jigsaw, opacityback=0, colframe=quarto-callout-note-color-frame, leftrule=.75mm, colback=white, toprule=.15mm, breakable, arc=.35mm, rightrule=.15mm, bottomrule=.15mm, left=2mm]

\textbf{\emph{Helpful Hint:}} \emph{The photographs of Walker Evans and
Dorethea Lange cited above are almost one-hundred years old. But they
reveal the way their heartfelt themes and beauty are universal and
timeless---the goal of excellent photos.}

\end{tcolorbox}

Log in your journal and share with a peer in your course what you
learned about photographic themes.

\end{tcolorbox}

\section*{Summary}\label{summary-2}
\addcontentsline{toc}{section}{Summary}

\markright{Summary}

In this unit, you learned about the use of elements, principles, and
theme in visual composition and photographic design.

Composition and subject matter are different but intimately related. We
can have a simple subject with little composition. A white ball on a
black table. Or we can have a complex composition with little subject
matter. The bokeh (blurred or out of focus) of rain splashes on a glass
window.

We have emphasized composition here because well composed shots are a
foundational building block of a good film. As you will learn, if the
head space of your shots is too much or too little, or there is a lack
of balance of how people are arranged, or if you do not give enough
leading entry space in a jogging shot, it will detract from the effect
of your film. (There are exceptions to these rules which will be
addressed later.) Too many flaws will mark you as an untrained or sloppy
filmmaker - something we want to avoid.

\begin{tcolorbox}[enhanced jigsaw, opacityback=0, colframe=quarto-callout-note-color-frame, leftrule=.75mm, arc=.35mm, rightrule=.15mm, colbacktitle=quarto-callout-note-color!10!white, titlerule=0mm, colback=white, toprule=.15mm, bottomtitle=1mm, breakable, toptitle=1mm, title={Checking Your Learning}, coltitle=black, bottomrule=.15mm, left=2mm, opacitybacktitle=0.6]

Before you move on to the next unit, you may want to check to make sure
that you are able to:

\begin{itemize}
\tightlist
\item
  Describe the elements used in visual compositions.
\item
  Define composition principles.
\item
  Study the works of noted photographers and apply composition elements
  and principles to photographs you create.
\item
  Create photographs that reveal your understanding of the chapter's
  content.
\end{itemize}

\end{tcolorbox}

\bookmarksetup{startatroot}

\chapter{Cinematic Motion}\label{cinematic-motion}

\section*{Overview}\label{overview-3}
\addcontentsline{toc}{section}{Overview}

\markright{Overview}

In Unit 3 you created visual compositions with your camera. We hope you
found it exciting to go out and explore your world through a lens and
design a creative still image.

In this Unit you begin your journey into the world of motion pictures.
At heart, film (and video) is a temporal medium. It involves a series of
images that unfold over time, most often at the rate of 24, 25, or 30
frames per second. This rate of frame-flow gives film its sense of
moving pictures where a series of single frame images of a galloping
horse look like the horse is actually running. This is why film is
considered a kinetic medium where moving people and objects take center
stage.

We began with still photos in the previous unit because they are easier
to control and to frame. In this unit we are going to add motion to what
we compositionally frame. This will bring a lot more variables to image
creation and make it more exciting for many people. As you focus on
motion you must not be tempted to forget the lessons learned in the
previous unit. Many of the same composition elements and principles will
apply and the wise student will refresh his or her self with Unit 3
before engaging the activities for this unit.

\subsection*{Topics}\label{topics-3}
\addcontentsline{toc}{subsection}{Topics}

This unit is divided into the following topics:

\begin{enumerate}
\def\labelenumi{\arabic{enumi}.}
\tightlist
\item
  Types of Cinematic Motion
\item
  Motivating Camera Movement
\end{enumerate}

\subsection*{Learning Outcomes}\label{learning-outcomes-3}
\addcontentsline{toc}{subsection}{Learning Outcomes}

When you have completed this unit, you will be able to:

\begin{itemize}
\tightlist
\item
  Describe the various types of film motion shots
\item
  Contrast the various types of film motion shots
\item
  Determine when and how to use cinematic motion
\item
  Create effective cinematic motion shots
\end{itemize}

\subsection*{Learning Activities}\label{learning-activities-3}
\addcontentsline{toc}{subsection}{Learning Activities}

Here is a checklist of learning activities you will benefit from in
completing this unit. You may find it useful for planning your work.

\begin{itemize}
\tightlist
\item
  Read and Study: Read and study Chapter 4 of course e-text:
  \emph{Digital Filmmaking: A Beginner's Guide to Mastering the Craft.}
\item
  Motion Shots Practice: Practice doing as many of the main types of
  motion shots as you can. Discuss what you did and why with your peers.
\item
  Proper Motivation Practice: Find someone to film and practice each of
  the cinematic motion techniques with your cell phone.
\end{itemize}

\begin{tcolorbox}[enhanced jigsaw, opacityback=0, colframe=quarto-callout-note-color-frame, leftrule=.75mm, colback=white, toprule=.15mm, breakable, arc=.35mm, rightrule=.15mm, bottomrule=.15mm, left=2mm]
\begin{minipage}[t]{5.5mm}
\textcolor{quarto-callout-note-color}{\faInfo}
\end{minipage}%
\begin{minipage}[t]{\textwidth - 5.5mm}

Working through course activities will help you to meet the learning
outcomes and successfully complete your assessments.

\end{minipage}%
\end{tcolorbox}

\subsection*{Assessment}\label{assessment-3}
\addcontentsline{toc}{subsection}{Assessment}

\textbf{Course Journal}

After completing this unit, including the learning activities, you are
asked to make sure you are doing journal entries and when appropriate to
share your responses with your facilitator and classmates when you meet.

Note that entries are expected after every unit. Your journal
reflections are submitted at the end of the course as part of the
Journal One: Personal Journal and self assessment.

\textbf{Motivated Camera Movement Exercise}

\emph{This activity is ungraded, but part of journal entry}

For this exercise on motion, you will shoot \textbf{5 different types}
of motivated kinetic shots, e.g., a pan, tilt, zoom, following action
shot, etc. that have smooth motion from the beginning to the end of the
shot and ``disguise'' the camera technique by matching the rate of
camera move and subject movement.

\begin{tcolorbox}[enhanced jigsaw, opacityback=0, colframe=quarto-callout-note-color-frame, leftrule=.75mm, colback=white, toprule=.15mm, breakable, arc=.35mm, rightrule=.15mm, bottomrule=.15mm, left=2mm]
\begin{minipage}[t]{5.5mm}
\textcolor{quarto-callout-note-color}{\faInfo}
\end{minipage}%
\begin{minipage}[t]{\textwidth - 5.5mm}

Please see the Assessment section in Moodle for assignment details as
well as the grading criteria.

\end{minipage}%
\end{tcolorbox}

\subsection*{Resources}\label{resources-3}
\addcontentsline{toc}{subsection}{Resources}

Here are the resources you will need to complete this unit.

\begin{itemize}
\tightlist
\item
  Chapter Four of the course text: \emph{Digital Filmmaking: A
  Beginner's Guide to Mastering the Craft}, by Ned Vankevich
\item
  \href{https://www.youtube.com/watch?v=g6zMtnLC50w}{8 Basic Types of
  Camera Movements}
\item
  \href{https://www.youtube.com/watch?v=h2c3JZ6X3f8}{5 Brilliant Moments
  of Camera Movement}
\item
  \href{https://www.youtube.com/watch?v=VPfKsdPsS5w}{Perfect your Film
  with Cinematic Motion}
\item
  Guidelines for Motion Exercises
\item
  Other online resources will be provided in the course text and unit.
\end{itemize}

\section{Types of Cinematic Motion
Shots}\label{types-of-cinematic-motion-shots}

In Unit 3 we explored how excellent or effective still (non-moving)
visual compositions are made. The elements and principles of such images
apply to filmmaking too. However, film and video add new variables and
techniques given that they deal with moving or motion pictures.

However, as we explore film motion, we must remember that film and video
involve a series of still images being projected and that persistence of
vision makes them appear connected. (See the first Helpful Hint in the
course text Chapter 4.) This is why we need to incorporate what we have
learned about visual composition as we create our ``moving'' pictures.

The main types of motion shots include:

\begin{itemize}
\tightlist
\item
  ZOOMS
\item
  PAN SHOTS
\item
  SWISH OR WHIP PAN
\item
  TILT SHOTS
\item
  DOLLY
\item
  DOLLY ZOOMS
\item
  TRACKING
\item
  ARCING
\item
  FOLLOW SHOTS
\item
  CRANE SHOTS
\item
  360-DEGREE TRACKING SHOTS
\item
  SLIDERS
\item
  GOPRO
\item
  DRONES
\item
  SHAKY CAMERA
\end{itemize}

\subsection{Activity: Motion Shots
Reading}\label{activity-motion-shots-reading}

\begin{tcolorbox}[enhanced jigsaw, opacityback=0, colframe=quarto-callout-note-color-frame, leftrule=.75mm, arc=.35mm, rightrule=.15mm, colbacktitle=quarto-callout-note-color!10!white, titlerule=0mm, colback=white, toprule=.15mm, bottomtitle=1mm, breakable, toptitle=1mm, title={Learning Activity}, coltitle=black, bottomrule=.15mm, left=2mm, opacitybacktitle=0.6]

Study Chapter 4 of course e-text: \emph{Digital Filmmaking: A Beginner's
Guide to Mastering the Craft.} This chapter will help you understand the
different types of motion shots. Be sure to take notes and begin logging
and identifying shot ideas you want to try and practice. Actively taking
notes as you read is a great way to absorb the material.

\end{tcolorbox}

\subsection{Activity: Motion Shots
Practice}\label{activity-motion-shots-practice}

\begin{tcolorbox}[enhanced jigsaw, opacityback=0, colframe=quarto-callout-note-color-frame, leftrule=.75mm, arc=.35mm, rightrule=.15mm, colbacktitle=quarto-callout-note-color!10!white, titlerule=0mm, colback=white, toprule=.15mm, bottomtitle=1mm, breakable, toptitle=1mm, title={Learning Activity}, coltitle=black, bottomrule=.15mm, left=2mm, opacitybacktitle=0.6]

Study the definitions of the main types of motion shots and how to do
them. Then practice doing as many of them as you can. (Obviously you
cannot do a GoPro or drone shot without that specific equipment, but you
might be imaginative and find an alternative to do a similar type of
shot.)

Write in your journal your observations about each type of shot and what
you learned from doing them.

\end{tcolorbox}

\subsection{Activity: Watch and Learn}\label{activity-watch-and-learn}

\begin{tcolorbox}[enhanced jigsaw, opacityback=0, colframe=quarto-callout-note-color-frame, leftrule=.75mm, arc=.35mm, rightrule=.15mm, colbacktitle=quarto-callout-note-color!10!white, titlerule=0mm, colback=white, toprule=.15mm, bottomtitle=1mm, breakable, toptitle=1mm, title={Learning Activity}, coltitle=black, bottomrule=.15mm, left=2mm, opacitybacktitle=0.6]

Before moving on to the next topic, watch the video
\href{https://www.youtube.com/watch?v=h2c3JZ6X3f8}{``5 Brilliant Moments
of Camera Movement''} to augment your understanding of camera movement

\url{https://www.youtube-nocookie.com/embed/h2c3JZ6X3f8}

\begin{tcolorbox}[enhanced jigsaw, opacityback=0, colframe=quarto-callout-note-color-frame, leftrule=.75mm, colback=white, toprule=.15mm, breakable, arc=.35mm, rightrule=.15mm, bottomrule=.15mm, left=2mm]

\textbf{Note:} \emph{Do not try doing zoom shots with your cell camera
unless you have a special app that makes it smooth. Squeezing your
finger in and out to move closer or further away from your subject will
result in a shaky shot. Our goal is to disguise movement, not to draw
attention to it.}

\end{tcolorbox}

\end{tcolorbox}

\section{Properly Motivated Camera Moving
Shots}\label{properly-motivated-camera-moving-shots}

A classical approach to filmmaking involves using shot and techniques
that are motivated by the characters and story and which do not
unnecessarily distract or pull the audience out of experiencing an event
in your film. Put another way, if you do something that is jarring or
not done well such as a shaky zoom shot it will draw attention to itself
and distract the audience and prevent them from engaging the flow of
your scene. What this means will become apparent as you progress in the
course.

The following sections will help you have properly motivated and framed
movement within the frame.

\begin{itemize}
\tightlist
\item
  Properly Motivated Movement Shots
\item
  Proper Placement in the Frame
\item
  Motivated Unmotivated Camera Movement
\end{itemize}

\subsection{Activity: Proper Motivation Exercise and
Practice}\label{activity-proper-motivation-exercise-and-practice}

\begin{tcolorbox}[enhanced jigsaw, opacityback=0, colframe=quarto-callout-note-color-frame, leftrule=.75mm, arc=.35mm, rightrule=.15mm, colbacktitle=quarto-callout-note-color!10!white, titlerule=0mm, colback=white, toprule=.15mm, bottomtitle=1mm, breakable, toptitle=1mm, title={Learning Activity}, coltitle=black, bottomrule=.15mm, left=2mm, opacitybacktitle=0.6]

After you have viewed the resources and studied the properly motivated
camera movements, find someone to film and practice each of the
techniques with your cell phone. For inspiration and how-to-do
techniques watch the video \href{https://youtu.be/6_p93J3OwfU}{Watch:
\emph{8 Cinematic Camera Moves For Video}}

\url{https://www.youtube-nocookie.com/embed/6_p93J3OwfU}

\end{tcolorbox}

\section*{Summary}\label{summary-3}
\addcontentsline{toc}{section}{Summary}

\markright{Summary}

In this unit, you learned about cinematic motion and how to motivate it
and do it properly. As such, you have gained understanding and practice
with another important basic of excellent filmmaking.

\begin{enumerate}
\def\labelenumi{\arabic{enumi}.}
\tightlist
\item
  Types of Cinematic Motion and how to use tracking shots, following
  shots, dolly shots, amount others to add professionalism and more
  meaning in your projects.
\item
  Motivating Camera Movement where you explore the classic cinema
  technique of discussing your moving shots by having the camera and
  subject movement in sync.
\end{enumerate}

\begin{tcolorbox}[enhanced jigsaw, opacityback=0, colframe=quarto-callout-note-color-frame, leftrule=.75mm, arc=.35mm, rightrule=.15mm, colbacktitle=quarto-callout-note-color!10!white, titlerule=0mm, colback=white, toprule=.15mm, bottomtitle=1mm, breakable, toptitle=1mm, title={Checking Your Learning}, coltitle=black, bottomrule=.15mm, left=2mm, opacitybacktitle=0.6]

Before you move on to the next unit, you may want to check to make sure
that you are able to:

\begin{itemize}
\tightlist
\item
  Describe the various types of film motion shots
\item
  Contrast the various types of film motion shots
\item
  Determine when and how to use cinematic motion
\item
  Create effective cinematic motion shots
\end{itemize}

\end{tcolorbox}

\bookmarksetup{startatroot}

\chapter{The Grammar of Film}\label{the-grammar-of-film}

\section*{Overview}\label{overview-4}
\addcontentsline{toc}{section}{Overview}

\markright{Overview}

We go back to theory in this unit and then have a chance to practice and
apply it.

When we watch a film or TV show most people have no idea how much work
and detail it took to make the production. Nor do they know the rules,
guidelines, principles, and practices that were followed to create it.
This unit and the ones following will help you explore what these best
practices are and understand how to apply them to your work.

This will be a fun chapter as you get to explore and experiment with the
fundamental building blocks that create meaning in cinema. It is a unit
that will challenge both your thinking and your intuition skills as you
learn the basic components of filmmaking and how to use them in creative
ways.

\subsection*{Topics}\label{topics-4}
\addcontentsline{toc}{subsection}{Topics}

This unit is divided into the following topics:

\begin{enumerate}
\def\labelenumi{\arabic{enumi}.}
\tightlist
\item
  Basic Grammar of Film
\item
  Basic Shots and Their Descriptions
\item
  Camera Angles and Heights
\item
  Camera Subject Angles
\end{enumerate}

\subsection*{Learning Outcomes}\label{learning-outcomes-4}
\addcontentsline{toc}{subsection}{Learning Outcomes}

When you have completed this unit, you will be able to:

\begin{itemize}
\tightlist
\item
  Define the basic elements of the grammar of film
\item
  Describe basic shots and their descriptions
\item
  Contrast camera angles and heights
\item
  Determine when and how to use subject angles
\item
  Create shots that reveal what you discovered in this unit
\end{itemize}

\subsection*{Learning Activities}\label{learning-activities-4}
\addcontentsline{toc}{subsection}{Learning Activities}

Here is a checklist of learning activities you will benefit from in
completing this unit. You may find it useful for planning your work.

\begin{itemize}
\tightlist
\item
  Find and consult a film glossary such as Brooklyn College Film
  Glossary. Read and study the one you like throughout this course. Set
  a goal to learn 5 new terms or concepts each day.
\item
  Watch two videos about Basic Shots in action
\item
  Read and Study Chapter 5 and explain how frames and shots become
  scenes and sequences and how they add up to a complete film story.
\item
  After thoroughly reading and studying Chapter 5 to grasp the concepts
  of shots, camera angles, heights, and camera subject angles, engage in
  practical application.
\end{itemize}

\begin{tcolorbox}[enhanced jigsaw, opacityback=0, colframe=quarto-callout-note-color-frame, leftrule=.75mm, colback=white, toprule=.15mm, breakable, arc=.35mm, rightrule=.15mm, bottomrule=.15mm, left=2mm]
\begin{minipage}[t]{5.5mm}
\textcolor{quarto-callout-note-color}{\faInfo}
\end{minipage}%
\begin{minipage}[t]{\textwidth - 5.5mm}

Working through course activities will help you to meet the learning
outcomes and successfully complete your assessments.

\end{minipage}%
\end{tcolorbox}

\subsection*{Assessment}\label{assessment-4}
\addcontentsline{toc}{subsection}{Assessment}

\textbf{Course Journal}

After completing this unit, including the learning activities, you are
asked to make sure you are doing journal entries and when appropriate to
share your responses with your facilitator and classmates when you meet.

Note that entries are expected after every unit. Your journal
reflections are submitted at the end of the course as part of the
Journal One: Personal Journal and self assessment.

\begin{tcolorbox}[enhanced jigsaw, opacityback=0, colframe=quarto-callout-note-color-frame, leftrule=.75mm, colback=white, toprule=.15mm, breakable, arc=.35mm, rightrule=.15mm, bottomrule=.15mm, left=2mm]
\begin{minipage}[t]{5.5mm}
\textcolor{quarto-callout-note-color}{\faInfo}
\end{minipage}%
\begin{minipage}[t]{\textwidth - 5.5mm}

Please see the Assessment section in Moodle for assignment details as
well as the grading criteria.

\end{minipage}%
\end{tcolorbox}

\subsection*{Resources}\label{resources-4}
\addcontentsline{toc}{subsection}{Resources}

Here are the resources you will need to complete this unit.

\begin{itemize}
\tightlist
\item
  Chapter 5 of course text, \emph{Digital Filmmaking: A Beginner's Guide
  to Mastering the Craft}.
\item
  \href{https://liveboldandbloom.com/04/self-improvement/develop-your-intuition}{21
  Eye-Opening Ways To Develop Your Intuition}
\item
  Studying film shots is an essential part of mastering the craft of
  filmmaking. Study this helpful overview:
  \href{https://www.youtube.com/watch?v=7y0ouVBcogU}{15 Essential Camera
  Shots, Angles and Movements in Filmmaking}
\item
  For an overview of the types of film shots and angles see:
  \href{https://www.studiobinder.com/blog/ultimate-guide-to-camera-shots/}{The
  Ultimate Guide to Camera Shots (over 50+ Types of Shots and Angles in
  Film)}
\item
  \href{http://userhome.brooklyn.cuny.edu/anthro/jbeatty/COURSES/glossary.htm}{Brooklyn
  College Film Glossary}
\item
  \href{https://erickimphotography.com/blog/2014/12/09/17-lessons-henri-cartier-bresson-taught-street-photography/}{17
  Lessons Henri Cartier-Bresson Has Taught Me About Street Photography}
\item
  Other online resources will be provided in the unit and text chapter.
\end{itemize}

\section{Basic Grammar of Film}\label{basic-grammar-of-film}

In the previous units we look at the basic building blocks of visual
composition and cinematic motion. In this unit we are going to learn how
to put them together with a variety of other shots to make cinematic
context and meaning. This is the grammar of film used to make cinematic
``sentences'' which convey meaning and context to an audience as we
create scenes, sequences, and short films.

However, as we focus on the components of filmmaking, we must not let
the emphasis only be on film logic. We also want to cultivate the role
that intuition and feeling play as we develop shots and put them
together. This reminder is important because filmmaking is as much art
as technical craft. We need to know the craft---the technical
aspects---but also the art---the aesthetic and emotional dimensions. We
need to keep this before us as we venture into the many elements and
guidelines of effective filmmaking.

In this first topic you will learn about: - Frames - Shots - Scenes -
Sequences - Acts

These components give you the big picture of how the many smaller
elements add up to create mood and meaning in a film project.

\subsection{Activity: Read, and
Explore}\label{activity-read-and-explore-1}

\begin{tcolorbox}[enhanced jigsaw, opacityback=0, colframe=quarto-callout-note-color-frame, leftrule=.75mm, arc=.35mm, rightrule=.15mm, colbacktitle=quarto-callout-note-color!10!white, titlerule=0mm, colback=white, toprule=.15mm, bottomtitle=1mm, breakable, toptitle=1mm, title={Learning Activity}, coltitle=black, bottomrule=.15mm, left=2mm, opacitybacktitle=0.6]

Find and consult a film glossary such as
\href{http://userhome.brooklyn.cuny.edu/anthro/jbeatty/COURSES/glossary.htm}{Brooklyn
College Film Glossary}. Continue to read and study the one you like
throughout this course and it will help you grow your film vocabulary
and your understanding of the many components and processes involved in
filmmaking. Set a goal to learn 5 new terms or concepts each day.

\end{tcolorbox}

\section{Basic Shots and Their
Descriptions}\label{basic-shots-and-their-descriptions}

This is a vital section. You need to know the following shots and how to
describe them because you will have to use this language to communicated
to your crew. Few things are more frustrating for a film crew than the
director or cinematographer not knowing what they want or how to name
and describe a shot.

\begin{itemize}
\tightlist
\item
  Extreme Long Shot (ELS)
\item
  Long Shot (LS) also called Wide Shot (WS)
\item
  Establishing Shot (ES)
\item
  Full Shot (FS)
\item
  Medium Long Shot (MLS) also known as Medium Wide Shot (MWS)
\item
  Cowboy Shot
\item
  Medium Shot (MS)
\item
  Medium Close Up (MCU)
\item
  Close Up (CU)
\item
  Extreme Close Up (ECU)
\end{itemize}

In addition to shot size we also designate film shots by who or what is
in the frame and how they are positioned:

\begin{itemize}
\tightlist
\item
  Single Shot
\item
  Two Shot
\item
  Three Shot
\item
  Group Shot
\item
  Over-The-Shoulder Shot
\item
  Reverse Shot
\item
  Point of View Shot (POV)
\item
  Reaction Shot
\item
  Cutaway
\item
  Cut-in and Insert
\item
  Combo Shot
\end{itemize}

\subsection{Activity: Watch and Learn}\label{activity-watch-and-learn-1}

\begin{tcolorbox}[enhanced jigsaw, opacityback=0, colframe=quarto-callout-note-color-frame, leftrule=.75mm, arc=.35mm, rightrule=.15mm, colbacktitle=quarto-callout-note-color!10!white, titlerule=0mm, colback=white, toprule=.15mm, bottomtitle=1mm, breakable, toptitle=1mm, title={Learning Activity}, coltitle=black, bottomrule=.15mm, left=2mm, opacitybacktitle=0.6]

As they say, a picture is worth a thousand words, and when you see the
shots in action in these resources they become more clear. Watch the
video resources below to learn more about basic shots.

\href{https://www.youtube.com/watch?v=lRo2IqYbEaE}{Watch:
\emph{Introduction to shot types and camera movement}}

\url{https://www.youtube-nocookie.com/embed/lRo2IqYbEaE}

\href{https://www.youtube.com/watch?v=7y0ouVBcogU}{Watch: \emph{15
Essential Camera Shots, Angles and Movements in Filmmaking}}

\url{https://www.youtube-nocookie.com/embed/7y0ouVBcogU}

\end{tcolorbox}

\section{Camera, Angles, and Heights}\label{camera-angles-and-heights}

Camera heights and angles reveal both text and subtext. That is, they
show us what is being filmed (the object or subject) and often convey
something about the meaning of the subject or object (the subtext).

In essence, there are no ``neutral'' camera angles. Our camera height
placement will convey something and it is important for the filmmaker to
know what that something is.

The different camera heights include:

\begin{itemize}
\tightlist
\item
  Eye-Level
\item
  Low Angle
\item
  High Angle
\item
  Shoulder Level
\item
  Hip Level
\item
  Knee Level Shot
\item
  Ground Level
\item
  Worm's Eye View
\item
  Bird's Eye View
\item
  God's Eye View
\item
  Dutch Tilt or Dutch Canted Angle
\end{itemize}

\subsection{Activity: Read and Study}\label{activity-read-and-study-1}

\begin{tcolorbox}[enhanced jigsaw, opacityback=0, colframe=quarto-callout-note-color-frame, leftrule=.75mm, arc=.35mm, rightrule=.15mm, colbacktitle=quarto-callout-note-color!10!white, titlerule=0mm, colback=white, toprule=.15mm, bottomtitle=1mm, breakable, toptitle=1mm, title={Learning Activity}, coltitle=black, bottomrule=.15mm, left=2mm, opacitybacktitle=0.6]

Read and study Chapter 5. In your Film Journal, explain how frames and
shots become scenes and sequences and how they add up to a complete film
story. Also, summarize the best thing you learned about camera heights
in this section.

Commit to doing one technique several times a day, and reflect on what
you learnt from this experience.

\textbf{Helpful Hint:} It is recommended but not mandatory that you
study a web essay on how to build your intuition such
as\href{https://liveboldandbloom.com/04/self-improvement/develop-your-intuition}{21
Eye-Opening Ways To Develop Your Intuition}

\end{tcolorbox}

\section{Camera Subject Angles}\label{camera-subject-angles}

Subject angle refers to the perspective the audience has of the person
being filmed. Subject angles are important because they affect the way
the audience interprets the character or scene.

The following are key subject angles:

\begin{itemize}
\tightlist
\item
  Profile
\item
  Full-Face
\item
  Three-quarter
\item
  Quarter angle
\item
  Full-Back
\item
  Combo
\item
  Silhouettes
\end{itemize}

Having completed this topic, you now have a firm grasp of the grammar of
film and its components.

\subsection{Activity: Practice}\label{activity-practice}

\begin{tcolorbox}[enhanced jigsaw, opacityback=0, colframe=quarto-callout-note-color-frame, leftrule=.75mm, arc=.35mm, rightrule=.15mm, colbacktitle=quarto-callout-note-color!10!white, titlerule=0mm, colback=white, toprule=.15mm, bottomtitle=1mm, breakable, toptitle=1mm, title={Learning Activity}, coltitle=black, bottomrule=.15mm, left=2mm, opacitybacktitle=0.6]

After thoroughly reading and studying Chapter 5, it's time to put theory
into practice. Engage in practical application using camera angles,
heights and camera subjects

\end{tcolorbox}

\subsection{Activity: Exercise for Applying Grammar of
Film}\label{activity-exercise-for-applying-grammar-of-film}

\begin{tcolorbox}[enhanced jigsaw, opacityback=0, colframe=quarto-callout-note-color-frame, leftrule=.75mm, arc=.35mm, rightrule=.15mm, colbacktitle=quarto-callout-note-color!10!white, titlerule=0mm, colback=white, toprule=.15mm, bottomtitle=1mm, breakable, toptitle=1mm, title={Learning Activity}, coltitle=black, bottomrule=.15mm, left=2mm, opacitybacktitle=0.6]

Now that you have done the theory, it is time to practice and apply what
you have learned. Be sure to use both sides of your brain---the
conscious, intentional shot logic part and the intuitive, feeling,
follow your gut and take a risk part. Learning to combine both will help
you grow immensely in your creativity.

For this exercise you will engage in street photography, that is, go out
into your local world and explore it through your cell phone camera lens
and capture photos that use at least ten different shot sizes, subject
framing, subject angles, and camera angles and heights.

For inspiration about street photography see:

\begin{itemize}
\tightlist
\item
  \href{https://erickimphotography.com/blog/2014/12/09/17-lessons-henri-cartier-bresson-taught-street-photography/}{17
  Lessons Henri Cartier-Bresson Has Taught Me About Street Photography}
\end{itemize}

Share your four best shots with your classmates and log in your journal
what learned from your street photography and exploring of camera
angles.

\end{tcolorbox}

\section*{Summary}\label{summary-4}
\addcontentsline{toc}{section}{Summary}

\markright{Summary}

In this unit, you learned about the importance of film grammar and the
multitude of shot sizes, framings, angles, etc. In doing so, you now
have a good foundational grasp of the basic components needed to make
excellent film projects.

\begin{tcolorbox}[enhanced jigsaw, opacityback=0, colframe=quarto-callout-note-color-frame, leftrule=.75mm, arc=.35mm, rightrule=.15mm, colbacktitle=quarto-callout-note-color!10!white, titlerule=0mm, colback=white, toprule=.15mm, bottomtitle=1mm, breakable, toptitle=1mm, title={Checking Your Learning}, coltitle=black, bottomrule=.15mm, left=2mm, opacitybacktitle=0.6]

Before you move on to the next unit, you may want to check to make sure
that you are able to:

\begin{itemize}
\tightlist
\item
  Define the basic elements of the grammar of film.
\item
  Describe basic shots and their descriptions.
\item
  Contrast camera angles and heights.
\item
  Determine when and how to use subject angles.
\item
  Create shots that reveal what you discovered in this unit.
\end{itemize}

\end{tcolorbox}

\bookmarksetup{startatroot}

\chapter{Visual Storytelling}\label{visual-storytelling}

\section*{Overview}\label{overview-5}
\addcontentsline{toc}{section}{Overview}

\markright{Overview}

Congratulations. You are now ready to make a rare type of film. Rare you
say? Yes, because you are going to shoot your film project in a way that
is not done very often, that is, in full sequential order. What this
means will be explained ahead.

We are now at a point in the course where you are going to be held
accountable for your work. The earlier exercises were not graded to give
you the chance to explore and make mistakes, without regard to a grade.
We now have to cross into that territory. The main reason for this is
that, as mentioned previously, film is a public medium and your work
will be critiqued when it is shown. So this will be a great opportunity
to learn to give and receive feedback.

But fear not. Based on your previous work and what you will learn in
this unit you will be ready to make a film that is ``public'' worthy. In
addition, this first film will only account for 10\% of your grade so
you are encouraged to explore and take risks.

\subsection*{Topics}\label{topics-5}
\addcontentsline{toc}{subsection}{Topics}

This unit is divided into the following topics:

\begin{enumerate}
\def\labelenumi{\arabic{enumi}.}
\tightlist
\item
  The In-Camera Project and Its Benefits
\item
  Secrets to a Simple Story
\item
  Brainstorming and Successful Creativity
\item
  Short Story Film Template
\item
  Direction Vectors and Eye-line Requirements
\end{enumerate}

\subsection*{Learning Outcomes}\label{learning-outcomes-5}
\addcontentsline{toc}{subsection}{Learning Outcomes}

When you have completed this unit, you will be able to:

\begin{itemize}
\tightlist
\item
  Describe what constitutes a liner story
\item
  Define what constitutes an ``In-Camera'' film
\item
  Analyze and apply a story template
\item
  Determine how to tell a visually-centered story well
\item
  Create a short simple story to film
\end{itemize}

\subsection*{Learning Activities}\label{learning-activities-5}
\addcontentsline{toc}{subsection}{Learning Activities}

Here is a checklist of learning activities you will benefit from in
completing this unit. You may find it useful for planning your work.

\begin{itemize}
\tightlist
\item
  Read, Study, and Reflect: Read and study Chapter 6 of course e-text:
  \emph{Digital Filmmaking: A Beginner's Guide to Mastering the Craft.}
  Write in your journal your initial thoughts regarding the benefits of
  doing an in-camera film
\item
  Brainstorming: Brainstorm at least five story ideas and use the
  ``Fairy Tale Template for a Short Film'' to write your story for your
  short film.
\end{itemize}

\begin{tcolorbox}[enhanced jigsaw, opacityback=0, colframe=quarto-callout-note-color-frame, leftrule=.75mm, colback=white, toprule=.15mm, breakable, arc=.35mm, rightrule=.15mm, bottomrule=.15mm, left=2mm]
\begin{minipage}[t]{5.5mm}
\textcolor{quarto-callout-note-color}{\faInfo}
\end{minipage}%
\begin{minipage}[t]{\textwidth - 5.5mm}

Working through course activities will help you to meet the learning
outcomes and successfully complete your assessments.

\end{minipage}%
\end{tcolorbox}

\subsection*{Assessment}\label{assessment-5}
\addcontentsline{toc}{subsection}{Assessment}

\textbf{Course Journal}

After completing this unit, including the learning activities, you are
asked to make sure you are doing journal entries and when appropriate to
share your responses with your facilitator and classmates when you meet.

Also, log in your journal what you learned from the creating and filming
of your In-Camera exercise. Log what you thought about the films of your
peers and what you learned from the feedback from them and your
instructor.

Be sure to make a note of what films stood out and who did them. You
will consult this at the end of the course when the class determines:
Best Overall Film. Most Imaginative Film, Best Story, Best
Cinematography, Best Editing, etc.

\textbf{In-Camera Exercise}

\emph{This activity is ungraded, but important}

Create a brief mini movie that shows your understanding of how to apply
composition, motivated camera movement, and a variety of shots to
visually tell a story. That is, images and not dialogue should drive
your story.

\begin{tcolorbox}[enhanced jigsaw, opacityback=0, colframe=quarto-callout-note-color-frame, leftrule=.75mm, colback=white, toprule=.15mm, breakable, arc=.35mm, rightrule=.15mm, bottomrule=.15mm, left=2mm]
\begin{minipage}[t]{5.5mm}
\textcolor{quarto-callout-note-color}{\faInfo}
\end{minipage}%
\begin{minipage}[t]{\textwidth - 5.5mm}

Please see the Assessment section in Moodle for assignment details as
well as the grading criteria.

\end{minipage}%
\end{tcolorbox}

\subsection*{Resources}\label{resources-5}
\addcontentsline{toc}{subsection}{Resources}

Here are the resources you will need to complete this unit.

\begin{itemize}
\tightlist
\item
  Chapter Six course text
\item
  \href{https://www.youtube.com/watch?v=iWQQgZh9EyE}{Visual Storytelling
  101}
\item
  \href{https://www.youtube.com/watch?v=4X5xvlTZpcY}{Visual Storytelling
  in Filmmaking}
\item
  \href{https://www.studiobinder.com/blog/short-film-ideas-you-can-actually-produce/}{``30
  Ways to Brainstorm Short Film Ideas You Can Actually Produce}
\item
  \href{https://www.indiewire.com/2015/08/19-great-ways-to-brainstorm-short-film-ideas-58785/}{19
  Great Ways to Brainstorm Short Film Ideas}
\item
  \href{https://www.youtube.com/watch?v=y_1H6V7uyYc}{The eyeline match}
\item
  \href{https://www.youtube.com/watch?v=9XOn5uxdSJc}{Screen Direction
  rule}
\item
  \href{https://www.youtube.com/watch?v=0pd0K2u1Bk8}{Filmmaking
  Tutorial: Head room, lead room \& Framing} YouTube, LightsFilmSchool.
\end{itemize}

\section{In-Camera Project and Its
Benefits}\label{in-camera-project-and-its-benefits}

Your first film project will be something you might never do again if
you go into the film and television world. You will shoot a short film
with no dialogue in the exact sequential order that it will appear when
you screen it for classs.

Almost always, films are shot out of sequence to save time and money by
doing all the scenes in one location at a time and then all the scenes
in another location at one time, etc.

Here you will be forced to shoot at a location and move to the next, and
if the first location is needed you will have to go back to it. Why do
this? There are multiple benefits, as the course text chapter
highlights. Some of them include:

\begin{itemize}
\tightlist
\item
  Forcing you to think of a simple story that fits these parameters.
\item
  Challenging you to plan your shots in a linear, well-thought out way.
\item
  Making you attentive to each shot and how relates to the previos ones.
\item
  Challenging your brainstorming and film logic skills.
\end{itemize}

\subsection{Activity: Read, Study and
Reflect}\label{activity-read-study-and-reflect}

\begin{tcolorbox}[enhanced jigsaw, opacityback=0, colframe=quarto-callout-note-color-frame, leftrule=.75mm, arc=.35mm, rightrule=.15mm, colbacktitle=quarto-callout-note-color!10!white, titlerule=0mm, colback=white, toprule=.15mm, bottomtitle=1mm, breakable, toptitle=1mm, title={Learning Activity}, coltitle=black, bottomrule=.15mm, left=2mm, opacitybacktitle=0.6]

Read and study Chapter 6. Write in your journal your initial thoughts
regarding the benefits of doing an in-camera film. Does it thrill and
excite you or make you apprehensive and anxious? Or maybe a combination
of these feelings and emotions. As artists we need to get in touch with
our feelings and learn to work with them not fear them.

To prime your imagination for the upcoming project, watch the video
\href{https://www.youtube.com/watch?v=iWQQgZh9EyE}{Watch: \emph{Visual
Storytelling 101}}

\url{https://www.youtube-nocookie.com/embed/iWQQgZh9EyE}

\begin{itemize}
\tightlist
\item
  What was your best take-away from it? Log this in your journal and if
  there is time, share your insights with your facilitator and
  classmates.
\end{itemize}

\end{tcolorbox}

\section{Secrets to a Simple Story}\label{secrets-to-a-simple-story}

In order to do this first project, you will need to come up with a story
to film. Something has to happened to someone and they must engage in
action to address it. But if we are to design a good story---one that
holds the attention of our audience and engages them---we need to focus
on several things in our short film project:

\begin{itemize}
\tightlist
\item
  One or two main characters.
\item
  A single problem or conflict.
\item
  Knowing your genre (will your film be funny, dramatic, scary, etc.?).
\item
  A simple setting or number of locations.
\item
  A satisfying ending.
\end{itemize}

You need to keep this simple formula in mind in the projects ahead.
There will be many forces and temptations that will distract you and
pull you away from this simple understanding. Resist them and go back to
basics if you get lost, confused, or frustrated.

\subsection{Activity: Read and
Reflect}\label{activity-read-and-reflect-1}

\begin{tcolorbox}[enhanced jigsaw, opacityback=0, colframe=quarto-callout-note-color-frame, leftrule=.75mm, arc=.35mm, rightrule=.15mm, colbacktitle=quarto-callout-note-color!10!white, titlerule=0mm, colback=white, toprule=.15mm, bottomtitle=1mm, breakable, toptitle=1mm, title={Learning Activity}, coltitle=black, bottomrule=.15mm, left=2mm, opacitybacktitle=0.6]

Study the next section in the course text Secrets to a Simple Story.
What stands out for you? Did it inspire you? Did it help you find a good
story concept to film? Log these observations and reflections in your
course journal.

\end{tcolorbox}

\section{Brainstorming and Successful
Creativity}\label{brainstorming-and-successful-creativity}

The secret to a story sounds simple and it is. We need to create a
character with a goal and stakes (the painful consequences that will
happen to the character) who has to overcome opposition to that goal and
an ending that reveal whether or not the character obtains the goal.

However, the challenge for most of us is, what story do I tell?

Brainstorming is a great tool and way to discover and find the right
story you want to tell. Brainstorming involves the spontaneous
development of ideas. Brainstorming can be done alone or within a group.

The great thing about brainstorming is that you do not have to judge and
criticize the process and results. In addition, you are not trying to
perfect. You are only looking for that one idea that sparks your
creativity and you ignore the others. This non-judgmentalism and
jettisoning of perfectionism is important because these two tendencies
stifle creativity.

Even if you already have a story idea that you want to film you are
strongly encouraged to engage in brainstorming activities during this
section and the rest of the course. If you do, you unleash more of your
creativity and you will most likely find a better story concept.

\subsection{Activity: Brainstorming}\label{activity-brainstorming}

\begin{tcolorbox}[enhanced jigsaw, opacityback=0, colframe=quarto-callout-note-color-frame, leftrule=.75mm, arc=.35mm, rightrule=.15mm, colbacktitle=quarto-callout-note-color!10!white, titlerule=0mm, colback=white, toprule=.15mm, bottomtitle=1mm, breakable, toptitle=1mm, title={Learning Activity}, coltitle=black, bottomrule=.15mm, left=2mm, opacitybacktitle=0.6]

Study the Brainstorming section in Chapter 6, including the Steps for
Successful Brainstorming and the Tips for Success Creativity. Then
consult these resources:

\begin{itemize}
\tightlist
\item
  \{target=``\_blank''\}30 Ways to Brainstorm Short Film Ideas You Can
  Actually Produce
\item
  \{target=``\_blank''\}19 Great Ways to Brainstorm Short Film Ideas
\end{itemize}

After you feel you have a good grasp of brainstorming, engage in the
process and come up with 5 story concepts that fit the criteria in the
previous topic:

\begin{itemize}
\tightlist
\item
  One or two main characters.
\item
  A single problem or conflict.
\item
  Knowing your genre (will your film be funny, dramatic, scary, etc.?).
\item
  A simple setting or number of locations.
\item
  A satisfying ending. After you have done this go through your concept
  list and choose your top one and apply it to the template in the next
  topic.
\end{itemize}

\begin{tcolorbox}[enhanced jigsaw, opacityback=0, colframe=quarto-callout-note-color-frame, leftrule=.75mm, colback=white, toprule=.15mm, breakable, arc=.35mm, rightrule=.15mm, bottomrule=.15mm, left=2mm]

\textbf{Helpful Hint:} If you find yourself stuck and can't come up with
a story concept, tell your facilitator and have a group brainstorming
session. Take one of the ideas and develop it. Share your concept if you
are stuck finding one of the criteria such as a goal or stakes and ask
for brainstorming help to solve the problem. Remember, film is mostly a
collaborative venture.

\end{tcolorbox}

\end{tcolorbox}

\section{Short Film Story Template}\label{short-film-story-template}

Once you have your core story concept you now need to make it into a
full story with a beginning, middle, and end. We also need to make sure
we do not make it too complicated or muddled.

Many of us have the tendency to overly complicate things: to add
extraneous detail or to have elements that are not clear well explained.
The template is this section is designed to prevent these issues as you
use a fairy tale structure to find your characters, the problem they are
facing, and show what happens in a clear way with a beginning, middle,
and end of the story.

Have fun with this template process. Also do not forget to use
brainstorming for each section of the template so that you come up with
the best characters, situations, locations, and conflict which are not
predicable or have a ``been there, done that'' (boring) quality.

\subsection{Activity: Planning Your
Film}\label{activity-planning-your-film}

\begin{tcolorbox}[enhanced jigsaw, opacityback=0, colframe=quarto-callout-note-color-frame, leftrule=.75mm, arc=.35mm, rightrule=.15mm, colbacktitle=quarto-callout-note-color!10!white, titlerule=0mm, colback=white, toprule=.15mm, bottomtitle=1mm, breakable, toptitle=1mm, title={Learning Activity}, coltitle=black, bottomrule=.15mm, left=2mm, opacitybacktitle=0.6]

Once you have brainstormed and have a clear and solid concept film,
e.g., a student who has the power to turn things in gold, use the
\href{assets/u6/Fairy_Tale_Template_for_a_Short_Film.pdf}{Fairy Tale
Template for a Short Film} and write your story for your short film by
filling in the blanks.

Study the illustration example given in the chapter and then apply the
process to the creation of your film story. In crafting your story you
are not allowed to use dialogue beyond ``yes,'' ``no,'' and ``okay.''
That is, you must find visual ways to establish your story's setup and
context and its build and payoff. For example, you cannot have your
character say ``I am on my way to the store.'' You will just show him or
her getting in the car, driving, arriving at the store parking lot, and
entering the store. (This has visual direction vectors discussed below
in Topic 3.)

\textbf{Fairy Tale Template for a Short Film}

(This template is inspired by the one developed by Alexander
Mackendrick. See \emph{On Filmmaking: An Introduction to the Craft of
the Director}. London: Faber \& Faber, 2005, pgs. 78-85.)

\textbf{Once Upon a Time \_\_\_\_\_\_\_\_\_.} \emph{(Establish where
your story will take place.)}

\textbf{There Lived a \_\_\_\_\_\_\_\_\_\_\_\_\_.} \emph{Establish the
main character, hero, or protagonist.}

\textbf{But there was a Problem in the Land \_\_\_\_\_\_\_\_\_\_\_\_.}
\emph{Establish the inciting incident that starts the story's main
conflict that the protagonist must face and overcome.}

\textbf{That the Hero had to Solve \_\_\_\_\_\_\_\_\_\_.}
\emph{Establish the main character's drive, motive, goal, and stakes.}

\textbf{By Defeating a Villain or Evil Force \_\_\_\_\_\_\_\_\_\_.}
\emph{Establish the chief antagonist the protagonist has to overcome.}

\textbf{That resulted in Battles and Obstacles \_\_\_\_\_\_\_\_\_\_\_.}
\emph{Establish the problems and difficulties the protagonist must
overcome to achieve the goal.}

\textbf{The Result of Which \_\_\_\_\_\_\_\_.} \emph{Establish the chain
of cause and effect conflict events that form the middle of the story.
These should escalate, that is, grow more intense and challenging as the
action develops.}

\textbf{Produced Twists and Turns \_\_\_\_\_\_\_\_\_\_.} \emph{Establish
the surprises and unexpected turn of events as the protagonist battles
his or her foe.}

\textbf{Until the Time came that \_\_\_\_\_\_\_\_.} \emph{Establish the
obligatory scene or major confrontation or climax of the action.}

\textbf{When Suddenly \_\_\_\_\_\_\_\_\_\_\_.} \emph{Establish the final
major twist or surprise that heightens the action of the climax. This is
optional.}

\textbf{And it Turns Out that \_\_\_\_\_\_.} \emph{Establish the
resolution or wrapping up of the story.}

\textbf{And Forever After (or Not) \_\_\_\_\_\_\_\_\_.} \emph{Establish
the end of the story and its closure which can be happy or sad.}

Are you happy with your story? Do you have to make revisions? You should
go through at least three drafts of your story to make sure it works.
Test your story out on you family and friends. Ask them if it works but
remember they are not filmmakers or professional critics so ignore
things when they are too petty or too subjective.

Read through the section in Chapter 6 on creating a Short Story Film
Template.

Be sure to lay out your story in single lines of action that can be
filmed as in the example of PANIC in Chapter 6.

\end{tcolorbox}

\section{Direction Vectors and Eyeline
Requirements}\label{direction-vectors-and-eyeline-requirements}

Now that you have your story and one-line action descriptions, and have
your actors, props, and locations set-up, you are ready to film your
story (the In-Camera film project) with your cell phone.

Before doing so, this exercise will require that you focus on several
things found in the Direction Vectors and Eyeline Requirements section:

\begin{itemize}
\tightlist
\item
  Proper Lead Room
\item
  Motivated Shot Movement
\item
  Consistent Screen Direction
\item
  Proper Eye-Line Vectors
\item
  Proper Headroom
\end{itemize}

These requirements should not worry you since you have explored and
tried many of these techniques in the composition and film movements
units.

\subsection{Activity: Resources on Direction Vectors and Eyeline
Requirements}\label{activity-resources-on-direction-vectors-and-eyeline-requirements}

\begin{tcolorbox}[enhanced jigsaw, opacityback=0, colframe=quarto-callout-note-color-frame, leftrule=.75mm, arc=.35mm, rightrule=.15mm, colbacktitle=quarto-callout-note-color!10!white, titlerule=0mm, colback=white, toprule=.15mm, bottomtitle=1mm, breakable, toptitle=1mm, title={Learning Activity}, coltitle=black, bottomrule=.15mm, left=2mm, opacitybacktitle=0.6]

In order to do well and follow the direction vectors and eyeline
requirements for this first film project, be sure to consult the
following resources which explain them in visual ways. As you do so,
pre-visualize how you will film your project:

\href{https://www.youtube.com/watch?v=y_1H6V7uyYc}{Watch: \emph{The
eyeline match}}

\url{https://www.youtube-nocookie.com/embed/y_1H6V7uyYc}

\href{https://www.youtube.com/watch?v=9XOn5uxdSJc}{Watch: \emph{Screen
Direction rule}}

\url{https://www.youtube-nocookie.com/embed/9XOn5uxdSJc}

In addition, be sure to review Units 3, 4, and 5 to make sure your shots
will be well composed and that you will incorporate a variety of camera
angles.

Happy filming!

\end{tcolorbox}

\section*{Summary}\label{summary-5}
\addcontentsline{toc}{section}{Summary}

\markright{Summary}

In this unit, you learned about:

\begin{itemize}
\tightlist
\item
  The In-Camera Project and how It is beneficial to help you to
  understand how to think of filmmaking in a linear way.
\item
  Secrets to a Simple Story and how limiting your story to a key single
  concept or theme with only a few characters and locations will help
  you create a stronger film.
\item
  Brainstorming and Successful Creativity and how the more ideas and
  concepts you imagine and generate, the better your film will capture
  an audience's attention.
\item
  Short Story Film Template where a fairy tale story formula can help
  you design the beginning, middle, and end of your film.
\item
  Direction Vectors and Eye-line Requirements which are important
  cinematic compositional devices that prevent the audience's confusion
  as you transition between shots.
\end{itemize}

\begin{tcolorbox}[enhanced jigsaw, opacityback=0, colframe=quarto-callout-note-color-frame, leftrule=.75mm, arc=.35mm, rightrule=.15mm, colbacktitle=quarto-callout-note-color!10!white, titlerule=0mm, colback=white, toprule=.15mm, bottomtitle=1mm, breakable, toptitle=1mm, title={Checking Your Learning}, coltitle=black, bottomrule=.15mm, left=2mm, opacitybacktitle=0.6]

Before you move on to the next unit, you may want to check to make sure
that you are able to:

\begin{itemize}
\tightlist
\item
  Describe what constitutes a good, basic, and simple story
\item
  Define what constitutes an ``In-Camera'' film
\item
  Analyze and apply a story template
\item
  Determine how to tell a visual-centered story well
\item
  Create a short simple story to film
\end{itemize}

\end{tcolorbox}

\bookmarksetup{startatroot}

\chapter{Editing and Montage}\label{editing-and-montage}

\section*{Overview}\label{overview-6}
\addcontentsline{toc}{section}{Overview}

\markright{Overview}

Congratulations. You've made your first mini-movie in this course. Don't
worry if it did not live up to your expectations. The important thing
right now is that you wrote, produced, directed, and shot a short film
and that you are improving your skills. The rest of your course will
give you a chance to explore more elements of filmmaking and gain more
experience.

In this unit you will focus on editing and creating a montage, a common
form of filmmaking.

Unlike the previous project, you can now use dialogue, voice over,
music, and sound effects to enhance your short film project. You will
also use editing which will allow you to shoot your film out of order
and then assemble it after you have all of the elements you need to make
it.

As you saw in Unit 2, editing is the third major phase of filmmaking and
a major skill to understand.

Don't worry if you do not want to be an editor. You will work in teams
this time and as long as the work on the film is shared equally between
yourself and your partner(s), you can offer editing suggestions but will
not have to push the buttons. (This is how it works in the industry
where the director and producer tell the editor what they want and the
editor does the technical work.)

\subsection*{Topics}\label{topics-6}
\addcontentsline{toc}{subsection}{Topics}

This unit is divided into the following topics covered in the course
text:

\begin{enumerate}
\def\labelenumi{\arabic{enumi}.}
\tightlist
\item
  Post-Production Workflow
\item
  Montage
\item
  Picture Cutting Techniques
\item
  Types of Montage Projects
\item
  Digital Video Editing Software
\item
  Pitching Your Montage Project
\item
  Steps for Creating Your Montage
\end{enumerate}

\subsection*{Learning Outcomes}\label{learning-outcomes-6}
\addcontentsline{toc}{subsection}{Learning Outcomes}

When you have completed this unit, you will be able to:

\begin{itemize}
\tightlist
\item
  Define the Post-Production workflow
\item
  Analyze the necessary elements needed for your montage
\item
  Determine the steps needed in making a montage
\item
  Create an effective montage
\item
  Evaluate the quality of a montage
\end{itemize}

\subsection*{Learning Activities}\label{learning-activities-6}
\addcontentsline{toc}{subsection}{Learning Activities}

Here is a checklist of learning activities you will benefit from in
completing this unit. You may find it useful for planning your work.

\begin{itemize}
\tightlist
\item
  Read and study Chapter7 of course etext: \emph{Digital Filmmaking: A
  Beginner's Guide to Mastering the Craft.} The readings are divided
  into 6 topics:

  \begin{itemize}
  \tightlist
  \item
    The Post-Production Workflow
  \item
    Montage
  \item
    Types of Montages
  \item
    Picture Cutting Techniques.
  \item
    Digital Editing Software
  \item
    Creating a Montage
  \end{itemize}
\item
  Watch videos related to the topics previously mentioned.
\item
  As you read Chapter 7 ``Digital Editing Software'' section, find the
  digital video editing program you want to use and learn how to do
  basic editing with it.
\item
  Practice doing your pitch before a friend or family member or to an
  imaginary audience before making the actual pitch to your course
  members.
\end{itemize}

\begin{tcolorbox}[enhanced jigsaw, opacityback=0, colframe=quarto-callout-note-color-frame, leftrule=.75mm, colback=white, toprule=.15mm, breakable, arc=.35mm, rightrule=.15mm, bottomrule=.15mm, left=2mm]
\begin{minipage}[t]{5.5mm}
\textcolor{quarto-callout-note-color}{\faInfo}
\end{minipage}%
\begin{minipage}[t]{\textwidth - 5.5mm}

Working through course activities will help you to meet the learning
outcomes and successfully complete your assessments.

\end{minipage}%
\end{tcolorbox}

\subsection*{Assessment}\label{assessment-6}
\addcontentsline{toc}{subsection}{Assessment}

\textbf{Course Journal}

After completing this unit, including the learning activities, you are
asked to make sure you are doing journal entries and when appropriate to
share your responses with your facilitator and classmates when you meet.

Note that entries are expected after every unit. Your journal
reflections are submitted at the end of the course as part of the
Journal One: Personal Journal and self assessment.

\textbf{Montage Short Project (30\%)}

This montage project will account for 30 \% of your grade and will be
broken down according to the following:

\begin{itemize}
\tightlist
\item
  Story/Montage Concept: 20\%
\item
  Cinematography: 20\%
\item
  Editing: 40\%
\item
  Overall all project quality: 20\%
\end{itemize}

\begin{tcolorbox}[enhanced jigsaw, opacityback=0, colframe=quarto-callout-note-color-frame, leftrule=.75mm, colback=white, toprule=.15mm, breakable, arc=.35mm, rightrule=.15mm, bottomrule=.15mm, left=2mm]
\begin{minipage}[t]{5.5mm}
\textcolor{quarto-callout-note-color}{\faInfo}
\end{minipage}%
\begin{minipage}[t]{\textwidth - 5.5mm}

Please see the Assessment section in Moodle for assignment details as
well as the grading criteria.

\end{minipage}%
\end{tcolorbox}

\subsection*{Resources}\label{resources-6}
\addcontentsline{toc}{subsection}{Resources}

Here are the resources you will need to complete this unit.

\begin{itemize}
\tightlist
\item
  Chapter 7 of course text: \emph{Digital Filmmaking: A Beginner's Guide
  to Mastering the Craft,} by Ned Vankevich (e-text)
\item
  \href{https://www.youtube.com/watch?v=IFjyVW21Vpw}{Stages of Post
  Production for Filmmaking in Cinema}
\item
  \href{https://www.refinery29.com/en-us/best-movie-montages}{``20
  Montages That Might Be The Best Part Of The Movie}
\item
  \href{https://www.youtube.com/watch?v=dak2DkfDTuU}{How To Pitch A
  Project}
\item
  \href{https://www.youtube.com/watch?v=OAH0MoAv2CI&t=31s}{Cuts \&
  Transitions 101}
\item
  \href{https://www.youtube.com/watch?v=Wv3Hmf2Dxlo}{9 Cuts Every Video
  Editor Should Know \textbar{} Filmmaking Tips}
\item
  \href{https://www.youtube.com/watch?v=bdpbYeoZKnk}{VIDEO EDITING
  TRANSITIONS (Taylor Cut Tutorial)}
\item
  \href{https://www.youtube.com/watch?v=x5ASDkOSIBE}{Critiquing Your
  Short Films}
\item
  Other online resources provided in the course text and this unit.
\end{itemize}

\section{Post-Production Workflow}\label{post-production-workflow}

There are many steps and elements involved in the post-production
process. They include:

\begin{itemize}
\tightlist
\item
  Importing Footage and elements
\item
  Sound syncing
\item
  Picture cutting
\item
  Transitions
\item
  Dialogue editing
\item
  Sound design
\item
  Adding music
\item
  Adding sound effects
\item
  Adding visual and special effects
\item
  Color correcting
\item
  Adding tiles and graphics
\item
  Etc.
\end{itemize}

\subsection{Activity: Planning Your
Film}\label{activity-planning-your-film-1}

\begin{tcolorbox}[enhanced jigsaw, opacityback=0, colframe=quarto-callout-note-color-frame, leftrule=.75mm, arc=.35mm, rightrule=.15mm, colbacktitle=quarto-callout-note-color!10!white, titlerule=0mm, colback=white, toprule=.15mm, bottomtitle=1mm, breakable, toptitle=1mm, title={Learning Activity}, coltitle=black, bottomrule=.15mm, left=2mm, opacitybacktitle=0.6]

Before completing your reading for this unit, log in your journal what
type of film you want to work on for your final film project and why.

Then, read the first section in Chapter 7, the Post-Production Workflow.
For a detailed overview of the post-production process, watch the video
\href{https://www.youtube.com/watch?v=IFjyVW21Vpw}{Watch: \emph{Stages
of Post Production for Filmmaking in Cinema}}

\url{https://www.youtube-nocookie.com/embed/IFjyVW21Vpw}

\end{tcolorbox}

\section{Montage}\label{montage}

Montage involves a type of editing where a series of images and sound
elements are most often used used to condense time, create emotion, tell
a story, reveal something from the past, promote something, or impart
information.

Montages can stand alone or be part of a larger documentary, film or
television story, or a stand-alone film.

You will focus on a montage project in this unit because they are a
great way to learn the art of editing when you shoot images and edit
them in timed rhythm to music, dialogue, and/or sound effects. In this
exercise you are going to let emotions, feelings, and mood guide you as
you learn to further develop your intuitive creative skills.

\subsection{Activity: Movie Montages}\label{activity-movie-montages}

\begin{tcolorbox}[enhanced jigsaw, opacityback=0, colframe=quarto-callout-note-color-frame, leftrule=.75mm, arc=.35mm, rightrule=.15mm, colbacktitle=quarto-callout-note-color!10!white, titlerule=0mm, colback=white, toprule=.15mm, bottomtitle=1mm, breakable, toptitle=1mm, title={Learning Activity}, coltitle=black, bottomrule=.15mm, left=2mm, opacitybacktitle=0.6]

Read the second section of your text on the Montage. To get a feeling
for how montages work and for how effective they can be view this
resource:

\href{https://www.refinery29.com/en-us/best-movie-montages}{``20
Montages That Might Be The Best Part Of The Movie''}

\end{tcolorbox}

\section{Picture Cutting Techniques}\label{picture-cutting-techniques}

This montage project is helpful because you have to focus on a few
elements: the images you will use and the music and/or sound effects
that will provide the beats and rhythm for your picture cutting.

Although the music will guide your editing as you cut on beats, the
images you edit will be important. Once again, good composition and
motivate motions shots will play a vital role as you shot your footage
for your montage project. You can also used ``found'' or archival
footage, or news stories as your visual sources in a montage.

There are many types of cuts and transitions that can be used in a
montage as you splice and stich your footage together including:

\begin{itemize}
\tightlist
\item
  Cuts
\item
  Jump-Cuts
\item
  Cross-Cutting
\item
  Dissolves
\item
  Wipes
\item
  Fades
\end{itemize}

You can use J-cut and L-cuts when you are working with dialogue or you
want to lead into a new shot or scene or carry the audio to a new shot
or scene. If you are interested in doing this, watch the video
\href{https://www.youtube.com/watch?v=PAvJevWUVsc}{Watch: \emph{Film
Making 101 Tutorial - L \& J Cuts}}

\url{https://www.youtube-nocookie.com/embed/PAvJevWUVsc}

\subsection{Activity: Picture Cutting
Techniques}\label{activity-picture-cutting-techniques}

\begin{tcolorbox}[enhanced jigsaw, opacityback=0, colframe=quarto-callout-note-color-frame, leftrule=.75mm, arc=.35mm, rightrule=.15mm, colbacktitle=quarto-callout-note-color!10!white, titlerule=0mm, colback=white, toprule=.15mm, bottomtitle=1mm, breakable, toptitle=1mm, title={Learning Activity}, coltitle=black, bottomrule=.15mm, left=2mm, opacitybacktitle=0.6]

Read the text section on Picture Cutting Techniques. For visual examples
of Picture Cutting Techniques, watch the videos

\href{https://www.youtube.com/watch?v=Wv3Hmf2Dxlo}{Watch: \emph{Cuts
Every Video Editor Should Know \textbar{} Filmmaking Tips}}

\url{https://www.youtube-nocookie.com/embed/Wv3Hmf2Dxlo}

\href{https://www.youtube.com/watch?v=Wv3Hmf2Dxlo}{Watch: \emph{Cuts \&
Transitions 101}}

\url{https://www.youtube-nocookie.com/embed/OAH0MoAv2CI?si=nUtixWWane1Ryrxl}

\textbf{Optional}: For an excellent example of a nature and time-lapse
montage where a series of images are synced into a powerful soundtrack ,
watch the video
\href{https://www.youtube.com/watch?v=oagszCmJLpU}{``Pursuit - A 4K
storm time-lapse film''.} As you view it, pay attention to the way the
types and rhythms of the images flow and work in sync with the music and
sound effects especially from 2: 29 onward.

\url{https://www.youtube-nocookie.com/embed/oagszCmJLpU}

Lastly, log in in your journal what you learn about this montage

\end{tcolorbox}

\section{Types of Montage Projects}\label{types-of-montage-projects}

You are given creative latitude and freedom for the type of montage you
will shoot and edit for this unit. Some of the most popular ones
include:

\begin{itemize}
\tightlist
\item
  Poetic Montages
\item
  Nature Montages
\item
  Street Montages
\item
  Sequence Sprint
\item
  Music Videos
\item
  Aesthetic Montages
\end{itemize}

You can also combine these types of montages. As you read the
description of these different types, have fun choosing the one you want
to do.

\subsection{Activity: Types of Montage
Projects}\label{activity-types-of-montage-projects}

\begin{tcolorbox}[enhanced jigsaw, opacityback=0, colframe=quarto-callout-note-color-frame, leftrule=.75mm, arc=.35mm, rightrule=.15mm, colbacktitle=quarto-callout-note-color!10!white, titlerule=0mm, colback=white, toprule=.15mm, bottomtitle=1mm, breakable, toptitle=1mm, title={Learning Activity}, coltitle=black, bottomrule=.15mm, left=2mm, opacitybacktitle=0.6]

Read the Types of Montage Projects in the textbook. For an excellent
example of a nature and time-lapse montage where a series of images are
synced in to a powerful sound track view this resource: \textbf{Pursuit
- A 4K storm time-lapse film}

\url{https://www.youtube-nocookie.com/embed/oagszCmJLpU}

As you view it, pay attention to the way the types and rhythms of the
images flow and work in sync with the music and sound effects especially
from 2: 29 onward.

What did you learn about this montage? Log this in your journal.

\end{tcolorbox}

\section{Digital Video Editing
Software}\label{digital-video-editing-software}

You are now at a point where you will have to import your cellphone or
DSLR footage into a software program that will allow you to edit it.
Digital video editing software is helpful because it allows you to edit
your images and sync them to the rhythm and beats of your sound track
and export your project for viewing or streaming on the Internet.

There are many types and brands of editing software in the marketplace
and you will have to choose the one that works for you. Some
applications only work with Android-based cellphones and others for
Apple ISO devices and some for both platforms such as imovie. If you
want to really get series about editing you should explore Final Cut
Pro, Adobe Premiere, Avid, DaVinci Resolve, or some other professional
program.

There are similarities among editing software and programs but you will
have to do a tutorial on YouTube or the manufacture's website of the one
you will use to see how to work it. The more popular the software, the
more tutorials will be available.

\subsection{Activity: Read, Study, and
Explore}\label{activity-read-study-and-explore}

\begin{tcolorbox}[enhanced jigsaw, opacityback=0, colframe=quarto-callout-note-color-frame, leftrule=.75mm, arc=.35mm, rightrule=.15mm, colbacktitle=quarto-callout-note-color!10!white, titlerule=0mm, colback=white, toprule=.15mm, bottomtitle=1mm, breakable, toptitle=1mm, title={Learning Activity}, coltitle=black, bottomrule=.15mm, left=2mm, opacitybacktitle=0.6]

Read and study Chapter 7, Digital Editing Software section. Next, Find
the digital video editing program you want to use and learn how to do
basic editing with it. Basic editing using cuts, dissolves, fades, and
wipes titling will be all you need to learn for your montage project.

The
\href{https://filmora.wondershare.net/filmora-video-editor.html?}{Filmora}
website gives an overview of the type of features you want to look for.
For free and inexpensive Android-based video software
see:\href{https://fossbytes.com/best-android-video-editor-apps/}{13 Best
Android Video Editor Apps Of 2022} or search for best cellphone or DSLR
video editing software such as
\href{https://www.tomsguide.com/best-picks/best-video-editing-apps}{Best
video editing apps 2023}

\textbf{\emph{Note:}} When you pitch your montage project, see who
already has editing software and who knows how to use it and try to join
with them if you are a bit of a ``tech-phobic.'' If you can't find
someone, challenge yourself and learn something new. You will learn and
grow a lot doing this.

\end{tcolorbox}

\section{Pitching Your Montage
Project}\label{pitching-your-montage-project}

Because the montage project is more complicated, you are encouraged to
work in teams to create an excellent montage. You can do the project
solo if you feel you have the experience and understanding of the gear
and software to create it.

The reason almost all larger-scale film projects are done in teams with
a lot of different crew positions is that over time, people specialize
in their skill-sets and bring better expertise to a project. This is why
film crews have a cinematographer, production designer, wardrobe, hair
and make-up specialists, prop masters, special effects experts, etc.
(You should be consulting your film terms website if you do not know
these positions.)

For this project, ideally you will have a team of two people. One to
come up with the concept and film it and the other to edit it. In
reality, both members will share the idea development, shooting, and
editing. The important thing is that you both share equally in the work.
(In some cases if the montage is highly complex or sophisticated you can
have 3 or 4 members on the team as long as everyone does their fair
share.)

In order to find the find teammate, you will pitch your project to the
class to see who wants to work on it.

Pitching is a process central to the film and television industry. It
involves an individual or team presenting their film project concept and
why it is important and to a special audience such as a producer, agent,
studio or television executive, distributor, film competition, etc.
Everyone, no matter how big or important, has to pitch to those in the
industry to find the money, resources, actors, crew, and/or distribution
for their project.

Pitching is a vital part of filmmaking and it is a great transferable
skill to learn since you will have to pitch in many professions and
industries today.

\subsection{Activity: Pitching your
Project}\label{activity-pitching-your-project}

\begin{tcolorbox}[enhanced jigsaw, opacityback=0, colframe=quarto-callout-note-color-frame, leftrule=.75mm, arc=.35mm, rightrule=.15mm, colbacktitle=quarto-callout-note-color!10!white, titlerule=0mm, colback=white, toprule=.15mm, bottomtitle=1mm, breakable, toptitle=1mm, title={Learning Activity}, coltitle=black, bottomrule=.15mm, left=2mm, opacitybacktitle=0.6]

You will have to develop a montage concept to pitch such as a day in the
life of street vendors, or a music video, or visualizing a poem or
something from the Bible. So review the brainstorming techniques in the
previous unit and come up with at least 10 concepts and then choose the
one that most appeals to you.

Then review the section on \emph{Pitching} in Chapter Two of the course
text. Study \emph{How to Pitch Your Montage Project} and \emph{Steps for
an Effective Pitch} in the Chapter 7 and practice doing your pitch
before a friend or family member or to an imaginary audience before
making the actual pitch to your course members.

\textbf{Note}: Do not skimp on practicing your pitch. The more you
practice it you will gain confidence and know how to present your
concept in an effective way.

\textbf{Guidelines for Pitching a Film Project}

\textbf{Purpose of a Pitch:} To get people to work on your project,
commission your project, fund your project, help with your project,
distribute your project, etc. Pitches also help you to clarity what you
want to do and why.

\textbf{Steps to a Good Pitch}

\begin{enumerate}
\def\labelenumi{\arabic{enumi}.}
\tightlist
\item
  If you can, open with a teaser or dramatic statement or question.
\item
  Meet and greet your audience---minimum chitchat, professionals are
  busy, busy people.
\item
  Dress appropriately---relative to the audience and the project.
  Artists dress ``artsy''; corporate people dress with business attire.
\item
  Project the proper persona (your personality and
  character)---knowledgeable, credible, trustworthy, smart, clever, etc.
\item
  Explain your qualifications, background, and why you are competent and
  the right person to do the project. If you do not have a demo reel
  then sell people with your passion, e.g., WHIPLASH filmmaker.
\item
  Give the title of the film/project---titles shape perception and
  expectations.
\item
  State the Genre---comedy, drama, documentary, etc. This also shapes
  perceptions and expectations. The basic concept of a comedy should
  make us laugh or at lease smile.
\item
  Give the Unique Angle or Hook of your project---what makes it stand
  out as something different and catchy/edgy/clever, etc. E.g., SHOE IN
  LOVE, a romantic comedy from the POV of a pair of cowboy boots and
  stilettos falling in love with each other. The special angle concept
  is to use foot ware as the main characters and the execution hook is
  to shoot POVs at foot level or from the shin down.
\item
  Briefly state the logline or basic concept: a one or two sentence
  description of the core story. E.g., a young, bored dirt farmer gets a
  message that he has to help rescue a faraway princes and save the
  universe. STAR WARS.
\item
  Project synopsis---succinctly tell the beginning/middle/end of your
  story or if a documentary the main message of your doc.
\item
  Summarize your end goal: to wow my audience with extraordinary nature
  photography or to make my audience laugh, cry, fall in love, etc.
\item
  Explain what you need---cast, crew, funding, etc.
\item
  Ask for Questions---also be open to criticism and constructive
  comments and feedback.
\item
  Gratitude---be thankful and gracious even if your project is not
  received well or as anticipated. Burning bridges is wrongheaded.
  Pitching can be a training ground for how to deal with rejection.
\end{enumerate}

\textbf{Examples of Types of Montage Projects to Pitch}

The goal of your Montage Short assignment (see the syllabus) will be to
explore how to make an effective montage-based film that uses picture
and sound editing to convey meaning, emotion, story and/or mood.

For this project, you can:

\begin{itemize}
\tightlist
\item
  Create a music video where you take a song and create images that
  express the mood, feeling, meaning, etc. of the song.
\item
  Find a poem, fairy tale, or passage from the Bible and ``visualize''
  and enhance it with music and sound effects.
\item
  Create a chase scene---someone steals something and is on the run and
  using music and sound effects to enhance the excitement and dynamism.
\item
  Create a love or horror or thriller story ---use can use dialogue but
  there must be an edited montage---a series of edited shots that convey
  the mood, meaning etc.
\item
  Etc.
\end{itemize}

\end{tcolorbox}

\section{Steps for Creating Your
Montage}\label{steps-for-creating-your-montage}

Once you have your montage project and team in place, you are ready to
make your montage. As we have seen, there are three main phases of film
production and they will apply here:

\begin{itemize}
\tightlist
\item
  Montage Pre-Production
\item
  Montage Production
\item
  Montage Post-Production
\end{itemize}

The lion's share of the emphasis in the chapter has been on
post-production, but this does not mean you must take your
pre-production and production work lightly. Your montage will only be as
good as the images you have for it.

\subsection{Activity: Creating a
Montage}\label{activity-creating-a-montage}

\begin{tcolorbox}[enhanced jigsaw, opacityback=0, colframe=quarto-callout-note-color-frame, leftrule=.75mm, arc=.35mm, rightrule=.15mm, colbacktitle=quarto-callout-note-color!10!white, titlerule=0mm, colback=white, toprule=.15mm, bottomtitle=1mm, breakable, toptitle=1mm, title={Learning Activity}, coltitle=black, bottomrule=.15mm, left=2mm, opacitybacktitle=0.6]

Study the next section of Chapter 7 on creating a montage. As you study
montage pre-production, production, and post-production make notes on
your shooting script of what to watch out for and not forget as you film
the shots and images you will use.

\end{tcolorbox}

\subsection{Activity: Feedback and Self-Assessment for Montage
Project}\label{activity-feedback-and-self-assessment-for-montage-project}

\begin{tcolorbox}[enhanced jigsaw, opacityback=0, colframe=quarto-callout-note-color-frame, leftrule=.75mm, arc=.35mm, rightrule=.15mm, colbacktitle=quarto-callout-note-color!10!white, titlerule=0mm, colback=white, toprule=.15mm, bottomtitle=1mm, breakable, toptitle=1mm, title={Learning Activity}, coltitle=black, bottomrule=.15mm, left=2mm, opacitybacktitle=0.6]

Once you have edited your project and exported it, you are ready to show
it to your classmates and instructor. This should be an exciting time as
you screen the fruits of your labor and creativity.

As mentioned previously, film is a public medium and subject to critique
and evaluation.

The following criteria can help us better evaluate our own work and that
of others and to give constructive criticism on how to improve: -
Overall did the montage work? - Did the music and/or sound effects work
well with the images? - Did the montage convey a mood, emotion, a story
and/or theme (some insight into life)? If so, what was it? - Were the
shots well composed, motivated, and appropriately smooth or in some
cases jarring depending on what is being communicated? - Were the cuts
timed well? If not, where did they not work well? - Were the transitions
appropriate and effective? - Was the action followed well? - Was there a
sufficient variety of shots? - Were some shots or images too repetitive?
- Were the shots and images interesting and noteworthy? - What did you
like best about the project? - What suggestions would you offer to make
it better?

Evaluate and critique the work of your peers helping them to see what
worked and did not work in their montage and why. When possible give
suggestions of how something could have been done better. Remember the
tone and substance of our critiques should be one that offers
encouraging constructive insight and that helps each other to grow and
improve their craft.

For insight into how to critique a short film watch the video
\href{https://www.youtube.com/watch?v=x5ASDkOSIBE}{Watch:
\emph{Critiquing Your Short Films}}

\url{https://www.youtube-nocookie.com/embed/x5ASDkOSIBE}

Write in your journal what you learned about your project and yourself
from the montage projects.

Use the following questions to guide you: - What was the best lesson I
learned? - What did I do well? Why? - What did not work out as well as I
planned? Why? - What can I do better next time? - What was the quality
of my experience working with a partner? - If I did not work with a
partner, could the project have been better if I had one? - How did this
experience help me grow as a person and as a professional? - What did I
learn new about myself?

Be sure to note which films and filmmakers had the best cinematography,
editing, story, creativity, etc. so that you can consult this when you
vote on the awards at the end of the course.

\end{tcolorbox}

\section*{Summary}\label{summary-6}
\addcontentsline{toc}{section}{Summary}

\markright{Summary}

In this unit, you learned about:

\begin{itemize}
\tightlist
\item
  Post-Production Workflow which helps you to organize the process of
  how you will edit your film project.
\item
  Montage which is a technique or type of film where pictures and sounds
  are interwoven to create a story, mood, and theme.
\item
  Picture Cutting Techniques that are the transitions you will need to
  move from shot to shot and scene to scene.
\item
  Types of Montage include sports action, fight and chase scenes,
  parallel stories, film poems, among others.
\item
  Digital Video Editing Software that you must find that is appropriate
  for your computer, tablet, or cell phone operating system.
\item
  Steps for Creating Your Montage where you develop your story, plan and
  shoot your shots, find the appropriate music and sound effects and
  create the appropriate transition for them.
\item
  Pitching Your Montage Project where you explain your story idea and
  why it is worth doing and how you will develop and produce it.
\item
  Feedback and Self-Assessment for Montage Project where your project
  will be critiqued by your professor and your peers; and you will
  explain what you learned from this experience.
\end{itemize}

\begin{tcolorbox}[enhanced jigsaw, opacityback=0, colframe=quarto-callout-note-color-frame, leftrule=.75mm, arc=.35mm, rightrule=.15mm, colbacktitle=quarto-callout-note-color!10!white, titlerule=0mm, colback=white, toprule=.15mm, bottomtitle=1mm, breakable, toptitle=1mm, title={Checking Your Learning}, coltitle=black, bottomrule=.15mm, left=2mm, opacitybacktitle=0.6]

Before you move on to the next unit, you may want to check to make sure
that you are able to:

\begin{itemize}
\tightlist
\item
  Define the Post-Production workflow
\item
  Describe how to pitch a project well
\item
  Analyze the necessary elements needed for your montage
\item
  Determine the steps needed in making a montage
\item
  Create an effective montage
\item
  Evaluate the quality of a montage
\end{itemize}

\end{tcolorbox}

\bookmarksetup{startatroot}

\chapter{Creating Narrative and Other Short
Films}\label{creating-narrative-and-other-short-films}

\section*{Overview}\label{overview-7}
\addcontentsline{toc}{section}{Overview}

\markright{Overview}

More kudos to you. You now have two film projects and a lot of film
exercises in your experience bank account. You will now be moving up a
level where you will have the opportunity to bring together all of what
you have learned so far into a longer narrative film.

In the previous units you focused on how to make shorter films. In this
unit and the next you will focus on creating longer forms where your
film will be five to ten minutes in length. This might not seem long but
the longer the film, the more you have to do to sustain interest and
continually engage the audience. Mini-films (2-minutes and under) have
less of this burden because an audience is not feeling like they are
wasting a bigger chunk of their time if the film is not well done.

Narrative films, whether fiction or non-fiction, are popular because
they tell stories. Even if you do not want to be a ``Hollywood''
filmmaker learning to tell film stories will aid you in your personal
and professional development. Human beings are hard-wired for
storytelling. It is a primary way we make sense of the world and pass
our knowledge and cultural values and traditions to others. Those who
control the narratives of a society have powerful influence and impact.

Before moving ahead, view this resource:

\href{https://www.youtube.com/watch?v=mYnsKATCrdw}{Watch: \emph{7 Things
to Know About Making Short Films! : FRIDAY 101}}

\url{https://www.youtube-nocookie.com/embed/mYnsKATCrdw}

\subsection*{Topics}\label{topics-7}
\addcontentsline{toc}{subsection}{Topics}

This unit is divided into the following topics:

\begin{enumerate}
\def\labelenumi{\arabic{enumi}.}
\tightlist
\item
  Types of Short Films
\item
  Film Genres
\item
  The Logistics for the Final Project
\item
  Script and Story Development
\item
  Creating a Scriptment
\end{enumerate}

\subsection*{Learning Outcomes}\label{learning-outcomes-7}
\addcontentsline{toc}{subsection}{Learning Outcomes}

When you have completed this unit, you will be able to:

\begin{itemize}
\tightlist
\item
  Describe the different types of short films
\item
  Define film genre and why it is important
\item
  Determine how to approach making a film script to shoot and edit.
\item
  Create a shooting script
\end{itemize}

\subsection*{Learning Activities}\label{learning-activities-7}
\addcontentsline{toc}{subsection}{Learning Activities}

Here is a checklist of learning activities you will benefit from in
completing this unit. You may find it useful for planning your work.

\begin{itemize}
\tightlist
\item
  Log in your journal what type of film you want to work on for your
  final film project and why.
\item
  Review popular short film genres.
\item
  Follow the steps in \emph{Story Research} section of Chapter Eight and
  ask and answer the questions in the \emph{What to Look for} section.
  View the video selected.
\item
  Brainstorm and develop a core story idea to pitch to the class.
\item
  Create a scriptment with your group.
\end{itemize}

\begin{tcolorbox}[enhanced jigsaw, opacityback=0, colframe=quarto-callout-note-color-frame, leftrule=.75mm, colback=white, toprule=.15mm, breakable, arc=.35mm, rightrule=.15mm, bottomrule=.15mm, left=2mm]
\begin{minipage}[t]{5.5mm}
\textcolor{quarto-callout-note-color}{\faInfo}
\end{minipage}%
\begin{minipage}[t]{\textwidth - 5.5mm}

Working through course activities will help you to meet the learning
outcomes and successfully complete your assessments.

\end{minipage}%
\end{tcolorbox}

\subsection*{Assessment}\label{assessment-7}
\addcontentsline{toc}{subsection}{Assessment}

\textbf{Course Journal}

After completing this unit, including the learning activities, you are
asked to make sure you are doing journal entries and when appropriate to
share your responses with your facilitator and classmates when you meet.

Also, log in your journal what you learned from the creating and filming
of your In-Camera exercise. Log what you thought about the films of your
peers and what you learned from the feedback from them and your
instructor.

Be sure to make a note of what films stood out and who did them. You
will consult this at the end of the course when the class determines:
Best Overall Film. Most Imaginative Film, Best Story, Best
Cinematography, Best Editing, etc.

\begin{tcolorbox}[enhanced jigsaw, opacityback=0, colframe=quarto-callout-note-color-frame, leftrule=.75mm, colback=white, toprule=.15mm, breakable, arc=.35mm, rightrule=.15mm, bottomrule=.15mm, left=2mm]
\begin{minipage}[t]{5.5mm}
\textcolor{quarto-callout-note-color}{\faInfo}
\end{minipage}%
\begin{minipage}[t]{\textwidth - 5.5mm}

Please see the Assessment section in Moodle for assignment details as
well as the grading criteria.

\end{minipage}%
\end{tcolorbox}

\subsection*{Resources}\label{resources-7}
\addcontentsline{toc}{subsection}{Resources}

Here are the resources you will need to complete this unit.

\begin{itemize}
\tightlist
\item
  Chapter Eight of the course text
\item
  \href{https://www.youtube.com/watch?v=mYnsKATCrdw}{7 Things to Know
  About Making Short Films! : FRIDAY 101}
\item
  \href{https://milnepublishing.geneseo.edu/exploring-movie-construction-and-production/chapter/2-what-is-genre-and-how-is-it-determined/}{What
  Is Genre and How Is It Determined?}
\item
  \href{https://www.youtube.com/watch?v=wMqIQcTMlA0}{How to Write a
  Short Film}
\item
  \href{https://www.youtube.com/watch?v=F7oi_V7JXCg}{Writing 101: Basic
  Story Structure}
\item
  \href{http://classics.mit.edu/Aristotle/poetics.mb.txt}{Poetics by
  Aristotle, translated by S. H. Butcher, The Internet Classics Archive}
\item
  \href{https://screencraft.org/2018/02/27/outlines-treatments-and-scriptments-oh-my/}{Outlines,
  Treatments, and Scriptments, Oh My! by Ken Miyamoto}
\item
  Other resources will be available online and in the course text.
\end{itemize}

\section{Types of Short Films}\label{types-of-short-films}

There are three main types of short films that we can do for the project
ahead:

\begin{itemize}
\tightlist
\item
  Classical Narrative
\item
  Documentary
\item
  Experimental, \emph{Avant-Garde}, and Surreal Cinema
\end{itemize}

Most of you will probably choose the classical narrative, but you are
free to do the other two with the caveat that
experimental/\emph{avant-garde} films are far more challenging than they
might appear.

As a transferable skill, like in many industries and businesses, smart
filmmakers look for a market before they commit too much time and money
to a film project. Others of course rely on their intuition and passion.
Either way, it is wise to see what those who distribute or stream and
screen short films, such as festivals and competitions, look for in
making their choices. Their criteria can you vet the quality of your
story and its execution. See Film Shortage's
\href{https://filmshortage.com/what-we-like/}{What Type Of Short Films
Do We Prefer?}

\section{Film Genres}\label{film-genres}

The main films genres include:

\begin{itemize}
\tightlist
\item
  Drama
\item
  Comedy
\item
  Romance
\item
  Action
\item
  Thriller
\item
  Horror
\item
  Gangster
\item
  Crime
\item
  Adventure
\item
  Westerns
\item
  Sci-Fi
\item
  Fantasy
\item
  Historical
\item
  Epic
\item
  War
\item
  Bio Pics (biographies)
\end{itemize}

Knowing your film genre is important because it embodies narrative
patterns and expectations audiences have, and if these are not fulfilled
you will lessen your impact on those watching your film. Put another
way, comedies need to be funny, horror films need to induce fear,
romantic films need to inspire us to love, etc. Commercially, if these
basic requirements are not met it can result in financial disaster and
loss of reputation. At your level it will mean your film is not as
effective as it could be.

Knowing your genre will also help you develop your story and script in
this and the following unit.

\subsection{Activity: Planning your Final Film
Project}\label{activity-planning-your-final-film-project}

\begin{tcolorbox}[enhanced jigsaw, opacityback=0, colframe=quarto-callout-note-color-frame, leftrule=.75mm, arc=.35mm, rightrule=.15mm, colbacktitle=quarto-callout-note-color!10!white, titlerule=0mm, colback=white, toprule=.15mm, bottomtitle=1mm, breakable, toptitle=1mm, title={Learning Activity}, coltitle=black, bottomrule=.15mm, left=2mm, opacitybacktitle=0.6]

Log in your journal what type of film you want to work on for your final
film project and why.

Next, view this list of loglines (short one line descriptions of a film
story) in various genres:
\href{https://www.slideshare.net/BigBadBoobyLady/popular-short-film-genres}{Popular
Short film genres.}

Which genre appeals to you most? Why? Log this reflection and start to
brainstorm genre-based story concepts.

To improve your grasp of genre consult this resource:

\href{https://milnepublishing.geneseo.edu/exploring-movie-construction-and-production/chapter/2-what-is-genre-and-how-is-it-determined/}{What
Is Genre and How Is It Determined?}

\end{tcolorbox}

\section{Logistics for the Final Film
Project}\label{logistics-for-the-final-film-project}

This last film is a major project and will account for a lot of your
grade given that it will reveal what you learned during this course. In
light of this, you will need to create a strong story and the
``blueprint'' or script for filming it.

To save time so that you can put more of your energy into making the
film, you will not have to develop a full screenplay (hence the
scriptment section below) for the final project, but you are strongly
encouraged to do so.

However, this does not mean you do not have to pitch a core story
concept (a basic story, its genre, and hook or unique angle) to the
class. Everyone will be required to do this so that you can gain more
experience with your creativity and your pitching skills.

Once you have created a story concept that inspires you, pitch it to the
class to see if other classmates want to join you in your project and to
get feedback on it. Once you have your team in place you will then
follow the process for creating a shooting script for pre-production and
production.

\subsection{Activity: Story Research}\label{activity-story-research}

\begin{tcolorbox}[enhanced jigsaw, opacityback=0, colframe=quarto-callout-note-color-frame, leftrule=.75mm, arc=.35mm, rightrule=.15mm, colbacktitle=quarto-callout-note-color!10!white, titlerule=0mm, colback=white, toprule=.15mm, bottomtitle=1mm, breakable, toptitle=1mm, title={Learning Activity}, coltitle=black, bottomrule=.15mm, left=2mm, opacitybacktitle=0.6]

In the next activity, you'll be asked to brainstorm and develop a core
story idea to pitch to the class.

To help you do this, follow the steps in Story Research Section of
Chapter Eight and ask and answer the questions in the What to Look for
Section. This should prime your creative pump for the next topic. Also
view this helpful resource:

\href{https://www.youtube.com/watch?v=wMqIQcTMlA0}{Watch: \emph{How to
Write a Short Film}}

\url{https://www.youtube-nocookie.com/embed/wMqIQcTMlA0}

\end{tcolorbox}

\section{Script and Story
Development}\label{script-and-story-development}

Our goal in this unit is to find and create a strong story to film. As
we have seen, brainstorming is a great method to help do this.

Once you have created a story concept that inspires you pitch it to the
class to see if other classmates want to join you in your project.

Once you have your team in place you will then follow the process for
creating a shooting script for pre-production and production.

\subsection{Activity: Brainstorm and Prepare Your
Pitch}\label{activity-brainstorm-and-prepare-your-pitch}

\begin{tcolorbox}[enhanced jigsaw, opacityback=0, colframe=quarto-callout-note-color-frame, leftrule=.75mm, arc=.35mm, rightrule=.15mm, colbacktitle=quarto-callout-note-color!10!white, titlerule=0mm, colback=white, toprule=.15mm, bottomtitle=1mm, breakable, toptitle=1mm, title={Learning Activity}, coltitle=black, bottomrule=.15mm, left=2mm, opacitybacktitle=0.6]

Brainstorm either alone or with a partner and come up with a story
concept to pitch to the class. The goal of this activity is to do find a
compelling story and to pitch it in a way that will attract the
teammates you will need to make your longer film.

\textbf{\emph{Helpful Hint:}} Sometimes a title for your film can lend
lots of inspiration. Professional screenwriters and Hollywood spend a
lot of time finding the right title that captures the spirit of the film
and can aid in marketing. Think \emph{Rebel Without A Cause, Slum Dog
Millionaire, Vertigo, Dumb and Dumber, The Haunting, Groundhog Day, When
Harry Met Sally}.

To prepare for your pitch make sure you: - Know your genre. The story
above could be a comedy, a drama, thriller, or a horror film. Which
genre you decide will determine the following. - Establish your main
characters. - Establish the main conflict. - Escalate the conflict. -
Create some of the obstacles that will have to be overcome. - Have a
clear beginning, middle, and end. - Know what crew members you will need
and what they will do, e.g., a screenwriter, producer, cinematographer,
editor, etc.

Make sure the story is compressed, a la Aristotle's \emph{Poetics}, with
a limited number of characters, locations, and action.

\textbf{\emph{Helpful Hint:}} Most of the short films that won or where
nominated for the Academy Award for Best Live Action Short Film follow
Aristotle's idea of great drama and comedy: a few characters and
locations and action that takes place over a brief amount of time. For a
list of nominees and winners see:
\href{https://en.wikipedia.org/wiki/Academy_Award_for_Best_Live_Action_Short_Film}{Academy
Award for Best Live Action Short Film}.

Also be prepared to explain what crew members you will need and what
they need to do.

Make your pitch to your classmates and then see who wants to join which
team.

If your project is not chosen, that is, no one wants to work on it with
you, you can still do it but know that it will be a lot of work.

\subsubsection{Guidelines for Pitching a Film
Project}\label{guidelines-for-pitching-a-film-project}

\textbf{Purpose of a Pitch:} To get people to work on your project,
commission your project, fund your project, help with your project,
distribute your project, etc. Pitches also help you to clarity what you
want to do and why.

\textbf{Steps to a Good Pitch}

\begin{enumerate}
\def\labelenumi{\arabic{enumi}.}
\tightlist
\item
  If you can, open with a teaser or dramatic statement or question.
\item
  Meet and greet your audience---minimum chitchat, professionals are
  busy, busy people.
\item
  Dress appropriately---relative to the audience and the project.
  Artists dress ``artsy''; corporate people dress with business attire.
\item
  Project the proper persona (your personality and
  character)---knowledgeable, credible, trustworthy, smart, clever, etc.
\item
  Explain your qualifications, background, and why you are competent and
  the right person to do the project. If you do not have a demo reel
  then sell people with your passion, e.g., WHIPLASH filmmaker.
\item
  Give the title of the film/project---titles shape perception and
  expectations.
\item
  State the Genre---comedy, drama, documentary, etc. This also shapes
  perceptions and expectations. The basic concept of a comedy should
  make us laugh or at lease smile.
\item
  Give the Unique Angle or Hook of your project---what makes it stand
  out as something different and catchy/edgy/clever, etc. E.g., SHOE IN
  LOVE, a romantic comedy from the POV of a pair of cowboy boots and
  stilettos falling in love with each other. The special angle concept
  is to use foot ware as the main characters and the execution hook is
  to shoot POVs at foot level or from the shin down.
\item
  Briefly state the logline or basic concept: a one or two sentence
  description of the core story. E.g., a young, bored dirt farmer gets a
  message that he has to help rescue a faraway princes and save the
  universe. STAR WARS.
\item
  Project synopsis---succinctly tell the beginning/middle/end of your
  story or if a documentary the main message of your doc.
\item
  Summarize your end goal: to wow my audience with extraordinary nature
  photography or to make my audience laugh, cry, fall in love, etc.
\item
  Explain what you need---cast, crew, funding, etc.
\item
  Ask for Questions---also be open to criticism and constructive
  comments and feedback.
\item
  Gratitude---be thankful and gracious even if your project is not
  received well or as anticipated. Burning bridges is wrongheaded.
  Pitching can be a training ground for how to deal with rejection.
\end{enumerate}

\emph{Examples of Types of Montage Projects to Pitch}

The goal of your Montage Short assignment (see the syllabus) will be to
explore how to make an effective montage-based film that uses picture
and sound editing to convey meaning, emotion, story and/or mood.

For this project, you can:

\begin{itemize}
\tightlist
\item
  Create a music video where you take a song and create images that
  express the mood, feeling, meaning, etc. of the song.
\item
  Find a poem, fairy tale, or passage from the Bible and ``visualize''
  and enhance it with music and sound effects.
\item
  Create a chase scene---someone steals something and is on the run and
  using music and sound effects to enhance the excitement and dynamism.
\item
  Create a love or horror or thriller story ---use can use dialogue but
  there must be an edited montage---a series of edited shots that convey
  the mood, meaning etc.
\item
  Etc.
\end{itemize}

\end{tcolorbox}

\section{Creating a Scriptment}\label{creating-a-scriptment}

Our goal in this Unit is to find and create a strong story to film to
shoot and edit so that it shines with excellence. The following steps
will help you prepare a shooting script that will make pre-production,
production, and post-production flow better.

\begin{itemize}
\tightlist
\item
  Create a step outline.
\item
  Create a scriptment.
\end{itemize}

Your scriptment should be written in a way that someone who reads it
should be able to visualize and hear your film.

\subsection{Activity: Creating a
Scriptment}\label{activity-creating-a-scriptment}

\begin{tcolorbox}[enhanced jigsaw, opacityback=0, colframe=quarto-callout-note-color-frame, leftrule=.75mm, arc=.35mm, rightrule=.15mm, colbacktitle=quarto-callout-note-color!10!white, titlerule=0mm, colback=white, toprule=.15mm, bottomtitle=1mm, breakable, toptitle=1mm, title={Learning Activity}, coltitle=black, bottomrule=.15mm, left=2mm, opacitybacktitle=0.6]

For an overview of scriptments, consult this resource:
\href{https://screencraft.org/blog/outlines-treatments-and-scriptments-oh-my/}{Outlines,
Treatments, and Scriptments, Oh My!} by Ken Miyamoto.

As you do other drafts (good writing involves a lot of re-writing) to
improve your story and polish your script so that everything is clear,
it is helpful to ask the following questions. (These questions are
designed for a fiction narrative film, though some of them can help vet
your documentary or experimental film.) You might not have answers to
all of them but you should for most of them. Again the emphasis in on
helping you to be clear about your project.

\begin{itemize}
\tightlist
\item
  Do I have the right genre?
\item
  Do I have a hook or is there something unique and fascinating about my
  story? It's special angle.
\item
  Is the story clear? A solid beginng, middle, and end?
\item
  Do I focus on only a few characters and limited locations?
\item
  Are my action and conflict unified and focused?
\item
  Does my main character have a goal?
\item
  Is there an antagonist who or which wants to thwart that goal?
  (Remember, weather or a dog can be an antagonist.)
\item
  Are their stakes or consequences for not obtaining the goal?
\item
  Are there interesting obstacles to obtaining the goal?
\item
  Do I have a set-up of the main character, problem, goal, and
  antagonist?
\item
  Do I have some unexpected and unpredictable twists and turns in the
  middle section of my story?
\item
  Is there a main conflict or battle that determines whether the main
  character gets his or her goal?
\item
  Do I have a satisfying ending? (Remember not all endings have to be
  happy or closed. Open-ended endings are not resolved and make the
  audience guess, debate, and discuss what happened or might happen
  after).
\end{itemize}

Once you have your team and have developed the scriptment for your
story, you can begin pre-production to get everything in place to film
your project: e.g., cast, locations, props, gear, shooting schedule,
etc.

\end{tcolorbox}

\section*{Summary}\label{summary-7}
\addcontentsline{toc}{section}{Summary}

\markright{Summary}

In this unit, you learned about:

\begin{itemize}
\tightlist
\item
  Types of short films
\item
  Film Genres
\item
  The Logistics for the Final Project
\item
  Script and Story Development
\item
  Creating A Scriptment
\end{itemize}

\begin{tcolorbox}[enhanced jigsaw, opacityback=0, colframe=quarto-callout-note-color-frame, leftrule=.75mm, arc=.35mm, rightrule=.15mm, colbacktitle=quarto-callout-note-color!10!white, titlerule=0mm, colback=white, toprule=.15mm, bottomtitle=1mm, breakable, toptitle=1mm, title={Checking Your Learning}, coltitle=black, bottomrule=.15mm, left=2mm, opacitybacktitle=0.6]

Before you move on to the next unit, you may want to check to make sure
that you are able to:

\begin{itemize}
\tightlist
\item
  Describe the different types of short films
\item
  Define film genre and why it is important
\item
  Determine how to approach making a film script to shoot and edit.
\item
  Create a shooting script.
\end{itemize}

\end{tcolorbox}

\bookmarksetup{startatroot}

\chapter{Making Your Short Film}\label{making-your-short-film}

\section*{Overview}\label{overview-8}
\addcontentsline{toc}{section}{Overview}

\markright{Overview}

In this unit we culminate this course as you make your final film
project and bring together all you have learned so far. Congratulations.
Your hard work is paying off.

Here you will get the other skills needed to make an excellent short
film as you take your scriptment from the last unit and shoot and edit
it.

These skills will focus on exploring how to use film coverage to create
meaning, emotions, and moods.

Many of the items and techniques addressed here have been covered in the
previous units. Go back to them if you need a refresher. But there are
important new elements that you will learn. The important thing is not
to get lost in the detail but to maintain the big picture perspective,
namely, communicating what you want to say in a creative and excellent
way.

Let's dig in.

\subsection*{Topics}\label{topics-8}
\addcontentsline{toc}{subsection}{Topics}

This unit is divided into the following topics:

\begin{enumerate}
\def\labelenumi{\arabic{enumi}.}
\tightlist
\item
  The Syntax of Film
\item
  The Importance and Types of Camera Coverage
\item
  Continuity
\item
  Storyboarding
\item
  Shot Lists
\item
  Casting
\item
  Other Types of Short Films
\item
  Sound Recording
\item
  The Final Steps
\item
  Guidelines for Doing and Evaluating the Final Film Project
\end{enumerate}

\subsection*{Learning Outcomes}\label{learning-outcomes-8}
\addcontentsline{toc}{subsection}{Learning Outcomes}

When you have completed this unit, you will be able to:

\begin{itemize}
\tightlist
\item
  Define film syntax and its key terms to describe your filmmaking
  process.
\item
  Describe camera coverage and contrast the types of Coverage.
\item
  Demonstrate How To Use Continuity.
\item
  Create Storyboards and Shot Lists.
\item
  Demonstrate Casting Skills.
\item
  Apply L Cuts and J Cuts.
\item
  Evaluate the final film project.
\end{itemize}

\subsection*{Learning Activities}\label{learning-activities-8}
\addcontentsline{toc}{subsection}{Learning Activities}

Here is a checklist of learning activities you will benefit from in
completing this unit. You may find it useful for planning your work.

\begin{itemize}
\tightlist
\item
  Study the next section of Chapter 9 and watch the videos related to
  it. The selected videos for this unit are extremely important for you
  to understand how to make your final project. Failure to study them
  could jeopardize your final project.
\item
  Please log in your journal your key takeaways from the watched videos
  along with your responses to specific questions.
\item
  Storyboard the scenes from your scriptment you plan to film.
\item
  Review Unit 7 to refresh your knowledge about what is involved in the
  post-production process.
\item
  Study the guidelines and the Evaluation Criteria closely for the final
  project. They will help you to do well.
\end{itemize}

\begin{tcolorbox}[enhanced jigsaw, opacityback=0, colframe=quarto-callout-note-color-frame, leftrule=.75mm, colback=white, toprule=.15mm, breakable, arc=.35mm, rightrule=.15mm, bottomrule=.15mm, left=2mm]
\begin{minipage}[t]{5.5mm}
\textcolor{quarto-callout-note-color}{\faInfo}
\end{minipage}%
\begin{minipage}[t]{\textwidth - 5.5mm}

Working through course activities will help you to meet the learning
outcomes and successfully complete your assessments.

\end{minipage}%
\end{tcolorbox}

\subsection*{Assessment}\label{assessment-8}
\addcontentsline{toc}{subsection}{Assessment}

\textbf{Course Journal}

After completing this unit, including the learning activities, you are
asked to make sure you are doing journal entries and when appropriate to
share your responses with your facilitator and classmates when you meet.

Note that entries are expected after every unit. Your journal
reflections are submitted at the end of the course as part of the
Journal One: Personal Journal and self assessment.

\textbf{Final Film Project}

This final short film project will account for 40\% of your grade.

\begin{tcolorbox}[enhanced jigsaw, opacityback=0, colframe=quarto-callout-note-color-frame, leftrule=.75mm, colback=white, toprule=.15mm, breakable, arc=.35mm, rightrule=.15mm, bottomrule=.15mm, left=2mm]
\begin{minipage}[t]{5.5mm}
\textcolor{quarto-callout-note-color}{\faInfo}
\end{minipage}%
\begin{minipage}[t]{\textwidth - 5.5mm}

Please see the Assessment section in Moodle for assignment details as
well as the grading criteria.

\end{minipage}%
\end{tcolorbox}

\subsection*{Resources}\label{resources-8}
\addcontentsline{toc}{subsection}{Resources}

Here are the resources you will need to complete this unit.

\begin{itemize}
\tightlist
\item
  Chapter 8 of course text: \emph{Digital Filmmaking: A Beginner's Guide
  to Mastering the Craft,} by Ned Vankevich (e-text)
\item
  \href{https://www.youtube.com/watch?v=IK2IAEO-FUI}{How to Shoot a
  Scene! - Film Riot}
\item
  \href{https://www.youtube.com/watch?v=9AGaECt9j4g}{Film Blocking
  Tutorial --- Filmmaking Techniques for Directors: Ep3}
\item
  \href{https://www.youtube.com/watch?v=okphB85lfjk}{FilmSkills.com -
  Getting the right shots and coverage}
\item
  \href{https://www.youtube.com/watch?v=cz3nBkIa9K0}{What Is A Master
  Shot?}
\item
  \href{https://www.youtube.com/watch?v=eou7A-e2e4I}{Watch: \emph{Match
  on Action technique}}
\item
  \href{https://www.youtube.com/watch?v=El28XrjtcMI}{Watch: \emph{Match
  Cuts in Film Editing}}
\item
  \href{https://www.youtube.com/watch?v=9XOn5uxdSJc}{Watch: \emph{Screen
  Direction rule}}
\item
  \href{https://www.youtube.com/watch?v=RogoUz_pk4Y}{Watch: \emph{Screen
  direction}}
\item
  \href{https://www.youtube.com/watch?v=HinUychY3sE}{Watch:
  \emph{Breaking Down the 180-Degree Rule}}
\item
  \href{https://www.youtube.com/watch?v=1K8EUc98VoQ}{Watch: \emph{The 30
  Degree Rule in Filmmaking \textbar{} Kaicreative \textbar{} Tips}}
\item
  \href{https://www.wyzowl.com/what-is-a-storyboard/What}{Is a
  Storyboard and How Do You Make One for Your Video?}
\item
  \href{https://www.youtube.com/watch?v=-rzJP_5L_yQ}{Basics of Creating
  a Shot List}
\item
  \href{https://www.youtube.com/watch?v=YpCkRPqsiJ4}{How to Cast an
  ACTOR for a Low Budget Film \textbar{} The Film Look}
\item
  \href{https://www.youtube.com/watch?v=x0G6n346m90}{Auditioning Actors}
\item
  \href{https://www.youtube.com/watch?v=eyH-a964kAs}{Watch: \emph{SFX
  Secrets: The J Cut \& The L Cut}}
\item
  and \href{https://www.youtube.com/watch?v=fT5rRPMnka0}{Video Editing
  Tips: J Cut vs L Cut}
\end{itemize}

\section{The Syntax of Film}\label{the-syntax-of-film}

The first principle of this unit will focus on film syntax, where you
take the grammar of film---the types of shots, shot sizes, shot angles
and height, shot framing, subject angles, etc. and order them in a way
that best communicates what you want to say. This is where higher-level
creativity takes place.

Filmmaking is like a language where you use the elements of grammar to
make sentences. In this case, your shots to make scenes and the meaning
the scenes will covey. The cumulation of sentences lead to paragraphs --
the film equivalent of sequences, and the cumulation of paragraphs leads
to sections (acts in film) and the cumulation of sections leads to your
essay (in this case, your final film). You get the metaphor.

\subsection{Activity: Read, Watch and
Reflect}\label{activity-read-watch-and-reflect}

\begin{tcolorbox}[enhanced jigsaw, opacityback=0, colframe=quarto-callout-note-color-frame, leftrule=.75mm, arc=.35mm, rightrule=.15mm, colbacktitle=quarto-callout-note-color!10!white, titlerule=0mm, colback=white, toprule=.15mm, bottomtitle=1mm, breakable, toptitle=1mm, title={Learning Activity}, coltitle=black, bottomrule=.15mm, left=2mm, opacitybacktitle=0.6]

Read and study the first section of Chapter 9: The Syntax of Film. Then,
before moving on to the next topic in this unit, watch the following
tutorials to re-enforce the big picture. Do not forget to log in your
journal your best take-aways from them.

\href{https://www.youtube.com/watch?v=IK2IAEO-FUI}{Watch: \emph{How to
Shoot a Scene! - Film Riot}}

\url{https://www.youtube-nocookie.com/embed/IK2IAEO-FUI}

\href{https://www.youtube.com/watch?v=9AGaECt9j4g}{Watch: \emph{Film
Blocking Tutorial --- Filmmaking Techniques for Directors: Ep3}}

\url{https://www.youtube-nocookie.com/embed/9AGaECt9j4g}

\end{tcolorbox}

\section{The Importance and Types of Camera
Coverage}\label{the-importance-and-types-of-camera-coverage}

Exploring and understanding camera coverage is vital for those who want
to create excellent films. Coverage refers to how a scene is captured.
It involves how many shots are used and their type and kind to capture a
scene in a film or video production. Coverage is thus the shot material
an editor or post-production team will use to assemble the scenes and
sequences of a movie. Having proper coverage is critical if a scene and
a film is to make sense.

The are many questions that can guide you as you break down your film
script to determine what coverage to use and how to shoot a scene.
(Remember a scene can be one shot such as driving to the store, or a
series of shots such as someone confronting a person to get information
from him or her.) Take the time to study the \textbf{Questions for
Discovering Coverage} section.

Coverage can be done well or poorly. Our goal is the former.

There are two main types of general coverage: \textbf{Master Shots and
Mini-Masters}. Knowing why these are important and when to use them will
help guide you to the more detailed coverage and shots you will employ
as as you develop the action of a scene.

Filmmaking is not a ``paint by numbers'' art. If you follow formulas,
most likely your film will be predicable and lack freshness. This is why
understanding shot progression is important. Choosing the types of shots
you will use and their order will form a large part of your visual
creativity in your film.

\subsection{Activity: Read and Watch}\label{activity-read-and-watch}

\begin{tcolorbox}[enhanced jigsaw, opacityback=0, colframe=quarto-callout-note-color-frame, leftrule=.75mm, arc=.35mm, rightrule=.15mm, colbacktitle=quarto-callout-note-color!10!white, titlerule=0mm, colback=white, toprule=.15mm, bottomtitle=1mm, breakable, toptitle=1mm, title={Learning Activity}, coltitle=black, bottomrule=.15mm, left=2mm, opacitybacktitle=0.6]

Study the next section of Chapter 9 ``The Importance and Types of Camera
Coverage''. Next, watch the video
\href{https://www.youtube.com/watch?v=cz3nBkIa9K0}{Watch: \emph{What Is
A Master Shot?}}

\url{https://www.youtube-nocookie.com/embed/cz3nBkIa9K0}

\textbf{Questions to Consider}

\begin{itemize}
\tightlist
\item
  Explain the value of why this type of shot can be beneficial as the
  first shot you do when you start your coverage of a scene? What is the
  downside of relying too much on a master shot?
\end{itemize}

\end{tcolorbox}

\section{Continuity}\label{continuity}

Continuity, or the non-distracting and motivated and smooth flow of
action, is another skill set essential to the filmmaker's took kit. Some
techniques you have already studied, but others are new and important.

The following are the major techniques of continuity that you must focus
on:

\begin{itemize}
\tightlist
\item
  Matching Action
\item
  Directional Continuity
\item
  Eyeline Continuity
\item
  180-Degree Rule
\item
  30-Degree Rule
\end{itemize}

Each of these individually and as a group will help immensely in
creating well-constructed, well-motivated, and audience-engaging films.

\subsection{Activity: Read and Watch}\label{activity-read-and-watch-1}

\begin{tcolorbox}[enhanced jigsaw, opacityback=0, colframe=quarto-callout-note-color-frame, leftrule=.75mm, arc=.35mm, rightrule=.15mm, colbacktitle=quarto-callout-note-color!10!white, titlerule=0mm, colback=white, toprule=.15mm, bottomtitle=1mm, breakable, toptitle=1mm, title={Learning Activity}, coltitle=black, bottomrule=.15mm, left=2mm, opacitybacktitle=0.6]

Read the Continuity section of Chapter 9.

There are many resources that can help you to understand the various
parts of continuity. It might seem like a lot but it will greatly
benefit you if you view the following resources.

\begin{itemize}
\tightlist
\item
  \href{https://www.youtube.com/watch?v=eou7A-e2e4I}{Watch: \emph{Match
  on Action technique}}
\end{itemize}

\url{https://www.youtube-nocookie.com/embed/eou7A-e2e4I}

\begin{itemize}
\tightlist
\item
  \href{https://www.youtube.com/watch?v=El28XrjtcMI}{Watch: \emph{Match
  Cuts in Film Editing}}
\end{itemize}

\url{https://www.youtube-nocookie.com/embed/El28XrjtcMI}

\end{tcolorbox}

\begin{tcolorbox}[enhanced jigsaw, opacityback=0, colframe=quarto-callout-note-color-frame, leftrule=.75mm, arc=.35mm, rightrule=.15mm, colbacktitle=quarto-callout-note-color!10!white, titlerule=0mm, colback=white, toprule=.15mm, bottomtitle=1mm, breakable, toptitle=1mm, title={Learning Activity}, coltitle=black, bottomrule=.15mm, left=2mm, opacitybacktitle=0.6]

\begin{itemize}
\tightlist
\item
  \href{https://www.youtube.com/watch?v=9XOn5uxdSJc}{Watch: \emph{Screen
  Direction rule}}
\end{itemize}

\url{https://www.youtube-nocookie.com/embed/9XOn5uxdSJc}

\begin{itemize}
\tightlist
\item
  \href{https://www.youtube.com/watch?v=RogoUz_pk4Y}{Watch: \emph{Screen
  direction}}
\end{itemize}

\url{https://www.youtube-nocookie.com/embed/RogoUz_pk4Y}

\begin{itemize}
\tightlist
\item
  \href{https://www.youtube.com/watch?v=HinUychY3sE}{Watch:
  \emph{Breaking Down the 180-Degree Rule}}
\end{itemize}

\url{https://www.youtube-nocookie.com/embed/HinUychY3sE}

\begin{itemize}
\tightlist
\item
  \href{https://www.youtube.com/watch?v=1K8EUc98VoQ}{Watch: \emph{The 30
  Degree Rule in Filmmaking \textbar{} Kaicreative \textbar{} Tips}}
\end{itemize}

\url{https://www.youtube-nocookie.com/embed/1K8EUc98VoQ}

Do you feel you have a stronger grasp of continuity and of each the
techniques addressed? If not, review the videos of those parts you do
not understand.

\end{tcolorbox}

\section{Storyboarding}\label{storyboarding}

By now you might feel overwhelmed with camera coverage and all that it
takes to shoot your story and script well. Storyboarding and storyboards
can help simplify the process and make it easier to visualize what you
want to do and how to do it.

It will help your scene coverage if you storyboard your shots. They do
not have to be elaborate. They can be simple stick figures such as this
one that shows a long shot of someone pushing or struggling against
something (the context of your story will determine this).

\begin{figure}

\caption{\label{fig-Picture1}\emph{Graphic Stick Figure Pushing}}

\centering{

\begin{figure}
\centering
\pandocbounded{\includegraphics[keepaspectratio]{assets/u9/Picture1.png}}
\caption{A graphic stick figure pushing}
\end{figure}

}

\end{figure}%

The important thing is not the quality but that your storyboards make
sense to you and your crew so you will know how to approach a shot and
why.

At the same time, do not become a slave to your storyboard. If you
discover something better or get inspired on the film set, try the new
thing. For safety, shoot it the storyboard way and then the new way and
decide during post-production which one will work best.

\section{Shot Lists}\label{shot-lists}

Once you have your scriptment describing all of the action and dialogue
you will film and have storyboarded your film, you can now create a shot
list which will help ensure you get all the shots needed for your
coverage.

It is better if your whole production team gives feedback on the shot
list. Ultimately, it is the director who will make the final choice but
hearing what other team members feel can help cut down on mistakes.

Study the Shot List Example in Chapter Nine to see what you need to
include.

\subsection{Activity: Read and Watch}\label{activity-read-and-watch-2}

\begin{tcolorbox}[enhanced jigsaw, opacityback=0, colframe=quarto-callout-note-color-frame, leftrule=.75mm, arc=.35mm, rightrule=.15mm, colbacktitle=quarto-callout-note-color!10!white, titlerule=0mm, colback=white, toprule=.15mm, bottomtitle=1mm, breakable, toptitle=1mm, title={Learning Activity}, coltitle=black, bottomrule=.15mm, left=2mm, opacitybacktitle=0.6]

\begin{itemize}
\tightlist
\item
  Read about Shot Lists in Chapter 9.
\item
  Watch the video
  \href{https://www.youtube.com/watch?v=-rzJP_5L_yQ}{Watch: \emph{Master
  the Art of Shot Lists: Boost Your Filmmaking Efficiency!}}
\end{itemize}

\url{https://www.youtube-nocookie.com/embed/-rzJP_5L_yQ}

\end{tcolorbox}

\section{Casting}\label{casting}

Once you have your locations, storyboards, shot list, and schedule you
are ready to cast your show. Casting is often done earlier in larger
professional productions because they need to lock in busy actors. For
this scale of micro or no budget filmmaking having your shooting
schedule and your storyboards and shot lists can show seasoned actors
that you know what you are doing and can help you land them.

This pre-production process emphasized here will also help you to know
if some special skills are needed for your actors. E.g., someone who can
dance, sing, play guitar, is good at soccer, etc.

Having good actors is critical to having a good film. Follow the tips
and guidelines in the chapter and work hard to find the best actors you
can. It will pay off a lot.

\subsection*{Activity: Read, View and
Listen}\label{activity-read-view-and-listen}
\addcontentsline{toc}{subsection}{Activity: Read, View and Listen}

\begin{tcolorbox}[enhanced jigsaw, opacityback=0, colframe=quarto-callout-note-color-frame, leftrule=.75mm, arc=.35mm, rightrule=.15mm, colbacktitle=quarto-callout-note-color!10!white, titlerule=0mm, colback=white, toprule=.15mm, bottomtitle=1mm, breakable, toptitle=1mm, title={Learning Activity}, coltitle=black, bottomrule=.15mm, left=2mm, opacitybacktitle=0.6]

Read the Casting section in Chapter 9.

Before auditioning and casting your film, watch the video
\href{https://www.youtube.com/watch?v=YpCkRPqsiJ4}{``How to Cast an
ACTOR for a No Budget Film''} for a simple overview of a casting process

\url{https://www.youtube-nocookie.com/embed/YpCkRPqsiJ4?si=IbsICPDDO42pTPbK}

For practical tips for auditioning actors watch the video
\href{https://www.youtube.com/watch?v=x0G6n346m90}{Watch:
\emph{Auditioning Actors}}

\url{https://www.youtube-nocookie.com/embed/x0G6n346m90}

For detailed information regarding how to work with actors and casting
for low budget films listen to this
podcast:\href{https://www.youtube.com/watch?v=-DwKilT0T34}{Watch:
\emph{How to Cast a No Budget Indie Film with Casting Director Veronika
Lee (Backstage Magazine)}}

\url{https://www.youtube-nocookie.com/embed/-DwKilT0T34}

\end{tcolorbox}

\section{Other Types of Films}\label{other-types-of-films}

Some of you might want to do a documentary or an experimental film.
Though this course centers on classical approaches to fictional
narrative film, you are free to work on such projects. It is important
that you pitch such projects ahead of time to your instructor to make
sure it is doable within the confines of this course.

\subsubsection{Activity: Other Types of
Film}\label{activity-other-types-of-film}

\begin{tcolorbox}[enhanced jigsaw, opacityback=0, colframe=quarto-callout-note-color-frame, leftrule=.75mm, arc=.35mm, rightrule=.15mm, colbacktitle=quarto-callout-note-color!10!white, titlerule=0mm, colback=white, toprule=.15mm, bottomtitle=1mm, breakable, toptitle=1mm, title={Learning Activity}, coltitle=black, bottomrule=.15mm, left=2mm, opacitybacktitle=0.6]

See the section on Other Types of Film if you want to do a documentary
or experimental film and view the resources for them.

\end{tcolorbox}

\section{Sound Recording}\label{sound-recording}

Audio can make or break a film, especially in no-budget and low-budget
filmmaking. Proper sound recording and sound editing is a course unto
itself and if you want to go into filmmaking professionally you should
take at least one course in this.

Study the Sound Recording section in Chapter Nine and the suggested
resources to help up get better quality sound for your project.

\subsection{Activity: Read and
Research}\label{activity-read-and-research}

\begin{tcolorbox}[enhanced jigsaw, opacityback=0, colframe=quarto-callout-note-color-frame, leftrule=.75mm, arc=.35mm, rightrule=.15mm, colbacktitle=quarto-callout-note-color!10!white, titlerule=0mm, colback=white, toprule=.15mm, bottomtitle=1mm, breakable, toptitle=1mm, title={Learning Activity}, coltitle=black, bottomrule=.15mm, left=2mm, opacitybacktitle=0.6]

How to capture great sound for a film without the use of professionals
and professional gear is a challenge. View the three resources for this
in Chapter Nine and find tutorials and instruction essays on the web
that are appropriate to your cell phone and DSLR.

\end{tcolorbox}

\section{The Final Steps}\label{the-final-steps}

Bravo. Once you have reached this stage of this Unit and the previous
one you should be well-equipped to produce and film your movie.

You will then have to edit it. This section adds some new techniques
that can help elevate your film project to a higher level, namely Split
Edits. If you can, add this to the editing of your film, as well as L
cuts and J cuts. They will make your film flow better and be more
engaging and enjoyable. It might also help win you an award in the
course.

\subsection{Activity: Review}\label{activity-review}

\begin{tcolorbox}[enhanced jigsaw, opacityback=0, colframe=quarto-callout-note-color-frame, leftrule=.75mm, arc=.35mm, rightrule=.15mm, colbacktitle=quarto-callout-note-color!10!white, titlerule=0mm, colback=white, toprule=.15mm, bottomtitle=1mm, breakable, toptitle=1mm, title={Learning Activity}, coltitle=black, bottomrule=.15mm, left=2mm, opacitybacktitle=0.6]

Review Unit 7 to refresh yourself about what is involved in the
post-production process. If possible, add Split Edits to elevate the
quality and sophistication of your final film project.

\end{tcolorbox}

\section{Guidelines for Doing and Evaluating The Final Film
Project}\label{guidelines-for-doing-and-evaluating-the-final-film-project}

Continuity or the non-distracting and motivated and smooth flow of
action is another skill set essential to the filmmaker's took kit.

Study these resources and practice applying them in your edited scenes
especially those with dialogue or where one scene transitions into
another:

\href{https://www.youtube.com/watch?v=eyH-a964kAs}{Watch: \emph{SFX
Secrets: The J Cut \& The L Cut}}

\url{https://www.youtube-nocookie.com/embed/eyH-a964kAs}

\href{https://www.youtube.com/watch?v=fT5rRPMnka0}{Watch: \emph{Video
Editing Tips: J Cut vs L Cut}}

\url{https://www.youtube-nocookie.com/embed/fT5rRPMnka0}

In addition, study the \emph{Steps to Making an Excellent Short Film}
section in the chapter. It is both a good recap and a good checklist to
help ensure you have the elements need to make an excellent final film.

\subsection{Activity: Final Project
Guidelines}\label{activity-final-project-guidelines}

\begin{tcolorbox}[enhanced jigsaw, opacityback=0, colframe=quarto-callout-note-color-frame, leftrule=.75mm, arc=.35mm, rightrule=.15mm, colbacktitle=quarto-callout-note-color!10!white, titlerule=0mm, colback=white, toprule=.15mm, bottomtitle=1mm, breakable, toptitle=1mm, title={Learning Activity}, coltitle=black, bottomrule=.15mm, left=2mm, opacitybacktitle=0.6]

Please study the guidelines and the Evaluation Criteria closely for the
final project. They will help you to do well. This criteria closely
tracks that which was used to evaluate your montages so the process
should be more comfortable for you now.

\begin{itemize}
\tightlist
\item
  Did the film make sense and work overall?
\item
  Did the film meet genre expectations?
\item
  Was the story interesting? Surprising, engaging, exciting?
\item
  Did the actors perform well?
\item
  Did the camera coverage work? If so, why? If not, why?
\item
  Did the editing work? If so why? If not, why?
\item
  What stood out in a good way?
\item
  What needed improvement?
\item
  What did you like best about the film?
\item
  What did you like least?
\item
  Other comments.
\end{itemize}

\end{tcolorbox}

\section*{Summary}\label{summary-8}
\addcontentsline{toc}{section}{Summary}

\markright{Summary}

In this unit, you learned about:

\begin{itemize}
\tightlist
\item
  The Syntax of Film
\item
  The Importance and Types of Coverage
\item
  Continuity
\item
  Storyboarding and Shot Lists
\item
  Casting
\item
  Sound Recording
\item
  Other Types of Short Films
\item
  The Final Steps
\item
  Guidelines for Doing and Evaluating the Final Film Project
\end{itemize}

\begin{tcolorbox}[enhanced jigsaw, opacityback=0, colframe=quarto-callout-note-color-frame, leftrule=.75mm, arc=.35mm, rightrule=.15mm, colbacktitle=quarto-callout-note-color!10!white, titlerule=0mm, colback=white, toprule=.15mm, bottomtitle=1mm, breakable, toptitle=1mm, title={Checking Your Learning}, coltitle=black, bottomrule=.15mm, left=2mm, opacitybacktitle=0.6]

Before you move on to the next unit, you may want to check to make sure
that you are able to:

\begin{itemize}
\tightlist
\item
  Define film syntax and its key terms to describe your filmmaking
  process.
\item
  Describe Camera Coverage and Contrast The Types of Coverage.
\item
  Demonstrate How To Use Continuity.
\item
  Create Storyboards and Shot Lists.
\item
  Demonstrate Casting Skills.
\item
  Apply L Cuts and J Cuts.
\item
  Evaluate the final film project.
\end{itemize}

\end{tcolorbox}

\bookmarksetup{startatroot}

\chapter{Course Summary and
Celebration}\label{course-summary-and-celebration}

\section*{Overview}\label{overview-9}
\addcontentsline{toc}{section}{Overview}

\markright{Overview}

Congratulations. You are at the end of your creative journey into the
world of filmmaking. We hope it has been a grand adventure.

In this unit you celebrate your accomplishments and hopefully win a
course award for them, and if they merit it, be considered for a film
festival. You will also have a chance to reflect on what this course
means for your future personal and professional develop and career. You
will also self-assess your work and be encouraged to look at the big
picture this course experience has offered.

\subsection*{Topics}\label{topics-9}
\addcontentsline{toc}{subsection}{Topics}

This unit is divided into the following topics:

\begin{enumerate}
\def\labelenumi{\arabic{enumi}.}
\tightlist
\item
  Course Awards
\item
  Film Festivals
\item
  Your Future
\end{enumerate}

\subsection*{Learning Outcomes}\label{learning-outcomes-9}
\addcontentsline{toc}{subsection}{Learning Outcomes}

When you have completed this unit, you will be able to:

\begin{itemize}
\tightlist
\item
  Describe what you have learned from this course
\item
  Determine who wins awards for their work
\item
  Create a self-assessment that encapsulates what you learned.
\item
  Think about the big picture of your life and share your reflections.
\end{itemize}

\subsection*{Learning Activities}\label{learning-activities-9}
\addcontentsline{toc}{subsection}{Learning Activities}

Here is a checklist of learning activities you will benefit from in
completing this unit. You may find it useful for planning your work.

\begin{itemize}
\tightlist
\item
  Decide on your film favourites.
\item
  Consider submitting to a film festival.
\item
  Study resources that deal with jobs in the film industry and careers
  that can use the filmmaking skills you are acquiring.
\end{itemize}

\begin{tcolorbox}[enhanced jigsaw, opacityback=0, colframe=quarto-callout-note-color-frame, leftrule=.75mm, colback=white, toprule=.15mm, breakable, arc=.35mm, rightrule=.15mm, bottomrule=.15mm, left=2mm]
\begin{minipage}[t]{5.5mm}
\textcolor{quarto-callout-note-color}{\faInfo}
\end{minipage}%
\begin{minipage}[t]{\textwidth - 5.5mm}

Working through course activities will help you to meet the learning
outcomes and successfully complete your assessments.

\end{minipage}%
\end{tcolorbox}

\subsection*{Assessment}\label{assessment-9}
\addcontentsline{toc}{subsection}{Assessment}

\textbf{Course Journal}

After completing this unit, including the learning activities, you are
asked to make sure you are doing journal entries and when appropriate to
share your responses with your facilitator and classmates when you meet.

Note that entries are expected after every unit. Your journal
reflections are submitted at the end of the course as part of the
Journal One: Personal Journal and self assessment.

\textbf{Final Exam - Self-Assessment / Course Journal}

The final exam for this course will involve a detailed self-assessment
of what you have learned as a person and as a professional.

Follow the guidelines in the self-assessment section of the final
chapter for your final exam submission. Be sure to use entries you have
made in your journal.

As a reminder, the following is the point scale for determining your
grade:

\begin{itemize}
\tightlist
\item
  Assignment 1: In-Camera Exercise (10 points)
\item
  Assignment 2: Montage Project (20 points)
\item
  Assignment 3: Final Film Project (40 points)
\item
  Final Exam: Self-Assessment / Course Journal (30 points)
\end{itemize}

\begin{tcolorbox}[enhanced jigsaw, opacityback=0, colframe=quarto-callout-note-color-frame, leftrule=.75mm, colback=white, toprule=.15mm, breakable, arc=.35mm, rightrule=.15mm, bottomrule=.15mm, left=2mm]
\begin{minipage}[t]{5.5mm}
\textcolor{quarto-callout-note-color}{\faInfo}
\end{minipage}%
\begin{minipage}[t]{\textwidth - 5.5mm}

Please see the Assessment section in Moodle for assignment details as
well as the grading criteria.

\end{minipage}%
\end{tcolorbox}

\subsection*{Resources}\label{resources-9}
\addcontentsline{toc}{subsection}{Resources}

Here are the resources you will need to complete this unit.

\begin{itemize}
\tightlist
\item
  Chapter Ten course text
\item
  How to Pick Which Festivals to Submit to (And Get the Most from Your
  Submissions).
\item
  For a list of festivals focusing on short films see: ``20 Film
  Festivals You Should Enter Your Short Film Into, Bridget Schneider
\item
  8 ESSENTIAL skills you need for a Career in filmmaking!
\item
  20+ FILM SET JOBS FOR YOU
\item
  ``The 20 Best Careers in the Film Industry, Joanna Zambas
\item
  Film Crew Positions And Why All Jobs on a Movie Set Matter,'' George
  Edelman
\item
  ``Film In Colorado, Job Descriptions,'' This is an example of a local
  film commission.
\item
  10 Life Skills You Need to Land Your Dream Career
\item
  Other resources will be provided in the course text.
\end{itemize}

\section{Course Awards}\label{course-awards}

The first order of business on your capstone day is to celebrate who did
outstanding work in the course. After you view all of the final film
projects, your will vote on who wins the following awards based on the
criteria of this course:

\begin{itemize}
\tightlist
\item
  Best Overall Film
\item
  Best Overall Director
\item
  Best Overall Script
\item
  Best Overall Cinematography
\item
  Best Overall Editing
\item
  Best Photographic Compositions
\item
  Best Motional Exercises
\item
  Best In-Camera Exercise
\item
  Best Montage Exercise
\item
  Best Final Film Project
\item
  Most Improved Filmmaker
\end{itemize}

You will also have the opportunity to give feedback on why you voted on
the filmmakers, films, and awards which won.

\subsection{Activity: Film Favourites}\label{activity-film-favourites}

\begin{tcolorbox}[enhanced jigsaw, opacityback=0, colframe=quarto-callout-note-color-frame, leftrule=.75mm, arc=.35mm, rightrule=.15mm, colbacktitle=quarto-callout-note-color!10!white, titlerule=0mm, colback=white, toprule=.15mm, bottomtitle=1mm, breakable, toptitle=1mm, title={Learning Activity}, coltitle=black, bottomrule=.15mm, left=2mm, opacitybacktitle=0.6]

Be sure to review the sections in your course journal where you made
comments about the films you liked during the course. Go back and view
them in the Course Folder to confirm your decision.

\end{tcolorbox}

\section{Film Festivals}\label{film-festivals}

Film festivals are great venues for building your reputation and
advancing your career.

If your film merits it, your instructor will recommend that your film be
adjudicated for entry into Cinergy, the annual TWU campus student film
festival held each spring.

It will help you grow as a filmmaker if you research festivals and
submit your work to them.

\section{Your Future}\label{your-future}

This course is a stepping stone in the long journey of preparing for
your career and life-learning.

It will help you immensely in this process if you engage now in
reflection on what you have learned and explore potential jobs and
industry professions. After the course, your life will motor on and you
might forget to do this.

\textbf{Helpful Hint:} Honestly assess your work and progress in the
course. What strengths did you discover about yourself? What weaknesses
do you have to work on? The important thing is not to condemn ourselves
for our mistakes but to learn from them. Make a to do list for what you
will do after this course to improve your personal and professional
life. Log these insights in your journal and revisit them in six months
to see where you stand then

\subsection{Activity: Applying Your Film
Skills}\label{activity-applying-your-film-skills}

\begin{tcolorbox}[enhanced jigsaw, opacityback=0, colframe=quarto-callout-note-color-frame, leftrule=.75mm, arc=.35mm, rightrule=.15mm, colbacktitle=quarto-callout-note-color!10!white, titlerule=0mm, colback=white, toprule=.15mm, bottomtitle=1mm, breakable, toptitle=1mm, title={Learning Activity}, coltitle=black, bottomrule=.15mm, left=2mm, opacitybacktitle=0.6]

Study resources like the following that deal with jobs in the film
industry and careers that can use the filmmaking skills you are
acquiring.

\begin{itemize}
\tightlist
\item
  \href{https://www.youtube.com/watch?v=1grxjdy831g}{Watch: \emph{8
  ESSENTIAL skills you need for a Career in filmmaking}}
\end{itemize}

\url{https://www.youtube-nocookie.com/embed/1grxjdy831g}

\begin{itemize}
\tightlist
\item
  \href{https://www.youtube.com/watch?v=Tvjg1ndrPlQ}{Watch: \emph{20+
  FILM SET JOBS FOR YOU}}
\end{itemize}

\url{https://www.youtube-nocookie.com/embed/Tvjg1ndrPlQ}

\textbf{Questions to Consider}

Which jobs appeal to you? Which careers appeal to you? Make a list, date
it, and post it on your wall, bathroom mirror, or computer. Check it
periodically to remind you of your goal and to assess how well you are
on the road to it.

\end{tcolorbox}

\section*{Summary}\label{summary-9}
\addcontentsline{toc}{section}{Summary}

\markright{Summary}

In this unit, you learned about:

\begin{itemize}
\tightlist
\item
  Course Awards
\item
  Film Festivals
\item
  Your Future
\end{itemize}

\begin{tcolorbox}[enhanced jigsaw, opacityback=0, colframe=quarto-callout-note-color-frame, leftrule=.75mm, arc=.35mm, rightrule=.15mm, colbacktitle=quarto-callout-note-color!10!white, titlerule=0mm, colback=white, toprule=.15mm, bottomtitle=1mm, breakable, toptitle=1mm, title={Checking Your Learning}, coltitle=black, bottomrule=.15mm, left=2mm, opacitybacktitle=0.6]

Before you move on to the next unit, you may want to check to make sure
that you are able to:

\begin{itemize}
\tightlist
\item
  Describe what you have learned from this course
\item
  Determine who wins awards for their work
\item
  Create a self-assessment that encapsulates what you learned.
\item
  Think about the big picture of your life and share your reflections.
\end{itemize}

\end{tcolorbox}




\end{document}

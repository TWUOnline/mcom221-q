% Options for packages loaded elsewhere
\PassOptionsToPackage{unicode}{hyperref}
\PassOptionsToPackage{hyphens}{url}
%
\documentclass[
]{book}
\usepackage{amsmath,amssymb}
\usepackage{iftex}
\ifPDFTeX
  \usepackage[T1]{fontenc}
  \usepackage[utf8]{inputenc}
  \usepackage{textcomp} % provide euro and other symbols
\else % if luatex or xetex
  \usepackage{unicode-math} % this also loads fontspec
  \defaultfontfeatures{Scale=MatchLowercase}
  \defaultfontfeatures[\rmfamily]{Ligatures=TeX,Scale=1}
\fi
\usepackage{lmodern}
\ifPDFTeX\else
  % xetex/luatex font selection
\fi
% Use upquote if available, for straight quotes in verbatim environments
\IfFileExists{upquote.sty}{\usepackage{upquote}}{}
\IfFileExists{microtype.sty}{% use microtype if available
  \usepackage[]{microtype}
  \UseMicrotypeSet[protrusion]{basicmath} % disable protrusion for tt fonts
}{}
\makeatletter
\@ifundefined{KOMAClassName}{% if non-KOMA class
  \IfFileExists{parskip.sty}{%
    \usepackage{parskip}
  }{% else
    \setlength{\parindent}{0pt}
    \setlength{\parskip}{6pt plus 2pt minus 1pt}}
}{% if KOMA class
  \KOMAoptions{parskip=half}}
\makeatother
\usepackage{xcolor}
\usepackage{longtable,booktabs,array}
\usepackage{calc} % for calculating minipage widths
% Correct order of tables after \paragraph or \subparagraph
\usepackage{etoolbox}
\makeatletter
\patchcmd\longtable{\par}{\if@noskipsec\mbox{}\fi\par}{}{}
\makeatother
% Allow footnotes in longtable head/foot
\IfFileExists{footnotehyper.sty}{\usepackage{footnotehyper}}{\usepackage{footnote}}
\makesavenoteenv{longtable}
\usepackage{graphicx}
\makeatletter
\def\maxwidth{\ifdim\Gin@nat@width>\linewidth\linewidth\else\Gin@nat@width\fi}
\def\maxheight{\ifdim\Gin@nat@height>\textheight\textheight\else\Gin@nat@height\fi}
\makeatother
% Scale images if necessary, so that they will not overflow the page
% margins by default, and it is still possible to overwrite the defaults
% using explicit options in \includegraphics[width, height, ...]{}
\setkeys{Gin}{width=\maxwidth,height=\maxheight,keepaspectratio}
% Set default figure placement to htbp
\makeatletter
\def\fps@figure{htbp}
\makeatother
\setlength{\emergencystretch}{3em} % prevent overfull lines
\providecommand{\tightlist}{%
  \setlength{\itemsep}{0pt}\setlength{\parskip}{0pt}}
\setcounter{secnumdepth}{5}
\usepackage{booktabs}
\usepackage{amsthm}
\makeatletter
\def\thm@space@setup{%
  \thm@preskip=8pt plus 2pt minus 4pt
  \thm@postskip=\thm@preskip
}
\makeatother

\usepackage{tcolorbox}


\newtcolorbox{blackbox}{
  colback=black,
  coltext=white,
  colframe=black,
  boxsep=5pt,
  arc=4pt}
\newtcolorbox{bonus}{
  colback=blue!15,
  colframe=blue!15,
  coltext=black!80,
  boxsep=5pt,
  arc=4pt}
\newtcolorbox{reflect}{
  colback=green!5,
  colframe=green!5,
  coltext=black!80,
  boxsep=5pt,
  arc=4pt}
\newtcolorbox{assessment}{
  colback=blue!5,
  colframe=blue!5,
  coltext=black!80,
  boxsep=5pt,
  arc=4pt}
\newtcolorbox{progress}{
  colback=purple!10,
  colframe=purple!10,
  coltext=black!80,
  boxsep=5pt,
  arc=4pt}
\newtcolorbox{video}{
  colback=yellow!5,
  colframe=yellow!5,
  coltext=black!80,
  boxsep=5pt,
  arc=4pt}
\newtcolorbox{caution}{
  colback=red!5,
  colframe=red!5,
  coltext=black!80,
  boxsep=5pt,
  arc=4pt}
\newtcolorbox{feedback}{
  colback=black!5,
  colframe=black!5,
  coltext=black!80,
  boxsep=5pt,
  arc=4pt}
\ifLuaTeX
  \usepackage{selnolig}  % disable illegal ligatures
\fi
\usepackage[]{natbib}
\bibliographystyle{apalike}
\IfFileExists{bookmark.sty}{\usepackage{bookmark}}{\usepackage{hyperref}}
\IfFileExists{xurl.sty}{\usepackage{xurl}}{} % add URL line breaks if available
\urlstyle{same}
\hypersetup{
  pdftitle={Digital Filmmaking},
  pdfauthor={Ned Vankevich},
  hidelinks,
  pdfcreator={LaTeX via pandoc}}

\title{Digital Filmmaking}
\author{Ned Vankevich}
\date{2023-06-09}

\begin{document}
\maketitle

{
\setcounter{tocdepth}{1}
\tableofcontents
}
\hypertarget{welcome}{%
\chapter*{Welcome}\label{welcome}}
\addcontentsline{toc}{chapter}{Welcome}

This is the course book for {[}insert{]}. This book is divided into 6 units of study to help you engage with the materials. The course resources and learning activities are designed not only to help prepare you for the course assessments, but also to give you opportinities to practice various skills.

Below you will find information about how to navigate this book. Please also refer the schedule in Moodle, as well as the Asseessment section in Moodle for instructions on required readings and assignments.

\hypertarget{course-notes}{%
\section*{Course Notes}\label{course-notes}}
\addcontentsline{toc}{section}{Course Notes}

You should be reading this information in the context of a Trinity Western University course offered via Moodle. If this is not the case, then this may be an unauthorized reproduction of the course. Please contact \href{mailto:elearning@twu.ca}{\nolinkurl{elearning@twu.ca}} if you have concerns.

These notes will be your guide through the learning activities and assessment strategies necessary for you to succeed in the course, so it is important for you to engage to the best of your ability and take advantage of the resources available to you through Trinity Western University.

Assessment tasks are managed in other sections of the Moodle course, so be sure to familiarize yourself with those requirements and resources.

\hypertarget{how-this-course-is-built}{%
\section*{How this Course is Built}\label{how-this-course-is-built}}
\addcontentsline{toc}{section}{How this Course is Built}

This course is primarily designed to be completed asynchronously, meaning that there are no scheduled times or places that you are required to meet, even online. You can work according to your own schedule \emph{within the six weeks you have to complete the course}. That said, this is a full university level course and there are timelines that we strongly recommend that you meet to ensure that you are succeeding in building your knowledge through the course.

It would be to your significant disadvantage to submit everything at the end of the course.

Asynchronous courses require learners to be well-organized and self-motivated, and we have included supports for you to help you develop strong learning habits that will ensure your success.

For example, there are several self-check quizzes throughout the course. These quizzes are not graded, but they can be powerful tools for you to ensure you understand key ideas and concepts. We suggest you take each quiz without the aid of your notes and textbook and multiple times until you have mastered the content. This strategy taps into three powerful learning structures that have been shown to be highly effective.

\begin{enumerate}
\def\labelenumi{\arabic{enumi}.}
\tightlist
\item
  \textbf{Effortful recall.} By intentionally trying to recall information without external aids, you are strengthening the neural pathways in your brain that lead to building new connections between ideas. One way to make recall easier is to connect key ideas to other things that you know or have experienced. For example, you might be studying World War II, and you connect the date that Canadians participated in the D-Day operation with something else meaningful to you that happened on June 6, like maybe the date you bought your first car.
\item
  \textbf{Spaced repetition.} By spreading out your attempts on the quiz (leaving a few days between attempts) you can maximize the effects of the first strategy (effortful recall) and ensure that your second or third attempts truly reflect what you know about the topic. We suggest leaving 1-3 days between attempt 1 and 2, then 4-5 days between attempt 2 and 3. You can use a tool like Trello, Notion, or Asana (free versions), or even a task list on your phone to set up a spaced repetition schedule.
\item
  \textbf{Interleaving.} This is the practice of studying a particular topic for a relatively short period of time (maybe 30-40 mins), then switching to a different topic for the same period, before going back to the original topic. We will help build this into your learning by including items from unit 1 in your unit 2-6 quizzes. You can also practice this by taking regular breaks in your work, or even by retaking a unit 1 quiz while you are working in unit 2.
\end{enumerate}

These three strategies are very effective at helping people \emph{remember} key facts about a particular topic, an important first step in learning at the university level. However, you will be asked to do much more than just remember facts. Your ultimate goal is to develop \textbf{evaluative judgement}, or the ability for you to judge for yourself the quality of your (or your peers') responses to prompts.

The discussion forums are a key way for you to do this. We have set up the forums in such a way that you will need to present a response to any given prompt before you see other learners' responses. We strongly encourage you to use this structure to formulate your own ideas before you present them in the forum, and then to use the responses of your peers to help you evaluate your own response.

Using these self-check activities in this way is designed to help you to succeed on the course assignments, upon which your final grade will be determined. These assignments will require you to \textbf{use} the facts of the course to generate unique responses to the prompts, based on your past experiences, knowledge, and ability to evaluate the quality of your own work.

\hypertarget{how-to-navigate-this-book}{%
\subsection*{How To Navigate This Book}\label{how-to-navigate-this-book}}
\addcontentsline{toc}{subsection}{How To Navigate This Book}

To move quickly to different portions of the book, click on the appropriate chapter or section in the table of contents on the left. The buttons at the top of the page allow you to show/hide the table of contents, search the book, change font settings, download a pdf or ebook copy of this book, or get hints on various sections of the book.
\includegraphics{assets/course-intro/menu.png}

The faint left and right arrows at the sides of each page (or bottom of the page if it's narrow enough) allow you to step to the next/previous section. Here's what they look like:
\includegraphics{assets/course-intro/left_arrow.png} \includegraphics{assets/course-intro/right_arrow.png}

You can also download an offline copy of this book in various formats, such as pdf or an ebook. If you are having any accessibility or navigation issues with this book, please reach out to your instructor or our online team at \href{mailto:elearning@twu.ca}{\nolinkurl{elearning@twu.ca}}.

\hypertarget{course-units}{%
\subsection*{Course Units}\label{course-units}}
\addcontentsline{toc}{subsection}{Course Units}

This course is organized into 6 units. Each unit of the course will provide you with the following information:

\begin{itemize}
\tightlist
\item
  A general overview of the key concepts that will be addressed during the unit.\\
\item
  Specific learning outcomes and topics for the unit.\\
\item
  Learning activities to help you engage with the concepts. These often include key readings, videos, and reflective prompts.\\
\item
  The Assessment section provides details on assignments you will need to complete throughout the course to demonstrate your understanding of the course learning outcomes.
\end{itemize}

\begin{caution}
Note that assessments, including assignments and discussion posts will be submitted in Moodle. See the Assessment tab in Moodle for the assignment dropboxes.
\end{caution}

\hypertarget{course-activities}{%
\subsection*{Course Activities}\label{course-activities}}
\addcontentsline{toc}{subsection}{Course Activities}

Below is some key information on features you will see throughout the course.~

\begin{reflect}
\textbf{\emph{Learning Activity}}\\
This box will prompt you to engage in course concepts, often by viewing resources and reflecting on your experience and/or learning. Most learning activities are ungraded and are designed to help prepare you for the assessment in this course.
\end{reflect}

\begin{assessment}
\textbf{\emph{Assessment}}\\
This box will signify an assignment or discussion post you will submit in Moodle. Note that these demonstrate your understanding of the course learning outcomes. Be sure to review the grading rubrics for each assignment.
\end{assessment}

\begin{progress}
\textbf{\emph{Checking Your Learning}}\\
This box is for checking your understanding, to make sure you are ready for what follows. Ways to check your learning might include self-check quizzes or questions for discussion. These activities are not graded but are critical for you to be able to begin to develop evaluative judgement in this domain of knowledge.
\end{progress}

\begin{caution}
\textbf{\emph{Note}}\\
This box signifies key notes. It may also warn you of possible problems or pitfalls you may encounter!
\end{caution}

\hypertarget{course-communities}{%
\chapter*{Course Communities}\label{course-communities}}
\addcontentsline{toc}{chapter}{Course Communities}

As you begin this course, how will you build community with your fellow learners?

In this course, we have the following tools available to help foster community in your course, including other students who have previously taken this course. Some of these tools will be prescribed and graded (e.g.~Moodle Discussion Forums), others will be up to you to take advantage of.

Check with your course syllabus for which community tools will be used, and consider building your own Community of Practice with your classmates and external colleagues.

\hypertarget{communication-tools}{%
\section*{Communication Tools}\label{communication-tools}}
\addcontentsline{toc}{section}{Communication Tools}

\textbf{Moodle Discussion Forums}: In this course, we ask you to discuss ideas with your colleagues, challenging one another and analyzing key course resources. Refer to the course syllabus for assessment details, as well as the unit Assessment section for discussion questions. Submit your responses in Moodle.

\textbf{Video Conferencing}: We will have scheduled online meetings (Zoom or Teams). Take advantage of these face-to-face conferences! Come prepared with your questions and assigned activities. Refer to the course syllabus and unit activity instructions for details.

\emph{Optional:}

Your cohort may want to engage in other informal discussions to build community and support each other. Consider using the following:

\textbf{Learning Cafe:} This discussion forum in Moodle is a place for you to interact about things going on, share resources, and generally get to know one another. Your posts don't have to be course related. Take this opportunity to connect with fellow learners and learn from one another!

\textbf{Teams:} Every TWU course has a Teams channel, mostly to manage videos. Feel free to use the messaging feature to connect with peers.

\textbf{Twitter hashtag \texttt{\#CRSE\#\#\#}:} You can tweet about this course using \texttt{\#CRSE\#\#\#}.

\textbf{What's App:} Feel free to use a platform that works for you!! What's App is a popular chat forum that learners use for discussions, class projects, etc.

A key takeaway\ldots make these forums work for you! Interact with your peers, learn from each other, and make connections that will stay with you beyond this course.

With that, let's begin the journey together!

\includegraphics{assets/community/luke-porter-NEqEC7qa9FM-unsplash.jpg}

\hypertarget{project-excellence-and-professionalism}{%
\chapter{Project Excellence and Professionalism}\label{project-excellence-and-professionalism}}

\hypertarget{overview}{%
\section*{Overview}\label{overview}}
\addcontentsline{toc}{section}{Overview}

Welcome to MCOM 221! You are about to embark on a creative adventure. The specific goal of this journey is to create attention-grabbing and alluring short films. However, the general goal is far more valuable as you learn things about yourself you never knew. This course will equip you with the knowledge to develop transferable life skills that will help you in your career and professional life after you complete your education.

In addition, there is a larger element of this course that connects us to people from the past, present, and future. Namely, storytelling.

According to communications theorist Walter Fischer, human beings are more than \emph{Homo sapiens}, we are Homo \emph{narrans}, man the storyteller. Storytelling is hardwired into our brains and our cultures and history. Stories are how we make sense of our life and our world. This is why religion, philosophy, literature, and myth have been vital to human culture and understanding. Even science is a story---it tells us how the world works. But it cannot tell us why. Only story-based meaning can reveal why life is important, how we should live our lives, and why there is something rather than nothing. Think of the Bible. If one were to take out the stories, it would be a thin book.

Don't let these ideas scare you. This is a creative course not a philosophic one. The point here is that as you increase your storytelling skills---a vital part of this course---you will join the community of narrators throughout human history who have shared their wisdom, insight, and understanding. This knowledge serves as the foundation for building identities, developing traditions, and discovering the types of meaning that make life worth living across all cultures.

This course will focus on one aspect of narratives, namely visual storytelling. How do we use pictures (and sounds) to create meaning and emotionally connect people to ourselves and each other? This will be the deeper theme operating beneath the surface of this course. (In film terms it will be the subtext beneath the text.)

In practical terms, the better you are at communicating stories (relative to each profession), the better you will be at your job. Fasten your seat belt, you are about to find out why we have focused on the above to start this course.

\hypertarget{topics}{%
\subsection*{Topics}\label{topics}}
\addcontentsline{toc}{subsection}{Topics}

This unit is divided into the following topics:

\begin{enumerate}
\def\labelenumi{\arabic{enumi}.}
\tightlist
\item
  Course Introduction\\
\item
  Excellence and Professional and Personal Development\\
\item
  15 Tips for Doing Your Best\\
\item
  Transferable Skills
\end{enumerate}

\hypertarget{learning-outcomes}{%
\subsection*{Learning Outcomes}\label{learning-outcomes}}
\addcontentsline{toc}{subsection}{Learning Outcomes}

When you have completed this unit, you should be able to:

\begin{itemize}
\tightlist
\item
  Describe excellence and why it is important\\
\item
  Define what constitutes an excellent film project\\
\item
  Articulate the big picture of why stories and creativity are important\\
\item
  Determine potential transferable life skills\\
\item
  Self-assess your strengths and weaknesses\\
\item
  Plan what you want to focus on during the course in terms of your professional and personal growth.
\end{itemize}

\hypertarget{activity-checklist}{%
\subsection*{Activity Checklist}\label{activity-checklist}}
\addcontentsline{toc}{subsection}{Activity Checklist}

\begin{reflect}
Here is a checklist of learning activities you will benefit from in completing this unit. You may find it useful for planning your work.

\begin{itemize}
\tightlist
\item
  Write your first two Film Journal entries for this unit.\\
\item
  Choose a glossary and start learning 5 new film terms a day.\\
\item
  Read the Introduction which sets up the course ahead and Chapter One ``Project Excellence and Professionalism'' in the course textbook.\\
\item
  Reflect on the 15 Tips for Doing Your Best.\\
\item
  Reflection on the transferable life skills presented.
\end{itemize}

\textbf{Assessment}

\begin{itemize}
\tightlist
\item
  Film Journal
\end{itemize}
\end{reflect}

\hypertarget{resources}{%
\subsection*{Resources}\label{resources}}
\addcontentsline{toc}{subsection}{Resources}

Here are the resources you will need to complete this unit.

\begin{itemize}
\tightlist
\item
  Introduction and Chapter One of the course text: \emph{Digital Filmmaking: A Beginner's Guide to Mastering the Craft}, by Ned Vankevich (e-text)\\
\item
  \href{https://karenbanes.com/how-to-start-a-creative-journal}{How to Start a Creative Journal} (Be sure to click on the internal links on this website)\\
\item
  \href{https://www.youtube.com/watch?v=hUTWo7_W0lc}{How to Journal Every Day for Increased Productivity, Clarity, and Mental Health}\\
\item
  \href{https://brandyourself.com/blog/guide/how-to-grow-professionally/}{How To Grow Professionally}\\
\item
  \href{https://www.goodtherapy.org/learn-about-therapy/issues/creative-blocks}{Creative blocks website}\\
\item
  Other resources will be provided in the unit.
\end{itemize}

\hypertarget{course-introduction}{%
\section{Course Introduction}\label{course-introduction}}

We begin Unit 1 marvelling at the magic of film and cinema. (Read the course text Preface.) The ability of visually-centered storytelling to cross cultures and to captivate, educate, and entertain audiences is a universal contemporary phenomenon. For instance, how a series of flickering lights and images projected at 24, 25, and 30 frames a second can create emotions and experiences we all share is a wondrous mystery.

Less mysterious is how this is done. There are techniques, rules, guidelines, and practices that can help us reach people in the ways that films, TV shows, streaming Internet programs, news, and the host of other visually-centered can make us laugh, cry, and emotionally move and engage us. The worst thing that can be said by an audience is that ``I want my five, ten, sixty, or ninety minutes back (depending on the length of what we have viewed). Learning the methods to avoid this and to engage our audience will be central to this course.

\textbf{WHAT LIES AHEAD}

This course will be divided up into ten segments or units. Each segment (or chapter) is self-contained but each section is inter-connected and vital for the others. A word to the wise---do note not skip a section, even if you think that you already know it. Each section of the course will build upon the previous one.

See the overview of the course and the ten units in the Introduction to the course text: \emph{Introduction to Digital Filmmaking: A Beginner's Guide to Mastering the Craft}.

Of special note, Unit Ten and Chapter Ten will be a summary and a celebration of what you have learned during the course. It will also be a time for awards to be handed out for outstanding and excellent work such as Awards for Best film, Best Director, Best Cinematography, Best Editing., Best Story, and Most Imaginative and Most Improved Filmmaker, etc. This is noted here to give you a goal to work towards: namely, to win one of the top awards.

As can be gleaned from the above, this course will start with the most basic elements and proceed to creating a short film with a strong beginning, middle, and end. As you move through the course keep the mindset that this will be a fun, enjoyable, and exciting adventure.

\hypertarget{learning-activities}{%
\subsection*{Learning Activities}\label{learning-activities}}
\addcontentsline{toc}{subsection}{Learning Activities}

\begin{reflect}
\hypertarget{the-importance-of-the-film-journal}{%
\subsubsection*{The Importance of the Film Journal}\label{the-importance-of-the-film-journal}}
\addcontentsline{toc}{subsubsection}{The Importance of the Film Journal}

During this course you are encouraged to keep a ``Film Journal.'' This is important for many reasons, including helping you to brainstorm ideas, as well as ponder and process what you are learning. The journal will also help you to keep a record of your course activities, and note feedback of what does and does not work in film projects of your fellow course members. In addition, some of the exercises for this course will not be graded (for reasons that will be explained ahead) but this does not mean you will not be accountable for doing them. Your journal will be submitted at the end of the course and will play a role in the grade you receive. In short, students who engage with the course well will in general receive higher marks.

Also, note that you may be asked to use your journal entries to participate in discussions, presentations, and other learning activities in the FAR Centre Facilitated Learning Labs. Please check with your facilitator about specific due dates for activities, including journal responses.

\hypertarget{film-journal-entry}{%
\subsubsection*{Film Journal Entry:}\label{film-journal-entry}}
\addcontentsline{toc}{subsubsection}{Film Journal Entry:}

Based on the importance of the course journal make your first two entries for this unit:

\begin{itemize}
\tightlist
\item
  \textbf{Entry One:} After reading the course text Introduction, describe your best take-aways of what you have learned.
\item
  \textbf{Entry Two:} Study the recommended how-to-write-a-journal resources and log what you learned from them and how it will help you during the course.

  \begin{itemize}
  \tightlist
  \item
    \href{https://karenbanes.com/how-to-start-a-creative-journal}{How to Start a Creative Journal} (Be sure to click on the internal links on this website)\\
  \item
    \href{https://www.youtube.com/watch?v=hUTWo7_W0lc}{How to Journal Every Day for Increased Productivity, Clarity, and Mental Health}
  \end{itemize}
\end{itemize}
\end{reflect}

\begin{caution}
\textbf{NOTE} : your journal entries do not have to be long but they should be detailed and specific enough to indicate that you have engaged the course material and projects.
\end{caution}

\hypertarget{learning-activities-1}{%
\subsection*{Learning Activities}\label{learning-activities-1}}
\addcontentsline{toc}{subsection}{Learning Activities}

\begin{reflect}
\hypertarget{film-terms-and-vocabulary-needed-for-this-course}{%
\subsubsection*{Film Terms and Vocabulary Needed for this Course}\label{film-terms-and-vocabulary-needed-for-this-course}}
\addcontentsline{toc}{subsubsection}{Film Terms and Vocabulary Needed for this Course}

The course text Introduction also draws attention to the fact that you will learn many new terms and concepts during this course and that it is important for you to find resources that you can consult throughout the course to re-enforce their meaning and application to the course. Several websites will be cited here such as:

\begin{itemize}
\tightlist
\item
  studiobinder movie film terms \\
\item
  A more comprehensive glossary can be found at Wikipedia's Glossary of Motion Picture terms ;\\
\item
  and AMC Film Terms Glossary
\end{itemize}

Commit to learning \textbf{\emph{5}} new film terms each day by starting with A and going to Z. The more you focus on this the better you will do in the course. Don't worry if you do not understand a term. We will be covering many of them during the course. The important thing is to become familiar with important film terms and vocabulary.

\hypertarget{questions-to-consider}{%
\subsubsection*{Questions to Consider}\label{questions-to-consider}}
\addcontentsline{toc}{subsubsection}{Questions to Consider}

After completing the activities above, consider the following questions:

\begin{itemize}
\tightlist
\item
  Why is this course important to you?\\
\item
  What do you hope to learn and how can it aid your future?
\end{itemize}
\end{reflect}

\begin{caution}
\textbf{Note} that while the above learning activity is ungraded, it is designed to help you succeed in your assessments in this course.
\end{caution}

\hypertarget{excellence-and-professional-and-personal-development}{%
\section{Excellence and Professional and Personal Development}\label{excellence-and-professional-and-personal-development}}

This course will focus on the classical approach to filmmaking, that is, making motivated film stories and executing them in ways that help audiences to fully engage the story and characters without unnecessary distractions. In doing so, the course has a lofty goal: to aim for excellence. Excellence means a high standard of being good. This is of course relative to being an Introductory course with students who are just beginning their filmmaking journey, or are taking this course as an elective. But like any course, if you shoot for an A you will do better than settling for average and mediocrity.

Excellence is important in filmmaking because our competition is great. Film is a public medium. It is designed to be shown to audiences and the quality of our projects, like many things we do in life, will be judged whether we want them to be or not.

One of the transferable life skills of this course is to help us be more professional. Thus the more we shoot for excellence, the more we will develop abilities that will serve us better in our careers ahead.

At the same time we must not confuse excellence with perfection. Our beginning films and projects will have lots of mistakes and things that do not work. This should not deter us from our goal of excellence. All great artists, athletes, business leaders, and a host of other professional make mistakes. The key is to learn from them and to keep improving our craft.

HELPFUL HINT: For insight into how great artists, athletes, and performers use mistakes to grow in their art and craft see the following book by Daniel Coyle: \href{http://danielcoyle.com/the-talent-code/}{THE TALENT CODE: GREATNESS ISN'T BORN. IT'S GROWN. HERE'S HOW}.

Our goal then is to learn how to seek to produce visually-driven, digital film stories that are technically and artistically proficient, meaningfully engaging, motivated, and fraught with excellence.

As you can see excellence and professionalism go hand in hand. In order to grow and excel in a craft we need to grow and excel at being professional. Implied here is that we also need to mature and grow personally as we develop the character traits necessary for being a mature, moral, and responsible person.

\hypertarget{learning-activities-2}{%
\subsection*{Learning Activities}\label{learning-activities-2}}
\addcontentsline{toc}{subsection}{Learning Activities}

\begin{reflect}
\hypertarget{read-and-reflect}{%
\subsubsection*{Read and Reflect}\label{read-and-reflect}}
\addcontentsline{toc}{subsubsection}{Read and Reflect}

Read Chapter One ``Project Excellence and Professionalism'' in the course textbook.

As you study and ponder what excellence and what professional and personal development mean, log in your journal why they are important and how you hope to focus on them during the course. Which character traits are your strongest and which do you need to work on? How will you do this? To help you with these entries study this resource:

\href{https://brandyourself.com/blog/guide/how-to-grow-professionally/}{How To Grow Professionally}
\end{reflect}

\hypertarget{tips-for-doing-your-best}{%
\section{15 Tips for Doing Your Best}\label{tips-for-doing-your-best}}

The following will help you to not just do well in this Film project, but will make it fun and more enjoyable and hopefully be one of your favourite courses ever:

\begin{enumerate}
\def\labelenumi{\arabic{enumi}.}
\item
  \textbf{\emph{Keep your eye on the prize.}} Creating an excellent project and growing as a professional and better person through the process.
\item
  \textbf{\emph{Meeting Deadlines.}} Being on time is vital in any profession, especially in filmmaking where the cost of feature film shoots can be thousands of dollars per hour. Plan your time and schedule accordingly so that you make your due date. Failure to do so will result in reduced marks on your assignment.
\item
  \textbf{\emph{Avoiding Distractions.}} In our Internet-cell phone era we are inundated and surrounded by 24/7 distractions. If we succumb to them we will not meet our deadlines and thus delay our professional and personal development. Learning to discern what is vital and what is the tyranny of the unimportant and unnecessary urgent is a critical skill to develop today. Focus on it and you will go further in the course. (\textbf{Tip:} \emph{turn off all notifications on your cell phone. In fact, turn off or put your cell phone away while you are working on your Film course. You will find that this will help you to focus on and finish your assignments quicker and with more creativity}).
\item
  \textbf{\emph{Focussing on the positive}}. Most of the assignments ahead will be challenging. Cultivate a ``can do attitude.'' For context think about your taking a piano lesson course. You would not expect to sit down and immediately play chords and songs. You would have to learn to play the notes, then chords, then songs over a period of time. Filmmaking is similar. There are notes to learn (film shots), chords to learn (film scenes), before we can create melodies (film sequences) and songs (full films).
\item
  \textbf{\emph{It takes time and practice to master film}}. If you find you have time, or if you are truly interested in learning the world of film or becoming a filmmaker, I encourage you to go beyond the course exercises. For example, continue to find and shoot interesting visual compositions (they are all around you) during the course, or practice fascinating and more complicated motion shots and two and three person shots. The more you do this the faster you will grow in your craft. Think about musicians. They practice over and over everyday.
\item
  \textbf{\emph{Study the practical tips and helpful hints in these units}}. Pay attention to them. They will help fast-track your skills growth.
\item
  \textbf{\emph{Find and do what you love}}. During this course you will discover things you love to do and those that you do not. As a filmmaker, every skill set is vital for the final production -- but you will discover which roles you are better at and enjoy the most. Even if you do not go into filmmaking as a career this self-insight will be invaluable as you make life decisions ahead.
\item
  \textbf{\emph{Make a firm commitment to succeed}}. When we are double-minded or not sure of what we want, we will waiver and fall prey to distraction and self-sabotage. Resolve not to give up as you make your projects despite what ``Murphy's Law'' tosses your way. (\emph{To be covered later.})
\item
  \textbf{\emph{Strive for balance}}. The emphasis on excellence and professional and personal development must not make us ``lose our soul.'' Too many people get caught up in goals and ambition and end up failing in their marriage and relationships. Remember we are physical, mental, spiritual, emotional, and relational beings and must find a balance that lets us grow in each part equally. This is the key to personal flourishing.
\item
  \textbf{\emph{Create strong and healthy relationships}}. During this course you will discover a lot about yourself and others. Some people will be easy to work with and some won't. Some will be diligent and some won't. Choose your teammates wisely and if you have a winning rapport with them, you might want to work with them on the next project. One of the keys to success in the professional world is cultivating strong and fruitful relationships. Start the process here.
\item
  \textbf{\emph{Follow your intuition}}. If you've never experimented with following your deeper instincts or ``gut feelings'' try it during this course. Often this will take you to new creative places. It might not always work but you will learn a lot from doing so.
\item
  \textbf{\emph{Take risks}}. We don't grow unless we try new things and fail. They key is not to get discouraged but to shake off a loss and fight to win again.
\item
  \textbf{\emph{Practice, practice, practice. Learn, learn, learn}}. Your best competition is continually practicing and learning their craft. So should you. You need to cultivate both a strong work ethic and a smart ethic. Take the time to think things through before launching into something.
\item
  \textbf{\emph{Take responsibility for yourself and your future}}. Playing the blame game or the victim or continually making excuses will not get you ahead in life.
\item
  \textbf{\emph{Find inspiring quotes and pin them to your computer or wall}}. Most of us need encouragement especially when the going gets tough. Seek out inspiration that helps keep you going. The following strike some of the themes of this course:
\end{enumerate}

\begin{itemize}
\item
  \emph{``Success is not final;~failure is not fatal: It is the courage to continue that counts.''} - Winston S. Churchill
\item
  \emph{``It is better to fail in originality than to succeed in imitation.''} - Herman Melville
\item
  \emph{``You will become clever through your mistakes.''} - German Proverb in ``The Talent Code''
\end{itemize}

\hypertarget{learning-activities-3}{%
\subsection*{Learning Activities}\label{learning-activities-3}}
\addcontentsline{toc}{subsection}{Learning Activities}

\begin{reflect}
\hypertarget{reflection}{%
\subsubsection*{Reflection}\label{reflection}}
\addcontentsline{toc}{subsubsection}{Reflection}

Which of the tips in this section appeal to you most? Which do you need to focus on during the course? Write your response in your journal and share with your facilitator and/or classmates which are most important and why. Find an inspiring quote and share it with your peers.
\end{reflect}

\hypertarget{transferable-skills}{%
\section{Transferable Skills}\label{transferable-skills}}

This class is about more than making films. If followed diligently, you will learn a lot about yourself, others, and the world around you. You will also learn skills and traits that will play an important role in your life journey ahead. Some of the transferable skills can include:

\begin{itemize}
\tightlist
\item
  Growing in your creative skills\\
\item
  Growing in your teamwork skills\\
\item
  Growing in your critiquing skills\\
\item
  Growing in your ability to overcome obstacles\\
\item
  Meeting deadlines and taking responsibility\\
\item
  Becoming a leader\\
\item
  Learning to take constructive criticism\\
\item
  Committing to continual learning and improvement\\
\item
  Developing a positive, can-do attitude\\
\item
  Crafting what you want to say and communicating it well\\
\item
  Learning the power of story and how it impacts many areas in our lives.
\end{itemize}

\hypertarget{learning-activities-4}{%
\subsection*{Learning Activities}\label{learning-activities-4}}
\addcontentsline{toc}{subsection}{Learning Activities}

\begin{reflect}
\hypertarget{reflection-on-life-skills}{%
\subsubsection*{Reflection on Life Skills}\label{reflection-on-life-skills}}
\addcontentsline{toc}{subsubsection}{Reflection on Life Skills}

Which of the life skills above are important to you and why? Log this in your journal. Which life skill(s) can you add to the list? Share your responses with your Facilitator and classmates.
\end{reflect}

\begin{caution}
\textbf{NOTE}: Be sure to revisit these journal entries on Tips to Do Well and Transferable Life Skills as you monitor your progress during the course. One of the actives in Unit Ten will involve your writing a summary of the best lessons you learned from the course and about yourself. Again, this is a reminder of how important our course journal will be.
\end{caution}

\hypertarget{summary}{%
\section*{Summary}\label{summary}}
\addcontentsline{toc}{section}{Summary}

In this first unit, you have had the opportunity to learn about what this course entails, why it is important, and the role that professional and personal development will play. You also had a chance to respond to tips of how to make the most of this course and what skills you can transfer from it to your life.

\hypertarget{assessment}{%
\section*{Assessment}\label{assessment}}
\addcontentsline{toc}{section}{Assessment}

\begin{assessment}
\hypertarget{film-journal}{%
\subsection*{Film Journal}\label{film-journal}}
\addcontentsline{toc}{subsection}{Film Journal}

After completing this unit, including the learning activities, you are asked to make sure you are doing journal entries and when appropriate to share your responses with your facilitator and classmates when you meet.

Note that entries are expected for every unit. Your journal reflections are submitted at the end of the course as part of the Final Exam: Self Assessment worth 30\% of your course grade.

\hypertarget{grading-criteria}{%
\subsubsection*{Grading Criteria:}\label{grading-criteria}}
\addcontentsline{toc}{subsubsection}{Grading Criteria:}

See the Assessments section for more details on submitting your journal, as well as the grading criteria.
\end{assessment}

\hypertarget{checking-your-learning}{%
\section*{Checking your Learning}\label{checking-your-learning}}
\addcontentsline{toc}{section}{Checking your Learning}

\begin{progress}
Before you move on to the next unit, you may want to check to make sure that you are able to:

\begin{itemize}
\tightlist
\item
  Describe excellence and why it is important.\\
\item
  Define what constitutes an excellent film project.\\
\item
  Articulate the big picture of why stories and creativity are important.\\
\item
  Determine potential transferable life skills.\\
\item
  Self-assessing your strengths and weaknesses.\\
\item
  Plan what you want to focus on during the course in terms of professional and personal growth.
\end{itemize}
\end{progress}

\hypertarget{title}{%
\chapter{Title}\label{title}}

\hypertarget{title-1}{%
\chapter{Title}\label{title-1}}

\hypertarget{title-2}{%
\chapter{Title}\label{title-2}}

\hypertarget{title-3}{%
\chapter{Title}\label{title-3}}

\hypertarget{title-4}{%
\chapter{Title}\label{title-4}}

\hypertarget{title-5}{%
\chapter{Title}\label{title-5}}

\hypertarget{title-6}{%
\chapter{Title}\label{title-6}}

\hypertarget{assessment-1}{%
\chapter*{Assessment}\label{assessment-1}}
\addcontentsline{toc}{chapter}{Assessment}

The following assignments are opportunities for learners to demonstrate their understanding of the course outcomes. Please confirm assignment details with your instructor, referring to the course syllabus.

Note that Assignment dropboxes are located in Moodle. Also refer to the Course Schedule in Moodle for the specific due dates.

\hypertarget{assignment}{%
\section*{Assignment:}\label{assignment}}
\addcontentsline{toc}{section}{Assignment:}

\begin{assessment}

\end{assessment}

\hypertarget{grading-criteria-1}{%
\subsection*{Grading Criteria}\label{grading-criteria-1}}
\addcontentsline{toc}{subsection}{Grading Criteria}

See the following rubric that explains how your assignment will be evaluated. Also available as a \href{assets/assessment/Identity-as-a-Teacher-RUBRIC.pdf}{pdf}

\#\#\#\# APA/WRITING \{-\}

\textbf{Unsatisfactory:} Paper does not model language and conventions used in scholarly literature. Writing is not well-organized. Several errors in grammar or composition. Sources are not cited. APA citations are not appropriately formatted.

\textbf{Developing:} Paper partially models language and conventions used in scholarly literature. Writing is somewhat well organized and includes some errors in grammar or composition. Not all sources cited. APA citations are generally formatted correctly, with several errors.

\textbf{Proficient:} \emph{Paper consistently models language and conventions used in scholarly literature. Writing is well-organized and includes few (if any) errors in grammar or composition. All resources are appropriately cited (including in-text citations and bibliography information). Few (if any) errors in APA citations.}

\textbf{Exemplary:} Paper is an exemplar of language and conventions used in scholarly literature. Writing is well-organized and free of errors in grammar or composition. All resources are appropriately cited. No errors in APA format.

\hypertarget{statement-of-teaching-identity}{%
\subsubsection*{STATEMENT OF TEACHING IDENTITY}\label{statement-of-teaching-identity}}
\addcontentsline{toc}{subsubsection}{STATEMENT OF TEACHING IDENTITY}

\textbf{Unsatisfactory:} Does not provide a statement about identity as a teacher/facilitator

\textbf{Developing:} Provides an unclear statement about identity as a teacher/facilitator.

\textbf{Proficient:} \emph{Provides a clear, concise, and powerful statement about identity as a teacher/facilitator.}

\textbf{Exemplary:} Provides a clear, concise, and powerful statement about identity as a teacher/facilitator. Statement incorporates theory or research from course materials.

\hypertarget{developing-a-cohesive-and-logical-academic-argument}{%
\subsubsection*{DEVELOPING A COHESIVE AND LOGICAL ACADEMIC ARGUMENT}\label{developing-a-cohesive-and-logical-academic-argument}}
\addcontentsline{toc}{subsubsection}{DEVELOPING A COHESIVE AND LOGICAL ACADEMIC ARGUMENT}

\textbf{Unsatisfactory:} Does not make a focused, cohesive, or logical academic argument. Paper is confusing, and is missing an introduction, body, or conclusion. Transitions between sections and ideas are missing.

\textbf{Developing:} Makes an academic argument that is only partially focused, cohesive and logical. Paper is generally organized, but is missing an introduction, body, or conclusion. Transitions between sections and ideas are unclear.

\textbf{Proficient:} \emph{Makes a focused, cohesive, logical academic argument. Paper is effectively organized and includes an introduction, body, and conclusion. Transitions between sections and ideas are clear.}

\textbf{Exemplary:} Makes a focused, cohesive, logical and compelling academic argument. Paper is effectively organized and includes an introduction, body, and conclusion. Transitions between sections and ideas are clear, and build on each other.

\hypertarget{analysis-of-identity-as-a-teacher}{%
\subsubsection*{ANALYSIS OF IDENTITY AS A TEACHER}\label{analysis-of-identity-as-a-teacher}}
\addcontentsline{toc}{subsubsection}{ANALYSIS OF IDENTITY AS A TEACHER}

\textbf{Unsatisfactory:} Does not include three important aspects of identity as a teacher/facilitator. Does not include an analysis.

\textbf{Developing:} Lists but does not discuss three important aspects of identity as a teacher/facilitator. Includes a partial analysis.

\textbf{Proficient:} \emph{Includes a detailed discussion of three important aspects of identity as a teacher/facilitator. Includes thoughtful analysis of each of the three elements.}

\textbf{Exemplary:} Includes a detailed discussion of three important aspects of identity as a teacher/facilitator. Includes a thoughtful analysis, integrating scholarly literature to support analysis and furthering scholarly thinking related to teacher identity.

\hypertarget{scholarly-integration}{%
\subsubsection*{SCHOLARLY INTEGRATION}\label{scholarly-integration}}
\addcontentsline{toc}{subsubsection}{SCHOLARLY INTEGRATION}

\textbf{Unsatisfactory:} Does not integrate references to support claims and assertions made in the paper.

\textbf{Developing:} Integrates references to support some of the claims and assertions made in the paper.

\textbf{Proficient:} \emph{Integrates references to support claims and assertions made in the paper.}

\textbf{Exemplary:} Integrates references to support claims and assertions made in the paper, effectively synthesizing different perspectives and research results from scholarly sources.

\begin{longtable}[]{@{}
  >{\raggedright\arraybackslash}p{(\columnwidth - 8\tabcolsep) * \real{0.2000}}
  >{\raggedright\arraybackslash}p{(\columnwidth - 8\tabcolsep) * \real{0.2000}}
  >{\raggedright\arraybackslash}p{(\columnwidth - 8\tabcolsep) * \real{0.2000}}
  >{\raggedright\arraybackslash}p{(\columnwidth - 8\tabcolsep) * \real{0.2000}}
  >{\raggedright\arraybackslash}p{(\columnwidth - 8\tabcolsep) * \real{0.2000}}@{}}
\toprule\noalign{}
\begin{minipage}[b]{\linewidth}\raggedright
\textbf{TOTAL}
\end{minipage} & \begin{minipage}[b]{\linewidth}\raggedright
\textbf{0 = 0\% (F)}
\end{minipage} & \begin{minipage}[b]{\linewidth}\raggedright
\textbf{10 = 50\% (C)}
\end{minipage} & \begin{minipage}[b]{\linewidth}\raggedright
\textbf{15 = 75 (B)}
\end{minipage} & \begin{minipage}[b]{\linewidth}\raggedright
\textbf{20 = 100\% (A+)}
\end{minipage} \\
\midrule\noalign{}
\endhead
\bottomrule\noalign{}
\endlastfoot
\end{longtable}

\begin{center}\rule{0.5\linewidth}{0.5pt}\end{center}

\hypertarget{assignment-company-website-analysis}{%
\section*{Assignment: Company Website Analysis}\label{assignment-company-website-analysis}}
\addcontentsline{toc}{section}{Assignment: Company Website Analysis}

\begin{assessment}
Investigate the Human Resources or Faculty Development portion of a
company's website, a higher education institution or adult learning
facility, preferably one with which you are familiar. Focus on the
faculty or employee development part of the website. In this assignment,
you will apply the theory of teaching in/for/with depth by analyzing the
learning culture of an organization.

In a 4-5 page APA formatted paper, analyze the website by responding to
the following questions in your report:

\begin{enumerate}
\def\labelenumi{\arabic{enumi}.}
\tightlist
\item
  What can you infer about the company's learning culture?
\item
  From what is visible on the public website, would you say it is an
  authentic learning community? Why or why not? Discuss whether the
  website reflects aspects of one or more of the learning community
  models explored in previous lessons.
\item
  Do you see evidence that interconnectedness and integrity are valued?
  Explain.
\item
  What traits and skills seem to be valued in employees?
\item
  How does the company develop skills in its employees (e.g., workshops,
  seminars, mentoring)? Are the methods based on the principles of
  andragogy? (see Smith YouTube video). What specific adult learning
  strategies do you see reflected in the development/training
  opportunities for employees?
\end{enumerate}

Your paper should be 4-5 pages and should incorporate references to at
least five scholarly sources you have studied in this course, or other
scholarly sources you have identified.

The paper should include:

\begin{enumerate}
\def\labelenumi{\arabic{enumi}.}
\tightlist
\item
  Introduction
\item
  Analysis (responding to the prompts)
\item
  Conclusion
\item
  Reference List
\end{enumerate}
\end{assessment}

\hypertarget{company-website-analysis-rubric}{%
\subsection*{Company Website Analysis Rubric}\label{company-website-analysis-rubric}}
\addcontentsline{toc}{subsection}{Company Website Analysis Rubric}

See the following rubric that explains how your assignment will be evaluated. Also available as a \href{assets/assessment/Company-Website-Analysis-RUBRIC.pdf}{pdf}

\hypertarget{apa-formatting}{%
\subsubsection*{APA Formatting}\label{apa-formatting}}
\addcontentsline{toc}{subsubsection}{APA Formatting}

\textbf{Unsatisfactory:} Paper does not model language and conventions used in scholarly literature.
Writing is not well-organized. Several errors in grammar or composition. Sources
are not cited. APA citations are not appropriately formatted.

\textbf{Developing:} Paper partially models language and conventions used in scholarly literature.
Writing is somewhat well organized and includes some errors in grammar or
composition. Not all sources cited. APA citations are generally formatted
correctly, with several errors.

\textbf{Proficient:} \emph{Paper consistently models language and conventions used in scholarly
literature. Writing is well-organized and includes few (if any) errors in
grammar or composition. All resources are appropriately cited (including in-text
citations and bibliography information). Few (if any) errors in APA citations.}

\textbf{Exemplary:} Paper is an exemplar of language and conventions used in scholarly literature.
Writing is well-organized and free of errors in grammar or composition. All
resources are appropriately cited. No errors in APA format.

\hypertarget{developing-a-cohesive-and-logical-academic-argument-1}{%
\subsubsection*{DEVELOPING a COHESIVE and LOGICAL ACADEMIC ARGUMENT}\label{developing-a-cohesive-and-logical-academic-argument-1}}
\addcontentsline{toc}{subsubsection}{DEVELOPING a COHESIVE and LOGICAL ACADEMIC ARGUMENT}

\textbf{Unsatisfactory:} Does not make a focused, cohesive, or logical academic argument. Paper is
confusing, and is missing an introduction, body, or conclusion. Transitions
between sections and ideas are missing.

\textbf{Developing:} Makes an academic argument that is only partially focused, cohesive and logical.
Paper is generally organized, but is missing an introduction, body, or
conclusion. Transitions between sections and ideas are unclear.

\textbf{Proficient:} \emph{Makes a focused, cohesive, logical academic argument. Paper is effectively
organized and includes an introduction, body, and conclusion. Transitions
between sections and ideas are clear.}

\textbf{Exemplary:} Makes a focused, cohesive, logical and compelling academic argument. Paper is
effectively organized and includes an introduction, body, and conclusion.
Transitions between sections and ideas are clear and build on each other.

\hypertarget{analysis-of-learning-culture}{%
\subsubsection*{ANALYSIS of LEARNING CULTURE}\label{analysis-of-learning-culture}}
\addcontentsline{toc}{subsubsection}{ANALYSIS of LEARNING CULTURE}

\textbf{Unsatisfactory:} Does not include an analysis of the company learning culture, and no evaluation
of the authenticity of the learning community.

\textbf{Developing:} Includes a partial analysis of the company learning culture, including a limited
evaluation of the authenticity of the learning community.

\textbf{Proficient:} \emph{Includes a detailed analysis of the company learning culture, including an
evaluation of the authenticity of the learning community.}

\textbf{Exemplary:} Includes a detailed analysis of the company learning culture, including an
evaluation of the authenticity of the learning community. Includes a thoughtful
analysis, integrating scholarly literature to support analysis and furthering
scholarly thinking related to teacher identity.

\hypertarget{evaluation-of-interconnectedness-and-integrity}{%
\subsubsection*{EVALUATION of INTERCONNECTEDNESS and INTEGRITY}\label{evaluation-of-interconnectedness-and-integrity}}
\addcontentsline{toc}{subsubsection}{EVALUATION of INTERCONNECTEDNESS and INTEGRITY}

\textbf{Unsatisfactory:} Does not include an evaluation of evidence of interconnectedness and integrity
on the company website. Does not integrate scholarly sources in the evaluation.

\textbf{Developing:} Includes a partial evaluation of evidence of interconnectedness and integrity on
the company website. Evaluation includes only limited reference to scholarly
sources.

\textbf{Proficient:} \emph{Includes a detailed evaluation of evidence of interconnectedness and integrity
on the company website. Evaluation integrates scholarly sources.}

\textbf{Exemplary:} Includes a detailed evaluation of evidence of interconnectedness and integrity
on the company website. Includes recommendations for ways in which to integrate
interconnectedness and integrity into employee development.

\hypertarget{analysis-of-adult-learning-strategies}{%
\subsubsection*{ANALYSIS of ADULT LEARNING STRATEGIES}\label{analysis-of-adult-learning-strategies}}
\addcontentsline{toc}{subsubsection}{ANALYSIS of ADULT LEARNING STRATEGIES}

\textbf{Unsatisfactory:} Does not include a detailed analysis of valued skills and evidence of adult
learning theory in employee development. Does not integrate scholarly sources.

\textbf{Developing:} Includes a partial analysis of valued skills and evidence of adult learning
theory in employee development. Analysis integrates few, if any, scholarly
sources.

\textbf{Proficient:} \emph{Includes a detailed analysis of valued skills and evidence of adult learning
theory in employee development. Analysis integrates scholarly sources.}

\textbf{Exemplary:} Includes a detailed analysis of valued skills and evidence of adult learning
theory in employee development. Includes recommendations for ways in which to
integrate adult learning theory into employee development.

\hypertarget{scholarly-integration-1}{%
\subsubsection*{SCHOLARLY INTEGRATION}\label{scholarly-integration-1}}
\addcontentsline{toc}{subsubsection}{SCHOLARLY INTEGRATION}

\textbf{Unsatisfactory:} Does not integrate scholarly references to support claims and assertions made in
the paper.

\textbf{Developing:} Integrates scholarly references to support some of the claims and assertions
made in the paper.

\textbf{Proficient:} \emph{Integrates scholarly references to support claims and assertions made in the
paper.}

\textbf{Exemplary:} Integrates scholarly references to support claims and assertions made in the
paper, effectively synthesizing different perspectives and research results from
scholarly sources.

\begin{longtable}[]{@{}
  >{\raggedright\arraybackslash}p{(\columnwidth - 8\tabcolsep) * \real{0.2000}}
  >{\raggedright\arraybackslash}p{(\columnwidth - 8\tabcolsep) * \real{0.2000}}
  >{\raggedright\arraybackslash}p{(\columnwidth - 8\tabcolsep) * \real{0.2000}}
  >{\raggedright\arraybackslash}p{(\columnwidth - 8\tabcolsep) * \real{0.2000}}
  >{\raggedright\arraybackslash}p{(\columnwidth - 8\tabcolsep) * \real{0.2000}}@{}}
\toprule\noalign{}
\begin{minipage}[b]{\linewidth}\raggedright
\textbf{TOTAL}
\end{minipage} & \begin{minipage}[b]{\linewidth}\raggedright
\textbf{0 = 0\% (F)}
\end{minipage} & \begin{minipage}[b]{\linewidth}\raggedright
\textbf{10 = 50\% (C)}
\end{minipage} & \begin{minipage}[b]{\linewidth}\raggedright
\textbf{15 = 75 (B)}
\end{minipage} & \begin{minipage}[b]{\linewidth}\raggedright
\textbf{20 = 100\% (A+)}
\end{minipage} \\
\midrule\noalign{}
\endhead
\bottomrule\noalign{}
\endlastfoot
\end{longtable}

\begin{center}\rule{0.5\linewidth}{0.5pt}\end{center}

\hypertarget{assignment-platform-paper}{%
\section*{Assignment: Platform Paper}\label{assignment-platform-paper}}
\addcontentsline{toc}{section}{Assignment: Platform Paper}

\begin{assessment}
For this assignment, you will write a contextualized Platform Paper in
which you discuss your ideal learning community and your role as
teacher/leader of that learning community. Select a context for your
paper (i.e.~facilitating in a FAR Centre in a specific country, teaching
adult learners, facilitating employee development workshops, etc.). Your
paper should be written and referenced in APA format and include
references to a minimum of 10 scholarly sources (this can include
literature you read in this course). You will write a draft of the
Platform Paper in Unit 8 and post for Peer Review. In Unit 9, you will
provide feedback to another learner on their paper. You will make
revisions based on the Peer Review and, in Unit 10, you will submit the
final Platform Paper. Peer reviewers will be assigned in advance.

\hypertarget{paper-outline}{%
\subsubsection{Paper Outline}\label{paper-outline}}

This paper will be 12-15 pages long, and should include: 1. Introduction
(1-2 pages) 2. Section 1: Ideal Learning Environment (5-7 pages) 3.
Section 2: Your Role as Teacher and Leader (5-7 pages) 4. Conclusion
(1-2 pages)

\hypertarget{paper-guidelines}{%
\subsubsection{Paper Guidelines}\label{paper-guidelines}}

\begin{itemize}
\tightlist
\item
  \textbf{Introduction}: Introduce the two sections in your paper,
  providing a brief description of the key points you will make in each
  section.
\item
  \textbf{Section 1}: In section one, you will describe your ideal
  education learning environment. This section should demonstrate your
  learning about authentic learning communities, incorporating scholarly
  sources and your own analysis to depict your ideal learning
  environment. Incorporate a discussion of the learning community
  environment, learning experiences, student learning outcomes, and
  personal beliefs about teaching and learning.
\item
  \textbf{Section 2}: In this section, describe your role as a teacher
  or leader within an authentic learning community. Incorporating
  scholarly literature, analyze your role as a facilitator/leader in
  planning learning experiences, facilitating student learning, and
  assessing student learning. Describe the actions, practices, and
  strategies you will engage in to achieve your vision of the learning
  community you described in section one.
\item
  \textbf{Conclusion}: Summarize the key points you made in each
  section.
\item
  \textbf{References}: Include a reference list with references to at
  least 10 scholarly sources.
\end{itemize}
\end{assessment}

\hypertarget{platform-paper-rubric}{%
\subsection*{Platform Paper Rubric}\label{platform-paper-rubric}}
\addcontentsline{toc}{subsection}{Platform Paper Rubric}

See the following rubric that explains how your assignment will be evaluated. Also available as a \href{assets/assessment/Platform-Paper-RUBRIC.pdf}{pdf}

\hypertarget{apawriting}{%
\subsubsection*{APA/WRITING}\label{apawriting}}
\addcontentsline{toc}{subsubsection}{APA/WRITING}

\textbf{Unsatisfactory:} Paper does not model language and conventions used in scholarly literature. Writing is not well-organized. Several errors in grammar or composition. Sources are not cited. APA citations are not appropriately formatted.

\textbf{Developing:} Paper partially models language and conventions used in scholarly literature. Writing is somewhat well organized and includes some errors in grammar or composition. Not all sources cited. APA citations are generally formatted correctly, with several errors.

\textbf{Proficient:} \emph{Paper consistently models language and conventions used in scholarly literature. Writing is well-organized and includes few (if any) errors in grammar or composition. All resources are appropriately cited (including in-text citations and bibliography information). Few (if any) errors in APA citations.}

\textbf{Exemplary:} Paper is an exemplar of language and conventions used in scholarly literature. Writing is well-organized and free of errors in grammar or composition. All resources are appropriately cited. No errors in APA format.

\hypertarget{developing-a-cohesive-and-logical-academic-argument-2}{%
\subsubsection*{DEVELOPING a COHESIVE and LOGICAL ACADEMIC ARGUMENT}\label{developing-a-cohesive-and-logical-academic-argument-2}}
\addcontentsline{toc}{subsubsection}{DEVELOPING a COHESIVE and LOGICAL ACADEMIC ARGUMENT}

\textbf{Unsatisfactory:} Does not make a focused, cohesive, or logical academic argument. Paper is confusing, and is missing an introduction, body, or conclusion. Transitions between sections and ideas are missing.

\textbf{Developing:} Makes an academic argument that is only partially focused, cohesive and logical. Paper is generally organized, but is missing an introduction, body, or conclusion. Transitions between sections and ideas are unclear.

\textbf{Proficient:} \emph{Makes a focused, cohesive, logical academic argument. Paper is effectively organized and includes an introduction, body, and conclusion. Transitions between sections and ideas are clear.}

\textbf{Exemplary:} Makes a focused, cohesive, logical and compelling academic argument. Paper is effectively organized and includes an introduction, body, and conclusion. Transitions between sections and ideas are clear, and build on each other.

\hypertarget{ideal-learning-environment}{%
\subsubsection*{IDEAL LEARNING ENVIRONMENT}\label{ideal-learning-environment}}
\addcontentsline{toc}{subsubsection}{IDEAL LEARNING ENVIRONMENT}

\textbf{Unsatisfactory:} Does not include a description of your ideal learning environment. Does not reference scholarly sources. Does note analyze key elements of an authentic learning community. Does not mention or describe the learning community environment, student learning outcomes, learning outcomes and personal beliefs about teaching and learning.

\textbf{Developing:} Includes a partial description of your ideal learning environment, referencing few scholarly sources and including a partial analysis of key elements of an authentic learning community. Mentions some elements, but does not fully describe the learning community environment, student learning outcomes, learning outcomes and personal beliefs about teaching and learning.

\textbf{Proficient:} \emph{Includes a detailed description of your ideal learning environment, referencing scholarly sources and analyzing key elements of an authentic learning community. Describes the learning community environment, student learning outcomes, learning outcomes and personal beliefs about teaching and learning.}

\textbf{Exemplary:} Includes a detailed description of your ideal learning environment, referencing scholarly sources and analyzing key elements of authentic learning communities. Provides a rationale for key elements of the learning community environment, student learning outcomes, learning outcomes and personal beliefs about teaching and learning. Advances scholarly thinking about authentic learning communities.

\hypertarget{your-role-as-teacher-and-leaders}{%
\subsubsection*{YOUR ROLE AS TEACHER AND LEADERS}\label{your-role-as-teacher-and-leaders}}
\addcontentsline{toc}{subsubsection}{YOUR ROLE AS TEACHER AND LEADERS}

\textbf{Unsatisfactory:} Does not include a description of your role as a teacher or leader within an authentic learning community, incorporating scholarly literature. Does not include an analysis of your role as a facilitator/leader in planning learning experiences, facilitating student learning, and assessing student learning. Does not include a description of the actions, practices, and strategies you will engage in to achieve your vision of the learning community you described in section one.

\textbf{Developing:} Includes a partial description of your role as a teacher or leader within an authentic learning community, incorporating scholarly literature. Describes but does not analyze your role as a facilitator/leader in planning learning experiences, facilitating student learning, and assessing student learning. Lists but does not describe the actions, practices, and strategies you will engage in to achieve your vision of the learning community you described in section one.

\textbf{Proficient:} \emph{Includes a detailed description of your role as a teacher or leader within an authentic learning community, incorporating scholarly literature. Includes a detailed analysis of your role as a facilitator/leader in planning learning experiences, facilitating student learning, and assessing student learning. Includes a detailed description of the actions, practices, and strategies you will engage in to achieve your vision of the learning community you described in section one.}

\textbf{Exemplary:} Includes a detailed analysis of your role as a teacher or leader within an authentic learning community, incorporating scholarly literature. Includes a detailed analysis of your role as a facilitator/leader in planning learning experiences, facilitating student learning, and assessing student learning. Includes a detailed description of the actions, practices, and strategies you will engage in to achieve your vision of the learning community you described in section one. Synthesizes scholarly thinking about the role of the teacher/leader.

\hypertarget{scholarly-integration-2}{%
\subsubsection*{SCHOLARLY INTEGRATION}\label{scholarly-integration-2}}
\addcontentsline{toc}{subsubsection}{SCHOLARLY INTEGRATION}

\textbf{Unsatisfactory:} Does not integrate many references to support the arguments made in the paper.

\textbf{Developing:} Integrates fewer than 10 scholarly sources to support arguments made in the paper.

\textbf{Proficient:} \emph{Integrates a minimum of 10 scholarly sources to support arguments made in each section of the paper.}

\textbf{Exemplary:} Integrates a minimum of 10 references to support the arguments made in each section, including several scholarly sources not included in course materials.

\begin{longtable}[]{@{}
  >{\raggedright\arraybackslash}p{(\columnwidth - 8\tabcolsep) * \real{0.2000}}
  >{\raggedright\arraybackslash}p{(\columnwidth - 8\tabcolsep) * \real{0.2000}}
  >{\raggedright\arraybackslash}p{(\columnwidth - 8\tabcolsep) * \real{0.2000}}
  >{\raggedright\arraybackslash}p{(\columnwidth - 8\tabcolsep) * \real{0.2000}}
  >{\raggedright\arraybackslash}p{(\columnwidth - 8\tabcolsep) * \real{0.2000}}@{}}
\toprule\noalign{}
\begin{minipage}[b]{\linewidth}\raggedright
\textbf{TOTAL}
\end{minipage} & \begin{minipage}[b]{\linewidth}\raggedright
\textbf{0 = 0\% (F)}
\end{minipage} & \begin{minipage}[b]{\linewidth}\raggedright
\textbf{10 = 50\% (C)}
\end{minipage} & \begin{minipage}[b]{\linewidth}\raggedright
\textbf{15 = 75 (B)}
\end{minipage} & \begin{minipage}[b]{\linewidth}\raggedright
\textbf{20 = 100\% (A+)}
\end{minipage} \\
\midrule\noalign{}
\endhead
\bottomrule\noalign{}
\endlastfoot
\end{longtable}

\hypertarget{references}{%
\chapter*{References}\label{references}}
\addcontentsline{toc}{chapter}{References}

The following are key references used in this course. \textbf{\emph{Check with your course syllabus for required readings.}}

  \bibliography{book.bib}

\end{document}

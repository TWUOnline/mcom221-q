% Options for packages loaded elsewhere
\PassOptionsToPackage{unicode}{hyperref}
\PassOptionsToPackage{hyphens}{url}
%
\documentclass[
]{book}
\usepackage{amsmath,amssymb}
\usepackage{iftex}
\ifPDFTeX
  \usepackage[T1]{fontenc}
  \usepackage[utf8]{inputenc}
  \usepackage{textcomp} % provide euro and other symbols
\else % if luatex or xetex
  \usepackage{unicode-math} % this also loads fontspec
  \defaultfontfeatures{Scale=MatchLowercase}
  \defaultfontfeatures[\rmfamily]{Ligatures=TeX,Scale=1}
\fi
\usepackage{lmodern}
\ifPDFTeX\else
  % xetex/luatex font selection
\fi
% Use upquote if available, for straight quotes in verbatim environments
\IfFileExists{upquote.sty}{\usepackage{upquote}}{}
\IfFileExists{microtype.sty}{% use microtype if available
  \usepackage[]{microtype}
  \UseMicrotypeSet[protrusion]{basicmath} % disable protrusion for tt fonts
}{}
\makeatletter
\@ifundefined{KOMAClassName}{% if non-KOMA class
  \IfFileExists{parskip.sty}{%
    \usepackage{parskip}
  }{% else
    \setlength{\parindent}{0pt}
    \setlength{\parskip}{6pt plus 2pt minus 1pt}}
}{% if KOMA class
  \KOMAoptions{parskip=half}}
\makeatother
\usepackage{xcolor}
\usepackage{longtable,booktabs,array}
\usepackage{calc} % for calculating minipage widths
% Correct order of tables after \paragraph or \subparagraph
\usepackage{etoolbox}
\makeatletter
\patchcmd\longtable{\par}{\if@noskipsec\mbox{}\fi\par}{}{}
\makeatother
% Allow footnotes in longtable head/foot
\IfFileExists{footnotehyper.sty}{\usepackage{footnotehyper}}{\usepackage{footnote}}
\makesavenoteenv{longtable}
\usepackage{graphicx}
\makeatletter
\def\maxwidth{\ifdim\Gin@nat@width>\linewidth\linewidth\else\Gin@nat@width\fi}
\def\maxheight{\ifdim\Gin@nat@height>\textheight\textheight\else\Gin@nat@height\fi}
\makeatother
% Scale images if necessary, so that they will not overflow the page
% margins by default, and it is still possible to overwrite the defaults
% using explicit options in \includegraphics[width, height, ...]{}
\setkeys{Gin}{width=\maxwidth,height=\maxheight,keepaspectratio}
% Set default figure placement to htbp
\makeatletter
\def\fps@figure{htbp}
\makeatother
\usepackage{soul}
\setlength{\emergencystretch}{3em} % prevent overfull lines
\providecommand{\tightlist}{%
  \setlength{\itemsep}{0pt}\setlength{\parskip}{0pt}}
\setcounter{secnumdepth}{5}
\usepackage{booktabs}
\usepackage{amsthm}
\makeatletter
\def\thm@space@setup{%
  \thm@preskip=8pt plus 2pt minus 4pt
  \thm@postskip=\thm@preskip
}
\makeatother

\usepackage{tcolorbox}


\newtcolorbox{blackbox}{
  colback=black,
  coltext=white,
  colframe=black,
  boxsep=5pt,
  arc=4pt}
\newtcolorbox{bonus}{
  colback=blue!15,
  colframe=blue!15,
  coltext=black!80,
  boxsep=5pt,
  arc=4pt}
\newtcolorbox{reflect}{
  colback=green!5,
  colframe=green!5,
  coltext=black!80,
  boxsep=5pt,
  arc=4pt}
\newtcolorbox{assessment}{
  colback=blue!5,
  colframe=blue!5,
  coltext=black!80,
  boxsep=5pt,
  arc=4pt}
\newtcolorbox{progress}{
  colback=purple!10,
  colframe=purple!10,
  coltext=black!80,
  boxsep=5pt,
  arc=4pt}
\newtcolorbox{video}{
  colback=yellow!5,
  colframe=yellow!5,
  coltext=black!80,
  boxsep=5pt,
  arc=4pt}
\newtcolorbox{caution}{
  colback=red!5,
  colframe=red!5,
  coltext=black!80,
  boxsep=5pt,
  arc=4pt}
\newtcolorbox{feedback}{
  colback=black!5,
  colframe=black!5,
  coltext=black!80,
  boxsep=5pt,
  arc=4pt}
\ifLuaTeX
  \usepackage{selnolig}  % disable illegal ligatures
\fi
\usepackage[]{natbib}
\bibliographystyle{apalike}
\IfFileExists{bookmark.sty}{\usepackage{bookmark}}{\usepackage{hyperref}}
\IfFileExists{xurl.sty}{\usepackage{xurl}}{} % add URL line breaks if available
\urlstyle{same}
\hypersetup{
  pdftitle={Digital Filmmaking},
  pdfauthor={Ned Vankevich},
  hidelinks,
  pdfcreator={LaTeX via pandoc}}

\title{Digital Filmmaking}
\author{Ned Vankevich}
\date{Last Updated Jun 2023}

\begin{document}
\maketitle

{
\setcounter{tocdepth}{1}
\tableofcontents
}
\hypertarget{welcome}{%
\chapter*{Welcome}\label{welcome}}
\addcontentsline{toc}{chapter}{Welcome}

This is the course book for {[}insert{]}. This book is divided into 6 units of study to help you engage with the materials. The course resources and learning activities are designed not only to help prepare you for the course assessments, but also to give you opportinities to practice various skills.

Below you will find information about how to navigate this book. Please also refer the schedule in Moodle, as well as the Asseessment section in Moodle for instructions on required readings and assignments.

\hypertarget{course-notes}{%
\section*{Course Notes}\label{course-notes}}
\addcontentsline{toc}{section}{Course Notes}

You should be reading this information in the context of a Trinity Western University course offered via Moodle. If this is not the case, then this may be an unauthorized reproduction of the course. Please contact \href{mailto:elearning@twu.ca}{\nolinkurl{elearning@twu.ca}} if you have concerns.

These notes will be your guide through the learning activities and assessment strategies necessary for you to succeed in the course, so it is important for you to engage to the best of your ability and take advantage of the resources available to you through Trinity Western University.

Assessment tasks are managed in other sections of the Moodle course, so be sure to familiarize yourself with those requirements and resources.

\hypertarget{how-this-course-is-built}{%
\section*{How this Course is Built}\label{how-this-course-is-built}}
\addcontentsline{toc}{section}{How this Course is Built}

This course is primarily designed to be completed asynchronously, meaning that there are no scheduled times or places that you are required to meet, even online. You can work according to your own schedule \emph{within the six weeks you have to complete the course}. That said, this is a full university level course and there are timelines that we strongly recommend that you meet to ensure that you are succeeding in building your knowledge through the course.

It would be to your significant disadvantage to submit everything at the end of the course.

Asynchronous courses require learners to be well-organized and self-motivated, and we have included supports for you to help you develop strong learning habits that will ensure your success.

For example, there are several self-check quizzes throughout the course. These quizzes are not graded, but they can be powerful tools for you to ensure you understand key ideas and concepts. We suggest you take each quiz without the aid of your notes and textbook and multiple times until you have mastered the content. This strategy taps into three powerful learning structures that have been shown to be highly effective.

\begin{enumerate}
\def\labelenumi{\arabic{enumi}.}
\tightlist
\item
  \textbf{Effortful recall.} By intentionally trying to recall information without external aids, you are strengthening the neural pathways in your brain that lead to building new connections between ideas. One way to make recall easier is to connect key ideas to other things that you know or have experienced. For example, you might be studying World War II, and you connect the date that Canadians participated in the D-Day operation with something else meaningful to you that happened on June 6, like maybe the date you bought your first car.
\item
  \textbf{Spaced repetition.} By spreading out your attempts on the quiz (leaving a few days between attempts) you can maximize the effects of the first strategy (effortful recall) and ensure that your second or third attempts truly reflect what you know about the topic. We suggest leaving 1-3 days between attempt 1 and 2, then 4-5 days between attempt 2 and 3. You can use a tool like Trello, Notion, or Asana (free versions), or even a task list on your phone to set up a spaced repetition schedule.
\item
  \textbf{Interleaving.} This is the practice of studying a particular topic for a relatively short period of time (maybe 30-40 mins), then switching to a different topic for the same period, before going back to the original topic. We will help build this into your learning by including items from unit 1 in your unit 2-6 quizzes. You can also practice this by taking regular breaks in your work, or even by retaking a unit 1 quiz while you are working in unit 2.
\end{enumerate}

These three strategies are very effective at helping people \emph{remember} key facts about a particular topic, an important first step in learning at the university level. However, you will be asked to do much more than just remember facts. Your ultimate goal is to develop \textbf{evaluative judgement}, or the ability for you to judge for yourself the quality of your (or your peers') responses to prompts.

The discussion forums are a key way for you to do this. We have set up the forums in such a way that you will need to present a response to any given prompt before you see other learners' responses. We strongly encourage you to use this structure to formulate your own ideas before you present them in the forum, and then to use the responses of your peers to help you evaluate your own response.

Using these self-check activities in this way is designed to help you to succeed on the course assignments, upon which your final grade will be determined. These assignments will require you to \textbf{use} the facts of the course to generate unique responses to the prompts, based on your past experiences, knowledge, and ability to evaluate the quality of your own work.

\hypertarget{how-to-navigate-this-book}{%
\subsection*{How To Navigate This Book}\label{how-to-navigate-this-book}}
\addcontentsline{toc}{subsection}{How To Navigate This Book}

To move quickly to different portions of the book, click on the appropriate chapter or section in the table of contents on the left. The buttons at the top of the page allow you to show/hide the table of contents, search the book, change font settings, download a pdf or ebook copy of this book, or get hints on various sections of the book.
\includegraphics{assets/course-intro/menu.png}

The faint left and right arrows at the sides of each page (or bottom of the page if it's narrow enough) allow you to step to the next/previous section. Here's what they look like:
\includegraphics{assets/course-intro/left_arrow.png} \includegraphics{assets/course-intro/right_arrow.png}

You can also download an offline copy of this book in various formats, such as pdf or an ebook. If you are having any accessibility or navigation issues with this book, please reach out to your instructor or our online team at \href{mailto:elearning@twu.ca}{\nolinkurl{elearning@twu.ca}}.

\hypertarget{course-units}{%
\subsection*{Course Units}\label{course-units}}
\addcontentsline{toc}{subsection}{Course Units}

This course is organized into 6 units. Each unit of the course will provide you with the following information:

\begin{itemize}
\tightlist
\item
  A general overview of the key concepts that will be addressed during the unit.\\
\item
  Specific learning outcomes and topics for the unit.\\
\item
  Learning activities to help you engage with the concepts. These often include key readings, videos, and reflective prompts.\\
\item
  The Assessment section provides details on assignments you will need to complete throughout the course to demonstrate your understanding of the course learning outcomes.
\end{itemize}

\begin{caution}
Note that assessments, including assignments and discussion posts will be submitted in Moodle. See the Assessment tab in Moodle for the assignment dropboxes.
\end{caution}

\hypertarget{course-activities}{%
\subsection*{Course Activities}\label{course-activities}}
\addcontentsline{toc}{subsection}{Course Activities}

Below is some key information on features you will see throughout the course.~

\begin{reflect}
\textbf{\emph{Learning Activity}}\\
This box will prompt you to engage in course concepts, often by viewing resources and reflecting on your experience and/or learning. Most learning activities are ungraded and are designed to help prepare you for the assessment in this course.
\end{reflect}

\begin{assessment}
\textbf{\emph{Assessment}}\\
This box will signify an assignment or discussion post you will submit in Moodle. Note that these demonstrate your understanding of the course learning outcomes. Be sure to review the grading rubrics for each assignment.
\end{assessment}

\begin{progress}
\textbf{\emph{Checking Your Learning}}\\
This box is for checking your understanding, to make sure you are ready for what follows. Ways to check your learning might include self-check quizzes or questions for discussion. These activities are not graded but are critical for you to be able to begin to develop evaluative judgement in this domain of knowledge.
\end{progress}

\begin{caution}
\textbf{\emph{Note}}\\
This box signifies key notes. It may also warn you of possible problems or pitfalls you may encounter!
\end{caution}

\hypertarget{course-communities}{%
\chapter*{Course Communities}\label{course-communities}}
\addcontentsline{toc}{chapter}{Course Communities}

As you begin this course, how will you build community with your fellow learners?

In this course, we have the following tools available to help foster community in your course, including other students who have previously taken this course. Some of these tools will be prescribed and graded (e.g.~Moodle Discussion Forums), others will be up to you to take advantage of.

Check with your course syllabus for which community tools will be used, and consider building your own Community of Practice with your classmates and external colleagues.

\hypertarget{communication-tools}{%
\section*{Communication Tools}\label{communication-tools}}
\addcontentsline{toc}{section}{Communication Tools}

\textbf{Moodle Discussion Forums}: In this course, we ask you to discuss ideas with your colleagues, challenging one another and analyzing key course resources. Refer to the course syllabus for assessment details, as well as the unit Assessment section for discussion questions. Submit your responses in Moodle.

\textbf{Video Conferencing}: We will have scheduled online meetings (Zoom or Teams). Take advantage of these face-to-face conferences! Come prepared with your questions and assigned activities. Refer to the course syllabus and unit activity instructions for details.

\emph{Optional:}

Your cohort may want to engage in other informal discussions to build community and support each other. Consider using the following:

\textbf{Learning Cafe:} This discussion forum in Moodle is a place for you to interact about things going on, share resources, and generally get to know one another. Your posts don't have to be course related. Take this opportunity to connect with fellow learners and learn from one another!

\textbf{Teams:} Every TWU course has a Teams channel, mostly to manage videos. Feel free to use the messaging feature to connect with peers.

\textbf{Twitter hashtag \texttt{\#CRSE\#\#\#}:} You can tweet about this course using \texttt{\#CRSE\#\#\#}.

\textbf{What's App:} Feel free to use a platform that works for you!! What's App is a popular chat forum that learners use for discussions, class projects, etc.

A key takeaway\ldots make these forums work for you! Interact with your peers, learn from each other, and make connections that will stay with you beyond this course.

With that, let's begin the journey together!

\includegraphics{assets/community/luke-porter-NEqEC7qa9FM-unsplash.jpg}

\hypertarget{project-excellence-and-professionalism}{%
\chapter{Project Excellence and Professionalism}\label{project-excellence-and-professionalism}}

\hypertarget{overview}{%
\section*{Overview}\label{overview}}
\addcontentsline{toc}{section}{Overview}

Welcome to MCOM 221! You are about to embark on a creative adventure. The specific goal of this journey is to create attention-grabbing and alluring short films. However, the general goal is far more valuable as you learn things about yourself you never knew. This course will equip you with the knowledge to develop transferable life skills that will help you in your career and professional life after you complete your education.

In addition, there is a larger element of this course that connects us to people from the past, present, and future. Namely, storytelling.

According to communications theorist Walter Fischer, human beings are more than \emph{Homo sapiens}, we are Homo \emph{narrans}, man the storyteller. Storytelling is hardwired into our brains and our cultures and history. Stories are how we make sense of our life and our world. This is why religion, philosophy, literature, and myth have been vital to human culture and understanding. Even science is a story---it tells us how the world works. But it cannot tell us why. Only story-based meaning can reveal why life is important, how we should live our lives, and why there is something rather than nothing. Think of the Bible. If one were to take out the stories, it would be a thin book.

Don't let these ideas scare you. This is a creative course not a philosophic one. The point here is that as you increase your storytelling skills---a vital part of this course---you will join the community of narrators throughout human history who have shared their wisdom, insight, and understanding. This knowledge serves as the foundation for building identities, developing traditions, and discovering the types of meaning that make life worth living across all cultures.

This course will focus on one aspect of narratives, namely visual storytelling. How do we use pictures (and sounds) to create meaning and emotionally connect people to ourselves and each other? This will be the deeper theme operating beneath the surface of this course. (In film terms it will be the subtext beneath the text.)

In practical terms, the better you are at communicating stories (relative to each profession), the better you will be at your job. Fasten your seat belt, you are about to find out why we have focused on the above to start this course.

\hypertarget{topics}{%
\subsection*{Topics}\label{topics}}
\addcontentsline{toc}{subsection}{Topics}

This unit is divided into the following topics:

\begin{enumerate}
\def\labelenumi{\arabic{enumi}.}
\tightlist
\item
  Course Introduction\\
\item
  Excellence and Professional and Personal Development\\
\item
  15 Tips for Doing Your Best\\
\item
  Transferable Skills
\end{enumerate}

\hypertarget{learning-outcomes}{%
\subsection*{Learning Outcomes}\label{learning-outcomes}}
\addcontentsline{toc}{subsection}{Learning Outcomes}

When you have completed this unit, you should be able to:

\begin{itemize}
\tightlist
\item
  Describe excellence and why it is important\\
\item
  Define what constitutes an excellent film project\\
\item
  Articulate the big picture of why stories and creativity are important\\
\item
  Determine potential transferable life skills\\
\item
  Self-assess your strengths and weaknesses\\
\item
  Plan what you want to focus on during the course in terms of your professional and personal growth.
\end{itemize}

\hypertarget{activity-checklist}{%
\subsection*{Activity Checklist}\label{activity-checklist}}
\addcontentsline{toc}{subsection}{Activity Checklist}

\begin{reflect}
Here is a checklist of learning activities you will benefit from in completing this unit. You may find it useful for planning your work.

\begin{itemize}
\tightlist
\item
  Write your first two Film Journal entries for this unit.\\
\item
  Choose a glossary and start learning 5 new film terms a day.\\
\item
  Read the Introduction which sets up the course ahead and Chapter One ``Project Excellence and Professionalism'' in the course textbook.\\
\item
  Reflect on the 15 Tips for Doing Your Best.\\
\item
  Reflection on the transferable life skills presented.
\end{itemize}

\textbf{Assessment}

\begin{itemize}
\tightlist
\item
  Film Journal
\end{itemize}
\end{reflect}

\hypertarget{resources}{%
\subsection*{Resources}\label{resources}}
\addcontentsline{toc}{subsection}{Resources}

Here are the resources you will need to complete this unit.

\begin{itemize}
\tightlist
\item
  Introduction and Chapter One of the course text: \emph{Digital Filmmaking: A Beginner's Guide to Mastering the Craft}, by Ned Vankevich (e-text)\\
\item
  \href{https://karenbanes.com/how-to-start-a-creative-journal}{How to Start a Creative Journal} (Be sure to click on the internal links on this website)\\
\item
  \href{https://www.youtube.com/watch?v=hUTWo7_W0lc}{How to Journal Every Day for Increased Productivity, Clarity, and Mental Health}\\
\item
  \href{https://brandyourself.com/blog/guide/how-to-grow-professionally/}{How To Grow Professionally}\\
\item
  \href{https://www.goodtherapy.org/learn-about-therapy/issues/creative-blocks}{Creative blocks website}\\
\item
  Other resources will be provided in the unit.
\end{itemize}

\hypertarget{course-introduction}{%
\section{Course Introduction}\label{course-introduction}}

We begin Unit 1 marvelling at the magic of film and cinema. (Read the course text Preface.) The ability of visually-centered storytelling to cross cultures and to captivate, educate, and entertain audiences is a universal contemporary phenomenon. For instance, how a series of flickering lights and images projected at 24, 25, and 30 frames a second can create emotions and experiences we all share is a wondrous mystery.

Less mysterious is how this is done. There are techniques, rules, guidelines, and practices that can help us reach people in the ways that films, TV shows, streaming Internet programs, news, and the host of other visually-centered can make us laugh, cry, and emotionally move and engage us. The worst thing that can be said by an audience is that ``I want my five, ten, sixty, or ninety minutes back (depending on the length of what we have viewed). Learning the methods to avoid this and to engage our audience will be central to this course.

\textbf{WHAT LIES AHEAD}

This course will be divided up into ten segments or units. Each segment (or chapter) is self-contained but each section is inter-connected and vital for the others. A word to the wise---do note not skip a section, even if you think that you already know it. Each section of the course will build upon the previous one.

See the overview of the course and the ten units in the Introduction to the course text: \emph{Introduction to Digital Filmmaking: A Beginner's Guide to Mastering the Craft}.

Of special note, Unit Ten and Chapter Ten will be a summary and a celebration of what you have learned during the course. It will also be a time for awards to be handed out for outstanding and excellent work such as Awards for Best film, Best Director, Best Cinematography, Best Editing., Best Story, and Most Imaginative and Most Improved Filmmaker, etc. This is noted here to give you a goal to work towards: namely, to win one of the top awards.

As can be gleaned from the above, this course will start with the most basic elements and proceed to creating a short film with a strong beginning, middle, and end. As you move through the course keep the mindset that this will be a fun, enjoyable, and exciting adventure.

\hypertarget{learning-activities}{%
\subsection*{Learning Activities}\label{learning-activities}}
\addcontentsline{toc}{subsection}{Learning Activities}

\begin{reflect}
\hypertarget{the-importance-of-the-film-journal}{%
\subsubsection*{The Importance of the Film Journal}\label{the-importance-of-the-film-journal}}
\addcontentsline{toc}{subsubsection}{The Importance of the Film Journal}

During this course you are encouraged to keep a ``Film Journal.'' This is important for many reasons, including helping you to brainstorm ideas, as well as ponder and process what you are learning. The journal will also help you to keep a record of your course activities, and note feedback of what does and does not work in film projects of your fellow course members. In addition, some of the exercises for this course will not be graded (for reasons that will be explained ahead) but this does not mean you will not be accountable for doing them. Your journal will be submitted at the end of the course and will play a role in the grade you receive. In short, students who engage with the course well will in general receive higher marks.

Also, note that you may be asked to use your journal entries to participate in discussions, presentations, and other learning activities in the FAR Centre Facilitated Learning Labs. Please check with your facilitator about specific due dates for activities, including journal responses.

\hypertarget{film-journal-entry}{%
\subsubsection*{Film Journal Entry:}\label{film-journal-entry}}
\addcontentsline{toc}{subsubsection}{Film Journal Entry:}

Based on the importance of the course journal make your first two entries for this unit:

\begin{itemize}
\tightlist
\item
  \textbf{Entry One:} After reading the course text Introduction, describe your best take-aways of what you have learned.
\item
  \textbf{Entry Two:} Study the recommended how-to-write-a-journal resources and log what you learned from them and how it will help you during the course.
\item
  \href{https://karenbanes.com/how-to-start-a-creative-journal}{How to Start a Creative Journal} (Be sure to click on the internal links on this website)\\
\item
  \href{https://www.youtube.com/watch?v=hUTWo7_W0lc}{How to Journal Every Day for Increased Productivity, Clarity, and Mental Health}
\end{itemize}
\end{reflect}

\begin{caution}
\textbf{NOTE} : your journal entries do not have to be long but they should be detailed and specific enough to indicate that you have engaged the course material and projects.
\end{caution}

\hypertarget{learning-activities-1}{%
\subsection*{Learning Activities}\label{learning-activities-1}}
\addcontentsline{toc}{subsection}{Learning Activities}

\begin{reflect}
\hypertarget{film-terms-and-vocabulary-needed-for-this-course}{%
\subsubsection*{Film Terms and Vocabulary Needed for this Course}\label{film-terms-and-vocabulary-needed-for-this-course}}
\addcontentsline{toc}{subsubsection}{Film Terms and Vocabulary Needed for this Course}

The course text Introduction also draws attention to the fact that you will learn many new terms and concepts during this course and that it is important for you to find resources that you can consult throughout the course to re-enforce their meaning and application to the course. Several websites will be cited here such as:

\begin{itemize}
\tightlist
\item
  studiobinder movie film terms \\
\item
  A more comprehensive glossary can be found at Wikipedia's Glossary of Motion Picture terms ;\\
\item
  and AMC Film Terms Glossary
\end{itemize}

Commit to learning \textbf{\emph{5}} new film terms each day by starting with A and going to Z. The more you focus on this the better you will do in the course. Don't worry if you do not understand a term. We will be covering many of them during the course. The important thing is to become familiar with important film terms and vocabulary.

\hypertarget{questions-to-consider}{%
\subsubsection*{Questions to Consider}\label{questions-to-consider}}
\addcontentsline{toc}{subsubsection}{Questions to Consider}

After completing the activities above, consider the following questions:

\begin{itemize}
\tightlist
\item
  Why is this course important to you?\\
\item
  What do you hope to learn and how can it aid your future?
\end{itemize}
\end{reflect}

\begin{caution}
\textbf{Note:} that while the above learning activity is ungraded, it is designed to help you succeed in your assessments in this course.
\end{caution}

\hypertarget{excellence-and-professional-and-personal-development}{%
\section{Excellence and Professional and Personal Development}\label{excellence-and-professional-and-personal-development}}

This course will focus on the classical approach to filmmaking, that is, making motivated film stories and executing them in ways that help audiences to fully engage the story and characters without unnecessary distractions. In doing so, the course has a lofty goal: to aim for excellence. Excellence means a high standard of being good. This is of course relative to being an Introductory course with students who are just beginning their filmmaking journey, or are taking this course as an elective. But like any course, if you shoot for an A you will do better than settling for average and mediocrity.

Excellence is important in filmmaking because our competition is great. Film is a public medium. It is designed to be shown to audiences and the quality of our projects, like many things we do in life, will be judged whether we want them to be or not.

One of the transferable life skills of this course is to help us be more professional. Thus the more we shoot for excellence, the more we will develop abilities that will serve us better in our careers ahead.

At the same time we must not confuse excellence with perfection. Our beginning films and projects will have lots of mistakes and things that do not work. This should not deter us from our goal of excellence. All great artists, athletes, business leaders, and a host of other professional make mistakes. The key is to learn from them and to keep improving our craft.

HELPFUL HINT: For insight into how great artists, athletes, and performers use mistakes to grow in their art and craft see the following book by Daniel Coyle: \href{http://danielcoyle.com/the-talent-code/}{THE TALENT CODE: GREATNESS ISN'T BORN. IT'S GROWN. HERE'S HOW}.

Our goal then is to learn how to seek to produce visually-driven, digital film stories that are technically and artistically proficient, meaningfully engaging, motivated, and fraught with excellence.

As you can see excellence and professionalism go hand in hand. In order to grow and excel in a craft we need to grow and excel at being professional. Implied here is that we also need to mature and grow personally as we develop the character traits necessary for being a mature, moral, and responsible person.

\hypertarget{learning-activities-2}{%
\subsection*{Learning Activities}\label{learning-activities-2}}
\addcontentsline{toc}{subsection}{Learning Activities}

\begin{reflect}
\hypertarget{read-and-reflect}{%
\subsubsection*{Read and Reflect}\label{read-and-reflect}}
\addcontentsline{toc}{subsubsection}{Read and Reflect}

Read Chapter One ``Project Excellence and Professionalism'' in the course textbook.

As you study and ponder what excellence and what professional and personal development mean, log in your journal why they are important and how you hope to focus on them during the course. Which character traits are your strongest and which do you need to work on? How will you do this? To help you with these entries study this resource:

\href{https://brandyourself.com/blog/guide/how-to-grow-professionally/}{How To Grow Professionally}
\end{reflect}

\hypertarget{tips-for-doing-your-best}{%
\section{15 Tips for Doing Your Best}\label{tips-for-doing-your-best}}

The following will help you to not just do well in this Film project, but will make it fun and more enjoyable and hopefully be one of your favourite courses ever:

\begin{enumerate}
\def\labelenumi{\arabic{enumi}.}
\item
  \textbf{\emph{Keep your eye on the prize.}} Creating an excellent project and growing as a professional and better person through the process.
\item
  \textbf{\emph{Meeting Deadlines.}} Being on time is vital in any profession, especially in filmmaking where the cost of feature film shoots can be thousands of dollars per hour. Plan your time and schedule accordingly so that you make your due date. Failure to do so will result in reduced marks on your assignment.
\item
  \textbf{\emph{Avoiding Distractions.}} In our Internet-cell phone era we are inundated and surrounded by 24/7 distractions. If we succumb to them we will not meet our deadlines and thus delay our professional and personal development. Learning to discern what is vital and what is the tyranny of the unimportant and unnecessary urgent is a critical skill to develop today. Focus on it and you will go further in the course. (\textbf{Tip:} \emph{turn off all notifications on your cell phone. In fact, turn off or put your cell phone away while you are working on your Film course. You will find that this will help you to focus on and finish your assignments quicker and with more creativity}).
\item
  \textbf{\emph{Focussing on the positive}}. Most of the assignments ahead will be challenging. Cultivate a ``can do attitude.'' For context think about your taking a piano lesson course. You would not expect to sit down and immediately play chords and songs. You would have to learn to play the notes, then chords, then songs over a period of time. Filmmaking is similar. There are notes to learn (film shots), chords to learn (film scenes), before we can create melodies (film sequences) and songs (full films).
\item
  \textbf{\emph{It takes time and practice to master film}}. If you find you have time, or if you are truly interested in learning the world of film or becoming a filmmaker, I encourage you to go beyond the course exercises. For example, continue to find and shoot interesting visual compositions (they are all around you) during the course, or practice fascinating and more complicated motion shots and two and three person shots. The more you do this the faster you will grow in your craft. Think about musicians. They practice over and over everyday.
\item
  \textbf{\emph{Study the practical tips and helpful hints in these units}}. Pay attention to them. They will help fast-track your skills growth.
\item
  \textbf{\emph{Find and do what you love}}. During this course you will discover things you love to do and those that you do not. As a filmmaker, every skill set is vital for the final production -- but you will discover which roles you are better at and enjoy the most. Even if you do not go into filmmaking as a career this self-insight will be invaluable as you make life decisions ahead.
\item
  \textbf{\emph{Make a firm commitment to succeed}}. When we are double-minded or not sure of what we want, we will waiver and fall prey to distraction and self-sabotage. Resolve not to give up as you make your projects despite what ``Murphy's Law'' tosses your way. (\emph{To be covered later.})
\item
  \textbf{\emph{Strive for balance}}. The emphasis on excellence and professional and personal development must not make us ``lose our soul.'' Too many people get caught up in goals and ambition and end up failing in their marriage and relationships. Remember we are physical, mental, spiritual, emotional, and relational beings and must find a balance that lets us grow in each part equally. This is the key to personal flourishing.
\item
  \textbf{\emph{Create strong and healthy relationships}}. During this course you will discover a lot about yourself and others. Some people will be easy to work with and some won't. Some will be diligent and some won't. Choose your teammates wisely and if you have a winning rapport with them, you might want to work with them on the next project. One of the keys to success in the professional world is cultivating strong and fruitful relationships. Start the process here.
\item
  \textbf{\emph{Follow your intuition}}. If you've never experimented with following your deeper instincts or ``gut feelings'' try it during this course. Often this will take you to new creative places. It might not always work but you will learn a lot from doing so.
\item
  \textbf{\emph{Take risks}}. We don't grow unless we try new things and fail. They key is not to get discouraged but to shake off a loss and fight to win again.
\item
  \textbf{\emph{Practice, practice, practice. Learn, learn, learn}}. Your best competition is continually practicing and learning their craft. So should you. You need to cultivate both a strong work ethic and a smart ethic. Take the time to think things through before launching into something.
\item
  \textbf{\emph{Take responsibility for yourself and your future}}. Playing the blame game or the victim or continually making excuses will not get you ahead in life.
\item
  \textbf{\emph{Find inspiring quotes and pin them to your computer or wall}}. Most of us need encouragement especially when the going gets tough. Seek out inspiration that helps keep you going. The following strike some of the themes of this course:
\end{enumerate}

\begin{itemize}
\item
  \emph{``Success is not final;~failure is not fatal: It is the courage to continue that counts.''} - Winston S. Churchill
\item
  \emph{``It is better to fail in originality than to succeed in imitation.''} - Herman Melville
\item
  \emph{``You will become clever through your mistakes.''} - German Proverb in ``The Talent Code''
\end{itemize}

\hypertarget{learning-activities-3}{%
\subsection*{Learning Activities}\label{learning-activities-3}}
\addcontentsline{toc}{subsection}{Learning Activities}

\begin{reflect}
\hypertarget{reflection}{%
\subsubsection*{Reflection}\label{reflection}}
\addcontentsline{toc}{subsubsection}{Reflection}

Which of the tips in this section appeal to you most? Which do you need to focus on during the course? Write your response in your journal and share with your facilitator and/or classmates which are most important and why. Find an inspiring quote and share it with your peers.
\end{reflect}

\hypertarget{transferable-skills}{%
\section{Transferable Skills}\label{transferable-skills}}

This class is about more than making films. If followed diligently, you will learn a lot about yourself, others, and the world around you. You will also learn skills and traits that will play an important role in your life journey ahead. Some of the transferable skills can include:

\begin{itemize}
\tightlist
\item
  Growing in your creative skills\\
\item
  Growing in your teamwork skills\\
\item
  Growing in your critiquing skills\\
\item
  Growing in your ability to overcome obstacles\\
\item
  Meeting deadlines and taking responsibility\\
\item
  Becoming a leader\\
\item
  Learning to take constructive criticism\\
\item
  Committing to continual learning and improvement\\
\item
  Developing a positive, can-do attitude\\
\item
  Crafting what you want to say and communicating it well\\
\item
  Learning the power of story and how it impacts many areas in our lives.
\end{itemize}

\hypertarget{learning-activities-4}{%
\subsection*{Learning Activities}\label{learning-activities-4}}
\addcontentsline{toc}{subsection}{Learning Activities}

\begin{reflect}
\hypertarget{reflection-on-life-skills}{%
\subsubsection*{Reflection on Life Skills}\label{reflection-on-life-skills}}
\addcontentsline{toc}{subsubsection}{Reflection on Life Skills}

Which of the life skills above are important to you and why? Log this in your journal. Which life skill(s) can you add to the list? Share your responses with your Facilitator and classmates.
\end{reflect}

\begin{caution}
\textbf{NOTE:} Be sure to revisit these journal entries on Tips to Do Well and Transferable Life Skills as you monitor your progress during the course. One of the actives in Unit Ten will involve your writing a summary of the best lessons you learned from the course and about yourself. Again, this is a reminder of how important our course journal will be.
\end{caution}

\hypertarget{summary}{%
\section*{Summary}\label{summary}}
\addcontentsline{toc}{section}{Summary}

In this first unit, you have had the opportunity to learn about what this course entails, why it is important, and the role that professional and personal development will play. You also had a chance to respond to tips of how to make the most of this course and what skills you can transfer from it to your life.

\hypertarget{assessment}{%
\section*{Assessment}\label{assessment}}
\addcontentsline{toc}{section}{Assessment}

\begin{assessment}
\hypertarget{film-journal}{%
\subsection*{Film Journal}\label{film-journal}}
\addcontentsline{toc}{subsection}{Film Journal}

After completing this unit, including the learning activities, you are asked to make sure you are doing journal entries and when appropriate to share your responses with your facilitator and classmates when you meet.

Note that entries are expected for every unit. Your journal reflections are submitted at the end of the course as part of the Final Exam: Self Assessment worth 30\% of your course grade.

\hypertarget{grading-criteria}{%
\subsubsection*{Grading Criteria:}\label{grading-criteria}}
\addcontentsline{toc}{subsubsection}{Grading Criteria:}

See the Assessments section for more details on submitting your journal, as well as the grading criteria.
\end{assessment}

\hypertarget{checking-your-learning}{%
\section*{Checking your Learning}\label{checking-your-learning}}
\addcontentsline{toc}{section}{Checking your Learning}

\begin{progress}
Before you move on to the next unit, you may want to check to make sure that you are able to:

\begin{itemize}
\tightlist
\item
  Describe excellence and why it is important.\\
\item
  Define what constitutes an excellent film project.\\
\item
  Articulate the big picture of why stories and creativity are important.\\
\item
  Determine potential transferable life skills.\\
\item
  Self-assessing your strengths and weaknesses.\\
\item
  Plan what you want to focus on during the course in terms of professional and personal growth.
\end{itemize}
\end{progress}

\hypertarget{the-filmmaking-process}{%
\chapter{The Filmmaking Process}\label{the-filmmaking-process}}

\hypertarget{overview-1}{%
\section*{Overview}\label{overview-1}}
\addcontentsline{toc}{section}{Overview}

Filmmaking is exciting. We get to create and challenge ourselves as we make projects that can wow and inspire people. It is also a challenging art and craft. Film as a medium incorporates many other arts such as acting (like theatre); sets and production design (like painting and architecture); rhythm and timing (like music); scripts (like literature); movement (like dance), plasticity of form (like sculpture); as well as its unique combination of these many other arts.

Given the many skills needed to make a great film over time, filmmakers have devised a system to make it easier for filmmakers. This process is so well honed that some filmmakers, like experimental filmmakers, can do it all: produce, direct, act, shoot, and edit their film. However, the larger the film production and the bigger the budget, the more complicated the process becomes and the more people need to do it.

Like in a story that has a beginning, middle and end, the filmmaking process is divided into three sections or segments: pre-production, production, and post-production. (There is a fourth component dealing with marketing and distribution, but this will not be covered in this course.) We start with an overall look at the filmmaking process. This will give us the big picture of what will lie ahead so that we do not get lost in the details of the many elements of filmmaking, and so that we can place what we will be doing and practicing in a larger framework.

\hypertarget{topics-1}{%
\subsection*{Topics}\label{topics-1}}
\addcontentsline{toc}{subsection}{Topics}

This unit is divided into the following topics:

\begin{enumerate}
\def\labelenumi{\arabic{enumi}.}
\tightlist
\item
  The Three Phases of Filmmaking\\
\item
  Pre-Production\\
\item
  Production\\
\item
  Post-Production
\end{enumerate}

\hypertarget{learning-outcomes-1}{%
\subsection*{Learning Outcomes}\label{learning-outcomes-1}}
\addcontentsline{toc}{subsection}{Learning Outcomes}

When you have completed this unit, you should be able to:

\begin{itemize}
\item
  Describe the big picture of the film production process
\item
  Determine what is involved in each phase and why it is important
\item
  Capture an overview of the processes you will do during the course
\item
  Define why excellence should be the benchmark of each phase of filmmaking.
\end{itemize}

\hypertarget{activity-checklist-1}{%
\subsection*{Activity Checklist}\label{activity-checklist-1}}
\addcontentsline{toc}{subsection}{Activity Checklist}

Here is a checklist of learning activities you will benefit from in completing this unit. You may find it useful for planning your work.

\begin{itemize}
\tightlist
\item
  View the video resource: {[}Shooting a Film- START to FINISH!{]}
\end{itemize}

\begin{itemize}
\tightlist
\item
  Read the Pre-Production section of the course text.
\item
  Next, watch
\end{itemize}

{[}Pitching and Pre-Production: Crash Course Film Production.{]}

\begin{itemize}
\item
  Read the Production section of the course text. Then, view the video on How to Shoot a Scene.
\item
  Share in your journal and with your Facilitator and classmates why knowing the overall process is important and what you learned most from this unit.
\end{itemize}

\textbf{Assessment} : Film Journal

```

\hypertarget{resources-1}{%
\subsection*{Resources}\label{resources-1}}
\addcontentsline{toc}{subsection}{Resources}

Here are the resources you will need to complete this unit.

\begin{itemize}
\tightlist
\item
  Chapter Two of Digital Filmmaking: A Beginner's Guide to Mastering the Craft, by Ned Vankevich (e-text)\\
\item
  \href{https://www.youtube.com/watch?v=JE53JL60ihc}{\textbf{Pitching and Pre-Production: Crash Course Film Production \#2}}\\
\item
  \href{https://www.youtube.com/watch?v=8NCLf9rF6IQ}{\textbf{Shooting a film - START to FINISH!}}\\
\item
  \href{https://www.youtube.com/watch?v=oj_Blr8JE1I}{\textbf{Filmmaking From Beginning to End: Preproduction to Production}} (Don't worry if you do not understand everything in these tutorials. You will at the right time as you follow this course.)\\
\item
  \href{https://www.youtube.com/watch?v=y9_LW5H2EC4}{\textbf{How to Shoot a Scene - Blocking Actors}}\\
\item
  Other resources will be provided in the unit.
\end{itemize}

\hypertarget{the-three-phases-of-filmmaking}{%
\section{The Three Phases of Filmmaking}\label{the-three-phases-of-filmmaking}}

In Unit 1 we focused on project excellence and great story telling and why they are important. This unit will help obtain these goals by breaking the filmmaking process down into three phases. Each phase depends upon the other and if we if fail to understand and execute each one well, the other phases and the film itself will suffer.

The overall filmmaking process has not changed much over the past century. In short, it involves three phases: Pre-Production---preparing to shoot a project; Production---shooting or filming the project; Post-Production---editing or putting all of the production elements together in a unified final project.

Understanding these three phases is vital to making a great film. You will not have to master each phase, no one can because there are too many elements and skills involved depending on the complexity of the film, its story, and its execution.

\hypertarget{learning-activities-5}{%
\subsection*{Learning Activities}\label{learning-activities-5}}
\addcontentsline{toc}{subsection}{Learning Activities}

\begin{reflect}
\hypertarget{shooting-a-film}{%
\subsubsection*{Shooting a Film}\label{shooting-a-film}}
\addcontentsline{toc}{subsubsection}{Shooting a Film}

To get a visual representation of the overall filmmaking process view the following resource:

\href{https://www.youtube.com/watch?v=8NCLf9rF6IQ}{plugin:youtube}

Don't worry if you do not know all of the terms he is using such as blocking, dolly, tilt up, etc. These terms will be covered in the units ahead. For now, just take in the whole process and you will be amazed how much easier it will be to understand the later units as you explore the details of each phase.

For now, just watch the video and enjoy the process.

Helpful Hint: Don't multi-task during the viewing of resources. It will divide your attention and you will not absorb as much content. As with most things in this course---be in the moment and focus on what is at hand.
\end{reflect}

\begin{caution}
\textbf{Note} that while this learning activity is ungraded, it is designed to help you succeed in your assessments in this course.
\end{caution}

\hypertarget{pre-production}{%
\section*{Pre-Production}\label{pre-production}}
\addcontentsline{toc}{section}{Pre-Production}

Like most things in life, if you do not have a strong foundation, what follows after will wobble. Proper pre-production is the foundation of an excellent film (and your up-coming exercises). The better you plan something, the better the result.

There are many elements to pre-production. When followed properly they enhance the 5 Cs of successful filmmaking (see the \textbf{Why Pre-Production is Important} section in Chapter 2 of the course text):

\begin{itemize}
\tightlist
\item
  Creativity\\
\item
  Calmness\\
\item
  Co-ordination\\
\item
  Coherence\\
\item
  Consistency
\end{itemize}

Proper pre-production involves many important general steps (see \textbf{What Proper Pre-Production Entails} section in Chapter 2 of the course text).

\begin{itemize}
\tightlist
\item
  Planning\\
\item
  Anticipation\\
\item
  Strategizing\\
\item
  Testing
\end{itemize}

As can be gleaned from above, pre-production helps ensure that a film project maximizes the creative process and minimizes chaos, confusion, and uncertainty---the enemies of a good film.

There are many steps in the prep-production process highlighted in Chapter 2. (See \textbf{Successful Steps of Pre-Production}.)

They involve:

\begin{itemize}
\item
  Creating a viable concept for a project
\item
  Knowing who the audience is and genre requirements (people who like comedy want to laugh, people who like horror want to be scared)
\item
  Creating a script to film
\item
  Creating storyboards and shot lists
\item
  Pitching your project to get funding, actors, crew members, etc.
\item
  Budgeting the cost of the film
\item
  Getting your actors and the right crew
\item
  Finding locations and props
\item
  Scheduling the shoot
\item
  Testing your gear to make sure you know how to use it and that it works properly.
\end{itemize}

As can be seen, there is a lot involved before we film. Luckily, we will start with short exercises which will take minimal pre-production and gradually build to the more detailed aspects later.

\begin{caution}
\textbf{Note:} The lion's share of this unit and Chapter Two in the course book is devoted to pre-production to emphasize how important this phase is. Most people might think production is the most important part of the filming process. In fact, each phase is. In Hollywood the development process (creating the screenplay) and pre-production for a film often takes far longer than production. Post-Production is also a longer process than production when a major film is involved.
\end{caution}

\hypertarget{learning-activities-6}{%
\subsection*{Learning Activities}\label{learning-activities-6}}
\addcontentsline{toc}{subsection}{Learning Activities}

\begin{reflect}
\hypertarget{pre-production-planning}{%
\subsubsection*{Pre-Production Planning}\label{pre-production-planning}}
\addcontentsline{toc}{subsubsection}{Pre-Production Planning}

Read the Pre-Production section of the course text. What part of the pre-production process appeals to you most? Why? Log this in your journal and share your responses with your facilitator and classmates when you meet for this unit.

View this resource and share with your facilitator and course mates why pitching is important:

\href{https://www.youtube.com/watch?v=JE53JL60ihc}{plugin:youtube}
//todo \#4

Murphy's Law states that ``what can go wrong will go wrong'' but many people forget the last component ``at the worse possible moment.'' Have you ever experienced it? Share this experience with your facilitator and classmates, the result and if and how it was overcome.
\end{reflect}

\hypertarget{production}{%
\section{Production}\label{production}}

The production phase involves the shooting of the film, what is often referred to as ``principle photography''. (Sometimes pick-up shots, re-dos, and B-roll footage are done during post-production.) For many filmmakers this is the most exciting part of filmmaking when they get a chance to go on location and watch actors do their magic.

Like pre-production, the production phase must be attended to thoughtfully and diligently or you will not have what you need to edit your film and make your story work. This is why a large portion of this course and the course text are dedicated to this phase of filmmaking, which will only be covered briefly here.

In addition to filming the actors, locations, action sequences, etc. production also includes recording on-location audio for the project. Capturing, recording, and creating good sound, like cinematography, takes lots of time and practice.

Most of the production process takes place on real locations such as streets, schoolyards, restaurants, etc., and sets which are built for the filming. If you have the budget, filming on sets such as kitchens, living, rooms, police stations, etc. is easier than real locations. The reason for this is that you can control the lighting and use of space better and will not have extraneous and disruptive noises occurring in the environment.

Production can also include green screen and CGI (computer generated imagery) work, but these are normally for bigger budget productions since they are specialized skills and it's costly to do them well. They are also time consuming. (If you have After Effects and other special effects software skills please consider using them in this course.)

Performance is a crucial part of the filming process. Many filmmakers forget this and get so caught up in the shooting process that they overlook or miss bad acting. Don't fall into this trap. You want to work on having your actors deliver believable, motivated, non-melodramatic performances. More on this later.

\hypertarget{learning-activities-7}{%
\subsection*{Learning Activities}\label{learning-activities-7}}
\addcontentsline{toc}{subsection}{Learning Activities}

\begin{reflect}
\hypertarget{shooting-a-scene}{%
\subsubsection*{Shooting a Scene}\label{shooting-a-scene}}
\addcontentsline{toc}{subsubsection}{Shooting a Scene}

Read the Production section of the course text.
What part of the production process appeals to you? Why?

View this resource to get a feel for what is involved in shooting a scene:

\href{https://www.youtube.com/watch?v=y9_LW5H2EC4}{plugin:youtube}

What did you learn from this? Share this with your Facilitator and classmates.
\end{reflect}

\hypertarget{post-production}{%
\section{Post-Production}\label{post-production}}

The final phase of the filmmaking process involves post-production where all elements of pre-production and production are ``cut'' or stitched together to create a finished film.

Like production, the post process is hands-on and labor intensive. If done digitally, it involves using software such as Adobe Premier Pro, Apple Final Cut, Avid, DaVinici Resolve, etc. to edit the footage and make transitions such as fades, dissolves, wipes, etc. (to be covered later). This can be easy but a lot of work goes into choosing the best and rights shots and editing them together.

There are four phases to cutting the picture of a film:

\begin{itemize}
\tightlist
\item
  Assembly Cut: where the shots are placed together in proper order without trimming them.\\
\item
  Rough Cut: where you begin to ``trim the fat'' from the beginning and end of shots to get a feel for how the story will flow.\\
\item
  Fine Cuts: which will involve multiple versions as you trim or cut shots and scenes that do not work. You can further hone the film of any ``excess'' fat that does not add to the effective presentation of the story and characters.\\
\item
  Final Cut: which is the ``locked'' picture version that the composer and sound design people need to add music and sound effects to precise moments in the story.
\end{itemize}

In addition to cutting the picture, post-production also entails:

\begin{itemize}
\item
  \textbf{Sound Design} where the right music and sound effects are chosen and added.
\item
  \textbf{Titling and Graphics} when the opening and ending film credits appear and any special written material that will appear on the screen such as: FIVE YEARS EARLIER.
\item
  \textbf{Test Screening} your film to those you trust to make sure the story works and is clear, and to test how an audience will respond.
\item
  \textbf{Color Correcting} where the flow of the color and light and dark nature of the images appears seamless and appropriate. This is where you correct over exposed and under expose shots or those with the wrong color temperature. (\emph{Look these terms up in the film glossary you are using during this course. Refer to Activity 1.2 in Unit 1})
\item
  \textbf{Audio Mixing} which involves finding the right levels and balance between the sound elements such as dialogue, music, sound effects, room tone, etc.
\item
  \textbf{Format Delivery}: What resolution will you use to export your film project? The wrong one can undermine the quality of your film.
\end{itemize}

As can be gleaned, there is a lot to post-production but by carefully studying and practicing the units ahead it will make it easier.

If you are feeling overwhelmed and intimidated, you will not be alone. This is a lot to do and learn and this is why the bigger film projects can be done with team members.

\hypertarget{learning-activities-8}{%
\subsection*{Learning Activities}\label{learning-activities-8}}
\addcontentsline{toc}{subsection}{Learning Activities}

\begin{assessment}
\hypertarget{reflect-and-share}{%
\subsubsection*{Reflect and Share}\label{reflect-and-share}}
\addcontentsline{toc}{subsubsection}{Reflect and Share}

Based on this unit and your reading of Chapter 2 in the course text, share in your journal and with your Facilitator and classmates why knowing the overall process is important and what you learned most from this unit.

Share also if you are feeling intimidated. Remember all but one of the graded assignments will be done in teams so you will not have to do it all alone.
\end{assessment}

\hypertarget{summary-1}{%
\section*{Summary}\label{summary-1}}
\addcontentsline{toc}{section}{Summary}

In this unit, you learned about the overall process of making a film as well as the steps involved in the following three phases:

\begin{itemize}
\tightlist
\item
  Pre-Production\\
\item
  Production\\
\item
  Post-Production
\end{itemize}

\hypertarget{assessment-1}{%
\section*{Assessment}\label{assessment-1}}
\addcontentsline{toc}{section}{Assessment}

\begin{assessment}
\hypertarget{assignment-1-film-journal}{%
\subsection*{Assignment 1: Film Journal}\label{assignment-1-film-journal}}
\addcontentsline{toc}{subsection}{Assignment 1: Film Journal}

After completing this unit, including the learning activities, you are asked to make sure you are doing journal entries and when appropriate to share your responses with your facilitator and classmates when you meet.

Note that entries are expected for every unit. Your journal reflections are submitted at the end of the course as part of the Final Exam: Self Assessment worth 30\% of your course grade.

\hypertarget{grading-criteria-1}{%
\subsubsection*{Grading Criteria:}\label{grading-criteria-1}}
\addcontentsline{toc}{subsubsection}{Grading Criteria:}

See the Assessments section for more details on submitting your journal, as well as the grading criteria.
\end{assessment}

\hypertarget{checking-your-learning-1}{%
\section*{Checking your Learning}\label{checking-your-learning-1}}
\addcontentsline{toc}{section}{Checking your Learning}

\begin{progress}
Before you move on to the next unit, you may want to check to make sure that you are able to:

\begin{itemize}
\item
  Describe the big picture of the film production process
\item
  Determine what is involved in each phase and why it is important
\item
  Capture an overview of what lies ahead
\item
  Define why excellence should be the bench mark of each phase
\end{itemize}
\end{progress}

\hypertarget{visual-composition}{%
\chapter{Visual Composition}\label{visual-composition}}

\hypertarget{overview-2}{%
\section*{Overview}\label{overview-2}}
\addcontentsline{toc}{section}{Overview}

We will now make our theory more practical. So far, we have focused on project excellence and the three phases of filmmaking we will use to create it. In this unit we will concentrate on where it all begins visually -- the single image and how to craft it well.

This unit will also give you a chance to apply what you are learning in a creative way. Don't worry if you think you are not creative or artistic We will focus on simple elements and reveal how they can help you see the world and capture it in exciting ways. We will do so by focusing on composition---the way in which the elements in a picture or image are arranged.

Visual composition will play an important role in this course, so taking the time to learn it is well worth the investment.

\hypertarget{topics-2}{%
\subsection{Topics}\label{topics-2}}

This unit is divided into the following topics:

\begin{enumerate}
\def\labelenumi{\arabic{enumi}.}
\tightlist
\item
  Elements of Composition\\
\item
  Principles of Composition\\
\item
  Photographic Themes
\end{enumerate}

\hypertarget{learning-outcomes-2}{%
\subsection*{Learning Outcomes}\label{learning-outcomes-2}}
\addcontentsline{toc}{subsection}{Learning Outcomes}

When you have completed this unit, you should be able to:

\begin{itemize}
\tightlist
\item
  Describe the elements used in visual compositions\\
\item
  Define composition principles\\
\item
  Study the works of noted photographers and apply composition elements and principles to photographs you create\\
\item
  Create photos that reveal your understanding of the chapter's content
\end{itemize}

\hypertarget{activity-checklist-2}{%
\subsection*{Activity Checklist}\label{activity-checklist-2}}
\addcontentsline{toc}{subsection}{Activity Checklist}

Here is a checklist of learning activities you will benefit from in completing this unit. You may find it useful for planning your work.

\begin{itemize}
\item
  Read Chapter 3 - Elements of Composition and \href{https://www.bhphotovideo.com/explora/photography/tips-and-solutions/11-thoughts-introduction-photographic-composition}{11 Thoughts: An Introduction to Photographic Composition}, by Todd Vorenkamp.
\item
  Study resources on techniques and principles of design. Read the next section of Chapter 3, Principles of Composition.
\item
  Explore images online. Read the next section of Chapter 3 in the course text - Photographic Themes.
\end{itemize}

\textbf{Assessment}

\begin{itemize}
\tightlist
\item
  Film Journal\\
\item
  Visual Composition Exercise (ungraded, but part of journal entry)
\end{itemize}

\hypertarget{resources-2}{%
\subsection*{Resources}\label{resources-2}}
\addcontentsline{toc}{subsection}{Resources}

Here are the resources you will need to complete this unit:

\begin{itemize}
\tightlist
\item
  Chapter Three of course text: \emph{Digital Filmmaking: A Beginner's Guide to Mastering the Craft}, by Ned Vankevich (e-text)\\
\item
  \href{http://www.artyfactory.com/art_appreciation/visual-elements/visual-elements.html}{The Visual Elements: The Building Blocks of Composition in Art}\\
\item
  \href{https://petapixel.com/2016/09/14/20-composition-techniques-will-improve-photos/}{20 Composition Techniques That Will Improve Your Photos}\\
\item
  \href{https://www.themuse.com/advice/taking-constructive-criticism-like-a-champ}{Taking Constructive Criticism Like a Champ}\\
\item
  \href{https://artplusmarketing.com/how-criticism-triggers-growth-52d775e97557}{How Criticism triggers Growth\ldots{}}\\
\item
  \href{https://www.getty.edu/education/teachers/building_lessons/principles_design.pdf}{Principles of Design}\\
\item
  Guidelines for Visual Composition Assignment.\\
\item
  Other resources will be provided in the unit.
\end{itemize}

\hypertarget{elements-of-composition}{%
\section{Elements of Composition}\label{elements-of-composition}}

At heart, visual composition means the intentional arrangement or conscious activity of constructing the ``ingredients'' of an image. As we grow in our filmmaking craft, we must continually be aware of what takes place within the frame (the border of a picture or a film shot). Focusing on a single image can help train us to do this.

\begin{caution}
\textbf{Important Tip}: Before continuing below, study this overview of the building blocks of composition in art: \href{http://www.artyfactory.com/art_appreciation/visual-elements/visual-elements.html}{The Visual Elements: The Building Blocks of Composition in Art}.
\end{caution}

The following are some of the formal elements that make up the design and structure of an image. However, it is important to remember that subject matter and composition are linked and therefore do not get locked in to rigidly adhering to what you are learning. Many masters break the rules but they know them well before they do so.

\begin{itemize}
\tightlist
\item
  Line\\
\item
  Shape (geometric, organic, natural)\\
\item
  Colour (hues, intensities, symbolism)\\
\item
  Texture (tactile feel)\\
\item
  Value (shadows and shading)\\
\item
  Form (3-D, 2-D)\\
\item
  Space (positive/objects); (negative/space between objects)\\
\item
  Depth (foreground/mid-ground/background)
\end{itemize}

\hypertarget{learning-activities-9}{%
\subsection*{Learning Activities}\label{learning-activities-9}}
\addcontentsline{toc}{subsection}{Learning Activities}

\hypertarget{viewing-and-studying-elements-of-visual-composition}{%
\subsubsection*{Viewing and Studying Elements of Visual Composition}\label{viewing-and-studying-elements-of-visual-composition}}
\addcontentsline{toc}{subsubsection}{Viewing and Studying Elements of Visual Composition}

Read the section Elements of Composition in Chapter 3. Be sure to study each of the images provided in this section and how the elements are used. This will increase your understanding of how they function in visual compositions.

Before exploring the rest of this chapter, study this essay to help guide you during the process we are exploring: \href{https://www.bhphotovideo.com/explora/photography/tips-and-solutions/11-thoughts-introduction-photographic-composition}{11 Thoughts: An Introduction to Photographic Composition}, by Todd Vorenkamp.

Helpful Hint: New York-based B and H Photo is a great resource for buying and learning about photographic and film and video gear.

Based on your study of this section, what are some observations you can share in your journal and with you facilitator and classmates?

\hypertarget{principles-of-composition}{%
\section*{Principles of Composition}\label{principles-of-composition}}
\addcontentsline{toc}{section}{Principles of Composition}

In addition to elements of design, there are also principles or general rules that help give structure to visual compositions and which can also lend meaning or a theme to an image. Elements are specific things whereas principles are more general. Here are four that the course text focuses on:

\textbf{1. Balance}\\
\textbf{2. Rule of Thirds}\\
\textbf{3. Repetition and Patterns}\\
\textbf{4. Combinations}

There are many other principles such as the use of triangles and frame within a frame in compositions.

\hypertarget{learning-activities-10}{%
\subsection*{Learning Activities}\label{learning-activities-10}}
\addcontentsline{toc}{subsection}{Learning Activities}

\hypertarget{readings-on-techniques-and-principles-of-composition}{%
\subsubsection*{Readings on Techniques and Principles of Composition}\label{readings-on-techniques-and-principles-of-composition}}
\addcontentsline{toc}{subsubsection}{Readings on Techniques and Principles of Composition}

Study these helpful resources: \href{https://petapixel.com/2016/09/14/20-composition-techniques-will-improve-photos/}{20 Composition Techniques That Will Improve Your Photos}, and ``\href{https://www.getty.edu/education/teachers/building_lessons/principles_design.pdf}{Principles of Design}.

Read the next section of Chapter 3, Principles of Composition, and see if you have a richer understanding of composition. Log your observations in your journal.

Find an example of the use of frame-within-a-frame and triangular composition and share them with your facilitator and classmates.

\hypertarget{photographic-themes}{%
\section{Photographic Themes}\label{photographic-themes}}

The subjects of a visual image includes the objects, people, or items in the frame. E.g., a mountainous landscape, a portrait of twins, etc.

A more advanced approach to composition involves the use of creating a theme or deeper meaning to a composition, or where you take a topic such as clothes on clotheslines and photograph a series of images that reveal their beauty or what they say about the culture, people, or environment where they were shot.

As you grow in your composition skills you will discover how to add more depth, interest, and meaning to your photographs and images.

\hypertarget{learning-activities-11}{%
\subsection*{Learning Activities}\label{learning-activities-11}}
\addcontentsline{toc}{subsection}{Learning Activities}

\hypertarget{read-and-explore}{%
\subsubsection*{Read and Explore}\label{read-and-explore}}
\addcontentsline{toc}{subsubsection}{Read and Explore}

You do not have to create a theme for your photographs in this unit. However, studying websites like the following can help enrich your understanding and find images that stand out in intriguing, startling, fascinating, and emotionally moving ways.

\begin{itemize}
\tightlist
\item
  \href{https://www.jotform.com/blog/35-powerful-photos-that-each-tells-a-story/}{34 Powerful Photos That Tell a Story}\\
\item
  \href{https://www.moma.org/artists/1777}{Walker Evans MoMA Exhibit}\\
\item
  The outstanding Depression Era works of Dorothea Lange are also rich with theme. See ``\href{http://www.historyplace.com/unitedstates/lange/}{The History Place Dorothea Lange}''.\\
\item
  \href{https://www.boredpanda.com/top-10-photographers-for-travel-portraits/?utm_source=google\&utm_medium=organic\&utm_campaign=organic}{Top 10 Most Famous Portrait Photographers In The World}
\end{itemize}

Helpful Hint: The photographs of Walker Evans and Dorethea Lange cited above are almost one-hundred years old. But they reveal the way their heartfelt themes and beauty are universal and timeless---the goal of excellent photos.

Study the next section of Chapter 3 in the course text - Photographic Themes.

Log in your journal and share with a peer in your course what you learned about photographic themes.

\hypertarget{summary-2}{%
\section*{Summary}\label{summary-2}}
\addcontentsline{toc}{section}{Summary}

In this unit, you learned about the use of elements, principles, and theme in visual composition and photographic design.

Composition and subject matter are different but intimately related. We can have a simple subject with little composition. A white ball on a black table. Or we can have a complex composition with little subject matter. The bokeh (blurred or out of focus) of rain splashes on a glass window.

We have emphasized composition here because well composed shots are a foundational building block of a good film. As you will learn, if the head space of your shots is too much or too little, or there is a lack balance of how people are arranged, or if you do not give enough leading entry space in a jogging shot, it will detract from the effect of your film. (There are exceptions to these rules which will be addressed later.) Too many flaws will mark you as an untrained or sloppy filmmaker - something we want to avoid.

\hypertarget{assessment-2}{%
\section*{Assessment}\label{assessment-2}}
\addcontentsline{toc}{section}{Assessment}

\hypertarget{assignment}{%
\subsection*{Assignment}\label{assignment}}
\addcontentsline{toc}{subsection}{Assignment}

\textbf{Film Journal}

After completing this unit, including the learning activities, you are asked to make sure you are doing journal entries and when appropriate to share your responses with your facilitator and classmates when you meet.

\textbf{Composition Exercise}

Congratulations! You have earned the right to shoot your first assignment for the course.

After you have done some composition research and explored great photographers and photos and their compositional techniques and aesthetics, review Chapter 3 and go out and find and create five outstanding visual compositions with your camera (cell phone, DSLR, etc.). Share these with your family, friends, and peers and get their feedback about what worked and did not work. When you have found your best images upload them to the course online folder where the class assignments will be stored.

Remember, the name of the game is to learn not to be perfect.

Refer to the specific guidelines in the assessment section. Be sure to study them carefully.

\hypertarget{grading-criteria-2}{%
\subsubsection*{Grading Criteria}\label{grading-criteria-2}}
\addcontentsline{toc}{subsubsection}{Grading Criteria}

See the Assessments section for more details on your composition, as well as the grading criteria.

Note that entries are expected for every unit. Your journal reflections are submitted at the end of the course as part of the Final Exam: Self Assessment worth 30\% of your course grade.

\hypertarget{checking-your-learning-2}{%
\section*{Checking your Learning}\label{checking-your-learning-2}}
\addcontentsline{toc}{section}{Checking your Learning}

\begin{progress}
Before you move on to the next unit, you may want to check to make sure that you are able to:

\begin{itemize}
\tightlist
\item
  Describe the elements used in visual compositions.
\item
  Define composition principles.
\item
  Study the works of noted photographers and apply composition elements and principles to photographs you create.
\item
  Create photographs that reveal your understanding of the chapter's content.
\end{itemize}
\end{progress}

\hypertarget{cinematic-motion}{%
\chapter{Cinematic Motion}\label{cinematic-motion}}

\hypertarget{overview-3}{%
\section*{Overview}\label{overview-3}}
\addcontentsline{toc}{section}{Overview}

In Unit 3 you created visual compositions with your camera. We hope you found it exciting to go out and explore your world through a lens and design a creative still image.

In this Unit you begin your journey into the world of motion pictures. At heart, film (and video) is a temporal medium. It involves a series of images that unfold over time, most often at the rate of 24, 25, or 30 frames per second. This rate of frame-flow gives film its sense of moving pictures where a series of single frame images of a galloping horse look like the horse is actually running. This is why film is considered a kinetic medium where moving people and objects take center stage.

We began with still photos in the previous unit because they are easier to control and to frame. In this unit we are going to add motion to what we compositionally frame. This will bring a lot more variables to image creation and make it more exciting for many people. As you focus on motion you must not be tempted to forget the lessons learned in the previous unit. Many of the same composition elements and principles will apply and the wise student will refresh his or her self with Unit 3 before engaging the activities for this unit.

\hypertarget{topics-3}{%
\subsection*{Topics}\label{topics-3}}
\addcontentsline{toc}{subsection}{Topics}

This unit is divided into the following topics:

\begin{enumerate}
\def\labelenumi{\arabic{enumi}.}
\tightlist
\item
  Types of Cinematic Motion\\
\item
  Motivating Camera Movement
\end{enumerate}

\hypertarget{learning-outcomes-3}{%
\subsection*{Learning Outcomes}\label{learning-outcomes-3}}
\addcontentsline{toc}{subsection}{Learning Outcomes}

When you have completed this unit, you should be able to:

\begin{itemize}
\tightlist
\item
  Describe and contrast the various types of film motion shots.\\
\item
  Determine when and how to use cinematic motion.\\
\item
  Create effective cinematic motion shots.\\
\item
  Observe and reflect on your progress as a filmmaker and the transferrable skills you are honing.
\end{itemize}

\hypertarget{activity-checklist-3}{%
\subsection*{Activity Checklist}\label{activity-checklist-3}}
\addcontentsline{toc}{subsection}{Activity Checklist}

\begin{reflect}
Here is a checklist of learning activities you will benefit from in completing this unit. You may find it useful for planning your work.

\begin{itemize}
\tightlist
\item
  Read the first section in Chapter 4 ``Types of Cinematic Motion Shots''\\
\item
  Practice doing as many of the main types of motion shots as you can. Discuss what you did and why with your peers in the Learning Lab.\\
\item
  Read the rest of Chapter 4 in your textbook. Find someone to film and practice each of the cinematic motion techniques with your cell phone.
\end{itemize}

\textbf{Assessment}

\begin{itemize}
\tightlist
\item
  Film Journal\\
\item
  Motivated Camera Movement Exercise (ungraded, but part of journal entry)
\end{itemize}
\end{reflect}

\hypertarget{resources-3}{%
\subsection*{Resources}\label{resources-3}}
\addcontentsline{toc}{subsection}{Resources}

Here are the resources you will need to complete this unit:

\begin{itemize}
\tightlist
\item
  Chapter Four of the course text: \emph{Digital Filmmaking: A Beginner's Guide to Mastering the Craft}, by Ned Vankevich
\item
  ``\href{https://www.youtube.com/watch?v=g6zMtnLC50w}{8 Basic Types of Camera Movements}''\\
\item
  ``\href{https://www.youtube.com/watch?v=h2c3JZ6X3f8}{5 Brilliant Moments of Camera Movement}''\\
\item
  ``\href{https://www.youtube.com/watch?v=VPfKsdPsS5w}{Perfect your Film with Cinematic Motion}''\\
\item
  Guidelines for Motion Exercises\\
\item
  Other online resources will be provided in the course text and unit.
\end{itemize}

\hypertarget{types-of-cinematic-motion-shots}{%
\section*{Types of Cinematic Motion Shots}\label{types-of-cinematic-motion-shots}}
\addcontentsline{toc}{section}{Types of Cinematic Motion Shots}

In Unit 3 we explored how excellent or effective still (non-moving) visual compositions are made. The elements and principles of such images apply to filmmaking too. However, film and video add new variables and techniques given that they deal with moving or motion pictures.

However, as we explore film motion, we must remember that film and video involve a series of still images being projected and that persistence of vision makes them appear connected. (See the first Helpful Hint in the course text Chapter 4.) This is why we need to incorporate what we have learned about visual composition as we create our ``moving'' pictures.

The main types of motion shots include:

\begin{itemize}
\tightlist
\item
  ZOOMS\\
\item
  PAN SHOTS\\
\item
  SWISH OR WHIP PAN\\
\item
  TILT SHOTS\\
\item
  DOLLY\\
\item
  DOLLY ZOOMS\\
\item
  TRACKING\\
\item
  ARCING\\
\item
  FOLLOW SHOTS\\
\item
  CRANE SHOTS\\
\item
  360-DEGREE TRACKING SHOTS\\
\item
  SLIDERS\\
\item
  GOPRO\\
\item
  DRONES\\
\item
  SHAKY CAMERA
\end{itemize}

\hypertarget{learning-activities-12}{%
\subsection*{Learning Activities}\label{learning-activities-12}}
\addcontentsline{toc}{subsection}{Learning Activities}

\begin{reflect}
\hypertarget{motion-shots-reading}{%
\subsubsection*{Motion Shots Reading}\label{motion-shots-reading}}
\addcontentsline{toc}{subsubsection}{Motion Shots Reading}

Study the first section in Chapter 4 ``Types of Cinematic Motion Shots'' to help you understand the different types of motion shots. Be sure to take notes and begin logging and identifying shot ideas you want to try and practice. Actively taking notes as you read is a great way to absorb the material.

\hypertarget{motion-shots-practice}{%
\subsubsection*{Motion Shots Practice}\label{motion-shots-practice}}
\addcontentsline{toc}{subsubsection}{Motion Shots Practice}

Study the definitions of the main types of motion shots and how to do them. Then practice doing as many of them as you can. (Obviously you cannot do a GoPro or drone shot without that specific equipment, but you might be imaginative and find an alternative to do a similar type of shot.)

Write in your journal your observations about each type of shot and what you learned from doing them. Share these insights and questions with your class mates when you meet.

Before moving on to the next topic, view the following resources to augment your understanding of camera movement
\end{reflect}

\begin{caution}
\textbf{Note:} that this learning activity is ungraded, but is designed to help you succeed in your assessments in this course.

\emph{In addition, do not try doing zoom shots with your cell camera unless you have a special app that makes it smooth. Squeezing your finger in and out to move closer or further away from your subject will result in a shaky shot. Our goal is to disguise movement, not to draw attention to it.}
\end{caution}

\hypertarget{properly-motivated-camera-moving-shots}{%
\section{Properly Motivated Camera Moving Shots}\label{properly-motivated-camera-moving-shots}}

A classical approach to filmmaking involves using shot and techniques that are motivated by the characters and story and which do not unnecessarily distract or pull the audience out of experiencing an event in your film. Put another way, if you do something that is jarring or not done well such as a shaky zoom shot it will draw attention to itself and distract the audience and prevent them from engaging the flow of your scene. What this means will become apparent as you progress in the course.

The following sections will help you have properly motivated and framed movement within the frame.

\begin{itemize}
\tightlist
\item
  \textbf{Properly Motivated Movement Shots}
\item
  \textbf{Proper Placement in the Frame}
\item
  \textbf{Motivated Unmotivated Camera Movement}
\end{itemize}

\hypertarget{learning-activities-13}{%
\subsection*{Learning Activities}\label{learning-activities-13}}
\addcontentsline{toc}{subsection}{Learning Activities}

\begin{reflect}
\hypertarget{proper-motivation-exercise-and-practice}{%
\subsubsection*{Proper Motivation Exercise and Practice}\label{proper-motivation-exercise-and-practice}}
\addcontentsline{toc}{subsubsection}{Proper Motivation Exercise and Practice}

Read the rest of Chapter 4 in your textbook.

\begin{itemize}
\tightlist
\item
  After you have viewed the resources and studied the properly motivated camera movements, find someone to film and practice each of the techniques with your cell phone.
\end{itemize}

For inspiration and how-to-do techniques view this resource: 8 Cinematic Camera Moves For Video

Bring your best shots to the Learning Lab to share with your facilitator and classmates and to get feedback.

See the Assessment section for more guidelines and criteria for evaluating this exercise.
\end{reflect}

\hypertarget{summary-3}{%
\section*{Summary}\label{summary-3}}
\addcontentsline{toc}{section}{Summary}

In this unit, you learned about cinematic motion and how to motivate it and do it properly. As such, you have gained understanding and practice with another important basic of excellent filmmaking.

\hypertarget{assessment-3}{%
\section*{Assessment}\label{assessment-3}}
\addcontentsline{toc}{section}{Assessment}

\begin{assessment}
\hypertarget{motivated-camera-movement-exercise}{%
\section*{Motivated Camera Movement Exercise}\label{motivated-camera-movement-exercise}}
\addcontentsline{toc}{section}{Motivated Camera Movement Exercise}

\emph{(ungraded, but part of journal entry)}

For this exercise on motion, you will shoot \textbf{5 different types} of motivated kinetic shots, e.g., a pan, tilt, zoom, following action shot, etc. that have smooth motion from the beginning to the end of the shot and ``disguise'' the camera technique by matching the rate of camera move and subject movement.

In our quest to learn the art of excellent filmmaking we will focus on creating motivated motion shots. In doing this we can use shots that are context-centered or context-free.

For example, a context-centered shot would use a pan shot to show the comradery or bond between a group of friends as they share and pass homemade cookies to
each other. The motion of the pan shot would match the rate of the characters' movement as they take a cookie and pass the plate to the next friend.

A context-free shot would be to do a pan shot of a car as you practice following the motion with enough lead space as you keep the car in the same part of the
frame as it passes. In other words, you will do a shot that follows motion for the sake of practicing how to do it.

Don't get too caught up in context-centered or context free shots. Just focus on mastering camera movement and motivating it such as using a follow zoom where
you match the rate of the camera move and the subject movement so that we are not away of the zoom in or the zoom out as you follow the action. For example, you would start with a wide shot of someone entering a room and as he crosses the room the use the zoom to move in closer and end with a waist to the top of the head shot as the person is seated. Notice that this example would also
include a motivated tilt down as the person sits. This is a great type of shot to practice to gain confidence in your motion shot skills.

The above examples might seem simple but they will take practice to execute smoothly and not have any distracting bumps or mars in the smoothness of the motion or not having characters bump their heads on the top of the frame as they move or bump up against the edge of the frame.

When you can control the action (unlike following an airplane across the sky) be sure to have a clean start and clear end to your shot. A clean start means
rather than launching immediately into the shot and action you take a beat or short moment before stating the movement. A clear end means you hold the last
shot for at least a second before the person leaves the frame and you turn off the camera. You will discover later that is important to have a clean open and clear end when editing shots together.

Be sure to rehearse the shots and to do multiple takes. Only bring your best example of each shot type to class. The more you practice the better you will get.
\end{assessment}

\begin{caution}
\textbf{Helpful Reminder}: Do not forget to create well composed shots. Review the principles of good composition and design from the last unit.
\end{caution}

\begin{assessment}
The following guidelines sheet will explain the specific requirements and logistics for this assignment.

\hypertarget{criteria-for-assessing-the-motion-shots}{%
\subsection*{Criteria for Assessing the Motion Shots}\label{criteria-for-assessing-the-motion-shots}}
\addcontentsline{toc}{subsection}{Criteria for Assessing the Motion Shots}

This exercise, like the previous one, will not be graded. You are learning the basics and it is not fair to penalize you at this early stage. However, as we are learning, film is a public medium and you will be showing your work to your peers so in some ways you will be accountable for the quality of your shots. In addition, if you do not practice these basics it will impact your grade in the next assignments if you do not execute the techniques well.

The following questions will be discussed as you show your work:

\begin{itemize}
\tightlist
\item
  Was the shot motivated?\\
\item
  Was the shot smooth or distracting?\\
\item
  Did it follow the action well?\\
\item
  Did the shot have a clean start and a clear end?\\
\item
  Was the shot well-composed in general?\\
\item
  Was the shot well-composed throughout from the beginning to the end?\\
\item
  What was interesting about the shot?\\
\item
  Was there anything that pulled us of being involved with the shot?\\
\item
  Did the shot combine more than one movement type? If so, how many and did they work well together? E.g., a zoom, pan, and tilt.
\end{itemize}

You will be wise to keep in mind the following as you design and practice your
shots:

\begin{itemize}
\tightlist
\item
  Your goal is to create smooth, seamless, cinematic camera motion and learn how to motivate and disguise it by having the camera move at the apparent rate of the moving subject.\\
\item
  A Zoom should follow the movement and not go in and out during the move.\\
\item
  A Pan should move at the apparent rate of the subject and not lead or follow it.\\
\item
  Head room is important---be careful of bumping heads against the top of the frame during the shot or cutting parts of the head that look weird or inappropriate.\\
\item
  Lead room---have space for the subject to move into without bumping into the frame or feeling too squeezed\ldots unless motivated by the situation or context. Eg., someone being pressured.\\
\item
  A Tilt should reveal something new and can be motivated by character movement, e.g.~tilting down at the rate which the characters sits down or tilting up at the rate that someone picks something up.\\
\item
  A Dolly shot should follow the action (side to side) or be led by the subject. E.g., dolly back while doing a walk and talk.\\
\item
  Depth (S and arcing curves are normally more aesthetically pleasing).\\
\item
  Tracking/Trucking (lateral/parallel movement) at the apparent rate of the subject, i.e., not faster or slower than the subject movement which draws attention to itself.
\item
  Arc (combo pan and track---curved movement) can be motivated by a character's move or situation.\\
\item
  Car/Helicopter/Camera Truck, etc. where we follow the subject at a smooth, non-attention grabbing rate.\\
\item
  Crane\\
\item
  Boom up/ Boom Down as a subject moves away.\\
\item
  Selfie stick\\
\item
  Drones\\
\item
  Follow the subject at the appropriate rate.\\
\item
  Gimbal shots\\
\item
  Follow the subject at the appropriate rate.
\end{itemize}

For this exercise we will avoid the shaky camera and fast-moving hand-held techniques for obvious reasons.
\end{assessment}

\begin{caution}
\textbf{Helpful Hint} : Before doing the following assignment read this resource for inspiration: ``\href{https://www.filmcomment.com/article/game-changers-camera-movement/}{Game Changers: Camera Movement},'' and \href{https://www.youtube.com/watch?v=6_p93J3OwfU}{8 Cinematic Camera Moves For Video}.
\end{caution}

\begin{assessment}
\hypertarget{guidelines-for-cinematic-motion-exercise}{%
\subsection*{Guidelines for Cinematic Motion Exercise}\label{guidelines-for-cinematic-motion-exercise}}
\addcontentsline{toc}{subsection}{Guidelines for Cinematic Motion Exercise}

\begin{itemize}
\tightlist
\item
  Your goal: to continue practicing basic filmmaking skills by exploring and experimenting with cinematic motion.\\
\item
  Before doing the exercise below go back to the previous unit and refresh your understanding of excellent visual composition.\\
\item
  You must shoot \textbf{5 different types} of motivated kinetic shots, e.g., a pan, tilt, zoom, following action shot, etc. that have smooth motion from the beginning to the end of the shot and ``disguise'' the camera technique by matching the rate of camera move and subject movement.\\
\item
  Your shots can be context-centered (reveal a character or story element) or context-free (merely follow action and show smooth camera work and motion).
\item
  You must have a clean start and a clear end of the shot as discussed during the course.
\item
  Given the parameters of this exercise \emph{you will not be allowed to use a shaky handheld camera effect or technique}. Later, when properly motivated you can, but not for this exercise.
\item
  We will go into cameras later but for now keep this assignment simple and you can use your cell phone or, if you have one, a DSLR for this assignment. You can consult tutorials on YouTube, UDEMY, or other learning websites for how to shoot video you specific phone or camera.
\item
  You are strongly encouraged to use a tripod, selfie stick, gimbal, monopod or other similar devices to help you make smooth camera moves.
\item
  Be sure to rehearse and to do multiple takes until you are satisfied that you have a smooth, motivated shot. Only submit your best example of each type of shot.
\item
  Place your shots in the Course Folder with your name on it. The upload site address will be handed out in class.
\item
  Label the image with the type of moving shot that you did so that we are clear what you intended and whether it worked or not.
\item
  As with the composition exercise do not take this assignment lightly. You will learn important things from it. Remember the goal is to disguise the movement of the camera by subject movement and matching the flow rate of each so that you technique follows the action in a non-distracted, seamless or disguised and ``invisible'' manner.
\item
  Do not get coaching or help for this exercise. Do it on your own. You will learn more from your personal successes and failures.
\item
  The due date for this activity will be given out during the course.
\item
  Those wanting to excel at this exercise will do some research on films that have exceptional motion shots and try re-creating one or try a technique we have not addressed. For example: \href{https://www.youtube.com/watch?v=oLFHdagIw6o\&t=61s}{12 Best Long Takes in Film History}.
\end{itemize}

Continue to cultivate a good learning perspective. Look upon the exercise and assignment for this unit as a fun adventure. Again, don't be crippled by fear or perfectionism. Practice, practice, practice, and do you best. The fruits of this process will show up later.

\hypertarget{grading-criteria-3}{%
\subsubsection*{Grading Criteria:}\label{grading-criteria-3}}
\addcontentsline{toc}{subsubsection}{Grading Criteria:}

See the Assessments section for more details, as well as the grading criteria.

\hypertarget{assignment-1-film-journal-1}{%
\subsection*{Assignment 1: Film Journal}\label{assignment-1-film-journal-1}}
\addcontentsline{toc}{subsection}{Assignment 1: Film Journal}

After completing this unit, including the learning activities, you are asked to make sure you are doing journal entries and when appropriate to share your responses with your facilitator and classmates when you meet.

Note that entries are expected for every unit. Your journal reflections are submitted at the end of the course as part of the Final Exam: Self Assessment worth 30\% of your course grade.
\end{assessment}

\hypertarget{checking-your-learning-3}{%
\section*{Checking your Learning}\label{checking-your-learning-3}}
\addcontentsline{toc}{section}{Checking your Learning}

\begin{progress}
Before you move on to the next unit, you may want to check to make sure that you are able to:

\begin{itemize}
\tightlist
\item
  Describe the various types of film motion shots
\item
  Contrast the various types of film motion shots
\item
  Determine when and how to use cinematic motion
\item
  Create effective cinematic motion shots
\item
  Make observations in your journal of what you have learned from this unit.
\end{itemize}
\end{progress}

\hypertarget{the-grammar-of-film}{%
\chapter{The Grammar of Film}\label{the-grammar-of-film}}

\hypertarget{overview-4}{%
\section*{Overview}\label{overview-4}}
\addcontentsline{toc}{section}{Overview}

We go back to theory in this unit and then have a chance to practice and apply it.

When we watch a film or TV show most people have no idea how much work and detail it took to make the production. Nor do they know the rules, guidelines, principles, and practices that were followed to create it. This unit and the ones following will help you explore what these best practices are and understand how to apply them to your work.

This will be a fun chapter as you get to explore and experiment with the fundamental building blocks that create meaning in cinema. It is a unit that will challenge both your thinking and your intuition skills as you learn the basic components of filmmaking and how to use them in creative ways.

\hypertarget{topics-4}{%
\subsection*{Topics}\label{topics-4}}
\addcontentsline{toc}{subsection}{Topics}

This unit is divided into the following topics:

\begin{enumerate}
\def\labelenumi{\arabic{enumi}.}
\tightlist
\item
  Basic Grammar of Film\\
\item
  Basic Shots and Their Descriptions\\
\item
  Camera Angles and Heights\\
\item
  Camera Subject Angles
\end{enumerate}

\hypertarget{learning-outcomes-4}{%
\subsection*{Learning Outcomes}\label{learning-outcomes-4}}
\addcontentsline{toc}{subsection}{Learning Outcomes}

When you have completed this unit, you should be able to:

\begin{itemize}
\tightlist
\item
  Define the basic elements of the grammar of film\\
\item
  Describe basic shots and their descriptions\\
\item
  Contrast camera angles and heights\\
\item
  Determine when and how to use subject angles\\
\item
  Create shots that reveal what you discovered in this unit
\end{itemize}

\hypertarget{activity-checklist-4}{%
\subsection*{Activity Checklist}\label{activity-checklist-4}}
\addcontentsline{toc}{subsection}{Activity Checklist}

\begin{reflect}
Here is a checklist of learning activities you will benefit from in completing this unit. You may find it useful for planning your work.

\begin{itemize}
\tightlist
\item
  Consult a film glossary, read the first section of Chapter 5, and study \href{https://liveboldandbloom.com/04/self-improvement/develop-your-intuition}{21 Eye-Opening Ways To Develop Your Intuition}.\\
\item
  Read the next section of Chapter 5 and view the video resources listed.\\
\item
  Read \href{http://timurcivan.com/2015/02/camera-work-101-camera-height/}{Camera work 101: Camera Height}
\item
  Read the next section of Chapter 5 and John Suler's \href{http://truecenterpublishing.com/photopsy/camera_angles.htm}{Photographic Psychology: Image and Psyche, Camera Angles}.\\
\item
  Complete the Street Photography exercise.
\end{itemize}

\textbf{Assessment}

\begin{itemize}
\tightlist
\item
  Film Journal
\end{itemize}
\end{reflect}

\hypertarget{resources-4}{%
\subsection*{Resources}\label{resources-4}}
\addcontentsline{toc}{subsection}{Resources}

Here are the resources you will need to complete this unit.

\begin{itemize}
\tightlist
\item
  Chapter 5 of course text, \emph{Digital Filmmaking: A Beginner's Guide to Mastering the Craft}.\\
\item
  \href{https://liveboldandbloom.com/04/self-improvement/develop-your-intuition}{21 Eye-Opening Ways To Develop Your Intuition}\\
\item
  Studying film shots is an essential part of mastering the craft of filmmaking. Study this helpful overview: \href{https://www.youtube.com/watch?v=7y0ouVBcogU}{15 Essential Camera Shots, Angles and Movements in Filmmaking}\\
\item
  For an overview of the types of film shots and angles see: \href{https://www.studiobinder.com/blog/ultimate-guide-to-camera-shots/}{The Ultimate Guide to Camera Shots (over 50+ Types of Shots and Angles in Film)}\\
\item
  \href{http://userhome.brooklyn.cuny.edu/anthro/jbeatty/COURSES/glossary.htm}{Brooklyn College Film Glossary}\\
\item
  \href{https://erickimphotography.com/blog/2014/12/09/17-lessons-henri-cartier-bresson-taught-street-photography/}{17 Lessons Henri Cartier-Bresson Has Taught Me About Street Photography}\\
\item
  Other online resources will be provided in the unit and text chapter.
\end{itemize}

\hypertarget{basic-grammar-of-film}{%
\section{Basic Grammar of Film}\label{basic-grammar-of-film}}

In the previous units we look at the basic building blocks of visual composition and cinematic motion. In this unit we are going to learn how to put them together with a variety of other shots to make cinematic context and meaning. This is the grammar of film used to make cinematic ``sentences'' which convey meaning and context to an audience as we create scenes, sequences, and short films.

However, as we focus on the components of filmmaking, we must not let the emphasis only be on film logic. We also want to cultivate the role that intuition and feeling play as we develop shots and put them together. This reminder is important because filmmaking is as much art as technical craft. We need to know the craft---the technical aspects---but also the art---the aesthetic and emotional dimensions. We need to keep this before us as we venture into the many elements and guidelines of effective filmmaking.

In this first topic you will learn about:

\begin{itemize}
\item
  \textbf{Frames}
\item
  \textbf{Shots}
\item
  \textbf{Scenes}
\item
  \textbf{Sequences}
\item
  \textbf{Acts}
\end{itemize}

These components give you the big picture of how the many smaller elements add up to create mood and meaning in a film project.

\hypertarget{learning-activities-14}{%
\subsection*{Learning Activities}\label{learning-activities-14}}
\addcontentsline{toc}{subsection}{Learning Activities}

\begin{reflect}
\hypertarget{study-read-and-explore}{%
\subsubsection*{Study, Read, and Explore}\label{study-read-and-explore}}
\addcontentsline{toc}{subsubsection}{Study, Read, and Explore}

Find and consult a film glossary such as \href{http://userhome.brooklyn.cuny.edu/anthro/jbeatty/COURSES/glossary.htm}{Brooklyn College Film Glossary}.

Continue to read and study the one you like throughout this course and it will help you grow your film vocabulary and your understanding of the many components and processes involved in filmmaking. Set a goal to learn \textbf{5 new terms or concepts} each day. Bring 5 of the new terms to class and share one of them with your classmates. Listen closely to their terms and how your peers define them. This will enhance and re-enforce the concepts for everyone.

Read the first section of Chapter 5.
In your Film Journal, explain how frames and shots become scenes and sequences and how they add up to a complete film story.

Finally, study a web essay on how to build your intuition such as \href{https://liveboldandbloom.com/04/self-improvement/develop-your-intuition}{21 Eye-Opening Ways To Develop Your Intuition}.

Commit to doing one technique several times a day. Share the technique you will focus on with your facilitator and peers. What did you learn from this experience?
\end{reflect}

\hypertarget{basic-shots-and-their-descriptions}{%
\section{Basic Shots and Their Descriptions}\label{basic-shots-and-their-descriptions}}

This is a vital section. You need to know the following shots and how to describe them because you will have to use this language to communicated to your crew. Few things are more frustrating for a film crew than the director or cinematographer not knowing what they want or how to name and describe a shot.

\begin{itemize}
\tightlist
\item
  Extreme Long Shot (ELS)\\
\item
  Long Shot (LS) also called Wide Shot (WS)\\
\item
  Establishing Shot (ES)\\
\item
  Full Shot (FS)\\
\item
  Medium Long Shot (MLS) also known as Medium Wide Shot (MWS)\\
\item
  Cowboy Shot\\
\item
  Medium Shot (MS)\\
\item
  Medium Close Up (MCU)\\
\item
  Close Up (CU)\\
\item
  Extreme Close Up (ECU)
\end{itemize}

In addition to shot size we also designate film shots by who or what is in the frame and how they are positioned:

\begin{itemize}
\tightlist
\item
  Single Shot\\
\item
  Two Shot\\
\item
  Three Shot\\
\item
  Group Shot\\
\item
  Over-The-Shoulder Shot\\
\item
  Reverse Shot\\
\item
  Point of View Shot (POV)\\
\item
  Reaction Shot\\
\item
  Cutaway\\
\item
  Cut-in and Insert\\
\item
  Combo Shot
\end{itemize}

\hypertarget{learning-activities-15}{%
\subsection*{Learning Activities}\label{learning-activities-15}}
\addcontentsline{toc}{subsection}{Learning Activities}

\begin{reflect}
\hypertarget{read-and-watch}{%
\subsubsection*{Read and Watch}\label{read-and-watch}}
\addcontentsline{toc}{subsubsection}{Read and Watch}

Read the next section of Chapter 5.

View the video resources below, as well as others listed in the course text.
//todo \#6

\begin{itemize}
\item
  \href{https://www.youtube.com/watch?v=lRo2IqYbEaE}{watch video here}
\item
\end{itemize}

As they say, a picture is worth a thousand words, and when you see the shots in action in these resources they become more clear.
\end{reflect}

\hypertarget{camera-angles-and-heights}{%
\section{Camera Angles and Heights}\label{camera-angles-and-heights}}

Camera heights and angles reveal both text and subtext. That is, they show us what is being filmed (the object or subject) and often convey something about the meaning of the subject or object (the subtext).

In essence, there are no ``neutral'' camera angles. Our camera height placement will convey something and it is important for the filmmaker to know what that something is.

The different camera heights include:

\begin{itemize}
\tightlist
\item
  Eye-Level\\
\item
  Low Angle\\
\item
  High Angle\\
\item
  Shoulder Level\\
\item
  Hip Level\\
\item
  Knee Level Shot\\
\item
  Ground Level\\
\item
  Worm's Eye View\\
\item
  Bird's Eye View\\
\item
  God's Eye View\\
\item
  Dutch Tilt or Dutch Canted Angle
\end{itemize}

\hypertarget{learning-activities-16}{%
\subsection*{Learning Activities}\label{learning-activities-16}}
\addcontentsline{toc}{subsection}{Learning Activities}

\begin{reflect}
\hypertarget{studying-camera-heights}{%
\subsubsection*{Studying Camera Heights}\label{studying-camera-heights}}
\addcontentsline{toc}{subsubsection}{Studying Camera Heights}

Read the next section of Chapter 5.

To help re-enforce how camera height affects what we see and how we interpret it study this essay and its visual representations:

\begin{itemize}
\tightlist
\item
  \href{http://timurcivan.com/2015/02/camera-work-101-camera-height/}{Camera work 101: Camera Height}
\end{itemize}

Summarize in your journal the best thing you learned about camera heights in this section.
\end{reflect}

\hypertarget{camera-subject-angles}{%
\section{Camera Subject Angles}\label{camera-subject-angles}}

Subject angle refers to the perspective the audience has of the person being filmed. Subject angles are important because they affect the way the audience interprets the character or scene.

The following are key subject angles:

\begin{itemize}
\tightlist
\item
  Profile\\
\item
  Full-Face\\
\item
  Three-quarter\\
\item
  Quarter angle\\
\item
  Full-Back\\
\item
  Combo\\
\item
  Silhouettes
\end{itemize}

Having completed this topic, you now have a firm grasp of the grammar of film and its components.

\hypertarget{learning-activities-17}{%
\subsection*{Learning Activities}\label{learning-activities-17}}
\addcontentsline{toc}{subsection}{Learning Activities}

\begin{reflect}
\hypertarget{exploring-camera-shots-and-angles}{%
\subsubsection*{Exploring Camera Shots and Angles}\label{exploring-camera-shots-and-angles}}
\addcontentsline{toc}{subsubsection}{Exploring Camera Shots and Angles}

Read the final section of Chapter 5.

Now that you know the overall basics of camera shots and angles, study the following detailed resource to enhance your understanding of the psychological and narrative impact they have on an audience:
- John Suler's \href{http://truecenterpublishing.com/photopsy/camera_angles.htm}{Photographic Psychology: Image and Psyche, Camera Angles}

What new or deeper insights did you learn? Log these observations in your journal and share them with your facilitator and peers.

\hypertarget{exercise-for-applying-grammar-of-film}{%
\subsubsection*{Exercise for Applying Grammar of Film}\label{exercise-for-applying-grammar-of-film}}
\addcontentsline{toc}{subsubsection}{Exercise for Applying Grammar of Film}

Now that you have done the theory, it is time to practice and apply what you have learned. Be sure to use both sides of your brain---the conscious, intentional shot logic part and the intuitive, feeling, follow your gut and take a risk part. Learning to combine both will help you grow immensely in your creativity.

For this exercise you will engage in street photography, that is, go out into your local world and explore it through your cell phone camera lens and capture photos that use at least ten different shot sizes, subject framing, subject angles, and camera angles and heights.

For inspiration about street photography see:

\begin{itemize}
\tightlist
\item
  \href{https://erickimphotography.com/blog/2014/12/09/17-lessons-henri-cartier-bresson-taught-street-photography/}{17 Lessons Henri Cartier-Bresson Has Taught Me About Street Photography}
\end{itemize}

Share your four best shots with your classmates and log in your journal what learned from your street photography and exploring of camera angles.
\end{reflect}

\hypertarget{summary-4}{%
\section*{Summary}\label{summary-4}}
\addcontentsline{toc}{section}{Summary}

In this unit, you learned about the importance of film grammar and the multitude of shot sizes, framings, angles, etc. In doing so, you now a good foundational grasp of the basic components needed to make excellent film projects.

\hypertarget{assessment-4}{%
\section*{Assessment}\label{assessment-4}}
\addcontentsline{toc}{section}{Assessment}

\begin{assessment}
\hypertarget{assignment-film-journal}{%
\subsection*{Assignment : Film Journal}\label{assignment-film-journal}}
\addcontentsline{toc}{subsection}{Assignment : Film Journal}

After completing this unit, including the learning activities, you are asked to make sure you are doing journal entries and when appropriate to share your responses with your facilitator and classmates when you meet.
\end{assessment}

\begin{caution}
\textbf{Note:} that entries are expected for every unit. Your journal reflections are submitted at the end of the course as part of the Final Exam: Self Assessment worth 30\% of your course grade.
\end{caution}

\hypertarget{checking-your-learning-4}{%
\section*{Checking your Learning}\label{checking-your-learning-4}}
\addcontentsline{toc}{section}{Checking your Learning}

\begin{progress}
Before you move on to the next unit, you may want to check to make sure that you are able to:

\begin{itemize}
\tightlist
\item
  Define the basic elements of the grammar of film.\\
\item
  Describe basic shots and their descriptions.\\
\item
  Contrast camera angles and heights.\\
\item
  Determine when and how to use subject angles.\\
\item
  Create shots that reveal what you discovered in this unit.
\end{itemize}
\end{progress}

\hypertarget{visual-storytelling}{%
\chapter{Visual Storytelling}\label{visual-storytelling}}

\hypertarget{overview-5}{%
\section*{Overview}\label{overview-5}}
\addcontentsline{toc}{section}{Overview}

Congratulations. You are now ready to make a rare type of film. Rare you say? Yes, because you are going to shoot your film project in a way that is not done very often, that is, in full sequential order. What this means will be explained ahead.

We are now at a point in the course where you are going to be held accountable for your work. The earlier exercises were not graded to give you the chance to explore and make mistakes, without regard to a grade. We now have to cross into that territory. The main reason for this is that, as mentioned previously, film is a public medium and your work will be critiqued when it is shown. So this will be a great opportunity to learn to give and receive feedback.

But fear not. Based on your previous work and what you will learn in this unit you will be ready to make a film that is ``public'' worthy. In addition, this first film will only account for 10\% of your grade so you are encouraged to explore and take risks.

\hypertarget{topics-5}{%
\subsection*{Topics}\label{topics-5}}
\addcontentsline{toc}{subsection}{Topics}

This unit is divided into the following topics:

\begin{enumerate}
\def\labelenumi{\arabic{enumi}.}
\tightlist
\item
  The In-Camera Project and Its Benefits\\
\item
  Secrets to a Simple Story\\
\item
  Brainstorming and Successful Creativity\\
\item
  Short Story Film Template\\
\item
  Direction Vectors and Eye-line Requirements
\end{enumerate}

\hypertarget{learning-outcomes-5}{%
\subsection*{Learning Outcomes}\label{learning-outcomes-5}}
\addcontentsline{toc}{subsection}{Learning Outcomes}

When you have completed this unit, you should be able to:

\begin{itemize}
\tightlist
\item
  Describe what constitutes a liner story\\
\item
  Define what constitutes an ``In-Camera'' film\\
\item
  Analyze and apply a story template\\
\item
  Determine how to tell a visually-centered story well\\
\item
  Create a short simple story to film
\end{itemize}

\hypertarget{activity-checklist-5}{%
\subsection*{Activity Checklist}\label{activity-checklist-5}}
\addcontentsline{toc}{subsection}{Activity Checklist}

\begin{reflect}
Here is a checklist of learning activities you will benefit from in completing this unit. You may find it useful for planning your work.

\begin{itemize}
\tightlist
\item
  Read the first section of Chapter 6 and watch \href{https://www.youtube.com/watch?v=iWQQgZh9EyE}{Visual Storytelling 101}.\\
\item
  Study the next section in the course text Secrets to a Simple Story.\\
\item
  Study the Brainstorming section in Chapter 6 and consult the resource listed.\\
\item
  Use the Fairy Tale Template to plan your film. Read through the section in Chapter 6 on creating a Short Story Film Template.\\
\item
  Read the next section in Chapter 6 and view the videos listed.
\end{itemize}

\textbf{Assessment}

\begin{itemize}
\tightlist
\item
  Film Journal
\end{itemize}

\textbf{Assignment}

\begin{itemize}
\item
  In-Camera Exercise (10\%)
\item
  Log in your journal what you learned from the creating and filming of your In-Camera exercise.
\end{itemize}
\end{reflect}

\hypertarget{resources-5}{%
\subsection*{Resources}\label{resources-5}}
\addcontentsline{toc}{subsection}{Resources}

Here are the resources you will need to complete this unit.

\begin{itemize}
\tightlist
\item
  Chapter Six course text\\
\item
  \href{https://www.youtube.com/watch?v=iWQQgZh9EyE}{Visual Storytelling 101}\\
\item
  \href{https://www.youtube.com/watch?v=4X5xvlTZpcY}{Visual Storytelling in Filmmaking}\\
\item
  \href{https://www.studiobinder.com/blog/short-film-ideas-you-can-actually-produce/}{``30 Ways to Brainstorm Short Film Ideas You Can Actually Produce}\\
\item
  \href{https://www.indiewire.com/2015/08/19-great-ways-to-brainstorm-short-film-ideas-58785/}{19 Great Ways to Brainstorm Short Film Ideas}\\
\item
  \href{https://www.youtube.com/watch?v=y_1H6V7uyYc}{The eyeline match}\\
\item
  \href{https://www.youtube.com/watch?v=9XOn5uxdSJc}{Screen Direction rule}
\end{itemize}

\begin{itemize}
\tightlist
\item
  \href{https://www.youtube.com/watch?v=0pd0K2u1Bk8}{Filmmaking Tutorial: Head room, lead room \& Framing} YouTube, LightsFilmSchool.
\end{itemize}

\hypertarget{in-camera-project-and-its-benefits}{%
\section{In-Camera Project and Its Benefits}\label{in-camera-project-and-its-benefits}}

Your first film project will be something you might never do again if you go into the film and television world. You will shoot a short film with no dialogue in the exact sequential order that it will appear when you screen it for classs.

Almost always, films are shot out of sequence to save time and money by doing all the scenes in one location at a time and then all the scenes in another location at one time, etc.

Here you will be forced to shoot at a location and move to the next, and if the first location is needed you will have to go back to it. Why do this? There are multiple benefits, as the course text chapter highlights. Some of them include:

\begin{itemize}
\tightlist
\item
  Forcing you to think of a simple story that fits these parameters.\\
\item
  Challenging you to plan your shots in a linear, well-thought out way.\\
\item
  Making you attentive to each shot and how relates to the previos ones.\\
\item
  Challenging your brainstorming and film logic skills.
\end{itemize}

\hypertarget{learning-activities-18}{%
\subsection*{Learning Activities}\label{learning-activities-18}}
\addcontentsline{toc}{subsection}{Learning Activities}

\begin{reflect}
\hypertarget{read-reflect-and-view}{%
\subsubsection*{Read, Reflect and View}\label{read-reflect-and-view}}
\addcontentsline{toc}{subsubsection}{Read, Reflect and View}

Read the first section of Chapter 6 about the benfits of doing an in-camera film. Write in your journal your initial thoughts regarding this assignment. Does it thrill and excite you or make you apprehensive and anxious? Or maybe a combination of these feelings and emotions. As artists we need to get in touch with our feelings and learn to work with them not fear them.

Share with your classmates your responses. This can help foster tighter bonds among you as you realize you are not alone in your feelings or need of support.

To prime your imagination for the upcoming project, view this resource:

What was your best take-away from it? Log this in your journal and if there is time, share your insights with your facilitator and classmates.
\end{reflect}

\begin{caution}
\textbf{Note:} that this learning activity is ungraded, but is designed to help you succeed in your assessments in this course.
\end{caution}

\hypertarget{secrets-to-a-simple-story}{%
\section{Secrets to a Simple Story}\label{secrets-to-a-simple-story}}

In order to do this first project, you will need to come up with a story to film. Something has to happened to someone and they must engage in action to address it. But if we are to design a good story---one that holds the attention of our audience and engages them---we need to focus on several things in our short film project:

\begin{itemize}
\tightlist
\item
  One or two main characters.\\
\item
  A single problem or conflict.\\
\item
  Knowing your genre (will your film be funny, dramatic, scary, etc.?).\\
\item
  A simple setting or number of locations.\\
\item
  A satisfying ending.
\end{itemize}

You need to keep this simple formula in mind in the projects ahead. There will be many forces and temptations that will distract you and pull you away from this simple understanding. Resist them and go back to basics if you get lost, confused, or frustrated.

\hypertarget{learning-activities-19}{%
\subsection*{Learning Activities}\label{learning-activities-19}}
\addcontentsline{toc}{subsection}{Learning Activities}

\begin{reflect}
\hypertarget{read-and-reflect-1}{%
\subsubsection*{Read and Reflect}\label{read-and-reflect-1}}
\addcontentsline{toc}{subsubsection}{Read and Reflect}

Study the next section in the course text Secrets to a Simple Story. What stands out for you? Did it inspire you? Did it help you find a good story concept to film? Log these observations and reflections in your course journal.
\end{reflect}

\hypertarget{brainstorming-and-successful-creativity}{%
\section{Brainstorming and Successful Creativity}\label{brainstorming-and-successful-creativity}}

The secret to a story sounds simple and it is. We need to create a character with a goal and stakes (the painful consequences that will happen to the character) who has to overcome opposition to that goal and an ending that reveal whether or not the character obtains the goal.

However, the challenge for most of us is, what story do I tell?

Brainstorming is a great tool and way to discover and find the right story you want to tell. Brainstorming involves the spontaneous development of ideas. Brainstorming can be done alone or within a group.

The great thing about brainstorming is that you do not have to judge and criticize the process and results. In addition, you are not trying to perfect. You are only looking for that one idea that sparks your creativity and you ignore the others. This non-judgmentalism and jettisoning of perfectionism is important because these two tendencies stifle creativity.

Even if you already have a story idea that you want to film you are strongly encouraged to engage in brainstorming activities during this section and the rest of the course. If you do, you unleash more of your creativity and you will most likely find a better story concept.

\hypertarget{learning-activities-20}{%
\subsection*{Learning Activities}\label{learning-activities-20}}
\addcontentsline{toc}{subsection}{Learning Activities}

\begin{reflect}
\hypertarget{brainstorming}{%
\subsubsection*{Brainstorming}\label{brainstorming}}
\addcontentsline{toc}{subsubsection}{Brainstorming}

Study the Brainstorming section in Chapter 6, including the Steps for Successful Brainstorming and the Tips for Success Creativity. Then consult these resources:

\begin{itemize}
\tightlist
\item
  30 Ways to Brainstorm Short Film Ideas You Can Actually Produce
\item
  19 Great Ways to Brainstorm Short Film Ideas
\end{itemize}

After you feel you have a good grasp of brainstorming, engage in the process and come up with \ul{20 story concepts} that fit the criteria in the previous topic:

\begin{itemize}
\tightlist
\item
  One or two main characters.\\
\item
  A single problem or conflict.\\
\item
  Knowing your genre (will your film be funny, dramatic, scary, etc.?).\\
\item
  A simple setting or number of locations.\\
\item
  A satisfying ending.
\end{itemize}

After you have done this go through your concept list and choose your top one and apply it to the template in the next topic.
\end{reflect}

\begin{caution}
\textbf{Helpful Hint}: If you find yourself stuck and can't come up with a story concept, tell your facilitator and have a group brainstorming session. Take one of the ideas and develop it. Share your concept if you are stuck finding one of the criteria such as a goal or stakes ask for brainstorming help to solve the problem. Remember, film is mostly a collaborative venture.
\end{caution}

\hypertarget{short-film-story-template}{%
\section{Short Film Story Template}\label{short-film-story-template}}

Once you have your core story concept you now need to make it into a full story with a beginning, middle, and end. We also need to make sure we do not make it too complicated or muddled.

Many of us have the tendency to overly complicate things: to add extraneous detail or to have elements that are not clear well explained. The template is this section is designed to prevent these issues as you use a fairy tale structure to find your characters, the problem they are facing, and show what happens in a clear way with a beginning, middle, and end of the story.

Have fun with this template process. Also do not forget to use brainstorming for each section of the template so that you come up with the best characters, situations, locations, and conflict which are not predicable or have a ``been there, done that'' (boring) quality.

\hypertarget{learning-activities-21}{%
\subsection*{Learning Activities}\label{learning-activities-21}}
\addcontentsline{toc}{subsection}{Learning Activities}

\begin{reflect}
\hypertarget{planning-your-film}{%
\subsubsection*{Planning Your Film}\label{planning-your-film}}
\addcontentsline{toc}{subsubsection}{Planning Your Film}

Once you have brainstormed and have a clear and solid concept film, e.g., a student who has the power to turn things in gold, use the template below and write your story for your short film by filling in the blanks.

Study the illustration example given in the chapter and then apply the process to the creation of your film story. In crafting your story you are not allowed to use dialogue beyond ``yes,'' ``no,'' and ``okay.'' That is, you must find visual ways to establish your story's setup and context and its build and payoff. For example, you cannot have you character say ``I am on my way to the store.'' You will just show him or her getting in the car, driving, arriving at the store parking lot, and entering the store. (This has visual direction vectors discussed below in Topic 5.)

\hypertarget{fairy-tale-template-for-a-short-film}{%
\subsubsection*{Fairy Tale Template for a Short Film}\label{fairy-tale-template-for-a-short-film}}
\addcontentsline{toc}{subsubsection}{Fairy Tale Template for a Short Film}

(This template is inspired by the one developed by Alexander Mackendrick. See \emph{On Filmmaking: An Introduction to the Craft of the Director}. London: Faber \& Faber, 2005, pgs. 78-85.)

\textbf{Once Upon a Time \_\_\_\_\_\_\_\_\_.}
\emph{(Establish where your story will take place.)}

\textbf{There Lived a \_\_\_\_\_\_\_\_\_\_\_\_\_.}
\emph{Establish the main character, hero, or protagonist.}

\textbf{But there was a Problem in the Land \_\_\_\_\_\_\_\_\_\_\_\_.}
\emph{Establish the inciting incident that starts the story's main conflict that the protagonist must face and overcome.}

\textbf{That the Hero had to Solve \_\_\_\_\_\_\_\_\_\_.}
\emph{Establish the main character's drive, motive, goal, and stakes.}

\textbf{By Defeating a Villain or Evil Force \_\_\_\_\_\_\_\_\_\_.}
\emph{Establish the chief antagonist the protagonist has to overcome.}

\textbf{That resulted in Battles and Obstacles \_\_\_\_\_\_\_\_\_\_\_.}
\emph{Establish the problems and difficulties the protagonist must overcome to achieve the goal.}

\textbf{The Result of Which \_\_\_\_\_\_\_\_.}
\emph{Establish the chain of cause and effect conflict events that form the middle of the story. These should escalate, that is, grow more intense and challenging as the action develops.}

\textbf{Produced Twists and Turns \_\_\_\_\_\_\_\_\_\_.}
\emph{Establish the surprises and unexpected turn of events as the protagonist battles his or her foe.}

\textbf{Until the Time came that \_\_\_\_\_\_\_\_.}
\emph{Establish the obligatory scene or major confrontation or climax of the action.}

\textbf{When Suddenly \_\_\_\_\_\_\_\_\_\_\_.}
\emph{Establish the final major twist or surprise that heightens the action of the climax. This is optional.}

\textbf{And it Turns Out that \_\_\_\_\_\_.}
\emph{Establish the resolution or wrapping up of the story.}

\textbf{And Forever After (or Not) \_\_\_\_\_\_\_\_\_.}
\emph{Establish the end of the story and its closure which can be happy or sad.}

Are you happy with your story? Do you have to make revisions? You should go through at least three drafts of your story to make sure it works. Test your story out on you family and friends. Ask them if it works but remember they are not filmmakers or professional critics so ignore things when they are too petty or too subjective.

Read through the section in Chapter 6 on creating a Short Story Film Template.

Be sure to lay out your story in single lines of action that can be filmed as in the example of PANIC in Chapter 6.
\end{reflect}

\hypertarget{direction-vectors-and-eyeline-requirements}{%
\section{Direction Vectors and Eyeline Requirements}\label{direction-vectors-and-eyeline-requirements}}

Now that you have your story and one-line action descriptions, and have your actors, props, and locations set-up, you are ready to film your story (the In-Camera film project) with your cell phone.

Before doing so, this exercise will require that you focus on several things found in the Direction Vectors and Eyeline Requirements section:

\begin{itemize}
\tightlist
\item
  Proper Lead Room\\
\item
  Motivated Shot Movement\\
\item
  Consistent Screen Direction\\
\item
  Proper Eye-Line Vectors\\
\item
  Proper Headroom
\end{itemize}

These requirements should not worry you since you have explored and tried many of these techniques in the composition and film movements units.

\hypertarget{learning-activities-22}{%
\subsection*{Learning Activities}\label{learning-activities-22}}
\addcontentsline{toc}{subsection}{Learning Activities}

\begin{reflect}
\hypertarget{resources-on-direction-vectors-and-eyeline-requirements}{%
\subsubsection*{Resources on Direction Vectors and Eyeline Requirements}\label{resources-on-direction-vectors-and-eyeline-requirements}}
\addcontentsline{toc}{subsubsection}{Resources on Direction Vectors and Eyeline Requirements}

Read the next section in Chapter 6. In order to do well and follow the direction vectors and eyeline requirements for this first film project, be sure to consult the following resources which explain them in visual ways. \ul{As you do so, pre-visualize how you will film your project}:

In addition, be sure to review Units 3, 4, and 5 to make sure your shots will be well composed and that you will incorporate a variety of camera angles.

Happy filming!
\end{reflect}

\hypertarget{summary-5}{%
\section*{Summary}\label{summary-5}}
\addcontentsline{toc}{section}{Summary}

In this unit, you learned about:

\begin{itemize}
\tightlist
\item
  The In-Camera Project and Its Benefits\\
\item
  Secrets to a Simple Story\\
\item
  Brainstorming and Successful Creativity\\
\item
  Short Story Film Template\\
\item
  Direction Vectors and Eye-line Requirements\\
\item
  In-Camera Exercise Guidelines
\end{itemize}

\hypertarget{assessment-5}{%
\section*{Assessment}\label{assessment-5}}
\addcontentsline{toc}{section}{Assessment}

\begin{assessment}
\hypertarget{assignment-1}{%
\subsection*{Assignment :}\label{assignment-1}}
\addcontentsline{toc}{subsection}{Assignment :}

\textbf{Film Journal}
After completing this unit, including the learning activities, you are asked to make sure you are doing journal entries and when appropriate to share your responses with your facilitator and classmates when you meet.
\end{assessment}

\begin{caution}
\textbf{Note:} that entries are expected for every unit. Your journal reflections are submitted at the end of the course as part of the Final Exam: Self Assessment worth 30\% of your course grade.
\end{caution}

\begin{assessment}
\textbf{In-Camera Exercise (10\%)}

Having practiced working on composition and motion shots and learning about the variety and types of shot sizes and camera angles, you are now ready to apply all this knowledge and make your first film. Though it will be challenging, do not fear. You can do it.

For this exercise project you make an in-camera mini-movie. That is, take the story you have developed in this chapter and film it. The catch is that you must do it in sequence: ``in camera.'' Put another way, you must shoot the film in the order that it will be screened and you cannot edit it beyond butting shots up against each other and taking out the call for action or call for cut at the end of the shot.

The reason for this constraint of shooting all of the shots in their proper order and not using any editing beyond cutting the shots together is to help you intentionally plan your shots so that their meaning and mood flow and connect well.

This will most likely be the last time you will film this way, given that screenplays are shot out of sequence according to location, crew, cast, and equipment needs. But here your job is to plan a simple mini-movie and work on using composition, motion shots, and film grammar to visually tell your story with no dialogue or added music or sound effects. This might sound intimidating, but if you have been diligent in your previous film exercises work, you will soar.

The following guidelines must be adhered for your In-Camera film project:

\textbf{Guidelines for In-Camera Project}

\begin{enumerate}
\def\labelenumi{\arabic{enumi}.}
\tightlist
\item
  You must explore and experiment with visual-centered storytelling. That is, make a film where the images and shots tell the story without the use of dialogue and verbal explanation.\\
\item
  There must be no dialogue. A character can say ``yes,'' ``no,'' or ``okay'' if receiving a phone call but you cannot have the character explain things. Work on letting your visuals do this.\\
\item
  This assignment must be an ``In-Camera'' project, that is, one that is shot in linear sequence from the beginning to the end. It should be about 1.5 to 2.5 minutes long. It can be a bit shorter or longer but not much.\\
\item
  Your mini-movie should have a clear beginning, middle, and end as per the template provided earlier.\\
\item
  You must choose a genre such as comedy, drama, horror, fantasy, etc. Nothing too elaborate, but we should have a feel for the type of mini film you have done. (We will discuss and cover genre in more detail later.)\\
\item
  You must use a variety of shot sizes, heights, angles, framing, perspectives, etc. that reveal your are understanding of film grammar and syntax. There should be multiple different types of shots and camera angles.\\
\item
  You must have some linear movement where you use motion shots to follow the action. Review the previous sections on motion shots.\\
\item
  Your shots must have:
\end{enumerate}

\begin{itemize}
\tightlist
\item
  Proper Lead Room\\
\item
  Motivated Shot Movement\\
\item
  Consistent Screen Direction\\
\item
  Proper Eye-Line Vectors\\
\item
  Proper Headroom
\end{itemize}

\begin{enumerate}
\def\labelenumi{\arabic{enumi}.}
\setcounter{enumi}{8}
\tightlist
\item
  You must make sure your shots are motivated. Remember motivated means the camera and subject motion are matched or are in sync in a way that draws attention to the subject action and not the camera movement. Motivated also means the story and character situations justify your choice of shots.\\
\item
  You are not allowed to edit the film and move shots into different positions on your timeline. You can trim the beginning of a shot where you call ``action'' and the end when the shot goes longer than needed. You are not allowed to use transitions like dissolves and fades (covered later). You must only use cuts to butt the shots up against each other. You can use Premiere, Final Cut, Imovie, Fiolmora, Windows Movie Maker, or other software. There are plenty of tutorials on Youtube to show you how to use most software. Don't waste your time on junky tutorials. You should be able to know a professional presentation when you see one.\\
\item
  The only music that can be used is that which is ``diegetic,'' that is, sound which has a source on-screen in your shots, e.g., a CD player in a car, Alex playing in the background, etc.\\
\item
  You must shoot this project yourself. Later you will divide into teams where someone shoots and someone directs but for this exercise you must produce, write, direct, shoot, and edit this project by yourself. However, you can have someone show you how to use editing software if you need it.\\
\item
  Given that you cannot edit this film, make sure you rehearse the shots to get them right. Your actors will need at least two or three takes to get the movement right.
\item
  This project can be shot on your cell phone or with a DSLR camera if you have one.\\
\item
  Remember: The important thing is to focus on your story and telling it in an interesting visual way through good compositions, motivated movement, a variety of shot sizes and camera angles, etc. The ``Fairy Tale'' narrative pattern can help you create a solid story. Go for a ``twist,'' or surprised ending, if you can.\\
\item
  Please don't wait until the last minute to do this assignment or you will waste your time and ours if you rush it. Creating story and filming it well will take more time than you think. And Murphy's Law loves to bite procrastinators.\\
\item
  You must submit a write-up of what you attempted to do and what you learned from this exercise, especially from the feedback you received from your instructor and peers. This write-up must be emailed to \href{mailto:ned.vankevich@twu.ca}{\nolinkurl{ned.vankevich@twu.ca}} by the due date which will be determined during the course.\\
\item
  Be sure to upload your film to the course Cloud folder (this link will be sent to you) and label your file with your name, course, and the name of the assignment. E.g., Alfred Hitchcock, MCOM 221, In-Camera Project N.B.\\
\item
  Backup this assignment and all others by either emailing your assignment and images to yourself or copying them to an external drive, flash drive, the Cloud, etc.\textbf{This course will not accept excuses for lost or destroyed assignments,} such as my computer died or my files disappeared. Backing up work is the sign of a professional.\\
\item
  You will be graded on this activity as per the following guidelines.
\end{enumerate}

\textbf{Grading Criteria for the In-Camera Assignment}

You will be given verbal feedback in the Learning Lab by your instructor and peers after you show your film. You will also receive written feedback by your instructor on your write-up submission.

The following questions will be discussed by your peers and considered by your instructor concerning your film:

\begin{itemize}
\tightlist
\item
  Did you communicate a story with a clear beginning, middle, and end?\\
\item
  Did you express a clear genre and was the coverage (type of shots used) appropriate for it?\\
\item
  Did your film reveal that you understand how to use film grammar and syntax?\\
\item
  Did you have a sufficient variety of shot sizes, angles, heights, framing, etc.?\\
\item
  Were your motion shots motivated?\\
\item
  Were the shots in focus and were they smooth or did they have bumps or unnecessary movements or elements?\\
\item
  Did you follow the linear action of the story well?\\
\item
  Were your shots well composed? Appropriate lead room, head room, eye vectors, etc.?
\end{itemize}

\begin{enumerate}
\def\labelenumi{\arabic{enumi}.}
\tightlist
\item
  You will be adjudicated on how well you follow the ``classical'' approach to camera work and storytelling we have covered, how creative and compelling your story and its execution are within the limits of an introductory course, and how well you follow and execute the guidelines above.\\
\item
  This project will count as 10\% of you grade and the breakdown will be:
\end{enumerate}

\begin{itemize}
\tightlist
\item
  Story 5\%\\
\item
  Visual execution 5\%\\
\end{itemize}

\begin{enumerate}
\def\labelenumi{\arabic{enumi}.}
\setcounter{enumi}{2}
\tightlist
\item
  Beyond following the guidelines above, your mini-film will merit the following grades:
\end{enumerate}

A \textbar{} Excellent camera shots and compositions that tell a compelling story. \textbar{}\\
- \textbar{} \textbar{}\\
B \textbar{} Above average camera shots, compositions, and story. \textbar{}\\
C \textbar{} Average camera shots, compositions, and story. \textbar{}\\
D \textbar{} Unacceptable camera shots, compositions, and poorly done story. \textbar{}\\
E \textbar{} Failure to do the activity at all levels. \textbar{}

See the course syllabus for more detail on the qualitative criteria for a grade.

Now that you know what must be done and how you will be evaluated, check to make sure your story can accomplish the above requirements and that you can execute it accordingly.

\hypertarget{assignment-reflection}{%
\subsubsection*{Assignment Reflection}\label{assignment-reflection}}
\addcontentsline{toc}{subsubsection}{Assignment Reflection}

Log in your journal what you learned from the creating and filming of your In-Camera exercise. Log what you thought about the films of your peers and what you learned from the feedback from them and your instructor.

Be sure to make a note of what films stood out and who did them. You will consult this at the end of the course when the class determines: Best Overall Film. Most Imaginative Film, Best Story, Best Cinematography, Best Editing, etc.
\end{assessment}

\hypertarget{checking-your-learning-5}{%
\section*{Checking your Learning}\label{checking-your-learning-5}}
\addcontentsline{toc}{section}{Checking your Learning}

\begin{progress}
Before you move on to the next unit, you may want to check to make sure that you are able to:

\begin{itemize}
\tightlist
\item
  Describe what constitutes a good, basic, and simple story\\
\item
  Define what constitutes an ``In-Camera'' film\\
\item
  Analyze and apply a story template\\
\item
  Determine how to tell a visual-centered story well\\
\item
  Create a short simple story to film
\end{itemize}
\end{progress}

\hypertarget{editing-and-montage}{%
\chapter{Editing and Montage}\label{editing-and-montage}}

\hypertarget{overview-6}{%
\section*{Overview}\label{overview-6}}
\addcontentsline{toc}{section}{Overview}

Congratulations. You've made your first mini-movie in this course. Don't worry if it did not live up to your expectations. The important thing for right now is that you wrote, produced, directed, and shot a short film and that you are improving your skills. The rest of your course will give you a chance to explore more elements of filmmaking and gain more experience.

In this unit you will focus on editing and creating a montage, a common form of filmmaking.

Unlike the previous project, you can now use dialogue, voice over, music, and sound effects to enhance your short film project. You will also use editing which will allow you to shoot your film out of order and then assemble it after you have all of the elements you need to make it.

As you saw in Unit 2, editing is the third major phase of filmmaking and a major skill to understand.

Don't worry if you do not want to be an editor. You will work in teams this time and as long as the work on the film is shared equally between yourself and your partner(s), you can offer editing suggestions but will not have to push the buttons. (This is how it works in the industry where the director and producer tell the editor what they want and the editor does the technical work.)

\hypertarget{topics-6}{%
\subsection*{Topics}\label{topics-6}}
\addcontentsline{toc}{subsection}{Topics}

This unit is divided into the following topics covered in the course text:

\begin{enumerate}
\def\labelenumi{\arabic{enumi}.}
\tightlist
\item
  Post-Production Workflow\\
\item
  Montage\\
\item
  Picture Cutting Techniques\\
\item
  Types of Montage\\
\item
  Digital Video Editing Software\\
\item
  Steps for Creating Your Montage\\
\item
  Pitching Your Montage Project\\
\item
  Steps for Creating Your Montage
\end{enumerate}

\hypertarget{learning-outcomes-6}{%
\subsection*{Learning Outcomes}\label{learning-outcomes-6}}
\addcontentsline{toc}{subsection}{Learning Outcomes}

When you have completed this unit, you should be able to:

\begin{itemize}
\item
  Define the Post-Production workflow
\item
  Describe how to pitch a project well
\item
  Analyze the necessary elements needed for your montage
\item
  Determine the steps needed in making a montage
\item
  Create an effective montage
\item
  Evaluate the quality of a montage
\end{itemize}

\hypertarget{activity-checklist-6}{%
\subsection*{Activity Checklist}\label{activity-checklist-6}}
\addcontentsline{toc}{subsection}{Activity Checklist}

\begin{reflect}
Here is a checklist of learning activities you will benefit from in completing this unit. You may find it useful for planning your work.

\begin{itemize}
\tightlist
\item
  Log in your journal what type of film you want to work on for your final film project and why.
\item
  Read the first section in Chapter 7, the Post-Production Work Flow, and watch the selected video.
\item
  Read about the Montage in the course text and view ``20 Montages That Might Be The Best Part Of The Movie.''
\item
  Read about Picture Cutting Techniques and view the examples listed.
\item
  Read about the types of montage projects in the textbook and view the example.
\item
  Read the Digital Editing Software section in the text and view the resources listed.
\item
  Review the resources on how to pitch your montage project.
\item
  Read the next section of Chapter 7 on creating a montage.
\item
  Feedback and Self-Assessment for Montage Project.
\end{itemize}

\textbf{Assessment:}

\begin{itemize}
\tightlist
\item
  Film Journal
\end{itemize}

\textbf{Assignment:}

\begin{itemize}
\tightlist
\item
  Montage Project (20\%)
\end{itemize}
\end{reflect}

\hypertarget{resources-6}{%
\subsection*{Resources}\label{resources-6}}
\addcontentsline{toc}{subsection}{Resources}

Here are the resources you will need to complete this unit.

\begin{itemize}
\tightlist
\item
  Chapter Seven course text\\
\item
  \href{https://www.youtube.com/watch?v=IFjyVW21Vpw}{Stages of Post Production for Filmmaking in Cinema}\\
\item
  \href{https://www.refinery29.com/en-us/best-movie-montages}{``20 Montages That Might Be The Best Part Of The Movie}\\
\item
  \href{https://www.youtube.com/watch?v=dak2DkfDTuU}{How To Pitch A Project}\\
\item
  \href{https://www.youtube.com/watch?v=OAH0MoAv2CI\&t=31s}{Cuts \& Transitions 101}\\
\item
  \href{https://www.youtube.com/watch?v=Wv3Hmf2Dxlo}{9 Cuts Every Video Editor Should Know \textbar{} Filmmaking Tips}\\
\item
  \href{https://www.youtube.com/watch?v=bdpbYeoZKnk}{VIDEO EDITING TRANSITIONS (Taylor Cut Tutorial)}\\
\item
  \href{https://www.youtube.com/watch?v=x5ASDkOSIBE}{Critiquing Your Short Films}\\
\item
  Other online resources provided in the course text and this unit.
\end{itemize}

\hypertarget{post-production-workflow}{%
\section{Post-Production Workflow}\label{post-production-workflow}}

There are many steps and elements involved in the post-production process. They include:

\begin{itemize}
\tightlist
\item
  Importing Footage and elements\\
\item
  Sound syncing\\
\item
  Picture cutting\\
\item
  Transitions\\
\item
  Dialogue editing\\
\item
  Sound design\\
\item
  Adding music\\
\item
  Adding sound effects\\
\item
  Adding visual and special effects\\
\item
  Color correcting\\
\item
  Adding tiles and graphics\\
\item
  Etc.
\end{itemize}

\hypertarget{learning-activities-23}{%
\subsection*{Learning Activities}\label{learning-activities-23}}
\addcontentsline{toc}{subsection}{Learning Activities}

\begin{reflect}
\hypertarget{planning-your-film-1}{%
\subsubsection*{Planning Your Film}\label{planning-your-film-1}}
\addcontentsline{toc}{subsubsection}{Planning Your Film}

Log in your journal what type of film you want to work on for your final film project and why.
\end{reflect}

\begin{caution}
\textbf{Note:} that this learning activity is ungraded, but is designed to help you succeed in your assessments in this course.
\end{caution}

\begin{reflect}
\hypertarget{read-reflect-and-view-1}{%
\subsubsection*{Read, Reflect and View}\label{read-reflect-and-view-1}}
\addcontentsline{toc}{subsubsection}{Read, Reflect and View}

Read the first section in Chapter 7, the Post-Production Work Flow. For a detailed overview of the post-production process view the following resource:
\end{reflect}

\begin{caution}
\textbf{Note:} that this learning activity is ungraded, but is designed to help you succeed in your assessments in this course.
\end{caution}

\hypertarget{montage}{%
\section{Montage}\label{montage}}

Montage involves a type of editing where a series of images and sound elements are most often used used to condense time, create emotion, tell a story, reveal something from the past, promote something, or impart information.

Montages can stand alone or be part of a larger documentary, film or television story, or a stand-alone film.

You will focus on a montage project in this unit because they are a great way to learn the art of editing when you shoot images and edit them in timed rhythm to music, dialogue, and/or sound effects. In this exercise you are going to let emotions, feelings, and mood guide you as you learn to further develop your intuitive creative skills.

\hypertarget{learning-activities-24}{%
\subsection*{Learning Activities}\label{learning-activities-24}}
\addcontentsline{toc}{subsection}{Learning Activities}

\begin{reflect}
\hypertarget{movie-montages}{%
\subsubsection*{Movie Montages}\label{movie-montages}}
\addcontentsline{toc}{subsubsection}{Movie Montages}

Read the second section of your text on the Montage. To get a feeling for how montages work and for how effective they can be view this resource:

\href{https://www.refinery29.com/en-us/best-movie-montages}{``20 Montages That Might Be The Best Part Of The Movie''}
\end{reflect}

\hypertarget{picture-cutting-techniques}{%
\section{Picture Cutting Techniques}\label{picture-cutting-techniques}}

This montage project is helpful because you have to focus on a few elements: the images you will use and the music and/or sound effects that will provide the beats and rhythm for your picture cutting.

Although the music will guide your editing as you cut on beats, the images you edit will be important. Once again, good composition and motivate motions shots will play a vital role as you shot your footage for your montage project. You can also used ``found'' or archival footage, or news stories as your visual sources in a montage.

There are many types of cuts and transitions that can be used in a montage as you splice and stich your footage together including:

\begin{itemize}
\tightlist
\item
  \textbf{Cuts}\\
\item
  \textbf{Jump-Cuts}\\
\item
  \textbf{Cross-Cutting}\\
\item
  \textbf{Dissolves}\\
\item
  \textbf{Wipes}\\
\item
  \textbf{Fades}
\end{itemize}

You can use J-cut and L-cuts when you are working with dialogue or you want to lead into a new shot or scene or carry the audio to a new shot or scene. If you are interested in doing this, see:

\hypertarget{learning-activities-25}{%
\subsection*{Learning Activities}\label{learning-activities-25}}
\addcontentsline{toc}{subsection}{Learning Activities}

\begin{reflect}
\hypertarget{picture-cutting-techniques-1}{%
\subsubsection*{Picture Cutting Techniques}\label{picture-cutting-techniques-1}}
\addcontentsline{toc}{subsubsection}{Picture Cutting Techniques}

Read the text section on Picture Cutting Techniques. For visual examples of the above techniques view this resource: \textbf{Cuts \& Transitions 101}
\href{https://youtu.be/OAH0MoAv2CI}{plugin:youtube}

See also: \textbf{9 Cuts Every Video Editor Should Know \textbar{} Filmmaking Tips}
\end{reflect}

\hypertarget{types-of-montage-projects}{%
\section{Types of Montage Projects}\label{types-of-montage-projects}}

You are given creative latitude and freedom for the type of montage you will shoot and edit for this unit. Some of the most popular ones include:

\begin{itemize}
\tightlist
\item
  \textbf{Poetic Montages}\\
\item
  \textbf{Nature Montages}\\
\item
  \textbf{Street Montages}\\
\item
  \textbf{Sequence Sprint}\\
\item
  \textbf{Music Videos}\\
\item
  \textbf{Aesthetic Montages}
\end{itemize}

You can also combine these types of montages. As you read the description of these different types, have fun choosing the one you want to do.

\hypertarget{learning-activities-26}{%
\subsection*{Learning Activities}\label{learning-activities-26}}
\addcontentsline{toc}{subsection}{Learning Activities}

\begin{reflect}
\hypertarget{types-of-montage-projects-1}{%
\subsubsection*{Types of Montage Projects}\label{types-of-montage-projects-1}}
\addcontentsline{toc}{subsubsection}{Types of Montage Projects}

Read the Types of Montage Projects in the textbook. For an excellent example of a nature and time-lapse montage where a series of images are synced in to a powerful sound track view this resource: \textbf{Pursuit - A 4K storm time-lapse film}

As you view it, pay attention to the way the types and rhythms of the images flow and work in sync with the music and sound effects especially from 2: 29 onward.

What did you learn about this montage? Log this in your journal.
\end{reflect}

\hypertarget{digital-video-editing-software}{%
\section{Digital Video Editing Software}\label{digital-video-editing-software}}

You are now at a point where you will have to import your cellphone or DSLR footage into a software program that will allow you to edit it. Digital video editing software is helpful because it allows you to edit your images and sync them to the rhythm and beats of your sound track and export your project for viewing or streaming on the Internet.

There are many types and brands of editing software in the marketplace and you will have to choose the one that works for you. Some applications only work with Android-based cellphones and others for Apple ISO devices and some for both platforms such as imovie. If you want to really get series about editing you should explore Final Cut Pro, Adobe Premiere, Avid, DaVinci Resolve, or some other professional program.

There are similarities among editing software and programs but you will have to do a tutorial on YouTube or the manufacture's website of the one you will use to see how to work it. The more popular the software, the more tutorials will be available.

\hypertarget{learning-activities-27}{%
\subsection*{Learning Activities}\label{learning-activities-27}}
\addcontentsline{toc}{subsection}{Learning Activities}

\begin{reflect}
\hypertarget{digital-editing-software}{%
\subsubsection*{Digital Editing Software}\label{digital-editing-software}}
\addcontentsline{toc}{subsubsection}{Digital Editing Software}

Read the Digital Editing Software section in the text.

Find the digital video editing program you want to use and learn how to do basic editing with it. Basic editing using cuts, dissolves, fades, and wipes titling will be all you need to learn for your montage project.

The Filmora website gives an overview of the type of features you want to look for in video editing software.

For free and inexpensive Android-based video software see:
10 Best Android Video Editor Apps Of 2019

Or search for best cellphone or DSLR video editing software such as Best Picks 12 Best Video Editing Apps for Phones and Tablets .
\end{reflect}

\begin{caution}
\textbf{Note}: When you pitch your montage project, see who already has editing software and who knows how to use it and try to join with them if you are a bit of a ``tech-phobic.'' If you can't find someone, challenge yourself and learn something new. You will learn and grow a lot doing this.
\end{caution}

\hypertarget{pitching-your-montage-project}{%
\section{Pitching Your Montage Project}\label{pitching-your-montage-project}}

Because the montage project is more complicated, you are encouraged to work in teams to create an excellent montage. You can do the project solo if you feel you have the experience and understanding of the gear and software to create it.

The reason almost all larger-scale film projects are done in teams with a lot of different crew positions is that over time, people specialize in their skill-sets and bring better expertise to a project. This is why film crews have a cinematographer, production designer, wardrobe, hair and make-up specialists, prop masters, special effects experts, etc. (You should be consulting your film terms website if you do not know these positions.)

For this project, ideally you will have a team of two people. One to come up with the concept and film it and the other to edit it. In reality, both members will share the idea development, shooting, and editing. The important thing is that you both share equally in the work. (In some cases if the montage is highly complex or sophisticated you can have 3 or 4 members on the team as long as everyone does their fair share.)

In order to find the find teammate, you will pitch your project to the class to see who wants to work on it.

Pitching is a process central to the film and television industry. It involves an individual or team presenting their film project concept and why it is important and to a special audience such as a producer, agent, studio or television executive, distributor, film competition, etc. Everyone, no matter how big or important, has to pitch to those in the industry to find the money, resources, actors, crew, and/or distribution for their project.

Pitching is a vital part of filmmaking and it is a great transferable skill to learn since you will have to pitch in many professions and industries today.

\hypertarget{learning-activities-28}{%
\subsection*{Learning Activities}\label{learning-activities-28}}
\addcontentsline{toc}{subsection}{Learning Activities}

\begin{reflect}
\hypertarget{pitching-your-project}{%
\subsubsection*{Pitching your Project}\label{pitching-your-project}}
\addcontentsline{toc}{subsubsection}{Pitching your Project}

You will have to develop a montage concept to pitch such as a day in the life of street vendors, or a music video, or visualizing a poem or something from the Bible. So review the brainstorming techniques in the previous unit and come up with at least 10 concepts and then choose the one that most appeals to you.

Then review the section on \emph{Pitching} in Chapter Two of the course text. Study \emph{How to Pitch Your Montage Project} and \emph{Steps for an Effective Pitch} in the Chapter 7 and practice doing your pitch before a friend or family member or to an imaginary audience before making the actual pitch to your course members.
\end{reflect}

\begin{caution}
\textbf{Note}: Do not skimp on practicing your pitch. The more you practice it you will gain confidence and know how to present your concept in an effective way.
\end{caution}

\begin{reflect}
\hypertarget{guidelines-for-pitching-a-film-project}{%
\subsection*{Guidelines for Pitching a Film Project}\label{guidelines-for-pitching-a-film-project}}
\addcontentsline{toc}{subsection}{Guidelines for Pitching a Film Project}

\textbf{Purpose of a Pitch:} To get people to work on your project, commission your
project, fund your project, help with your project, distribute your project,
etc. Pitches also help you to clarity what you want to do and why.

\textbf{Steps to a Good Pitch}

\begin{enumerate}
\def\labelenumi{\arabic{enumi}.}
\tightlist
\item
  If you can, open with a teaser or dramatic statement or question.\\
\item
  Meet and greet your audience---minimum chitchat, professionals are busy, busy people.\\
\item
  Dress appropriately---relative to the audience and the project. Artists dress ``artsy''; corporate people dress with business attire.\\
\item
  Project the proper persona (your personality and character)---knowledgeable,credible, trustworthy, smart, clever, etc.\\
\item
  Explain your qualifications, background, and why you are competent and the right person to do the project. If you do not have a demo reel then sell people with your passion, e.g., WHIPLASH filmmaker.\\
\item
  Give the title of the film/project---titles shape perception and expectations.\\
\item
  State the Genre---comedy, drama, documentary, etc. This also shapes perceptions and expectations. The basic concept of a comedy should make us laugh or at lease smile.\\
\item
  Give the Unique Angle or Hook of your project---what makes it stand out as something different and catchy/edgy/clever, etc. E.g., SHOE IN LOVE, a romantic comedy from the POV of a pair of cowboy boots and stilettos falling in love with each other. The special angle concept is to use foot ware as the main characters and the execution hook is to shoot POVs at foot level or from the shin down.\\
\item
  Briefly state the logline or basic concept: a one or two sentence description of the core story. E.g., a young, bored dirt farmer gets a message that he has to help rescue a faraway princes and save the universe. STAR WARS.\\
\item
  Project synopsis---succinctly tell the beginning/middle/end of your story or if a documentary the main message of your doc.\\
\item
  Summarize your end goal: to wow my audience with extraordinary nature photography or to make my audience laugh, cry, fall in love, etc.\\
\item
  Explain what you need---cast, crew, funding, etc.\\
\item
  Ask for Questions---also be open to criticism and constructive comments and feedback.\\
\item
  Gratitude---be thankful and gracious even if your project is not received well or as anticipated. Burning bridges is wrongheaded. Pitching can be a training ground for how to deal with rejection.
\end{enumerate}

\emph{Examples of Types of Montage Projects to Pitch}

The goal of your Montage Short assignment (see the syllabus) will be to explore how to make an effective montage-based film that uses picture and sound editing to convey meaning, emotion, story and/or mood.

For this project, you can:

\begin{itemize}
\tightlist
\item
  Create a music video where you take a song and create images that express the mood, feeling, meaning, etc. of the song.
\item
  Find a poem, fairy tale, or passage from the Bible and ``visualize'' an enhance it with music and sound effects.
\item
  Create a chase scene---someone steals something and is on the run and using music and sound effects to enhance the excitement and dynamism.
\item
  Create a love or horror or thriller story ---use can use dialogue but there must be an edited montage---a series of edited shots that convey the mood, meaning etc.
\end{itemize}
\end{reflect}

\hypertarget{steps-for-creating-your-montage}{%
\section{Steps for Creating Your Montage}\label{steps-for-creating-your-montage}}

Once you have your montage project and team in place, you are ready to make your montage. As we have seen, there are three main phases of film production and they will apply here:

\begin{itemize}
\tightlist
\item
  Montage Pre-Production\\
\item
  Montage Production\\
\item
  Montage Post-Production
\end{itemize}

The lion's share of the emphasis in the chapter has been on post-production, but this does not mean you must take lightly your pre-production and production work. Your montage will only be as good as the images you have for it.

\hypertarget{learning-activities-29}{%
\subsection*{Learning Activities}\label{learning-activities-29}}
\addcontentsline{toc}{subsection}{Learning Activities}

\begin{reflect}
\hypertarget{creating-a-montage}{%
\subsubsection*{Creating a Montage}\label{creating-a-montage}}
\addcontentsline{toc}{subsubsection}{Creating a Montage}

Read the next section of Chapter 7 on creating a montage. As you study montage pre-production, production, and post-production make notes on your shooting script of what to watch out for and not forget as you film the shots and images you will use.

\hypertarget{feedback-and-self-assessment-for-montage-project}{%
\subsubsection*{Feedback and Self-Assessment for Montage Project}\label{feedback-and-self-assessment-for-montage-project}}
\addcontentsline{toc}{subsubsection}{Feedback and Self-Assessment for Montage Project}

Once you have edited your project and exported it, you are ready to show it to your classmates and instructor. This should be an exciting time as you screen the fruits of your labor and creativity.

As mentioned previously, film is a public medium and subject to critique and evaluation.

The following criteria can help us better evaluate our own work and that of others and to give constructive criticism on how to improve:

\begin{itemize}
\tightlist
\item
  Overall did the montage work?\\
\item
  Did the music and/or sound effects work well with the images?\\
\item
  Did the montage convey a mood, emotion, a story and/or theme (some insight into life)? If so, what was it?\\
\item
  Were the shots well composed, motivated, and appropriately smooth or in some cases jarring depending on what is being communicated?\\
\item
  Were the cuts timed well? If not, where did they not work well?\\
\item
  Were the transitions appropriate and effective?\\
\item
  Was the action followed well?\\
\item
  Was there a sufficient variety of shots?\\
\item
  Were some shots or images too repetitive?\\
\item
  Were the shots and images interesting and noteworthy?\\
\item
  What did you like best about the project?\\
\item
  What suggestions would you offer to make it better?
\end{itemize}

Evaluate and critique the work of your peers helping them to see what worked and did not work in their montage and why. When possible give suggestions of how something could have been done better. Remember the tone and substance of our critiques should be one that offers encouraging constructive insight and that helps each other to grow and improve their craft.

\textbf{Helpful Tip}: For insight into how to critique a short film see: \textbf{Critiquing Your Short Films}

Write in your journal what you learned about your project and yourself from the montage projects.

Use the following questions to guide you:

\begin{itemize}
\tightlist
\item
  What was the best lesson I learned?\\
\item
  What did I do well? Why?\\
\item
  What did not work out as well as I planned? Why?\\
\item
  What can I do better next time?\\
\item
  What was the quality of my experience working with a partner?\\
\item
  If I did not work with a partner, could the project have been better if I had one?\\
\item
  How did this experience help me grow as a person and as a professional?\\
\item
  What did I learn new about myself?
\end{itemize}

Be sure to note which films and filmmakers had the best cinematography, editing, story, creativity, etc. so that you can consult this when you vote on the awards at the end of the course.
\end{reflect}

\hypertarget{summary-6}{%
\section*{Summary}\label{summary-6}}
\addcontentsline{toc}{section}{Summary}

In this unit, you learned about\ldots{}

\begin{itemize}
\tightlist
\item
  Post-Production Workflow\\
\item
  Montage\\
\item
  Picture Cutting Techniques\\
\item
  Types of Montage\\
\item
  Digital Video Editing Software\\
\item
  Steps for Creating Your Montage\\
\item
  Pitching Your Montage Project\\
\item
  Steps for Creating Your Montage\\
\item
  Feedback and Self-Assessment for Montage Project
\end{itemize}

\hypertarget{assessment-6}{%
\section*{Assessment}\label{assessment-6}}
\addcontentsline{toc}{section}{Assessment}

\begin{assessment}
\hypertarget{assignment-2}{%
\subsection*{Assignment:}\label{assignment-2}}
\addcontentsline{toc}{subsection}{Assignment:}

\hypertarget{film-journal-1}{%
\subsubsection*{\texorpdfstring{\textbf{Film Journal}}{Film Journal}}\label{film-journal-1}}
\addcontentsline{toc}{subsubsection}{\textbf{Film Journal}}

After completing this unit, including the learning activities, you are asked to make sure you are doing journal entries and when appropriate to share your responses with your facilitator and classmates when you meet.
\end{assessment}

\begin{caution}
\textbf{Note:} that entries are expected for every unit. Your journal reflections are submitted at the end of the course as part of the Final Exam: Self Assessment worth 30\% of your course grade.
\end{caution}

\begin{assessment}
\hypertarget{assignment-montage-project-20}{%
\subsection*{Assignment: Montage Project (20\%)}\label{assignment-montage-project-20}}
\addcontentsline{toc}{subsection}{Assignment: Montage Project (20\%)}

This montage project will account for 20 percent of your grade and will be broken down according to the following:

\begin{itemize}
\tightlist
\item
  Story/Montage Concept: 20\%\\
\item
  Cinematography: 20\%\\
\item
  Editing: 40\%\\
\item
  Overall all project quality: 20\%
\end{itemize}

See the Assessment section of the course for more details.

\hypertarget{grading-criteria-4}{%
\subsubsection*{Grading Criteria:}\label{grading-criteria-4}}
\addcontentsline{toc}{subsubsection}{Grading Criteria:}
\end{assessment}

\hypertarget{checking-your-learning-6}{%
\section*{Checking your Learning}\label{checking-your-learning-6}}
\addcontentsline{toc}{section}{Checking your Learning}

\begin{progress}
Before you move on to the next unit, you may want to check to make sure that you are able to:

\begin{itemize}
\tightlist
\item
  Define the Post-Production workflow\\
\item
  Describe how to pitch a project well\\
\item
  Analyze the necessary elements needed for your montage\\
\item
  Determine the steps needed in making a montage\\
\item
  Create an effective montage\\
\item
  Evaluate the quality of a montage
\end{itemize}
\end{progress}

\hypertarget{creating-narrative-and-other-short-films}{%
\chapter{Creating Narrative and Other Short Films}\label{creating-narrative-and-other-short-films}}

\hypertarget{overview-7}{%
\section*{Overview}\label{overview-7}}
\addcontentsline{toc}{section}{Overview}

More kudos to you. You now have two film projects and a lot of film exercises in your experience bank account. You will now be moving up a level where you will have the opportunity to bring together all of what you have learned so far into a longer narrative film.

In the previous units you focused on how to make shorter films. In this unit and the next you will focus on creating longer forms where your film will be five to ten minutes in length. This might not seem long but the longer the film, the more you have to do to sustain interest and continually engage the audience. Mini-films (2-minutes and under) have less of this burden because an audience is not feeling like they are wasting a bigger chunk of their time if the film is not well done.

Narrative films, whether fiction or non-fiction, are popular because they tell stories. Even if you do not want to be a ``Hollywood'' filmmaker learning to tell film stories will aid you in your personal and professional development. Human beings are hard-wired for storytelling. It is a primary way we make sense of the world and pass our knowledge and cultural values and traditions to others. Those who control the narratives of a society have powerful influence and impact.

Before moving ahead, view this resource:

\textbf{7 Things to Know About Making Short Films! : FRIDAY 101}

\href{https://www.youtube.com/watch?v=mYnsKATCrdw}{plugin:youtube}

\hypertarget{topics-7}{%
\subsection*{Topics}\label{topics-7}}
\addcontentsline{toc}{subsection}{Topics}

This unit is divided into the following topics:

\begin{enumerate}
\def\labelenumi{\arabic{enumi}.}
\tightlist
\item
  Types of short films\\
\item
  Film Genres\\
\item
  The Logistics for the Final Project\\
\item
  Script and Story Development\\
\item
  Creating a Scriptment
\end{enumerate}

\hypertarget{learning-outcomes-7}{%
\subsection*{Learning Outcomes}\label{learning-outcomes-7}}
\addcontentsline{toc}{subsection}{Learning Outcomes}

When you have completed this unit, you should be able to:

\begin{itemize}
\tightlist
\item
  Describe the different types of short films\\
\item
  Define film genre and why it is important\\
\item
  Determine how to approach making a film script to shoot and edit.\\
\item
  Create a shooting script
\end{itemize}

\hypertarget{activity-checklist-7}{%
\subsection*{Activity Checklist}\label{activity-checklist-7}}
\addcontentsline{toc}{subsection}{Activity Checklist}

\begin{reflect}
Here is a checklist of learning activities you will benefit from in completing this unit. You may find it useful for planning your work.

Log in your journal what type of film you want to work on for your final film project and why.

\begin{itemize}
\tightlist
\item
  Review popular short film genres.
\item
  Follow the steps in \emph{Story Research} section of Chapter Eight and ask and answer the questions in the \emph{What to Look for} section. View the video selected.
\item
  Brainstorm and develop a core story idea to pitch to the class.
\item
  Create a scriptment with your group.
\end{itemize}

\textbf{Assessment}

\begin{itemize}
\tightlist
\item
  Film Journal
\end{itemize}

\textbf{Assignment:}

\begin{itemize}
\tightlist
\item
  Planning for the Final Film Project (Due end of Unit 9)
\end{itemize}
\end{reflect}

\hypertarget{resources-7}{%
\subsection*{Resources}\label{resources-7}}
\addcontentsline{toc}{subsection}{Resources}

Here are the resources you will need to complete this unit.

\begin{itemize}
\tightlist
\item
  Chapter Eight of the course text\\
\item
  7 Things to Know About Making Short Films! : FRIDAY 101\\
\item
  What Is Genre and How Is It Determined? \\
\item
  How to Write a Short Film \\
\item
  Writing 101: Basic Story Structure \\
\item
  \ul{Poetics} by Aristotle, translated by S. H. Butcher, The Internet Classics Archive \\
\item
  Outlines, Treatments, and Scriptments, Oh My! by Ken Miyamoto \\
\item
  Other resources will be available online and in the course text.
\end{itemize}

\hypertarget{types-of-film}{%
\section{Types of Film}\label{types-of-film}}

There are three main types of short films that we can do for the project ahead:

\begin{itemize}
\item
  \textbf{Classical Narrative}
\item
  \textbf{Documentary}
\item
  \textbf{Experimental, \emph{Avant-Garde}, and Surreal Cinema}
\end{itemize}

Most of you will probably choose the classical narrative, but you are free to do the other two with the caveat that experimental/\emph{avant-garde} films are far more challenging than they might appear. If you choose this type discuss it with your instructor.

As a transferable skill, like in many industries and businesses, smart filmmakers look for a market before they commit too much time and money to a film project. Others of course rely on their intuition and passion. Either way, it is wise to see what those who distribute or stream and screen short films, such as festivals and competitions, look for in making their choices. Their criteria can you vet the quality of your story and its execution.
See Film Shortage's \href{https://filmshortage.com/what-we-like/}{What Type Of Short Films Do We Prefer?}

\hypertarget{learning-activities-30}{%
\subsection*{Learning Activities}\label{learning-activities-30}}
\addcontentsline{toc}{subsection}{Learning Activities}

\begin{reflect}
\hypertarget{planning-your-final-film-project}{%
\subsubsection*{Planning your Final Film Project}\label{planning-your-final-film-project}}
\addcontentsline{toc}{subsubsection}{Planning your Final Film Project}

Log in your journal what type of film you want to work on for your final film project and why.
\end{reflect}

\begin{caution}
\textbf{Note:} that this learning activity is ungraded, but is designed to help you succeed in your assessments in this course.*
\end{caution}

\hypertarget{film-genres}{%
\section{Film Genres}\label{film-genres}}

The main films genres include:

\begin{itemize}
\tightlist
\item
  Drama\\
\item
  Comedy\\
\item
  Romance\\
\item
  Action\\
\item
  Thriller\\
\item
  Horror\\
\item
  Gangster\\
\item
  Crime\\
\item
  Adventure\\
\item
  Westerns\\
\item
  Sci-Fi\\
\item
  Fantasy\\
\item
  Historical\\
\item
  Epic\\
\item
  War\\
\item
  Bio Pics (biographies)
\end{itemize}

Knowing your film genre is important because it embodies narrative patterns and expectations audiences have, and if these are not fulfilled you will lessen your impact on those watching your film. Put another way, comedies need to be funny, horror films need to induce fear, romantic films need to inspire us to love, etc. Commercially, if these basic requirements are not met it can result in financial disaster and loss of reputation. At your level it will mean your film is not as effective as it could be.

Knowing your genre will also help you develop your story and script in this and the following unit.

\hypertarget{learning-activities-31}{%
\subsection*{Learning Activities}\label{learning-activities-31}}
\addcontentsline{toc}{subsection}{Learning Activities}

\begin{reflect}
\hypertarget{popular-short-film-genres}{%
\subsubsection*{Popular Short Film Genres}\label{popular-short-film-genres}}
\addcontentsline{toc}{subsubsection}{Popular Short Film Genres}

View this list of loglines (short one line descriptions of a film story) in various genres: Popular Short film genres.

Which genre appeals to you most? Why? Log this reflection and start to brainstorm genre-based story concepts.

To improve your grasp of genre consult this resource:

What Is Genre and How Is It Determined?
\end{reflect}

\hypertarget{logistics-for-the-final-film-project}{%
\section*{Logistics for the Final Film Project}\label{logistics-for-the-final-film-project}}
\addcontentsline{toc}{section}{Logistics for the Final Film Project}

This last film is a major project and will account for a lot of your grade given that it will reveal what you learned during this course. In light of this, you will need to create a strong story and the ``blueprint'' or script for filming it.

To save time so that you can put more of your energy into making the film, you will not have to develop a full screenplay (hence the scriptment section below) for the final project, but you are strongly encouraged to do so.

However, this does not mean you do not have to pitch a core story concept (a basic story, its genre, and hook or unique angle) to the class. Everyone will be required to do this so that you can gain more experience with your creativity and your pitching skills.

\hypertarget{learning-activities-32}{%
\subsection*{Learning Activities}\label{learning-activities-32}}
\addcontentsline{toc}{subsection}{Learning Activities}

\begin{reflect}
\hypertarget{story-research}{%
\subsubsection*{Story Research}\label{story-research}}
\addcontentsline{toc}{subsubsection}{Story Research}

In the next activity, you'll be asked to brainstorm and develop a core story idea to pitch to the class.

To help you do this, follow the steps in Story Research Section of Chapter Eight and ask and answer the questions in the What to Look for Section. This should prime your creative pump for the next topic. Also view this helpful resource:
\end{reflect}

\hypertarget{script-and-story-development}{%
\section{Script and Story Development}\label{script-and-story-development}}

Our goal in this unit is to find and create a strong story to film. As we have seen, brainstorming is a great method to help do this.

Once you have created a story concept that inspires you pitch it to the class to see if other classmates want to join you in your project.

Once you have your team in place you will then follow the process for creating a shooting script for pre-production and production.

\hypertarget{learning-activities-33}{%
\subsection*{Learning Activities}\label{learning-activities-33}}
\addcontentsline{toc}{subsection}{Learning Activities}

\begin{reflect}
\hypertarget{brainstorm-and-prepare-your-pitch}{%
\subsubsection*{Brainstorm and Prepare Your Pitch}\label{brainstorm-and-prepare-your-pitch}}
\addcontentsline{toc}{subsubsection}{Brainstorm and Prepare Your Pitch}

Brainstorm either alone or with a partner and come up with a story concept to pitch to the class. The goal of this activity is to do find a compelling story and to pitch it in a way that will attract the teammates you will need to make your longer film.

Helpful Hint: Sometimes a title for your film can lend lots of inspiration. Professional screenwriters and Hollywood spend a lot of time finding the right title that captures the spirit of the film and can aid in marketing. Think \emph{Rebel Without A Cause, Slum Dog Millionaire, Vertigo, Dumb and Dumber, The
Haunting, Groundhog Day, When Harry Met Sally.}

To prepare for your pitch make sure you:

\begin{itemize}
\tightlist
\item
  Know your genre. The story above could be a comedy, a drama, thriller, or a horror film. Which genre you decide will determine the following.\\
\item
  Establish your main characters.\\
\item
  Establish the main conflict.\\
\item
  Escalate the conflict.\\
\item
  Create some of the obstacles that will have to be overcome.\\
\item
  Have a clear beginning, middle, and end.\\
\item
  Know what crew members you will need and what they will do, e.g., a screenwriter, producer, cinematographer, editor, etc.
\end{itemize}

Make sure the story is compressed, a la Aristotle's \emph{Poetics}, with a limited number of characters, locations, and action.
\end{reflect}

\begin{caution}
\textbf{Helpful Hint:} Most of the short films that won or where nominated for the Academy Award for Best Live Action Short Film follow Aristotle's idea of great drama and comedy: a few characters and locations and action that takes place over a brief amount of time. For a list of nominees and winners see: Academy Award for Best Live Action Short Film.
\end{caution}

\begin{reflect}
Also be prepared to explain what crew members you will need and what they need to do.

Make your pitch to your classmates and then see who wants to join which team.

If your project is not chosen, that is, no one wants to work on it with you, you can still do it but know that it will be a lot of work.

\hypertarget{guidelines-for-pitching-a-film-project-1}{%
\subsubsection*{Guidelines for Pitching a Film Project}\label{guidelines-for-pitching-a-film-project-1}}
\addcontentsline{toc}{subsubsection}{Guidelines for Pitching a Film Project}

\textbf{Purpose of a Pitch:} To get people to work on your project, commission your
project, fund your project, help with your project, distribute your project,
etc. Pitches also help you to clarity what you want to do and why.

\textbf{Steps to a Good Pitch}

\begin{enumerate}
\def\labelenumi{\arabic{enumi}.}
\item
  If you can, open with a teaser or dramatic statement or question.
\item
  Meet and greet your audience---minimum chitchat, professionals are busy, busy people.
\item
  Dress appropriately---relative to the audience and the project. Artists dress
  ``artsy''; corporate people dress with business attire.
\item
  Project the proper persona (your personality and character)---knowledgeable, credible, trustworthy, smart, clever, etc.
\item
  Explain your qualifications, background, and why you are competent and the right person to do the project. If you do not have a demo reel then sell people with your passion, e.g., WHIPLASH filmmaker.
\item
  Give the title of the film/project---titles shape perception and expectations.
\item
  State the Genre---comedy, drama, documentary, etc. This also shapes perceptions and expectations. The basic concept of a comedy should make us laugh or at lease smile.
\item
  Give the Unique Angle or Hook of your project---what makes it stand out as something different and catchy/edgy/clever, etc. E.g., SHOE IN LOVE, a romantic comedy from the POV of a pair of cowboy boots and stilettos falling in love with each other. The special angle concept is to use foot ware as the main characters and the execution hook is to shoot POVs at foot level or from the shin down.
\item
  Briefly state the logline or basic concept: a one or two sentence description of the core story. E.g., a young, bored dirt farmer gets a message that he has to help rescue a faraway princes and save the universe. STAR WARS.
\item
  Project synopsis---succinctly tell the beginning/middle/end of your story or if a documentary the main message of your doc.
\item
  Summarize your end goal: to wow my audience with extraordinary nature photography or to make my audience laugh, cry, fall in love, etc.
\item
  Explain what you need---cast, crew, funding, etc.
\item
  Ask for Questions---also be open to criticism and constructive comments and feedback.
\item
  Gratitude---be thankful and gracious even if your project is not received well or as anticipated. Burning bridges is wrongheaded. Pitching can be a training ground for how to deal with rejection.
\end{enumerate}

\emph{Examples of Types of Montage Projects to Pitch}

The goal of your Montage Short assignment (see the syllabus) will be to explore how to make an effective montage-based film that uses picture and sound editing to convey meaning, emotion, story and/or mood.

For this project, you can:

\begin{itemize}
\tightlist
\item
  Create a music video where you take a song and create images that express the mood, feeling, meaning, etc. of the song.\\
\item
  Find a poem, fairy tale, or passage from the Bible and ``visualize'' and enhance it with music and sound effects.\\
\item
  Create a chase scene---someone steals something and is on the run and using music and sound effects to enhance the excitement and dynamism.\\
\item
  Create a love or horror or thriller story ---use can use dialogue but there must be an edited montage---a series of edited shots that convey the mood, meaning etc.
\end{itemize}
\end{reflect}

\hypertarget{creating-a-scriptment}{%
\section{Creating a Scriptment}\label{creating-a-scriptment}}

Our goal in this Unit is to find and create a strong story to film to shoot and edit so that it shines with excellence. The following steps will help you prepare a shooting script that will make pre-production, production, and post-production flow better.

\begin{itemize}
\item
  Create a step outline.
\item
  Create a scriptment.
\end{itemize}

Your scriptment should be written in a way that someone who reads it should be able to visualize and hear your film.

\hypertarget{learning-activities-34}{%
\subsection*{Learning Activities}\label{learning-activities-34}}
\addcontentsline{toc}{subsection}{Learning Activities}

\begin{reflect}
\hypertarget{creating-a-scriptment-1}{%
\subsubsection*{Creating a Scriptment}\label{creating-a-scriptment-1}}
\addcontentsline{toc}{subsubsection}{Creating a Scriptment}

For an overview of scriptments, consult this resource: Outlines, Treatments, and Scriptments, Oh My! by Ken Miyamoto.

As you do other drafts (good writing involves a lot of re-writing) to improve your story and polish your script so that everything is clear, it is helpful to ask the following questions. (These questions are designed for a fiction narrative film, though some of them can help vet your documentary or experimental film.) You might not have answers to all of them but you should for most of them. Again the emphasis in on helping you to be clear about your project.

\begin{itemize}
\tightlist
\item
  Do I have the right genre?\\
\item
  Do I have a hook or is there something unique and fascinating about my story? It's special angle.\\
\item
  Is the story clear? A solid beginng, middle, and end?\\
\item
  Do I focus on only a few characters and limited locations?\\
\item
  Are my action and conflict unified and focused?\\
\item
  Does my main character have a goal?\\
\item
  Is there an antagonist who or which wants to thwart that goal? (Remember, weather or a dog can be an antagonist.)\\
\item
  Are their stakes or consequences for not obtaining the goal?\\
\item
  Are there interesting obstacles to obtaining the goal?\\
\item
  Do I have a set-up of the main character, problem, goal, and antagonist?\\
\item
  Do I have some unexpected and unpredictable twists and turns in the middle section of my story?\\
\item
  Is there a main conflict or battle that determines whether the main character gets his or her goal?\\
\item
  Do I have a satisfying ending? (Remember not all endings have to be happy or closed. Open-ended endings are not resolved and make the audience guess, debate, and discuss what happened or might happen after).
\end{itemize}

Once you have your team and have developed the scriptment for your story, you can begin pre-production to get everything in place to film your project: e.g., cast, locations, props, gear, shooting schedule, etc.
\end{reflect}

\hypertarget{summary-7}{%
\section*{Summary}\label{summary-7}}
\addcontentsline{toc}{section}{Summary}

In this unit, you learned about:

\begin{itemize}
\tightlist
\item
  Types of short films\\
\item
  Film Genres\\
\item
  The Logistics for the Final Project\\
\item
  Script and Story Development\\
\item
  Creating A Scriptment
\end{itemize}

\hypertarget{assessment-7}{%
\section*{Assessment}\label{assessment-7}}
\addcontentsline{toc}{section}{Assessment}

\begin{assessment}
\hypertarget{assignment-3}{%
\subsection*{Assignment:}\label{assignment-3}}
\addcontentsline{toc}{subsection}{Assignment:}

\textbf{Film Journal}
After completing this unit, including the learning activities, you are asked to make sure you are doing journal entries and when appropriate to share your responses with your facilitator and classmates when you meet.

Note that entries are expected for every unit. Your journal reflections are submitted at the end of the course as part of the Final Exam: Self Assessment worth 30\% of your course grade.

\textbf{Planning for the Final Film Project}

Given that you cannot have a good final short film without a good story, the work of this unit will play a vital role in determining how good your final project will be.

After completing this unit, including the learning activities, you are asked to use the activities here as the basis for the next unit.

\hypertarget{grading-criteria-5}{%
\subsection*{Grading Criteria:}\label{grading-criteria-5}}
\addcontentsline{toc}{subsection}{Grading Criteria:}

There is no assessment for this unit, as your grade will be for your final project. But the activities here are essential if you are to make a strong final film.
\end{assessment}

\hypertarget{checking-your-learning-7}{%
\section*{Checking your Learning}\label{checking-your-learning-7}}
\addcontentsline{toc}{section}{Checking your Learning}

\begin{progress}
Before you move on to the next unit, you may want to check to make sure that you are able to:

\begin{itemize}
\tightlist
\item
  Describe the different types of short films\\
\item
  Define film genre and why it is important\\
\item
  Determine how to approach making a film script to shoot and edit.\\
\item
  Create a Scriptment
\end{itemize}
\end{progress}

\hypertarget{summary-making-your-short-film}{%
\chapter{Summary Making Your Short Film}\label{summary-making-your-short-film}}

\hypertarget{overview-8}{%
\section*{Overview}\label{overview-8}}
\addcontentsline{toc}{section}{Overview}

In this unit we culminate this course as you make your final film project and bring together all you have learned so far. Congratulations. Your hard work is paying off.

Here you will get the other skills needed to make an excellent short film as you take your scriptment from the last unit and shoot and edit it.

These skills will focus on exploring how to use film coverage to create meaning, emotions, and moods.

Many of the items and techniques addressed here have been covered in the previous units. Go back to them if you need a refresher. But there are important new elements that you will learn. The important thing is not to get lost in the detail but to maintain the big picture perspective, namely, communicating what you want to say in a creative and excellent way.

Let's dig in.

\hypertarget{topics-8}{%
\subsection*{Topics}\label{topics-8}}
\addcontentsline{toc}{subsection}{Topics}

This unit is divided into the following topics:

\begin{enumerate}
\def\labelenumi{\arabic{enumi}.}
\tightlist
\item
  The Syntax of Film\\
\item
  The Importance and Types of Camera Coverage\\
\item
  Continuity\\
\item
  Storyboarding\\
\item
  Shot Lists\\
\item
  Casting\\
\item
  Other Types of Short Films\\
\item
  Sound Recording\\
\item
  The Final Steps
\item
  Guidelines for Doing and Evaluating the Final Film Project
\end{enumerate}

\hypertarget{learning-outcomes-8}{%
\subsection*{Learning Outcomes}\label{learning-outcomes-8}}
\addcontentsline{toc}{subsection}{Learning Outcomes}

When you have completed this unit, you should be able to:

\begin{itemize}
\tightlist
\item
  Define film syntax and its key terms to describe your filmmaking process.\\
\item
  Describe camera coverage and contrast the types of Coverage.\\
\item
  Demonstrate How To Use Continuity.\\
\item
  Create Storyboards and Shot Lists.\\
\item
  Demonstrate Casting Skills.\\
\item
  Apply L Cuts and J Cuts.\\
\item
  Evaluate the final film project.
\end{itemize}

\hypertarget{activity-checklist-8}{%
\subsection*{Activity Checklist}\label{activity-checklist-8}}
\addcontentsline{toc}{subsection}{Activity Checklist}

\begin{reflect}
Here is a checklist of learning activities you will benefit from in completing this unit. You may find it useful for planning your work.

\begin{itemize}
\tightlist
\item
  Read the first section of Chapter 9: The Syntax of Film and view the video resources.\\
\item
  Study the next section of Chapter 9 The Importance and Types of Camera Coverage and then view the video resource.\\
\item
  Read the Continuity section of Chapter 9 and view the video resources.\\
\item
  Read the section on Storyboarding in Chapter 9 and consult the resource listed.\\
\item
  Read about Shot Lists in Chapter 9 and view the resource listed.\\
\item
  Read the Casting section in Chapter 9 and watch the videos and podcast.\\
\item
  See the section on Other Types of Film in your text.
\item
  View the three resources for Sound Recording in Chapter Nine and find tutorials and instruction essays on the web that are appropriate to your cell phone and DSLR.\\
\item
  Review Unit 7 to refresh yourself about what is involved in the post-production process.\\
\item
  Study the guidelines and the Evaluation Criteria closely for the final project.
\end{itemize}

\hypertarget{assessment-8}{%
\subsubsection*{Assessment:}\label{assessment-8}}
\addcontentsline{toc}{subsubsection}{Assessment:}

\begin{itemize}
\tightlist
\item
  \textbf{Film Journal}\\
\item
  \textbf{Assignment : Final Film Project} (40\%)
\end{itemize}
\end{reflect}

\hypertarget{resources-8}{%
\section*{Resources}\label{resources-8}}
\addcontentsline{toc}{section}{Resources}

Here are the resources you will need to complete this unit.

\begin{itemize}
\tightlist
\item
  Chapter Nine course text\\
\item
  \href{https://www.youtube.com/watch?v=IK2IAEO-FUI}{How to Shoot a Scene! - Film Riot}\\
\item
  \href{https://www.youtube.com/watch?v=9AGaECt9j4g}{Film Blocking Tutorial --- Filmmaking Techniques for Directors: Ep3}\\
\item
  \href{https://www.youtube.com/watch?v=okphB85lfjk}{FilmSkills.com - Getting the right shots and coverage}\\
\item
  \href{https://www.youtube.com/watch?v=cz3nBkIa9K0}{What Is A Master Shot?}\\
\item
  \href{https://www.youtube.com/watch?v=eou7A-e2e4I}{Match on Action technique}\\
\item
  \href{https://www.youtube.com/watch?v=El28XrjtcMI}{Match Cuts in Film Editing}\\
\item
  \href{https://www.youtube.com/watch?v=9XOn5uxdSJc}{Screen Direction rule}\\
\item
  \href{https://www.youtube.com/watch?v=RogoUz_pk4Y}{Screen direction}\\
\item
  \href{https://www.youtube.com/watch?v=HinUychY3sE}{Breaking Down the 180-Degree Rule}\\
\item
  \href{https://www.youtube.com/watch?v=1K8EUc98VoQ}{The 30 Degree Rule in Filmmaking \textbar{} Kaicreative \textbar{} Tips}\\
\item
  \href{https://www.wyzowl.com/what-is-a-storyboard/What}{Is a Storyboard and How Do You Make One for Your Video?}\\
\item
  \href{https://www.youtube.com/watch?v=-rzJP_5L_yQ}{Basics of Creating a Shot List}\\
\item
  \href{https://www.youtube.com/watch?v=YpCkRPqsiJ4}{How to Cast an ACTOR for a Low Budget Film \textbar{} The Film Look}\\
\item
  \href{https://www.youtube.com/watch?v=x0G6n346m90}{Auditioning Actors}\\
\item
  \href{https://www.youtube.com/watch?v=eyH-a964kAs}{SFX Secrets: The J Cut \& The L Cut}\\
\item
  and \href{https://www.youtube.com/watch?v=fT5rRPMnka0}{Video Editing Tips: J Cut vs L Cut}\\
\item
  Other resources will be available online and in the course text.
\end{itemize}

\hypertarget{the-syntax-of-film}{%
\section*{The Syntax of Film}\label{the-syntax-of-film}}
\addcontentsline{toc}{section}{The Syntax of Film}

The first principle of this unit will focus on film syntax, where you take the grammar of film---the types of shots, shot sizes, shot angles and height, shot framing, subject angles, etc. and order them in a way that best communicates what you want to say. This is where higher-level creativity takes place.

Filmmaking is like a language where you use the elements of grammar to make sentences. In this case, your shots to make scenes and the meaning the scenes will covey. The cumulation of sentences lead to paragraphs -- the film equivalent of sequences, and the cumulation of paragraphs leads to sections (acts in film) and the cumulation of sections leads to your essay (in this case, your final film). You get the metaphor.

\hypertarget{learning-activities-35}{%
\subsection*{Learning Activities}\label{learning-activities-35}}
\addcontentsline{toc}{subsection}{Learning Activities}

\begin{reflect}
\hypertarget{read-view-and-reflect}{%
\subsubsection*{Read, View, and Reflect}\label{read-view-and-reflect}}
\addcontentsline{toc}{subsubsection}{Read, View, and Reflect}

Read the first section of Chapter 9: The Syntax of Film. Before moving on to the next topic in this unit, view these resources to re-enforce the big picture:

What were your best take-aways from these tutorials? Log this in your journal.
\end{reflect}

\begin{caution}
\textbf{Note:} that this learning activity is ungraded, but is designed to help you succeed in your assessments in this course.
\end{caution}

\hypertarget{the-importance-and-types-of-camera-coverage}{%
\section*{The Importance and Types of Camera Coverage}\label{the-importance-and-types-of-camera-coverage}}
\addcontentsline{toc}{section}{The Importance and Types of Camera Coverage}

Exploring and understanding camera coverage is vital for those who want to create excellent films. Coverage refers to how a scene is captured. It involves how many shots are used and their type and kind to capture a scene in a film or video production. Coverage is thus the shot material an editor or post-production team will use to assemble the scenes and sequences of a movie. Having proper coverage is critical if a scene and a film is to make sense.

The are many questions that can guide you as you break down your film script to determine what coverage to use and how to shoot a scene. (Remember a scene can be one shot such as driving to the store, or a series of shots such as someone confronting a person to get information from him or her.) Take the time to study the \textbf{Questions for Discovering Coverage} section.

Coverage can be done well or poorly. Our goal is the former.

There are two main types of general coverage: \textbf{Master Shots and Mini-Masters}. Knowing why these are important and when to use them will help guide you to the more detailed coverage and shots you will employ as as you develop the action of a scene.

Filmmaking is not a ``paint by numbers'' art. If you follow formulas, most likely your film will be predicable and lack freshness. This is why understanding shot progression is important. Choosing the types of shots you will use and their order will form a large part of your visual creativity in your film.

\hypertarget{learning-activities-36}{%
\subsection*{Learning Activities}\label{learning-activities-36}}
\addcontentsline{toc}{subsection}{Learning Activities}

\begin{reflect}
\hypertarget{read-and-view}{%
\subsubsection*{Read and View}\label{read-and-view}}
\addcontentsline{toc}{subsubsection}{Read and View}

Study the next section of Chapter 9 The Importance and Types of Camera Coverage and then view this resource:

Explain the value of why this type of shot can be beneficial as the first shot you do when you start your coverage of a scene? What is the downside of relying too much on a master shot?
\end{reflect}

\hypertarget{continuity}{%
\section*{Continuity}\label{continuity}}
\addcontentsline{toc}{section}{Continuity}

Continuity, or the non-distracting and motivated and smooth flow of action, is another skill set essential to the filmmaker's took kit. Some techniques you have already studied, but others are new and important.

The following are the major techniques of continuity that you must focus on:

\begin{itemize}
\item
  Matching Action
\item
  Directional Continuity
\item
  Eyeline Continuity
\item
  180-Degree Rule
\item
  30-Degree Rule
\end{itemize}

Each of these individually and as a group will help immensely in creating well-constructed, well-motivated, and audience-engaging films.

\hypertarget{learning-activities-37}{%
\subsection*{Learning Activities}\label{learning-activities-37}}
\addcontentsline{toc}{subsection}{Learning Activities}

\begin{reflect}
\hypertarget{read-and-view-1}{%
\subsubsection*{Read and View}\label{read-and-view-1}}
\addcontentsline{toc}{subsubsection}{Read and View}

Read the Continuity section of Chapter 9.

There are many resources that can help you to understand the various parts of continuity. It might seem like a lot but it will greatly benefit you if you view the following resources.

\textbf{Match on Action technique}

\textbf{Match Cuts in Film Editing}

\textbf{Screen Direction rule}

\textbf{Screen direction}

\textbf{Breaking Down the 180-Degree Rule}

\textbf{The 30 Degree Rule in Filmmaking \textbar{} Kaicreative \textbar{} Tips}

Do you feel you have a stronger grasp of continuity and of each the techniques addressed? If not, review the videos of those parts you do not understand.
\end{reflect}

\hypertarget{storyboarding}{%
\section*{Storyboarding}\label{storyboarding}}
\addcontentsline{toc}{section}{Storyboarding}

By now you might feel overwhelmed with camera coverage and all that it takes to shoot your story and script well. Storyboarding and storyboards can help simplify the process and make it easier to visualize what you want to do and how to do it.

It will help your scene coverage if you storyboard your shots. They do not have to be elaborate. They can be simple stick figures such as this one that shows a long shot of someone pushing or struggling against something (the context of your story will determine this).

\begin{figure}
\centering
\includegraphics{assets/unit5/Picture1.png}
\caption{graphic stick figure pushing}
\end{figure}

The important thing is not the quality but that your storyboards make sense to you and your crew so you will know how to approach a shot and why.

At the same time, do not become a slave to your storyboard. If you discover something better or get inspired on the film set, try the new thing. For safety, shoot it the storyboard way and then the new way and decide during post-production which one will work best.

\hypertarget{learning-activities-38}{%
\subsection*{Learning Activities}\label{learning-activities-38}}
\addcontentsline{toc}{subsection}{Learning Activities}

\begin{reflect}
\hypertarget{read-and-view-2}{%
\subsubsection*{Read and View}\label{read-and-view-2}}
\addcontentsline{toc}{subsubsection}{Read and View}

Read the section on Storyboarding in Chapter 9.

Consult this resource to help you see and better understand storyboards:

\url{https://www.wyzowl.com/what-is-a-storyboard/“What} Is a Storyboard and How Do You Make One for Your Video?

Storyboard the scenes from your scriptment you plan to film. You can do this alone but it will be better if at least the director and cinematographer for you project work on this together. Ideally everyone on the team should work on this so that you will bond and have a unified vision of how to stage and shoot your shots.
\end{reflect}

\begin{caution}
\textbf{Helpful Hint:} Remember your locations will determine a lot of how you can and cannot shoot a scene. A drone will not work in a small room. Therefore, be sure to scout your locations and take pictures of them as you desing your storyboards and shot lists. Be sure to also confirm that the locations will be available for your shooting schedule.
\end{caution}

\hypertarget{shot-lists}{%
\section*{Shot Lists}\label{shot-lists}}
\addcontentsline{toc}{section}{Shot Lists}

Once you have your scriptment describing all of the action and dialogue you will film and have storyboarded your film, you can now create a shot list which will help ensure you get all the shots needed for your coverage.

It is better if your whole production team gives feedback on the shot list. Ultimately, it is the director who will make the final choice but hearing what other team members feel can help cut down on mistakes.

Study the Shot List Example in Chapter Nine to see what you need to include.

\begin{reflect}
\hypertarget{shot-lists-1}{%
\subsubsection*{Shot Lists}\label{shot-lists-1}}
\addcontentsline{toc}{subsubsection}{Shot Lists}

Read about Shot Lists in Chapter 9.

In preparation, view this resource:

\textbf{Basics of Creating a Shot List}
\end{reflect}

\hypertarget{casting}{%
\section*{Casting}\label{casting}}
\addcontentsline{toc}{section}{Casting}

Once you have your locations, storyboards, shot list, and schedule you are ready to cast your show. Casting is often done earlier in larger professional productions because they need to lock in busy actors. For this scale of micro or no budget filmmaking having your shooting schedule and your storyboards and shot lists can show seasoned actors that you know what you are doing and can help you land them.

This pre-production process emphasized here will also help you to know if some special skills are needed for your actors. E.g., someone who can dance, sing, play guitar, is good at soccer, etc.

Having good actors is critical to having a good film. Follow the tips and guidelines in the chapter and work hard to find the best actors you can. It will pay off a lot.

\begin{reflect}
\hypertarget{read-view-and-listen}{%
\subsubsection*{Read, View and Listen}\label{read-view-and-listen}}
\addcontentsline{toc}{subsubsection}{Read, View and Listen}

Read the Casting section in Chapter 9.

Before auditioning and casting your film see this simple overview of the casting process:

\textbf{How to Cast an ACTOR for a Low Budget Film \textbar{} The Film Look}

For practical tips for auditioning actors see:

\textbf{Auditioning Actors}

For detailed information regarding how to work with actors and casting for low budget films listen to this podcast:

\textbf{How to Cast a No Budget Indie Film with Casting Director Veronika Lee (Backstage Magazine),}
\end{reflect}

\hypertarget{other-types-of-films}{%
\section*{Other Types of Films}\label{other-types-of-films}}
\addcontentsline{toc}{section}{Other Types of Films}

Some of you might want to do a documentary or an experimental film. Though this course centers on classical approaches to fictional narrative film, you are free to work on such projects. It is important that you pitch such projects ahead of time to your instructor to make sure it is doable within the confines of this course.

\begin{reflect}
\hypertarget{other-types-of-film}{%
\subsubsection*{Other Types of Film}\label{other-types-of-film}}
\addcontentsline{toc}{subsubsection}{Other Types of Film}

See the section on Other Types of Film if you want to do a documentary or experimental film and view the resources for them.
\end{reflect}

\hypertarget{sound-recording}{%
\section*{Sound Recording}\label{sound-recording}}
\addcontentsline{toc}{section}{Sound Recording}

Audio can make or break a film, especially in no-budget and low-budget filmmaking. Proper sound recording and sound editing is a course unto itself and if you want to go into filmmaking professionally you should take at least one course in this.

Study the Sound Recording section in Chapter Nine and the suggested resources to help up get better quality sound for your project.

\begin{reflect}
\hypertarget{chapter-resources}{%
\subsubsection*{Chapter Resources}\label{chapter-resources}}
\addcontentsline{toc}{subsubsection}{Chapter Resources}

How to capture great sound for a film without the use of professionals and professional gear is a challenge. View the three resources for this in Chapter Nine and find tutorials and instruction essays on the web that are appropriate to your cell phone and DSLR.
\end{reflect}

\hypertarget{the-final-steps}{%
\section*{The Final Steps}\label{the-final-steps}}
\addcontentsline{toc}{section}{The Final Steps}

Bravo. Once you have reached this stage of this Unit and the previous one you should be well-equipped to produce and film your movie.

You will then have to edit it. This section adds some new techniques that can help elevate your film project to a higher level, namely Split Edits. If you can, add this to the editing of your film, as well as L cuts and J cuts. They will make your film flow better and be more engaging and enjoyable. It might also help win you an award in the course.

\begin{reflect}
\hypertarget{review}{%
\subsubsection*{Review}\label{review}}
\addcontentsline{toc}{subsubsection}{Review}

Review Unit 7 to refresh yourself about what is involved in the post-production process. If possible, add Split Edits to elevate the quality and sophistication of your final film project.
\end{reflect}

\hypertarget{guidelines-for-doing-and-evaluating-the-final-film-project}{%
\section*{Guidelines for Doing and Evaluating The Final Film Project}\label{guidelines-for-doing-and-evaluating-the-final-film-project}}
\addcontentsline{toc}{section}{Guidelines for Doing and Evaluating The Final Film Project}

Continuity or the non-distracting and motivated and smooth flow of action is another skill set essential to the filmmaker's took kit.

Study these resources and practice applying them in your edited scenes especially those with dialogue or where one scene transitions into another:

\begin{reflect}
\textbf{SFX Secrets: The J Cut \& The L Cut}

\textbf{Video Editing Tips: J Cut vs L Cut}

In addition, study the \emph{Steps to Making an Excellent Short Film} section in the chapter. It is both a good recap and a good checklist to help ensure you have the elements need to make an excellent final film.

\hypertarget{final-project-guidelines}{%
\subsubsection*{Final Project Guidelines}\label{final-project-guidelines}}
\addcontentsline{toc}{subsubsection}{Final Project Guidelines}

Study the guidelines and the Evaluation Criteria closely for the final project. They will help you to do well. This criteria closely tracks that which was used to evaluate your montages so the process should be more comfortable for you now.

\begin{itemize}
\tightlist
\item
  Did the film make sense and work overall?\\
\item
  Did the film meet genre expectations?\\
\item
  Was the story interesting? Surprising, engaging, exciting?\\
\item
  Did the actors perform well?\\
\item
  Did the camera coverage work? If so, why? If not, why?\\
\item
  Did the editing work? If so why? If not, why?\\
\item
  What stood out in a good way?\\
\item
  What needed improvement?\\
\item
  What did you like best about the film?\\
\item
  What did you like least?\\
\item
  Other comments.
\end{itemize}
\end{reflect}

\hypertarget{summary-8}{%
\section*{Summary}\label{summary-8}}
\addcontentsline{toc}{section}{Summary}

In this unit, you learned about:

\begin{itemize}
\tightlist
\item
  The Syntax of Film\\
\item
  The Importance and Types of Coverage\\
\item
  Continuity\\
\item
  Storyboarding\\
\item
  Shot Lists\\
\item
  Casting\\
\item
  Sound Recording\\
\item
  Other Types of Short Films\\
\item
  The Final Steps\\
\item
  Guidelines for Doing and Evaluating the Final Film Project
\end{itemize}

\hypertarget{assessment-9}{%
\section*{Assessment}\label{assessment-9}}
\addcontentsline{toc}{section}{Assessment}

\begin{assessment}
\hypertarget{assignment-4}{%
\subsection*{Assignment:}\label{assignment-4}}
\addcontentsline{toc}{subsection}{Assignment:}

\hypertarget{film-journal-2}{%
\subsection*{Film Journal}\label{film-journal-2}}
\addcontentsline{toc}{subsection}{Film Journal}

After completing this unit, including the learning activities, you are asked to make sure you are doing journal entries and when appropriate to share your responses with your facilitator and classmates when you meet.

\hypertarget{final-film-project-40}{%
\subsubsection*{Final Film Project (40\%)}\label{final-film-project-40}}
\addcontentsline{toc}{subsubsection}{Final Film Project (40\%)}

This final film project will account for 40\% of your grade.

See the details in the Assessment section.

\hypertarget{grading-criteria-6}{%
\subsubsection*{Grading Criteria:}\label{grading-criteria-6}}
\addcontentsline{toc}{subsubsection}{Grading Criteria:}

\textbf{Grading Breakdown for Final Film Project}

\begin{itemize}
\tightlist
\item
  Film Story and Script 10 Points\\
\item
  Cinematography and Coverage 10 Points\\
\item
  Editing 10 Points\\
\item
  Acting 10 Points
\end{itemize}

Total 40 points.=

\begin{itemize}
\tightlist
\item
  10 points = Excellent work\\
\item
  8 Points = Above Average work\\
\item
  6-7 points Average work\\
\item
  4-5 points Below Average Work\\
\item
  1-3 points failure to meet the criteria
\end{itemize}
\end{assessment}

\begin{caution}
\textbf{Note:} that entries are expected for every unit. Your journal reflections are submitted at the end of the course as part of the Final Exam: Self Assessment worth 30\% of your course grade.
\end{caution}

\hypertarget{checking-your-learning-8}{%
\section*{Checking your Learning}\label{checking-your-learning-8}}
\addcontentsline{toc}{section}{Checking your Learning}

\begin{progress}
Before you move on to the next unit, you may want to check to make sure that you are able to:

\begin{itemize}
\tightlist
\item
  Define film syntax and its key terms to describe your filmmaking process.\\
\item
  Describe Camera Coverage and Contrast The Types of Coverage.\\
\item
  Demonstrate How To Use Continuity.\\
\item
  Create Storyboards and Shot Lists.\\
\item
  Demonstrate Casting Skills.\\
\item
  Apply L Cuts and J Cuts.\\
\item
  Evaluate the final film project.
\end{itemize}
\end{progress}

\hypertarget{course-summary-and-celebration}{%
\chapter{Course Summary and Celebration}\label{course-summary-and-celebration}}

\hypertarget{overview-9}{%
\section*{Overview}\label{overview-9}}
\addcontentsline{toc}{section}{Overview}

Congratulations. You are at the end of your creative journey into the world of filmmaking. We hope it has been a grand adventure.

In this unit you celebrate your accomplishments and hopefully win a course award for them, and if they merit it, be considered for a film festival. You will also have a chance to reflect on what this course means for your future personal and professional develop and career. You will also self-assess your work and be encouraged to look at the big picture this course experience has offered.

\hypertarget{topics-9}{%
\subsection*{Topics}\label{topics-9}}
\addcontentsline{toc}{subsection}{Topics}

This unit is divided into the following topics:

\begin{enumerate}
\def\labelenumi{\arabic{enumi}.}
\tightlist
\item
  Course Awards\\
\item
  Film Festivals\\
\item
  Your Future
\end{enumerate}

\hypertarget{learning-outcomes-9}{%
\section*{Learning Outcomes}\label{learning-outcomes-9}}
\addcontentsline{toc}{section}{Learning Outcomes}

When you have completed this unit, you should be able to:

\begin{itemize}
\tightlist
\item
  Describe what you have learned from this course.\\
\item
  Determine who wins awards for their work.\\
\item
  Create a self-assessment that encapsulates what you learned.\\
\item
  Think about the big picture of your life and share your reflections.
\end{itemize}

\hypertarget{activity-checklist-9}{%
\subsection*{Activity Checklist}\label{activity-checklist-9}}
\addcontentsline{toc}{subsection}{Activity Checklist}

\begin{reflect}
Here is a checklist of learning activities you will benefit from in completing this unit. You may find it useful for planning your work.

\begin{itemize}
\tightlist
\item
  Decide on your film favourites.\\
\item
  Consider submitting to a film festival.\\
\item
  Study resources that deal with jobs in the film industry and careers that can use the filmmaking skills you are acquiring.
\end{itemize}

\textbf{Assessment:}

\begin{itemize}
\tightlist
\item
  Film Journal\\
\item
  Final Exam: Self-Assessment / Journal (30\%)
\end{itemize}
\end{reflect}

\hypertarget{resources-9}{%
\subsection*{Resources}\label{resources-9}}
\addcontentsline{toc}{subsection}{Resources}

Here are the resources you will need to complete this unit.

\begin{itemize}
\tightlist
\item
  Chapter Ten course text\\
\item
  \href{https://medium.com/filmfreeway/how-to-pick-which-festivals-to-submit-to-5c230989c607}{How to Pick Which Festivals to Submit to (And Get the Most from Your Submissions}\\
\item
  \href{http://resourcemagonline.com/2017/11/20-film-festivals-you-should-enter-your-short-film-into/82548/}{For a list of festivals focusing on short films see: ``20 Film Festivals You Should Enter Your Short Film Into, Bridget Schneider}\\
\item
  \href{https://www.youtube.com/watch?v=1grxjdy831g}{ESSENTIAL skills you need for a Career in filmmaking!}\\
\item
  \href{https://www.youtube.com/watch?v=Tvjg1ndrPlQ}{20+ FILM SET JOBS FOR YOU}\\
\item
  \href{https://www.careeraddict.com/film-jobs}{The 20 Best Careers in the Film Industry, Joanna Zambas}\\
\item
  \href{https://nofilmschool.com/film-crew-jobs}{Film Crew Positions And Why All Jobs on a Movie Set Matter,'' George Edelman}\\
\item
  \href{https://filmincolorado.com/resources/job-descriptions/}{Film In Colorado, Job Descriptions} This is an example of a local film commission.\\
\item
  \href{https://career.noomii.com/life-skills-need-land-dream-career/}{10 Life Skills You Need to Land Your Dream Career}\\
\item
  Other resources will be provided in the course text.
\end{itemize}

\hypertarget{course-awards}{%
\section{Course Awards}\label{course-awards}}

The first order of business on your capstone day is to celebrate who did outstanding work in the course. After you view all of the final film projects, your will vote on who wins the following awards based on the criteria of this course:

\begin{itemize}
\tightlist
\item
  Best Overall Film\\
\item
  Best Overall Director\\
\item
  Best Overall Script\\
\item
  Best Overall Cinematography\\
\item
  Best Overall Editing\\
\item
  Best Photographic Compositions\\
\item
  Best Motional Exercises\\
\item
  Best In-Camera Exercise\\
\item
  Best Montage Exercise\\
\item
  Best Final Film Project\\
\item
  Most Improved Filmmaker
\end{itemize}

You will also have the opportunity to give feedback on why you voted on the filmmakers, films, and awards which won.

\hypertarget{learning-activities-39}{%
\subsection*{Learning Activities}\label{learning-activities-39}}
\addcontentsline{toc}{subsection}{Learning Activities}

\begin{reflect}
\hypertarget{film-favourites}{%
\subsubsection*{Film Favourites}\label{film-favourites}}
\addcontentsline{toc}{subsubsection}{Film Favourites}

Be sure to review the sections in your course journal where you made comments about the films you liked during the course. Go back and view them in the Course Folder to confirm your decision.
\end{reflect}

\hypertarget{film-festivals}{%
\section*{Film Festivals}\label{film-festivals}}
\addcontentsline{toc}{section}{Film Festivals}

Film festivals are great venues for building your reputation and advancing your career.

If your film merits it, your instructor will recommend that your film be adjudicated for entry into Cinergy, the annual TWU campus student film festival held each spring.

It will help you grow as a filmmaker if you research festivals and submit your work to them.

\hypertarget{learning-activities-40}{%
\subsection*{Learning Activities}\label{learning-activities-40}}
\addcontentsline{toc}{subsection}{Learning Activities}

\begin{reflect}
\hypertarget{film-festival-options}{%
\subsubsection*{Film Festival Options}\label{film-festival-options}}
\addcontentsline{toc}{subsubsection}{Film Festival Options}

Explore these resources and others and see if you can find a film festival to submit to. Please let us know if you win!

\begin{itemize}
\tightlist
\item
  \href{https://medium.com/filmfreeway/how-to-pick-which-festivals-to-submit-to-5c230989c607}{How to Pick Which Festivals to Submit to And Get the Most from Your Submissions}\\
\item
  \href{http://resourcemagonline.com/2017/11/20-film-festivals-you-should-enter-your-short-film-into/82548/}{20 Film Festivals You Should Enter Your Short Film Into, Bridget Schneider}
\end{itemize}
\end{reflect}

\hypertarget{your-future}{%
\section{Your Future}\label{your-future}}

This course is a stepping stone in the long journey of preparing for your career and life-learning.

It will help you immensely in this process if you engage \textbf{now} in reflection on what you have learned and explore potential jobs and industry professions. After the course, your life will motor on and you might forget to do this.

\begin{caution}
\textbf{Helpful Hint:} Honestly assess your work and progress in the course. What strengths did you discover about yourself? What weaknesses do you have to work on? The important thing is not to condemn ourselves for our mistakes but to learn from them. Make a to do list for what you will do after this course to improve your personal and professional life. Log these insights in your journal and revisit them in six months to see where you stand then
\end{caution}

\hypertarget{learning-activities-41}{%
\subsection*{Learning Activities}\label{learning-activities-41}}
\addcontentsline{toc}{subsection}{Learning Activities}

\begin{reflect}
\hypertarget{applying-your-film-skills}{%
\subsubsection*{Applying Your Film Skills}\label{applying-your-film-skills}}
\addcontentsline{toc}{subsubsection}{Applying Your Film Skills}

Study resources like the following that deal with jobs in the film industry and careers that can use the filmmaking skills you are acquiring.

\begin{itemize}
\tightlist
\item
  \textbf{8 ESSENTIAL skills you need for a Career in filmmaking!,}
\end{itemize}

\begin{itemize}
\tightlist
\item
  \textbf{20+ FILM SET JOBS FOR YOU}
\end{itemize}

\begin{itemize}
\tightlist
\item
  \href{https://www.careeraddict.com/film-jobs}{The 20 Best Careers in the Film Industry, Joanna Zambas}
\item
  \href{https://nofilmschool.com/film-crew-jobs}{Film Crew Positions And Why All Jobs on a Movie Set Matter,'' George Edelman}
\end{itemize}

Which jobs appeal to you? Which careers appeal to you? Make a list, date it, and post it on your wall, bathroom mirror, or computer. Check it periodically to remind you of your goal and to assess how well you are on the road to it.
\end{reflect}

\hypertarget{summary-9}{%
\section*{Summary}\label{summary-9}}
\addcontentsline{toc}{section}{Summary}

\begin{itemize}
\tightlist
\item
  Course Awards\\
\item
  Film Festivals\\
\item
  Your Future\\
\item
  Self-Assessment
\end{itemize}

\hypertarget{assessment-10}{%
\section*{Assessment}\label{assessment-10}}
\addcontentsline{toc}{section}{Assessment}

\begin{assessment}
\hypertarget{assignment-5}{%
\subsection*{Assignment :}\label{assignment-5}}
\addcontentsline{toc}{subsection}{Assignment :}

\textbf{Film Journal}
After completing this unit, including the learning activities, you are asked to make sure you are doing journal entries and when appropriate to share your responses with your facilitator and classmates when you meet.
\end{assessment}

\begin{caution}
\textbf{Note:} that entries are expected for every unit. Your journal reflections are submitted at the end of the course as part of the Final Exam: Self Assessment worth 30\% of your course grade.
\end{caution}

\hypertarget{final-exam---self-assessment-course-journal}{%
\section*{Final Exam - Self-Assessment / Course Journal}\label{final-exam---self-assessment-course-journal}}
\addcontentsline{toc}{section}{Final Exam - Self-Assessment / Course Journal}

\begin{assessment}
The final exam for this course will involve a detailed self-assessment of what you have learned as a person and as a professional.

Follow the guidelines in the self-assessment section of the final chapter for your final exam submission. Be sure to use entries you have made in your journal.

As a reminder, the following is the point scale for determining your grade:

\textbf{Assignment} \textbar{} \textbf{Points} \textbar{} \textbf{Due} \textbar{}\\
\hspace*{0.333em}\textbar{} - \textbar{} -- \textbar{}\\
Assignment 1: In-Camera Exercise \textbar{} 10 points \textbar{} End of Unit \ldots{} \textbar{}\\
Assignment 2: Montage Project \textbar{} 20 points \textbar{} End of Unit \ldots{} \textbar{}\\
Assignment 3: Final Film Project \textbar{} 40 points \textbar{} End of Unit 10 \textbar{}\\
Final Exam: Self-Assessment / Course Journal \textbar{} 30 points \textbar{} End of Unit 10 \textbar{}\\
TOTAL \textbar{} 100 points \textbar{} \textbar{}
\end{assessment}

\hypertarget{checking-your-learning-9}{%
\section*{Checking your Learning}\label{checking-your-learning-9}}
\addcontentsline{toc}{section}{Checking your Learning}

\begin{progress}
Before you move on to the next unit, you may want to check to make sure that you are able to:

\begin{itemize}
\tightlist
\item
  Describe what you have learned from this course.\\
\item
  Determine who wins awards for their work.\\
\item
  Create a self-assessment that encapsulates what you learned.\\
\item
  Think about the big picture of your life and share your reflections.
\end{itemize}
\end{progress}

\hypertarget{assessment-11}{%
\chapter*{Assessment}\label{assessment-11}}
\addcontentsline{toc}{chapter}{Assessment}

The following assignments are opportunities for learners to demonstrate their understanding of the course outcomes. Please confirm assignment details with your instructor, referring to the course syllabus.

Note that Assignment dropboxes are located in Moodle. Also refer to the Course Schedule in Moodle for the specific due dates.

\hypertarget{assignment-6}{%
\section*{Assignment:}\label{assignment-6}}
\addcontentsline{toc}{section}{Assignment:}

\begin{assessment}

\end{assessment}

\hypertarget{grading-criteria-7}{%
\subsection*{Grading Criteria}\label{grading-criteria-7}}
\addcontentsline{toc}{subsection}{Grading Criteria}

See the following rubric that explains how your assignment will be evaluated. Also available as a \href{assets/assessment/Identity-as-a-Teacher-RUBRIC.pdf}{pdf}

\#\#\#\# APA/WRITING \{-\}

\textbf{Unsatisfactory:} Paper does not model language and conventions used in scholarly literature. Writing is not well-organized. Several errors in grammar or composition. Sources are not cited. APA citations are not appropriately formatted.

\textbf{Developing:} Paper partially models language and conventions used in scholarly literature. Writing is somewhat well organized and includes some errors in grammar or composition. Not all sources cited. APA citations are generally formatted correctly, with several errors.

\textbf{Proficient:} \emph{Paper consistently models language and conventions used in scholarly literature. Writing is well-organized and includes few (if any) errors in grammar or composition. All resources are appropriately cited (including in-text citations and bibliography information). Few (if any) errors in APA citations.}

\textbf{Exemplary:} Paper is an exemplar of language and conventions used in scholarly literature. Writing is well-organized and free of errors in grammar or composition. All resources are appropriately cited. No errors in APA format.

\hypertarget{statement-of-teaching-identity}{%
\subsubsection*{STATEMENT OF TEACHING IDENTITY}\label{statement-of-teaching-identity}}
\addcontentsline{toc}{subsubsection}{STATEMENT OF TEACHING IDENTITY}

\textbf{Unsatisfactory:} Does not provide a statement about identity as a teacher/facilitator

\textbf{Developing:} Provides an unclear statement about identity as a teacher/facilitator.

\textbf{Proficient:} \emph{Provides a clear, concise, and powerful statement about identity as a teacher/facilitator.}

\textbf{Exemplary:} Provides a clear, concise, and powerful statement about identity as a teacher/facilitator. Statement incorporates theory or research from course materials.

\hypertarget{developing-a-cohesive-and-logical-academic-argument}{%
\subsubsection*{DEVELOPING A COHESIVE AND LOGICAL ACADEMIC ARGUMENT}\label{developing-a-cohesive-and-logical-academic-argument}}
\addcontentsline{toc}{subsubsection}{DEVELOPING A COHESIVE AND LOGICAL ACADEMIC ARGUMENT}

\textbf{Unsatisfactory:} Does not make a focused, cohesive, or logical academic argument. Paper is confusing, and is missing an introduction, body, or conclusion. Transitions between sections and ideas are missing.

\textbf{Developing:} Makes an academic argument that is only partially focused, cohesive and logical. Paper is generally organized, but is missing an introduction, body, or conclusion. Transitions between sections and ideas are unclear.

\textbf{Proficient:} \emph{Makes a focused, cohesive, logical academic argument. Paper is effectively organized and includes an introduction, body, and conclusion. Transitions between sections and ideas are clear.}

\textbf{Exemplary:} Makes a focused, cohesive, logical and compelling academic argument. Paper is effectively organized and includes an introduction, body, and conclusion. Transitions between sections and ideas are clear, and build on each other.

\hypertarget{analysis-of-identity-as-a-teacher}{%
\subsubsection*{ANALYSIS OF IDENTITY AS A TEACHER}\label{analysis-of-identity-as-a-teacher}}
\addcontentsline{toc}{subsubsection}{ANALYSIS OF IDENTITY AS A TEACHER}

\textbf{Unsatisfactory:} Does not include three important aspects of identity as a teacher/facilitator. Does not include an analysis.

\textbf{Developing:} Lists but does not discuss three important aspects of identity as a teacher/facilitator. Includes a partial analysis.

\textbf{Proficient:} \emph{Includes a detailed discussion of three important aspects of identity as a teacher/facilitator. Includes thoughtful analysis of each of the three elements.}

\textbf{Exemplary:} Includes a detailed discussion of three important aspects of identity as a teacher/facilitator. Includes a thoughtful analysis, integrating scholarly literature to support analysis and furthering scholarly thinking related to teacher identity.

\hypertarget{scholarly-integration}{%
\subsubsection*{SCHOLARLY INTEGRATION}\label{scholarly-integration}}
\addcontentsline{toc}{subsubsection}{SCHOLARLY INTEGRATION}

\textbf{Unsatisfactory:} Does not integrate references to support claims and assertions made in the paper.

\textbf{Developing:} Integrates references to support some of the claims and assertions made in the paper.

\textbf{Proficient:} \emph{Integrates references to support claims and assertions made in the paper.}

\textbf{Exemplary:} Integrates references to support claims and assertions made in the paper, effectively synthesizing different perspectives and research results from scholarly sources.

\begin{longtable}[]{@{}
  >{\raggedright\arraybackslash}p{(\columnwidth - 8\tabcolsep) * \real{0.2000}}
  >{\raggedright\arraybackslash}p{(\columnwidth - 8\tabcolsep) * \real{0.2000}}
  >{\raggedright\arraybackslash}p{(\columnwidth - 8\tabcolsep) * \real{0.2000}}
  >{\raggedright\arraybackslash}p{(\columnwidth - 8\tabcolsep) * \real{0.2000}}
  >{\raggedright\arraybackslash}p{(\columnwidth - 8\tabcolsep) * \real{0.2000}}@{}}
\toprule\noalign{}
\begin{minipage}[b]{\linewidth}\raggedright
\textbf{TOTAL}
\end{minipage} & \begin{minipage}[b]{\linewidth}\raggedright
\textbf{0 = 0\% (F)}
\end{minipage} & \begin{minipage}[b]{\linewidth}\raggedright
\textbf{10 = 50\% (C)}
\end{minipage} & \begin{minipage}[b]{\linewidth}\raggedright
\textbf{15 = 75 (B)}
\end{minipage} & \begin{minipage}[b]{\linewidth}\raggedright
\textbf{20 = 100\% (A+)}
\end{minipage} \\
\midrule\noalign{}
\endhead
\bottomrule\noalign{}
\endlastfoot
\end{longtable}

\begin{center}\rule{0.5\linewidth}{0.5pt}\end{center}

\hypertarget{assignment-company-website-analysis}{%
\section*{Assignment: Company Website Analysis}\label{assignment-company-website-analysis}}
\addcontentsline{toc}{section}{Assignment: Company Website Analysis}

\begin{assessment}
Investigate the Human Resources or Faculty Development portion of a
company's website, a higher education institution or adult learning
facility, preferably one with which you are familiar. Focus on the
faculty or employee development part of the website. In this assignment,
you will apply the theory of teaching in/for/with depth by analyzing the
learning culture of an organization.

In a 4-5 page APA formatted paper, analyze the website by responding to
the following questions in your report:

\begin{enumerate}
\def\labelenumi{\arabic{enumi}.}
\tightlist
\item
  What can you infer about the company's learning culture?
\item
  From what is visible on the public website, would you say it is an
  authentic learning community? Why or why not? Discuss whether the
  website reflects aspects of one or more of the learning community
  models explored in previous lessons.
\item
  Do you see evidence that interconnectedness and integrity are valued?
  Explain.
\item
  What traits and skills seem to be valued in employees?
\item
  How does the company develop skills in its employees (e.g., workshops,
  seminars, mentoring)? Are the methods based on the principles of
  andragogy? (see Smith YouTube video). What specific adult learning
  strategies do you see reflected in the development/training
  opportunities for employees?
\end{enumerate}

Your paper should be 4-5 pages and should incorporate references to at
least five scholarly sources you have studied in this course, or other
scholarly sources you have identified.

The paper should include:

\begin{enumerate}
\def\labelenumi{\arabic{enumi}.}
\tightlist
\item
  Introduction
\item
  Analysis (responding to the prompts)
\item
  Conclusion
\item
  Reference List
\end{enumerate}
\end{assessment}

\hypertarget{company-website-analysis-rubric}{%
\subsection*{Company Website Analysis Rubric}\label{company-website-analysis-rubric}}
\addcontentsline{toc}{subsection}{Company Website Analysis Rubric}

See the following rubric that explains how your assignment will be evaluated. Also available as a \href{assets/assessment/Company-Website-Analysis-RUBRIC.pdf}{pdf}

\hypertarget{apa-formatting}{%
\subsubsection*{APA Formatting}\label{apa-formatting}}
\addcontentsline{toc}{subsubsection}{APA Formatting}

\textbf{Unsatisfactory:} Paper does not model language and conventions used in scholarly literature.
Writing is not well-organized. Several errors in grammar or composition. Sources
are not cited. APA citations are not appropriately formatted.

\textbf{Developing:} Paper partially models language and conventions used in scholarly literature.
Writing is somewhat well organized and includes some errors in grammar or
composition. Not all sources cited. APA citations are generally formatted
correctly, with several errors.

\textbf{Proficient:} \emph{Paper consistently models language and conventions used in scholarly
literature. Writing is well-organized and includes few (if any) errors in
grammar or composition. All resources are appropriately cited (including in-text
citations and bibliography information). Few (if any) errors in APA citations.}

\textbf{Exemplary:} Paper is an exemplar of language and conventions used in scholarly literature.
Writing is well-organized and free of errors in grammar or composition. All
resources are appropriately cited. No errors in APA format.

\hypertarget{developing-a-cohesive-and-logical-academic-argument-1}{%
\subsubsection*{DEVELOPING a COHESIVE and LOGICAL ACADEMIC ARGUMENT}\label{developing-a-cohesive-and-logical-academic-argument-1}}
\addcontentsline{toc}{subsubsection}{DEVELOPING a COHESIVE and LOGICAL ACADEMIC ARGUMENT}

\textbf{Unsatisfactory:} Does not make a focused, cohesive, or logical academic argument. Paper is
confusing, and is missing an introduction, body, or conclusion. Transitions
between sections and ideas are missing.

\textbf{Developing:} Makes an academic argument that is only partially focused, cohesive and logical.
Paper is generally organized, but is missing an introduction, body, or
conclusion. Transitions between sections and ideas are unclear.

\textbf{Proficient:} \emph{Makes a focused, cohesive, logical academic argument. Paper is effectively
organized and includes an introduction, body, and conclusion. Transitions
between sections and ideas are clear.}

\textbf{Exemplary:} Makes a focused, cohesive, logical and compelling academic argument. Paper is
effectively organized and includes an introduction, body, and conclusion.
Transitions between sections and ideas are clear and build on each other.

\hypertarget{analysis-of-learning-culture}{%
\subsubsection*{ANALYSIS of LEARNING CULTURE}\label{analysis-of-learning-culture}}
\addcontentsline{toc}{subsubsection}{ANALYSIS of LEARNING CULTURE}

\textbf{Unsatisfactory:} Does not include an analysis of the company learning culture, and no evaluation
of the authenticity of the learning community.

\textbf{Developing:} Includes a partial analysis of the company learning culture, including a limited
evaluation of the authenticity of the learning community.

\textbf{Proficient:} \emph{Includes a detailed analysis of the company learning culture, including an
evaluation of the authenticity of the learning community.}

\textbf{Exemplary:} Includes a detailed analysis of the company learning culture, including an
evaluation of the authenticity of the learning community. Includes a thoughtful
analysis, integrating scholarly literature to support analysis and furthering
scholarly thinking related to teacher identity.

\hypertarget{evaluation-of-interconnectedness-and-integrity}{%
\subsubsection*{EVALUATION of INTERCONNECTEDNESS and INTEGRITY}\label{evaluation-of-interconnectedness-and-integrity}}
\addcontentsline{toc}{subsubsection}{EVALUATION of INTERCONNECTEDNESS and INTEGRITY}

\textbf{Unsatisfactory:} Does not include an evaluation of evidence of interconnectedness and integrity
on the company website. Does not integrate scholarly sources in the evaluation.

\textbf{Developing:} Includes a partial evaluation of evidence of interconnectedness and integrity on
the company website. Evaluation includes only limited reference to scholarly
sources.

\textbf{Proficient:} \emph{Includes a detailed evaluation of evidence of interconnectedness and integrity
on the company website. Evaluation integrates scholarly sources.}

\textbf{Exemplary:} Includes a detailed evaluation of evidence of interconnectedness and integrity
on the company website. Includes recommendations for ways in which to integrate
interconnectedness and integrity into employee development.

\hypertarget{analysis-of-adult-learning-strategies}{%
\subsubsection*{ANALYSIS of ADULT LEARNING STRATEGIES}\label{analysis-of-adult-learning-strategies}}
\addcontentsline{toc}{subsubsection}{ANALYSIS of ADULT LEARNING STRATEGIES}

\textbf{Unsatisfactory:} Does not include a detailed analysis of valued skills and evidence of adult
learning theory in employee development. Does not integrate scholarly sources.

\textbf{Developing:} Includes a partial analysis of valued skills and evidence of adult learning
theory in employee development. Analysis integrates few, if any, scholarly
sources.

\textbf{Proficient:} \emph{Includes a detailed analysis of valued skills and evidence of adult learning
theory in employee development. Analysis integrates scholarly sources.}

\textbf{Exemplary:} Includes a detailed analysis of valued skills and evidence of adult learning
theory in employee development. Includes recommendations for ways in which to
integrate adult learning theory into employee development.

\hypertarget{scholarly-integration-1}{%
\subsubsection*{SCHOLARLY INTEGRATION}\label{scholarly-integration-1}}
\addcontentsline{toc}{subsubsection}{SCHOLARLY INTEGRATION}

\textbf{Unsatisfactory:} Does not integrate scholarly references to support claims and assertions made in
the paper.

\textbf{Developing:} Integrates scholarly references to support some of the claims and assertions
made in the paper.

\textbf{Proficient:} \emph{Integrates scholarly references to support claims and assertions made in the
paper.}

\textbf{Exemplary:} Integrates scholarly references to support claims and assertions made in the
paper, effectively synthesizing different perspectives and research results from
scholarly sources.

\begin{longtable}[]{@{}
  >{\raggedright\arraybackslash}p{(\columnwidth - 8\tabcolsep) * \real{0.2000}}
  >{\raggedright\arraybackslash}p{(\columnwidth - 8\tabcolsep) * \real{0.2000}}
  >{\raggedright\arraybackslash}p{(\columnwidth - 8\tabcolsep) * \real{0.2000}}
  >{\raggedright\arraybackslash}p{(\columnwidth - 8\tabcolsep) * \real{0.2000}}
  >{\raggedright\arraybackslash}p{(\columnwidth - 8\tabcolsep) * \real{0.2000}}@{}}
\toprule\noalign{}
\begin{minipage}[b]{\linewidth}\raggedright
\textbf{TOTAL}
\end{minipage} & \begin{minipage}[b]{\linewidth}\raggedright
\textbf{0 = 0\% (F)}
\end{minipage} & \begin{minipage}[b]{\linewidth}\raggedright
\textbf{10 = 50\% (C)}
\end{minipage} & \begin{minipage}[b]{\linewidth}\raggedright
\textbf{15 = 75 (B)}
\end{minipage} & \begin{minipage}[b]{\linewidth}\raggedright
\textbf{20 = 100\% (A+)}
\end{minipage} \\
\midrule\noalign{}
\endhead
\bottomrule\noalign{}
\endlastfoot
\end{longtable}

\begin{center}\rule{0.5\linewidth}{0.5pt}\end{center}

\hypertarget{assignment-platform-paper}{%
\section*{Assignment: Platform Paper}\label{assignment-platform-paper}}
\addcontentsline{toc}{section}{Assignment: Platform Paper}

\begin{assessment}
For this assignment, you will write a contextualized Platform Paper in
which you discuss your ideal learning community and your role as
teacher/leader of that learning community. Select a context for your
paper (i.e.~facilitating in a FAR Centre in a specific country, teaching
adult learners, facilitating employee development workshops, etc.). Your
paper should be written and referenced in APA format and include
references to a minimum of 10 scholarly sources (this can include
literature you read in this course). You will write a draft of the
Platform Paper in Unit 8 and post for Peer Review. In Unit 9, you will
provide feedback to another learner on their paper. You will make
revisions based on the Peer Review and, in Unit 10, you will submit the
final Platform Paper. Peer reviewers will be assigned in advance.

\hypertarget{paper-outline}{%
\subsubsection{Paper Outline}\label{paper-outline}}

This paper will be 12-15 pages long, and should include: 1. Introduction
(1-2 pages) 2. Section 1: Ideal Learning Environment (5-7 pages) 3.
Section 2: Your Role as Teacher and Leader (5-7 pages) 4. Conclusion
(1-2 pages)

\hypertarget{paper-guidelines}{%
\subsubsection{Paper Guidelines}\label{paper-guidelines}}

\begin{itemize}
\tightlist
\item
  \textbf{Introduction}: Introduce the two sections in your paper,
  providing a brief description of the key points you will make in each
  section.
\item
  \textbf{Section 1}: In section one, you will describe your ideal
  education learning environment. This section should demonstrate your
  learning about authentic learning communities, incorporating scholarly
  sources and your own analysis to depict your ideal learning
  environment. Incorporate a discussion of the learning community
  environment, learning experiences, student learning outcomes, and
  personal beliefs about teaching and learning.
\item
  \textbf{Section 2}: In this section, describe your role as a teacher
  or leader within an authentic learning community. Incorporating
  scholarly literature, analyze your role as a facilitator/leader in
  planning learning experiences, facilitating student learning, and
  assessing student learning. Describe the actions, practices, and
  strategies you will engage in to achieve your vision of the learning
  community you described in section one.
\item
  \textbf{Conclusion}: Summarize the key points you made in each
  section.
\item
  \textbf{References}: Include a reference list with references to at
  least 10 scholarly sources.
\end{itemize}
\end{assessment}

\hypertarget{platform-paper-rubric}{%
\subsection*{Platform Paper Rubric}\label{platform-paper-rubric}}
\addcontentsline{toc}{subsection}{Platform Paper Rubric}

See the following rubric that explains how your assignment will be evaluated. Also available as a \href{assets/assessment/Platform-Paper-RUBRIC.pdf}{pdf}

\hypertarget{apawriting}{%
\subsubsection*{APA/WRITING}\label{apawriting}}
\addcontentsline{toc}{subsubsection}{APA/WRITING}

\textbf{Unsatisfactory:} Paper does not model language and conventions used in scholarly literature. Writing is not well-organized. Several errors in grammar or composition. Sources are not cited. APA citations are not appropriately formatted.

\textbf{Developing:} Paper partially models language and conventions used in scholarly literature. Writing is somewhat well organized and includes some errors in grammar or composition. Not all sources cited. APA citations are generally formatted correctly, with several errors.

\textbf{Proficient:} \emph{Paper consistently models language and conventions used in scholarly literature. Writing is well-organized and includes few (if any) errors in grammar or composition. All resources are appropriately cited (including in-text citations and bibliography information). Few (if any) errors in APA citations.}

\textbf{Exemplary:} Paper is an exemplar of language and conventions used in scholarly literature. Writing is well-organized and free of errors in grammar or composition. All resources are appropriately cited. No errors in APA format.

\hypertarget{developing-a-cohesive-and-logical-academic-argument-2}{%
\subsubsection*{DEVELOPING a COHESIVE and LOGICAL ACADEMIC ARGUMENT}\label{developing-a-cohesive-and-logical-academic-argument-2}}
\addcontentsline{toc}{subsubsection}{DEVELOPING a COHESIVE and LOGICAL ACADEMIC ARGUMENT}

\textbf{Unsatisfactory:} Does not make a focused, cohesive, or logical academic argument. Paper is confusing, and is missing an introduction, body, or conclusion. Transitions between sections and ideas are missing.

\textbf{Developing:} Makes an academic argument that is only partially focused, cohesive and logical. Paper is generally organized, but is missing an introduction, body, or conclusion. Transitions between sections and ideas are unclear.

\textbf{Proficient:} \emph{Makes a focused, cohesive, logical academic argument. Paper is effectively organized and includes an introduction, body, and conclusion. Transitions between sections and ideas are clear.}

\textbf{Exemplary:} Makes a focused, cohesive, logical and compelling academic argument. Paper is effectively organized and includes an introduction, body, and conclusion. Transitions between sections and ideas are clear, and build on each other.

\hypertarget{ideal-learning-environment}{%
\subsubsection*{IDEAL LEARNING ENVIRONMENT}\label{ideal-learning-environment}}
\addcontentsline{toc}{subsubsection}{IDEAL LEARNING ENVIRONMENT}

\textbf{Unsatisfactory:} Does not include a description of your ideal learning environment. Does not reference scholarly sources. Does note analyze key elements of an authentic learning community. Does not mention or describe the learning community environment, student learning outcomes, learning outcomes and personal beliefs about teaching and learning.

\textbf{Developing:} Includes a partial description of your ideal learning environment, referencing few scholarly sources and including a partial analysis of key elements of an authentic learning community. Mentions some elements, but does not fully describe the learning community environment, student learning outcomes, learning outcomes and personal beliefs about teaching and learning.

\textbf{Proficient:} \emph{Includes a detailed description of your ideal learning environment, referencing scholarly sources and analyzing key elements of an authentic learning community. Describes the learning community environment, student learning outcomes, learning outcomes and personal beliefs about teaching and learning.}

\textbf{Exemplary:} Includes a detailed description of your ideal learning environment, referencing scholarly sources and analyzing key elements of authentic learning communities. Provides a rationale for key elements of the learning community environment, student learning outcomes, learning outcomes and personal beliefs about teaching and learning. Advances scholarly thinking about authentic learning communities.

\hypertarget{your-role-as-teacher-and-leaders}{%
\subsubsection*{YOUR ROLE AS TEACHER AND LEADERS}\label{your-role-as-teacher-and-leaders}}
\addcontentsline{toc}{subsubsection}{YOUR ROLE AS TEACHER AND LEADERS}

\textbf{Unsatisfactory:} Does not include a description of your role as a teacher or leader within an authentic learning community, incorporating scholarly literature. Does not include an analysis of your role as a facilitator/leader in planning learning experiences, facilitating student learning, and assessing student learning. Does not include a description of the actions, practices, and strategies you will engage in to achieve your vision of the learning community you described in section one.

\textbf{Developing:} Includes a partial description of your role as a teacher or leader within an authentic learning community, incorporating scholarly literature. Describes but does not analyze your role as a facilitator/leader in planning learning experiences, facilitating student learning, and assessing student learning. Lists but does not describe the actions, practices, and strategies you will engage in to achieve your vision of the learning community you described in section one.

\textbf{Proficient:} \emph{Includes a detailed description of your role as a teacher or leader within an authentic learning community, incorporating scholarly literature. Includes a detailed analysis of your role as a facilitator/leader in planning learning experiences, facilitating student learning, and assessing student learning. Includes a detailed description of the actions, practices, and strategies you will engage in to achieve your vision of the learning community you described in section one.}

\textbf{Exemplary:} Includes a detailed analysis of your role as a teacher or leader within an authentic learning community, incorporating scholarly literature. Includes a detailed analysis of your role as a facilitator/leader in planning learning experiences, facilitating student learning, and assessing student learning. Includes a detailed description of the actions, practices, and strategies you will engage in to achieve your vision of the learning community you described in section one. Synthesizes scholarly thinking about the role of the teacher/leader.

\hypertarget{scholarly-integration-2}{%
\subsubsection*{SCHOLARLY INTEGRATION}\label{scholarly-integration-2}}
\addcontentsline{toc}{subsubsection}{SCHOLARLY INTEGRATION}

\textbf{Unsatisfactory:} Does not integrate many references to support the arguments made in the paper.

\textbf{Developing:} Integrates fewer than 10 scholarly sources to support arguments made in the paper.

\textbf{Proficient:} \emph{Integrates a minimum of 10 scholarly sources to support arguments made in each section of the paper.}

\textbf{Exemplary:} Integrates a minimum of 10 references to support the arguments made in each section, including several scholarly sources not included in course materials.

\begin{longtable}[]{@{}
  >{\raggedright\arraybackslash}p{(\columnwidth - 8\tabcolsep) * \real{0.2000}}
  >{\raggedright\arraybackslash}p{(\columnwidth - 8\tabcolsep) * \real{0.2000}}
  >{\raggedright\arraybackslash}p{(\columnwidth - 8\tabcolsep) * \real{0.2000}}
  >{\raggedright\arraybackslash}p{(\columnwidth - 8\tabcolsep) * \real{0.2000}}
  >{\raggedright\arraybackslash}p{(\columnwidth - 8\tabcolsep) * \real{0.2000}}@{}}
\toprule\noalign{}
\begin{minipage}[b]{\linewidth}\raggedright
\textbf{TOTAL}
\end{minipage} & \begin{minipage}[b]{\linewidth}\raggedright
\textbf{0 = 0\% (F)}
\end{minipage} & \begin{minipage}[b]{\linewidth}\raggedright
\textbf{10 = 50\% (C)}
\end{minipage} & \begin{minipage}[b]{\linewidth}\raggedright
\textbf{15 = 75 (B)}
\end{minipage} & \begin{minipage}[b]{\linewidth}\raggedright
\textbf{20 = 100\% (A+)}
\end{minipage} \\
\midrule\noalign{}
\endhead
\bottomrule\noalign{}
\endlastfoot
\end{longtable}

\hypertarget{references}{%
\chapter*{References}\label{references}}
\addcontentsline{toc}{chapter}{References}

The following are key references used in this course. \textbf{\emph{Check with your course syllabus for required readings.}}

  \bibliography{book.bib}

\end{document}
